% override specific chktex warnings
% chktex-file 46 - don't use $ instead of \(, etc)
% chktex-file 1 - ignore commands followed by a space, e.g. \\ new line here
% chktex-file 9 - sometimes messes up with ( and {

\documentclass[10pt,onesided]{amsart}
\usepackage[showboxes]{textpos}

\usepackage{amsmath}
\usepackage{amsthm}
\usepackage{amsfonts}
\usepackage{amssymb}
\usepackage{mathpazo}
\usepackage{booktabs}
\usepackage[usenames,dvipsnames]{color}
\usepackage[many]{tcolorbox}
\usepackage{tikz}
\usepackage{textcomp}
\usepackage[letterpaper]{geometry}
\geometry{verbose,tmargin=0.5in,bmargin=0.5in,lmargin=0.75in,rmargin=0.75in}
\usepackage{multicol}
\usepackage{bm}
\usepackage{comment}
\usepackage{cancel}
\pagestyle{empty}
\raggedright
\renewcommand{\familydefault}{\sfdefault}
\setlength{\parskip}{\medskipamount}
\setlength{\columnsep}{2cm}

\everymath{\displaystyle}
\setlength{\parskip}{\bigskipamount}

% counter for resuming enumerated list numbers
\newcounter{resumeenumi}
\newcommand{\suspend}{\setcounter{resumeenumi}{\theenumi}}
\newcommand{\resume}{\setcounter{enumi}{\theresumeenumi}}

\newcommand\lb{\linebreak}
\newcommand\pars{\par\smallskip}
\newcommand\parm{\par\medskip}
\newcommand\parb{\par\bigskip}

\makeatletter
\providecommand{\gettikzxy}[3]{%
	\tikz@scan@one@point\pgfutil@firstofone#1\relax
	\edef#2{\the\pgf@x}%
	\edef#3{\the\pgf@y}%
}
\makeatother



% full width colored block but color specifiable
%\cb[body bg strength]{header bg}{header text}{body text}
\newcommand{\cb}[4][15]{
	\setbeamercolor{block title}{bg = #2}
	\setbeamercolor{block body}{bg = #2!#1}
	\setbeamercolor{item projected}{bg=#2, fg=white}
	\begin{center}
		\begin{block}{#3}
			#4
		\end{block}
	\end{center}
}

% colored block with width specified
% \cbw[body bg strength]{header bg}{width}{header text}{body text}
\newcommand{\cbw}[5][15]{
	\begin{center}
		%\vspace{-0.35cm}
		\begin{minipage}{#3\textwidth}
			\setbeamercolor{block title}{bg= #2}
			\setbeamercolor{block body}{bg= #2!#1}
			\setbeamercolor{item projected}{bg=#2, fg=white}
			\begin{block}{#4}
				\raggedright
				#5
			\end{block}
		\end{minipage}
	\end{center}
}

% centered minipage with text \raggedright
%\cmini[width]{content}
\newcommand{\cmini}[2][0.8]{
	\begin{center}
		\begin{minipage}{#1\columnwidth}
			\raggedright
			#2
		\end{minipage}
	\end{center}
}

%left flushed minipage
\newcommand{\mini}[2][0.8]{
	\begin{minipage}{#1\columnwidth}
		\raggedright
		#2
	\end{minipage}
}

%left flushed minipage, top aligned
\newcommand{\minit}[2][0.8]{
	\begin{minipage}[t]{#1\columnwidth}
		\raggedright
		#2
	\end{minipage}
}

%left flushed minipage
% \newcommand{\miniT}[2][0.8]{
%  \begin{minipage}[T]{#1\columnwidth}
%   \raggedright
%   #2
%  \end{minipage}
% }

%left flushed minipage
\newcommand{\minib}[2][0.8]{
	\begin{minipage}[b]{#1\columnwidth}
		\raggedright
		#2
	\end{minipage}
}

\newcommand{\cfig}[2][1]{% centred, scaled graphic
	\begin{center}
		\includegraphics[scale=#1]{#2}
	\end{center}
}
% figure with tight border for photos
% \cfigb[saitMaroon]{borderwidth with unit}{scale}{image}
\newcommand{\cfigb}[4][structure]{
	% \usepackage{adjustbox}
	\setlength{\fboxrule}{1pt}
	\begin{center}
		\includegraphics[scale=#3, cframe= #1 #2]{#4}
	\end{center}
}

\newcommand{\imgbox}[3]{
	% \setlength{\fboxsep}{12pt}
	\includegraphics[scale=#1, cframe= structure #3]{#2}
}

% \imgboxbg[bg color=white]{scale}{path/to/img}{border color}{border, e.g. 2pt}{margin, e.g. 4pt}
\newcommand{\imgboxbg}[6][white]{
	\setlength{\fboxrule}{#5}
	\setlength{\fboxsep}{#6}
	\centering
	\fcolorbox{#4}{#1}{\includegraphics[scale=#2]{#3}}
}

\newcommand{\fig}[2][1]{% scaled graphic
	\includegraphics[scale=#1]{#2}
}

% centred framed  box black border
%\cbox[width]{content}
\newcommand{\cbox}[2][0.9]{% framed centered  box
	\setlength\fboxsep{0.042\columnwidth}
	\setlength\fboxrule{0.0015\columnwidth}
	\begin{center}
		\fcolorbox{black}{white}{
			\vspace{-0.5cm}
			\begin{minipage}{#1\columnwidth}
				\raggedright
				#2
			\end{minipage}
		}
	\end{center}
	\setlength\fboxsep{0cm}
}



\newtcolorbox{mybox}[1][]
{
	colback=white,
	top=0.25cm,
	bottom=0.25cm,
	left=0.25cm,
	right=0.25cm,
	colframe=structure,
	fonttitle=\bfseries,
	enhanced, drop fuzzy shadow,
	% attach boxed title to top left={yshift=-2mm, xshift=5mm},
	attach boxed title to top left={yshift=-2mm, xshift=5mm}, colbacktitle=structure!80!white, #1}

\newtcolorbox{plainbox}[1][]{colback=white, sharp corners, top=0.125cm, bottom=0.125cm, left=0pt, right=0pt, boxrule=0.5pt,colframe=structure,fonttitle=\bfseries, colbacktitle=structure, arc=0mm, #1}
%
\newtcbtheorem{myexam}{Example}%
{
	enhanced,
	colback=white,
	top=0.375cm,
	bottom=0.25cm,
	left=0.375cm,
	right=0.375cm,
	colframe=structure,
	fonttitle=\bfseries,
	drop fuzzy shadow,
	%description font=\mdseries\itshape,
	attach boxed title to top left={yshift=-2mm, xshift=5mm},
	colbacktitle=structure!80!white
	}{exam}% then \pageref{exer:theoexample} references the theo

\newcommand{\myexample}[2][red]{
	% \tcb\tcbset{theostyle/.style={colframe=red,colbacktitle=yellow}}
	\begin{myexam}{}{}
		\raggedright
		#2
	\end{myexam}
	% \tcbset{colframe=structure,colbacktitle=structure}
}

\newtcbtheorem{myexer}{Exercise}%
{
	enhanced,
	colback=white,
	top=0.375cm,
	bottom=0.25cm,
	left=0.375cm,
	right=0.375cm,
	colframe=structure,
	fonttitle=\bfseries,
	drop fuzzy shadow,
	%description font=\mdseries\itshape,
	attach boxed title to top left={yshift=-2mm, xshift=5mm},
	colbacktitle=structure!80!white
	}{exer}

\newcommand{\myexercise}[2][red]{
	% \tcb\tcbset{theostyle/.style={colframe=red,colbacktitle=yellow}}
	\begin{myexer}{}{}
		\raggedright
		#2
	\end{myexer}
	% \tcbset{colframe=structure,colbacktitle=structure}
}


\begin{document}

\thispagestyle{empty}
\vspace{-7cm}
\centering

\textbf{\Large Module 2: Forces Due To Static Fluids (CIVL 318)}
\parb
\begin{center}
	\begin{tabular}{rrcl}
		\toprule
		\addlinespace
		\textbf{Pressure and forces on plane areas}: & $P_{avg}$ & = & $\gamma{h_C}$      \\
		\addlinespace
		                                             & $F_R$     & = & $\gamma{h_C}A$     \\
		\addlinespace
		\midrule
		\addlinespace
		\textbf{Centre of pressure for plane areas}: & $L_p-L_c$ & = & $\frac{I_c}{L_cA}$ \\
		\addlinespace
		\bottomrule
	\end{tabular}
\end{center}


%%%%%%%%%%%%%%%%%%%%%%%%%%%%%%%%%%%%%%%%%%%%%%%%%%%%%%%%%%%%%%%%%%%%%%%%%%%%%%%%%%%%%%%%%%%%%%%%%%%%%%%%%%%%%%%%%%%%%

\rule{\columnwidth}{0.02in}
\parb
\raggedright

%%%%%%%%%%%%%%%%%%%%%%%%%%%%%%%%%%%%%%%%%%%%%%%%%%%%%%%%%%%%%%%%%%%%%%%%%%%%%%%%%%%%%%%%%%%%%%%%%%%%%%%%%
\begin{minipage}{0.35\columnwidth}
	\raggedright
	\textbf{Example 1}:
	\begin{cfig}[0.4]{../../figs/02StaticForces/03staticforces01}\end{cfig}
	Determine the force exerted by the oil and water upon the bottom plane surface of the barrel
\end{minipage}

\newpage
%%%%%%%%%%%%%%%%%%%%%%%%%%%%%%%%%%%%%%%%%%%%%%%%%%%%%%%%%%%%%%%%%%%%%%%%%%%%%%%%%%%%%%%%%%%%%%%%%%%%%%%%%

\begin{minipage}[t]{0.5\textwidth}
	\raggedright
	\textbf{Example 2}:\parb
	\parb
	\begin{cfig}[0.45]{../../figs/02StaticForces/03staticforces02}\end{cfig}
	A pressurized tank contains liquid with sg= $1.59$. \\
	There are rectangular inspection hatches at $A$ ($400\text{ mm}\times 250\text{ mm}$) and at $B$ ($500\text{ mm}\times 750\text{ mm}$).
	\parb
	Determine the force exerted by the fluid on the hatch at $A$.
	\vspace{4cm}
\end{minipage}
% \hfill
% \begin{minipage}[t]{0.45\textwidth}
% 	\textbf{Solution}:
% 	\parb
% 	\cbox[0.9]{
% 		\Large
% 		\begin{eqnarray*}
% 			p_A & = & 133\text{ kPa}\\
% 			&  & \quad+\;(1.59)(9.81\text{ kN/m}^3)(3.0\text{ m})\\
% 			& = & 133\text{ kPa}+46.794\text{ kPa}\\
% 			& = & 179.79\text{ kPa}\\\\
% 			F_A & = & p_A\cdot A_A\\
% 			& = & (179.79\text{ kN/m}^2)(0.4\text{ m}\times0.25\text{ m})\\
% 			& = & 17.979\text{ kN}\\\\
% 			F_A & \approx & 17.98\text{ kN}
% 		\end{eqnarray*}
% 	}
% \end{minipage}
\parb
\rule{\textwidth}{0.02in}
\parb

\begin{minipage}[t]{0.5\textwidth}
	\raggedright
	\textbf{Exercise 1}:\parb
	
	Determine the force exerted by the fluid on the hatch at $B$ for the pressurized tank in the previous example.
\end{minipage}
% \hfill
% \begin{minipage}[t]{0.45\textwidth}
% 	\textbf{Solution}:
% 	\parb
% 	\cbox[0.9]{
% 		\Large
% 		\begin{eqnarray*}
% 			p_B & = & 133\text{ kPa}\\
% 			&  & \quad+\;(1.59)(9.81\text{ kN/m}^3)(6.37\text{ m})\\
% 			& = & 133\text{ kPa}+99.358\text{ kPa}\\
% 			& = & 232.36\text{ kPa}\\\\
% 			F_B & = & p_B\cdot A_B\\
% 			& = & (232.36\text{ kN/m}^2)(0.5\text{ m}\times0.75\text{ m})\\
% 			& = & 87.134\text{ kN}\\\\
% 			F_B & \approx & 87.1\text{ kN}
% 		\end{eqnarray*}
% 	}
% \end{minipage}

\newpage
%%%%%%%%%%%%%%%%%%%%%%%%%%%%%%%%%%%%%%%%%%%%%%%%%%%%%%%%%%%%%%%%%%%%%%%%%%%%%%%%%%%%%%%%%%%%%%%%%%%%%%%%%

\begin{minipage}{0.4\textwidth}
	\raggedright
	\textbf{Example 3}:\\
	\vspace*{-1cm}
	\begin{cfig}[0.6]{../../figs/02StaticForces/03staticforcesEx3}\end{cfig}
	The tie-bars have cross-sectional dimension of $90\text{mm}\times 90\text{mm}$.
	Determine the normal stress in the tie-bars, and the bearing stress
	if the vertical posts are cut halfway into each tie-bar.		\par
	(Assume a pinned connection at the bottom of the sidewalls and treat pressure as though the fluid is static.)
	\parb
	% \textbf{Solution}:
	% \parb
	% \begin{cfig}[0.6]{../../figs/02StaticForces/03staticforcesEx3b}\end{cfig}
	% \cbox[0.9]{
	% 	\Large
	% 	\begin{align*}
	% 		p_{avg} & = \gamma\frac{h}{2}                                                              \\
	% 		        & = \left(9.81\,\text{kN/m}^3\right)\left(\frac{1.65}{2}\,\text{m}\right)          \\
	% 		        & = 	8.0933\,\text{kPa}                                                            \\\\
	% 		F_R     & = p_{avg}A                                                                       \\
	% 		        & = \left(8.0933\,\text{kPa}\right)\left(0.6\,\text{m}\times 1.65\,\text{m}\right) \\
	% 		        & = 8.0123\thinspace\text{kN}
	% 	\end{align*}
	% }
\end{minipage}
% \hfill
% \begin{minipage}{0.5\textwidth}
% 	\vspace{-5cm}
% 	\begin{cfig}[0.6]{../../figs/02StaticForces/03staticforcesEx3c}\end{cfig}
% 	\cbox[0.9]{
% 		\Large
% 		Draw a free body diagram of the forces acting on one wall and take moments about the pinned
% 		connection at the bottom of the wall
% 		\begin{align*}
% 			\Sigma M_A & = (1.95\,\text{m})T-(8.0123\,\text{kN})(0.55\,\text{m})=0 \\
% 			T          & = \frac{8.0123\times 0.55}{1.95}\,\text{kN}               \\
% 			           & = 2.2599\,\text{kN}
% 		\end{align*}
% 		The normal stress is given by $\sigma=F/A$:
% 		\begin{align*}
% 			\sigma & = \frac{2.2599\,\text{kN}}{(0.09\,\text{m}\times 0.09\,\text{m})} \\
% 			       & = 279.0\,\text{kPa}
% 		\end{align*}
% 		The bearing stress is given by $\sigma_b=F/A_b$. Here, the bearing area is half the cross-section of the tie-bar:
% 		\begin{align*}
% 			\sigma_b & = \frac{2.2599\,\text{kN}}{(0.09\,\text{m}\times 0.045\,\text{m})} \\
% 			         & = 558.0\,\text{kPa}
% 		\end{align*}
% 	}
%
% \end{minipage}
\newpage
%%%%%%%%%%%%%%%%%%%%%%%%%%%%%%%%%%%%%%%%%%%%%%%%%%%%%%%%%%%%%%%%

\begin{minipage}[t]{0.45\textwidth}
	\raggedright
	\textbf{Example 4}:\\
	\begin{cfig}[0.6]{../../figs/02StaticForces/03staticforcesEx04}\end{cfig}
	The wall has a rectangular plane area in contact with the water, is
	inclined at $60^\circ$ to the horizontal and is $17$ m long.
	\par\smallskip
	Determine the force exerted on the dam plane area by the water.
	\vspace{2cm}
\end{minipage}
\hfill
% \begin{minipage}[t]{0.5\textwidth}
% 	\textbf{Solution}:
% 	\parm
% 	\cbox[0.9]{
% 		\Large
% 		The average pressure for the area in contact with the water is at $h_C$, i.e. at half-depth:
% 		\begin{align*}
% 			p_{avg} & =\gamma h_C                                                             \\
% 			        & = \left(9.81\,\text{kN/m}^3\right)\left(\frac{5.25\,\text{m}}{2}\right)
% 			        & = 25.751\,\text{kN/m}^2
% 		\end{align*}
% 		The area $A$ of the dam wall is:
% 		\begin{align*}
% 			A & =\left(\frac{5.25}{\sin 60^\circ}\,\text{m}\right)(17\,\text{m}) \\
% 			  & = 103.06\,\text{m}^2
% 		\end{align*}
% 		The resultant force, $F_R$, is:
% 		\begin{align*}
% 			F_R & = 25.751\,\text{kN/m}^2 \times 103.06\,\text{m}^2 \\
% 			    & = 2653.9\,\text{kN}                               \\
% 			    & \approx 2650\,\text{kN}
% 		\end{align*}
% 	}
% \end{minipage}
\newpage
%%%%%%%%%%%%%%%%%%%%%%%%%%%%%%%%%%%%%%%%%%%%%%%%%%%%%%%%%%%%%%%%



\begin{minipage}[t]{0.45\textwidth}
	\raggedright
	\textbf{Example 5}:
	\parb
	A vertical retaining wall supports water to a depth of 4.75 m. There
	is a rectangular hatch in the wall. The top of the hatch is at a depth of 1.25~m; the hatch is 2.25~m wide
	$\times$ 1.5~m high.
	\parb
	What is the magnitude of the force exerted upon the hatch by the water?
	% 			\item A second hatch has dimensions $3.75$ m wide$\times 1.6$ m high. At
	% 			what depth can the top of this hatch be placed below the surface if
	% 			the maximum allowable force for the hatch is $128$ kN?
	% 			\end{enumerate}
	% 			\parb
	%
	%
	% 			%\begin{cfig}[0.55]{../../figs/02StaticForces/03staticforcesEx05}\end{cfig}
	%
\end{minipage}
% \hfill
% \begin{minipage}[t]{0.5\textwidth}
% 	\textbf{Solution }
% 	\cbox[0.9]{
% 		\Large
% 		The average pressure on the hatch is at centre-height of the hatch which is at:
% 		$$ h=1.25\,\text{m}+\frac{1.5\,\text{m}}{2}=2.0\,\text{m} $$
%
% 		\begin{align*}
% 			p_{avg} & = \gamma h                                                                          \\
% 			        & = \left(9.81\,\text{kN/m}^3\right)(2.0\,\text{m})                                   \\
% 			        & = 19.62\,\text{kN/m}^2                                                              \\\\
% 			F_R     & =p_{avg}A                                                                           \\
% 			        & = \left(19.62\,\text{kN/m}^2\right)\left(2.25\,\text{m}\times 1.50\,\text{m}\right) \\
% 			        & = 66.218\,\text{kN}                                                                 \\
% 			        & \approx 66.2\,\text{kN}                                                             \\
% 		\end{align*}
% 	}
% \end{minipage}
\parb\vspace{6cm}
\rule{\textwidth}{0.02in}

%%%%%%%%%%%%%%%%%%%%%%%%%%%%%%%%%%%%%%%%%%%%%%%%%%%%%%%%%%%%%%%%

\begin{minipage}[t]{0.35\textwidth}
	\raggedright
	\textbf{Exercise 2}:
	\parb
	A vertical retaining wall supports water to a depth of 6.25 m. There
	is a rectangular hatch in the wall. The hatch has dimensions $3.75\,\text{m wide}\times 1.6\,\text{m height}$.
	\parb
	At what depth below the surface can the top of this hatch be placed if the maximum allowable force on the hatch is $128\,$kN?
\end{minipage}
% \hfill
% \begin{minipage}[t]{0.6\textwidth}
% 	\textbf{Solution }
% 	\cbox[0.9]{
% 		\Large
% 		The average pressure on the hatch is at $h=d+0.8\,$m, where $d$ is the depth of the top of the hatch.
% 		The force on the hatch is:
% 		\begin{align*}
% 			F               & =p_{avg}A                                                                                                     \\
% 			128\,\text{kN}  & = \left(9.81\,\text{kN/m}^3\right)\left(d+0.8\,\text{m}\right)\left(1.6\,\text{m}\times 3.75\,\text{m}\right) \\
% 			d+0.8\,\text{m} & = \frac{128\,\text{kN}}{\left(9.81\,\text{kN/m}^3\right)\left(3.75\,\text{m}\times 1.6\,\text{m}\right)}      \\
% 			                & = 2.1724\,\text{m}                                                                                            \\\\
% 			d               & \approx 1.372\,\text{m}
% 		\end{align*}
% 	}
% \end{minipage}
\newpage
%%%%%%%%%%%%%%%%%%%%%%%%%%%%%%%%%%%%%%%%%%%%%%%%%%%%%%%%%%%%%%%%

\begin{minipage}[t]{0.45\textwidth}
	\textbf{Example 6}:\\
	\begin{cfig}[0.65]{../../figs/02StaticForces/03staticforcesEx06a}\end{cfig}
	\raggedright
	A tank containing kerosene (sg=$0.823$) has a triangular inspection hatch in a vertical sidewall. The hatch has
	a base of $400\,\text{mm}$ and a height of $600\,\text{mm}$. The top of the hatch is located at a depth of $700\,\text{mm}$.
	\par\smallskip
	Determine the force exerted on the hatch by the kerosene.
\end{minipage}
% \hfill
% \begin{minipage}[t]{0.5\textwidth}
% 	\textbf{Solution}:
% 	\parb
% 	\cbox[0.9]{
% 		The height of the triangular hatch is $600$ mm so the location of the centroidal $x$-axis is $200$ mm above the base\par
% 		Thus the average pressure on the hatch is the pressure at a depth of $700 + (600-200)=1100$ mm below the surface.\par
% 		Then
% 		\begin{align*}
% 			p_{avg} & = \gamma h_C                                                                                 \\
% 			        & = (0.823)\left(9.81\,\text{kN/m}^3\right)(1.10\,\text{m})                                    \\
% 			        & = 8.8810\,\text{kPa}                                                                         \\\\
% 			F_R     & = p_{avg}A                                                                                   \\
% 			        & = \left(8.8810\,\text{kN/m}^2\right)\left(0.5\times 0.4\,\text{m}\times 0.6\,\text{m}\right) \\
% 			        & \approx 1.066\,\text{kN}
% 		\end{align*}
% 	}
% \end{minipage}

\newpage
%%%%%%%%%%%%%%%%%%%%%%%%%%%%%%%%%%%%%%%%%%%%%%%%%%%%%%%%%%%%%%%%


\begin{minipage}[t]{0.45\textwidth}
	\textbf{Example 7}:\\
	\begin{cfig}[0.35]{../../figs/02StaticForces/03staticforcesEx07a}\end{cfig}
	\raggedright
	A water tank has a semi-circular inspection hatch, as illustrated.
	\par\smallskip
	Determine the force exerted on the hatch by the water. \par
	\begin{center}
		(For a semi-circle, $\bar{y}=\frac{4r}{3\pi}$.)
	\end{center}
\end{minipage}
% \hfill
% \begin{minipage}[t]{0.5\textwidth}
% 	\textbf{Solution}:
% 	\parb
% 	\cbox[0.9]{
% 		\Large
% 		\[\bar{y} = \frac{4(600\,\text{mm})}{3\pi} = 254.65\,\text{mm}\]
% 	}
% 	\parb
% 	\begin{cfig}[0.5]{../../figs/02StaticForces/03staticforcesEx07b}\end{cfig}
% 	\parb
% 	\cbox[0.9]{
% 		\Large
% 		\begin{align*}
% 			h_c     & = (345.35\,\text{mm}+250\,\text{mm})\cos 40^\circ + 750\,\text{mm}     \\
% 			        & = 1206.1\,\text{mm}                                                    \\\\
% 			p_{avg} & = \gamma h_c = \left(9.81\,\text{kN/m}^3\right)\times 1.2061\,\text{m} \\
% 			        & = 11.832\,\text{kPa}                                                   \\\\
% 			A       & = \pi(0.6\,\text{m})^2/2 = 0.56549\,\text{m}^2                         \\\\
% 			F_R     & = p_{avg}A                                                             \\
% 			        & = \left(11.832\,\text{kN/m}^2\right)\times(0.56549\,\text{m}^2)        \\
% 			        & = 6.6908\,\text{kN}                                                    \\
% 			        & = 6.69\,\text{kN}                                                      \\
% 		\end{align*}
% 	}
% \end{minipage}

\newpage
\begin{minipage}[t]{0.45\textwidth}
	\textbf{Example 8:}
	\parb
	\raggedright
	
	\begin{cfig}[0.65]{../../figs/02StaticForces/03staticforcesEx09a}\end{cfig}
	A tank containing castor oil has a $1.0\,\text{m wide}\times 600 \,\text{mm}$ high rectangular inspection hatch.\parb
	The top of the hatch is $2.5\,\text{m}$ below the surface of the castor oil. The hatch cover is attached to
	the tank by $8$ bolts, four at the top of the hatch and four at the bottom.\parb
	The bolts are offset the from the hatch opening by 	$100\,\text{mm}$, as shown.\parb
	Calculate the tension in each of the top and in each of the bottom bolts.\parb
	(Assume that all the top bolts have the same tension and that all the bottom bolts have the same tension.)
	% \parb
	% \textbf{Solution}:
	% \cbox[0.9]{
	% 	\Large
	% 	From the hatch calculations:
	% 	\begin{align*}
	% 		A   & = 1.0\,\text{m}\times 0.6\,\text{m} = 0.6\,\text{m}^2                                  \\
	% 		I_c & = \frac{bh^3}{12}=\frac{(1.0\,\text{m})\times (0.6\,\text{m})^3}{12}=0.018\,\text{m}^4 \\
	% 		L_c & = 2.5\,\text{m}+\frac{0.6\,\text{m}}{2}=2.8\,\text{m}
	% 	\end{align*}
	% 	Then
	% 	\begin{align*}
	% 		L_p -L_c & = \frac{I_c}{L_c\cdot A}                                            \\
	% 		         & = \frac{0.018\,\text{m}^4}{(2.8\,\text{m})\times (0.6\,\text{m}^2)} \\
	% 		         & = 0.010714\,\text{m}
	% 	\end{align*}
	% 	The centre of pressure is $10.714\,\text{mm}$ below the centroid of the hatch.
	% }
\end{minipage}
% \hfill
% \begin{minipage}[c]{0.5\textwidth}
% 	\raggedright
% 	\cbox[0.9]{
% 		\Large
% 		The average pressure on the hatch:
% 		\begin{align*}
% 			p_{avg} & = \gamma L_c                                            \\
% 			        & = (0.96)\left(9.81\,\text{kN/m}^3\right)(2.8\,\text{m}) \\
% 			        & = 26.369\,\text{kN/m}^2
% 		\end{align*}
% 		Then the magnitude of the resultant force on the hatch is:
% 		\begin{align*}
% 			F_R & = p_{avg}A                                                              \\
% 			    & = \left(26.369\,\text{kN/m}^2\right)(1.0\,\text{m}\times 0.6\,\text{m}) \\
% 			    & = 15.822\,\text{kN}
% 		\end{align*}
% 	}
% 	\par\vspace{0.5cm}
% 	\cbox[0.9]{
% 		\Large
% 		The upper row of bolts is at a depth of $2.4\,\text{m}$ and the lower row is at $3.2\,\text{m}$.\parb
% 		The upper row of bolts is $0.41071\,\text{m}$ above the centre of pressure\parb
% 		The lower row of bolts is $0.38929\,\text{m}$ below the centre of pressure\parb
% 	}
% 	\par\vspace{0.5cm}
% 	\begin{cfig}[0.75]{../../figs/02StaticForces/03staticforcesEx09b}\end{cfig}
% 	\par\vspace{0.5cm}
% 	\cbox[0.9]{
% 		\large
% 		Let $S$ and $T$ be the tensions in each upper and each lower bolt respectively. Then
% 		\begin{align*}
% 			\Sigma M_B & = 4S\times(0.8\,\text{m})-(0.38929\,\text{m})\times(15.822\,\text{kN})=0 \\
% 			S          & = 1.925\,\text{kN}                                                       \\\\
% 			\Sigma M_A & = -4T\times(0.8\,\text{m})-(0.4107\,\text{m})\times(15.822\,\text{kN})=0 \\
% 			S          & = 2.03\,\text{kN}                                                        \\\\
% 		\end{align*}
% 	}
% \end{minipage}



\newpage

%%%%%%%%%%%%%%%%%%%%%%%%%%%%%%%%%%%%%%%%%%%%%%%%%%%%%%%%%%%%%%%%


\begin{minipage}[t]{0.4\textwidth}
	\textbf{Example 9}:\\
	\begin{cfig}[0.75]{../../figs/02StaticForces/03staticforcesEx08a}\end{cfig}
	\raggedright
	This is an example of a ``self-levelling'' gate. It is hinged along its top edge \parb
	When the water exceeds a certain height, the hydrostatic force on the gate is sufficient to
	open the gate. Water drains until the level allows the gate to close again. \parb
	Find the value $d$ for which the gate opens.
	% \parb
	% \begin{cfig}[0.9]{../../figs/02StaticForces/03staticforcesEx08c}\end{cfig}
\end{minipage}
% \hfill
% \begin{minipage}[t]{0.55\textwidth}
% 	\textbf{Solution}:
% 	\parb
% 	\cbox[0.9]{
% 		\Large
% 		We don't know $L_c$ but we can calculate a few results in terms of $L_c$
% 		\begin{align*}
% 			%L_c &= (d+0.5)\,\text{m}\\\\
% 			A       & = (0.9\,\text{m})\times (1.0\,\text{m})=0.9\,\text{m}^2                                          \\\\
% 			I_c     & = \frac{bh^3}{12}=\frac{(0.9\,\text{m})\left( 1.0\,\text{m}\right)^3}{12}=0.075\,\text{m}^4      \\\\
% 			L_p-L_c & = \frac{I_c}{L_cA}=\frac{0.075\,\text{m}^4}{L_c(0.9\,\text{m})^2}=\frac{0.083333}{L_c}\,\text{m} \\
% 			p_{avg} & = \gamma h = \left(9.81\,\text{kN/m}^3\right)(L_c\,\text{m})=9.81L_c\,\text{kN/m}^2              \\
% 			F_R     & = \left(9.81L_c\,\text{kN/m}^2\right)(0.9\,\text{m})=8.8290 L_c\,\text{kN}
% 		\end{align*}
% 		When the gate just begins to open, the reaction at $B$ is $0$. Then, if we take moments about $A$, there are only
% 		two moments to consider:
% 		the moment due to the resultant force, $F_R$, of the water pressure and the moment due to the $10\,\text{kN}$ weight:
% 		\begin{align*}
% 			\Sigma M_A & = F_R\left(0.5+\left(L_p-L_c\right)\right)-10(0.75)  \\
% 			           & = 8.8290L_c\left(0.5+\frac{0.083333}{L_c}\right)-7.5 \\
% 			           & = 4.4145L_c+0.73572-7.5                              \\
% 			           & =0                                                   \\\\
% 			L_c        & = 1.5322\,\text{m}                                   \\
% 		\end{align*}
% 		The centroid of the gate is $1.5322L_c\,\text{m}$ from the surface when the gate opens.\par
% 		Therefore,
% 		$$ d \approx 1.032\,\text{m} $$
% 	}
% \end{minipage}

\newpage



\begin{minipage}[t]{0.45\textwidth}
	\textbf{Example 10}:\\
	\begin{cfig}[0.7]{../../figs/02StaticForces/03staticforcesEx10}\end{cfig}
	\raggedright
	Determine the tension $T$ in the upper bolt and tension $S$ in each of the lower bolts.
	% \parb
	% \textbf{Solution}:
	% \parb
	% \cbox[0.9]{
	% 	\begin{align*}
	% 		L_c     & = \frac{3.05\,\text{m}}{\sin54^\circ}+0.6\,\text{m}       \\
	% 		        & = 4.3700\,\text{m}                                        \\\\
	% 		h_c     & = L_c\sin54^\circ                                         \\
	% 		        & =3.5354\,\text{m}                                         \\\\
	% 		p_{avg} & = \gamma h_c                                              \\
	% 		        & = 0.823\left(9.81\,\text{kN/m}^3\right)(3.5354\,\text{m}) \\
	% 		        & = 28.544\,\text{kPa}                                      \\\\
	% 		A       & = \frac{(0.9\,\text{m})\times (1.0\,\text{m})}{2}         \\
	% 		        & =0.45\,\text{m}^2                                         \\\\
	% 		F_R     & = \left(28.544\,\text{kN/m}^2\right)(0.45\,\text{m}^2)    \\
	% 		        & = 12.845\,\text{kN}
	% 	\end{align*}
	% }
\end{minipage}
% \hfill
% \begin{minipage}[t]{0.5\textwidth}
%
% 	\parb
% 	\cbox[0.9]{
% 		\begin{align*}
% 			I_c     & = \frac{bh^3}{36}                                                             \\
% 			        & =\frac{(1.0\,\text{m})\left( 0.9\,\text{m}\right)^3}{36}                      \\
% 			        & =0.020250\,\text{m}^4                                                         \\\\
% 			L_p-L_c & = \frac{I_c}{L_cA}                                                            \\
% 			        & =\frac{0.020250\,\text{m}^4}{(4.3700\,\text{m})\left(0.45\,\text{m}^2\right)} \\
% 			        & =0.010297\,\text{m}
% 		\end{align*}
% 	}
% 	\begin{cfig}{../../figs/02StaticForces/03staticforcesEx10b}\end{cfig}
% 	\cbox[0.9]{
% 		\parb
% 		Sum the moments about the lower bolts at $A$:
% 		\begin{align*}
% 			\Sigma M_A    & = (T\,\text{kN})(1.2\,\text{m})-(12.845\,\text{kN})(0.43970\,\text{m}) \\
% 			              & = 0                                                                    \\
% 			\Rightarrow	T & = \frac{(12.845\,\text{kN})(0.43970\,\text{m})}{(1.2\,\text{m})}       \\
% 			              & = 4.7066\,\text{kN}                                                    \\
% 			              & \approx 4.71\,\text{kN}                                                \\
% 		\end{align*}
% 		\parb
% 		Sum the moments about the upper bolt at $B$:
% 		\begin{align*}
% 			\Sigma M_B    & = (12.845\,\text{kN})(0.76030\,\text{m})-(2S\,\text{kN})(1.2\,\text{m}) \\
% 			              & = 0                                                                     \\
% 			\Rightarrow	S & = \frac{(12.845\,\text{kN})(0.76030\,\text{m})}{2(1.2\,\text{m})}       \\
% 			              & = 4.0692\,\text{kN}                                                     \\
% 			              & \approx 4.07\,\text{kN}                                                 \\
% 		\end{align*}
% 	}
% \end{minipage}
\newpage
%%%%%%%%%%%%%%%%%%%%%%%%%%%%%%%%%%%%%%%%%%%%%%%%%%%%%%%%%%%%%%%%%%%%%%%%%%%%%%%%%%%%%%%%%%%%%%%%%%%%%%%%%%%%%%%%%%%%%%%%%%%%%%%%%%


\begin{minipage}[t]{0.4\textwidth}
	\textbf{Example 11}:\\
	\begin{cfig}[0.5]{../../figs/02StaticForces/03staticforcesEx11}\end{cfig}
	\raggedright
	A rectangular steel gate ($1.5\,\text{m}\times2.15\,\text{m}$) is used to regulate the level of a water storage pond.
	The gate has a weight of $1.9\,\text{kN}$ which can be thought of as acting through the centre of the gate.\par \bigskip
	Determine the force $T$ required to begin to open the gate when the pond has a depth of $3.1\,\text{m}$.
	% \parb
	% \textbf{Solution}:
	% \parb
	% \begin{cfig}[0.5]{../../figs/02StaticForces/03staticforcesEx11b}\end{cfig}
	% \cbox[0.9]{
	% 	\Large
	% 	\begin{align*}
	% 		\frac{3.1}{AB} & =\frac{\text{opp}}{\text{hyp}}=\sin52^\circ   \\
	% 		\Rightarrow AB & = \frac{3.1}{\sin52^\circ} = 3.9340\,\text{m} \\\\
	% 		\Rightarrow BH & = 3.9340-2.15=1.7840\,\text{m}                \\\\
	% 		\Rightarrow BC & = 3.9340-1.075=2.8590\,\text{m}               \\\\
	% 		\Rightarrow DC & = 2.8590\sin52^\circ= 2.2529\,\text{m}
	% 	\end{align*}
	% }
\end{minipage}
% \hfill
% \begin{minipage}[t]{0.55\textwidth}
%
% 	\cbox[0.9]{
% 		\Large
% 		\begin{align*}
% 			h_c          & = DC = 2.2529\,\text{m}                                                                         \\
% 			L_c          & = BC =2.859\,\text{m}                                                                           \\\\
% 			p_{avg}      & = \gamma h_c                                                                                    \\
% 			             & = \left(9.81\,\text{kN/m}^3\right)(2.2529\,\text{m})                                            \\
% 			             & = 22.101\,\text{kPa}                                                                            \\\\
% 			A            & = (1.5\,\text{m}\times2.15\,\text{m}) =3.225\,\text{m}^2                                        \\\\
% 			F_R          & = p_{avg}A                                                                                      \\
% 			             & = \left(22.101\,\text{kN/m}^2\right)\left(3.225\,\text{m}^2\right)                              \\
% 			             & = 71.275\,\text{kN}                                                                             \\\\
% 			I_c          & = \frac{bh^3}{12}=\frac{(1.5\,\text{m})(2.15\,\text{m})^2}{12}=1.2423\,\text{m}^2               \\
% 			\implies L_p & = L_c+\frac{I_c}{L_cA}                                                                          \\
% 			             & = 2.8590\,\text{m} +\frac{1.2423\,\text{m}^4}{(2.8590\,\text{m})\left(3.225\,\text{m}^2\right)} \\
% 			             & = 2.9937\,\text{m}
% 		\end{align*}
%
% 		Sum moments about the hinge, noting that the reaction at $A$ is $0$ when the gate begins to open:
% 		\begin{align*}
% 			\Sigma M_H & =(71.275\,\text{kN})(2.9937\,\text{m}-1.7840\,\text{m})                                             \\
% 			           & \qquad +(1.9\,\text{kN})(1.075\cos52^\circ\,\text{m})                                               \\
% 			           & \qquad -(T\,\text{kN})(2.15\cos52^\circ\,\text{m})                                                  \\
% 			           & = 0                                                                                                 \\
% 			T          & = \frac{(71.275\,\text{kN})(1.2097\,\text{m}+(1.9\,\text{kN})(0.66184\,\text{m})}{1.3237\,\text{m}} \\
% 			           & = 66.087\,\text{kN}                                                                                 \\
% 			           & \approx 66.1\,\text{kN}
% 		\end{align*}
% 	}
% \end{minipage}
\newpage
%%%%%%%%%%%%%%%%%%%%%%%%%%%%%%%%%%%%%%%%%%%%%%%%%%%%%%%%%%%%%%%%%%%%%%%%%%%%%%%%%%%%%%%%%%%%%%%%%%%%%%%%%%%%%%%%%%%%%%


% \begin{minipage}[c]{0.4\textwidth}
% 	\textbf{Example 12}:\\
% 	\begin{cfig}[0.45]{../../figs/02StaticForces/03staticforcesEx12}\end{cfig}
% 	\raggedright
% 	A $L$-shaped steel gate $ABC$ is used to regulate the level of a water storage pond.
% 	Both the vertical and the horizontal parts of the gate are rectangular with a width of $1.75\,\text{m}$.
% 	When the water level exceeds a certain level, the whole gate $ABC$ rotates clockwise about the frictionless hinge at $B$.
% 	\par \bigskip
% 	Determine the value of $h$ at which the gate begins to open.
% 	\parb
% 	\textbf{Solution}:
% 	\parb
% 	\cbox[0.9]{
% 		The gate opens when the clockwise moment of the hydrostatic force on the vertical section of the gate exceeds
% 		the counterclockwise moment of the hydrostatic force on the horizontal section of the gate.\parb
% 		The dimensions of the vertical rectangle are $1.75\,\text{m}\times1.5\,\text{m}$ and the dimensions of the
% 		horizontal rectangle are  $1.75\,\text{m}\times1.2\,\text{m}$\parb
% 		\underline{For the vertical section}:
% 		\begin{align*}
% 			h_c     & = d+0.75\,\text{m}                                   \\\\
% 			p_{avg} & = \gamma h_c                                         \\
% 			        & = \left(9.81\,\text{kN/m}^3\right)(d+0.75\,\text{m}) \\\\
% 			A       & = (1.75\,\text{m}\times 1.5\,\text{m})               \\
% 			        & = 2.625\,\text{m}^2
% 		\end{align*}
% 	}
% \end{minipage}
% \hfill
% \begin{minipage}[c]{0.58\textwidth}
%
% 	\cbox[0.9]{
% 		\begin{align*}
% 			F_{R1}  & = p_{avg}A                                                                                       \\
% 			        & = \left(9.81\,\text{kN/m}^3\right)(d+0.75\,\text{m})(2.625\,\text{m}^2)                          \\
% 			        & = 25.751(h+0.75)\,\text{kN}                                                                      \\\\
% 			L_p-L_c & = \frac{I_c}{L_cA}                                                                               \\
% 			        & = \frac{(1.75\,\text{m})(1.5\,\text{m})^3/12}{(h+0.75\,\text{m})(1.75\,\text{m})(1.5\,\text{m})} \\
% 			        & = \frac{0.1875}{h+0.75}\,\text{m}                                                                \\\\
% 			M_{cw}  & = 25.751(h+0.75)\left(0.75-\frac{0.1875}{h+0.75}\right)\,\text{kN}\cdot\text{m}
% 		\end{align*}
% 		\parb
% 		\underline{For the horizontal section}:
%
% 		\begin{align*}
% 			p_{avg} & = \gamma h_c                                                         \\
% 			        & = \left(9.81\,\text{kN/m}^3\right)(h+1.5\,\text{m})                  \\\\
% 			A       & = (1.75\,\text{m}\times 1.2\,\text{m})                               \\
% 			        & = 2.1\,\text{m}^2                                                    \\\\
% 			F_{R2}  & = p_{avg}A                                                           \\
% 			        & = \left(9.81\,\text{kN/m}^3\right)(h+1.5\,\text{m})(2.1\,\text{m}^2) \\
% 			        & = 20.601(h+1.5)\,\text{kN}                                           \\\\
% 			M_{ccw} & = 20.601(h+1.5)(0.6)\,\text{kN}\cdot\text{m}                         \\
% 			        & = 12.361(h+1.5)\,\text{kN}\cdot\text{m}
% 		\end{align*}
% 		\parb
% 		\underline{Equating the moments}
% 		\parb
% 		When the gate begins to open, the moments are equal:
% 		\begin{align*}
% 			25.751(h+0.75)\left(0.75-\frac{0.1875}{h+0.75}\right) & = 12.361(h+1.5)    \\
% 			25.751(h+0.75)(0.75)-25.751(0.1875)                   & = 12.361(h+1.5)    \\
% 			19.313(h+0.75)-25.751(0.1875)                         & = 12.361(h+1.5)    \\
% 			19.313h+14.485-4.8283                                 & = 12.361h+18.542   \\
% 			6.952h                                                & = 8.8851           \\
% 			h                                                     & = 1.2781\,\text{m}
% 		\end{align*}
% 	}
%
% \end{minipage}



\end{document}
