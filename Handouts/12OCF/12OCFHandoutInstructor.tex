% override specific chktex warnings

% chktex-file 1 - ignore commands followed by a space, e.g. \\ new line here
% chktex-file 3 - enclose previous parentheses wit {}
% chktex-file 9 - sometimes messes up with ( and {
% chktex-file 36 - put a space in front of parentheses
% chktex-file 45 - don't use $$ instead of \[, etc
% chktex-file 46 - don't use $ instead of \(, etc


\documentclass[10pt, oneside]{amsart}
% \usepackage[showboxes]{textpos}

\usepackage[absolute,overlay]{textpos}
\setlength{\TPHorizModule}{1.0cm}
\setlength{\TPVertModule}{\TPHorizModule}
\textblockorigin{0.0cm}{0.0cm}  %start all at upper left corner

\usepackage{amsmath}
\usepackage{amsthm}
\usepackage{amsfonts}
\usepackage{amssymb}
\usepackage{mathpazo}
\usepackage{booktabs}
\usepackage[usenames,x11names]{xcolor}
\usepackage{tikz}
\usepackage{textcomp}
\usepackage[letterpaper]{geometry}
\geometry{verbose,tmargin=0.5in,bmargin=0.5in,lmargin=1in,rmargin=0.5in}
\usepackage{multicol}
\usepackage{bm}
\usepackage{comment}
\usepackage{cancel}
\usepackage{array}
\usepackage{gensymb}
\usepackage{enumerate}
\usepackage[many]{tcolorbox}

\pagestyle{plain}
\raggedright
\renewcommand{\familydefault}{\sfdefault}
\setlength{\parskip}{\medskipamount}
\setlength{\columnsep}{1cm}

\everymath{\displaystyle}
\setlength{\parskip}{\bigskipamount}
\definecolor{structure}{RGB}{100,100,100}
% counter for resuming enumerated list numbers
\newcounter{resumeenumi}
\newcommand{\suspend}{\setcounter{resumeenumi}{\theenumi}}
\newcommand{\resume}{\setcounter{enumi}{\theresumeenumi}}

\newcommand\lb{\linebreak}
\newcommand\pars{\par\smallskip}
\newcommand\parm{\par\medskip}
\newcommand\parb{\par\bigskip}

\makeatletter
\providecommand{\gettikzxy}[3]{%
	\tikz@scan@one@point\pgfutil@firstofone#1\relax
	\edef#2{\the\pgf@x}%
	\edef#3{\the\pgf@y}%
}
\makeatother



% full width colored block but color specifiable
%\cb[body bg strength]{header bg}{header text}{body text}
\newcommand{\cb}[4][15]{
	\setbeamercolor{block title}{bg = #2}
	\setbeamercolor{block body}{bg = #2!#1}
	\setbeamercolor{item projected}{bg=#2, fg=white}
	\begin{center}
		\begin{block}{#3}
			#4
		\end{block}
	\end{center}
}

% colored block with width specified
% \cbw[body bg strength]{header bg}{width}{header text}{body text}
\newcommand{\cbw}[5][15]{
	\begin{center}
		%\vspace{-0.35cm}
		\begin{minipage}{#3\textwidth}
			\setbeamercolor{block title}{bg= #2}
			\setbeamercolor{block body}{bg= #2!#1}
			\setbeamercolor{item projected}{bg=#2, fg=white}
			\begin{block}{#4}
				\raggedright
				#5
			\end{block}
		\end{minipage}
	\end{center}
}

% centered minipage with text \raggedright
%\cmini[width]{content}
\newcommand{\cmini}[2][0.8]{
	\begin{center}
		\begin{minipage}{#1\columnwidth}
			\raggedright
			#2
		\end{minipage}
	\end{center}
}

%left flushed minipage
\newcommand{\mini}[2][0.8]{
	\begin{minipage}{#1\columnwidth}
		\raggedright
		#2
	\end{minipage}
}

%left flushed minipage, top aligned
\newcommand{\minit}[2][0.8]{
	\begin{minipage}[t]{#1\columnwidth}
		\raggedright
		#2
	\end{minipage}
}

%left flushed minipage
% \newcommand{\miniT}[2][0.8]{
%  \begin{minipage}[T]{#1\columnwidth}
%   \raggedright
%   #2
%  \end{minipage}
% }

%left flushed minipage
\newcommand{\minib}[2][0.8]{
	\begin{minipage}[b]{#1\columnwidth}
		\raggedright
		#2
	\end{minipage}
}

\newcommand{\cfig}[2][1]{% centred, scaled graphic
	\begin{center}
		\includegraphics[scale=#1]{#2}
	\end{center}
}
% figure with tight border for photos
% \cfigb[saitMaroon]{borderwidth with unit}{scale}{image}
\newcommand{\cfigb}[4][structure]{
	% \usepackage{adjustbox}
	\setlength{\fboxrule}{1pt}
	\begin{center}
		\includegraphics[scale=#3, cframe= #1 #2]{#4}
	\end{center}
}

\newcommand{\imgbox}[3]{
	% \setlength{\fboxsep}{12pt}
	\includegraphics[scale=#1, cframe= structure #3]{#2}
}

% \imgboxbg[bg color=white]{scale}{path/to/img}{border color}{border, e.g. 2pt}{margin, e.g. 4pt}
\newcommand{\imgboxbg}[6][white]{
	\setlength{\fboxrule}{#5}
	\setlength{\fboxsep}{#6}
	\centering
	\fcolorbox{#4}{#1}{\includegraphics[scale=#2]{#3}}
}

\newcommand{\fig}[2][1]{% scaled graphic
	\includegraphics[scale=#1]{#2}
}

% centred framed  box black border
%\cbox[width]{content}
\newcommand{\cbox}[2][0.9]{% framed centered  box
	\setlength\fboxsep{0.042\columnwidth}
	\setlength\fboxrule{0.0015\columnwidth}
	\begin{center}
		\fcolorbox{black}{white}{
			\vspace{-0.5cm}
			\begin{minipage}{#1\columnwidth}
				\raggedright
				#2
			\end{minipage}
		}
	\end{center}
	\setlength\fboxsep{0cm}
}



\newtcolorbox{mybox}[1][]
{
	colback=white,
	top=0.25cm,
	bottom=0.25cm,
	left=0.25cm,
	right=0.25cm,
	colframe=structure,
	fonttitle=\bfseries,
	enhanced, drop fuzzy shadow,
	% attach boxed title to top left={yshift=-2mm, xshift=5mm},
	attach boxed title to top left={yshift=-2mm, xshift=5mm}, colbacktitle=structure!80!white, #1}

\newtcolorbox{plainbox}[1][]{colback=white, sharp corners, top=0.125cm, bottom=0.125cm, left=0pt, right=0pt, boxrule=0.5pt,colframe=structure,fonttitle=\bfseries, colbacktitle=structure, arc=0mm, #1}
%
\newtcbtheorem{myexam}{Example}%
{
	enhanced,
	colback=white,
	top=0.375cm,
	bottom=0.25cm,
	left=0.375cm,
	right=0.375cm,
	colframe=structure,
	fonttitle=\bfseries,
	drop fuzzy shadow,
	%description font=\mdseries\itshape,
	attach boxed title to top left={yshift=-2mm, xshift=5mm},
	colbacktitle=structure!80!white
	}{exam}% then \pageref{exer:theoexample} references the theo

\newcommand{\myexample}[2][red]{
	% \tcb\tcbset{theostyle/.style={colframe=red,colbacktitle=yellow}}
	\begin{myexam}{}{}
		\raggedright
		#2
	\end{myexam}
	% \tcbset{colframe=structure,colbacktitle=structure}
}

\newtcbtheorem{myexer}{Exercise}%
{
	enhanced,
	colback=white,
	top=0.375cm,
	bottom=0.25cm,
	left=0.375cm,
	right=0.375cm,
	colframe=structure,
	fonttitle=\bfseries,
	drop fuzzy shadow,
	%description font=\mdseries\itshape,
	attach boxed title to top left={yshift=-2mm, xshift=5mm},
	colbacktitle=structure!80!white
	}{exer}

\newcommand{\myexercise}[2][red]{
	% \tcb\tcbset{theostyle/.style={colframe=red,colbacktitle=yellow}}
	\begin{myexer}{}{}
		\raggedright
		#2
	\end{myexer}
	% \tcbset{colframe=structure,colbacktitle=structure}
}


\begin{document}

\thispagestyle{empty}
\vspace{-7cm}
\centering

\textbf{\Large Module 12: Open Channel Flow (CIVL 318)}
\par\medskip


%%%%%%%%%%%%%%%%%%%%%%%%%%%%%%%%%%%%%%%%%%%%%%%%%%%%%%%%%%%%%%%%%%%%%%%%%%%%%%%%%%%%%%%%%%%%%%%%%%%%%%%%%%%%%%%%%%%%%

\begin{center}
	\begin{tabular}{r >{$}r<{$} >{$}c<{$} >{$}l<{$}}
		\toprule
		\addlinespace
		\textbf{ Hydraulic Radius}:            & R                              & =           & A/WP                                                    \\
		\addlinespace
		\midrule
		\addlinespace
		%\textbf{ Rectangular Channel}: & A &=& by \\
		%\addlinespace
		%& WP &=& b+2y \\
		%\addlinespace
		%& R &=& \frac{by}{b+2y}\ \\
		%\addlinespace
		%\midrule
		%\addlinespace
		%\textbf{ Triangular Channel}: & A &=&  zy^2 \\
		%\addlinespace
		%& WP &=& 2y\sqrt{1+z^2} \\
		%\addlinespace
		%& R &=& \frac{zy}{2\sqrt{1+z^2}}\ \\
		%\addlinespace
		%\midrule
		%\addlinespace
		%\textbf{ Trapezoidal Channel}: & A &=& y(b+yz) \\
		%\addlinespace
		%& WP &=& b+2y\sqrt{1+z^2} \\
		%\addlinespace
		%& R &=& \frac{y(b+yz)}{b+2y\sqrt{1+z^{2}}}  \\
		%\addlinespace
		%\midrule
		%\addlinespace
		\textbf{ Circular Channel}:            & A                              & =           & \frac{\left(\theta-\sin\theta\right)D^2}{8}             \\
		\addlinespace
		                                       & WP                             & =           & \theta D/2                                              \\
		\addlinespace
		                                       & R                              & =           & \left(\frac{\theta - \sin\theta}{\theta}\right)\frac D4 \\
		\addlinespace
		\midrule
		\textbf{ Manning's Equation}:          & v                              & =           & \frac{1}{n}R^{2/3}S^{1/2}                               \\
		\addlinespace
		                                       & Q                              & =           & \frac{1}{n}AR^{2/3}S^{1/2}                              \\
		\addlinespace
		\midrule
		\addlinespace
		\textbf{Open Channel Energy Equation}: & y_1 + \frac{v_1^2}{2g}+z_1-h_L & =           & y_2 + \frac{v_2^2}{2g}+z_2-h_L                          \\
		\addlinespace
		\midrule
		\addlinespace
		\textbf{ Specific Energy}:             & E_S                            & =           & y+\frac{v^2}{2g}                                        \\
		\addlinespace
		\midrule
		\textbf{ Froude Number}:               & N_F                            & =           & \frac{v}{\sqrt{g(A/T)}}                                 \\
		\addlinespace
		%$N_F$ for rectangular channel: & N_F &=& \frac{v}{\sqrt{gy}} \\
		%\addlinespace
		                                       & N_F>1                          & \Rightarrow & \text{Super-critical flow}                              \\
		\addlinespace
		                                       & N_F=1                          & \Rightarrow & \text{Critical flow}                                    \\
		\addlinespace
		                                       & N_F<1                          & \Rightarrow & \text{Sub-critical flow}                                \\
		\addlinespace
		\midrule
		\textbf{ Hydraulic Jump}:              & y_2                            & =           & \frac{y_1}{2}\left[\sqrt{1+8N_{F_1}^2}-1\ \right]       \\
		\addlinespace
		                                       & \Delta E                       & =           & \left(y_2 - y_1\right)^3/4y_1y_2                        \\
		\addlinespace
		\midrule
		\addlinespace
		\bottomrule
	\end{tabular}
	
	\vspace{2cm}
	
	%%%%%%%%%%%%%%%%%%%%%%%%%%%%%%%%%%%%%%%%%%%%%%%%%%%%%%%%%%%%%%%%%%%%%%%%%%%%%%%%%%%%%%%%%%%%%%%%%%%%%%%%%%%%%%%%%%%%%%%%%%
	
	\begin{center}\Large\textbf{Critical Depth, Velocity and Energy}\end{center}
	\begin{tabular}{ccc}
		\toprule
		\addlinespace
		Channel Shape & Critical Depth                                                                & Critical Velocity                                            \\
		\addlinespace
		\toprule
		\addlinespace
		Rectangular   & $y_{c}=\frac{2}{3}E_{min}$                                                    & $v_{c}=\sqrt{gy_{c}}$                                        \\
		\addlinespace
		\midrule
		\addlinespace
		Triangular    & $y_{c}=\frac{4}{5}E_{min}$                                                    & $v_{c}=\sqrt{\frac{gy_{c}}{2}}$                              \\
		% 		\addlinespace
		% 		 \midrule
		% 		 \addlinespace
		% 		 Circular \\
		\addlinespace
		\midrule
		\addlinespace
		Trapezoidal   & $y_{c}=\frac{4zE_{min}-3b+\sqrt{16z^{2}E_{min}^{2}+16zE_{min}b+9b^{2}}}{10z}$ & $v_{c}=\sqrt{\frac{gy_{c}\left(b+zy_{c}\right)}{b+2zy_{c}}}$ \\
		\addlinespace
		\bottomrule
	\end{tabular}
	
	
	
	\newpage
	\textbf{Manning's Equation - Resistance Factors}
	
	\begin{tabular}{rc}
		\toprule
		
		\textbf{Channel Description}                                       & \textbf{\emph{n}-value} \\
		\toprule
		
		Plastic - continuous                                               & 0.009                   \\ 	\midrule
		
		Plastic - pipe with joints                                         & 0.011                   \\ \midrule
		
		Smooth metal                                                       & 0.010                   \\ \midrule
		
		Concrete - steel trowelled or slip-formed                          & 0.012                   \\ \midrule
		
		Concrete - pipe with joints                                        & 0.013                   \\ \midrule
		
		Concrete - shotcreted                                              & 0.017                   \\ \midrule
		
		Riveted Steel                                                      & 0.018                   \\ \midrule
		
		Corrugated Steel                                                   & 0.022                   \\ \midrule
		
		Earth - firm clay                                                  & 0.020                   \\ \midrule
		
		Earth - alluvial silt and clay                                     & 0.021                   \\ \midrule
		
		Earth - medium sand                                                & 0.023                   \\ \midrule
		
		Earth - fine gravel                                                & 0.024                   \\ \midrule
		
		Earth - coarse gravel                                              & 0.028                   \\ \midrule
		
		Earth - cobbles                                                    & 0.030                   \\
		Rock cuts - smooth                                                 & 0.030                   \\ \midrule
		
		Rock cuts - jagged                                                 & 0.040                   \\ \midrule
		
		Highway Ditches - cut grass                                        & 0.040                   \\ \midrule
		
		Highway Ditches - uncut grass                                      & 0.090                   \\ \midrule
		
		Highway Ditches - uncut grass and small bushes                     & 0.125                   \\ \midrule
		
		Natural Streams - straight, uniform cross-section, no growth       & 0.025                   \\ \midrule
		
		Natural Streams - fairly straight, moderate changes           &   \\ in cross-section, some growth & 0.033 \\ \midrule
		
		Natural Streams - curving, moderate changes in cross-section, &   \\some growth & 0.040 \\ \midrule
		
		Flood Plains - with high grass and bushes, some developed channels & 0.070                   \\ \midrule
		
		Flood Plains - with bushes, trees and pools                        & 0.100                   \\ \midrule
		
		Mountain Creeks - with pools and waterfalls                        & 0.130                   \\ \bottomrule
	\end{tabular}
\end{center}
\vspace{1.5cm}
\begin{center}
	\begin{tabular}{l >{$}c<{$} >{$}c<{$}}
		\toprule
		
		\textbf{Soil Type}        & \textbf{ParticleSize}  & \textbf{ Maximum Velocity} \\
		\toprule
		
		
		Fine sand, no clay        & 62-250\, \mu\text{m}   & 0.45\,\text{m/s}           \\
		\midrule
		
		Medium sand, no clay      & 250-500\, \mu\text{m}  & 0.50\,\text{m/s}           \\
		\midrule
		
		Coarse sand, no clay      & 500-2000\, \mu\text{m} & 0.60\,\text{m/s}           \\
		\midrule
		
		Fine gravel               & 4-8\, \text{mm}        & 0.74\,\text{m/s}           \\
		\midrule
		
		Coarse gravel             & 8-64\, \text{mm}       & 1.25\,\text{m/s}           \\
		\midrule
		
		Cobbles                   & 64-256\, \text{mm}     & 1.55\,\text{m/s}           \\
		\midrule
		
		Firm clay                 &                        & 1.15\,\text{m/s}           \\
		\midrule
		
		Alluvial silt (high clay) &                        & 1.15\,\text{m/s}           \\
		\midrule
		
		Alluvial silt (low clay)  &                        & 0.60\,\text{m/s}           \\
		\midrule
		
	\end{tabular}
\end{center}

\newpage
%\rule{\textwidth}{0.02in}
%%%%%%%%%%%%%%%%%%%%%%%%%%%%%%%%%%%%%%%%%%%%%%%%%%%%%%%%%%%%%%%%%%%%%%%%%%%%%%%%%%%%%%%%%%%%%%%%%%%%%%%%%%%%%%%%%%%%%
\raggedright



%%%%%%%%%%%%%%%%%%%%%%%%%%%%%%%%%%%%%%%%%%%%%%%%%%%%%%%%%%%%%%%%%%%%%%%%%%%%%%%%%%%%%%%%%%%%%%%%%%%%%%%%%


\cmini[0.575]{
	\begin{myexam}{}{}
		\raggedright
		The Pontcysyllte Aqueduct is $3.4$ m wide, $1.60$ m deep and is constructed of cast iron with a resistance factor of
		$n=0.018$.
		\parb
		If the flow velocity is $0.1$ m/s, determine the slope.
		
	\end{myexam}
}
\mini[0.5]{
	\large
	\parb
	\textbf{Solution}:\parb
	
	\cbox[0.9]{
		\begin{align*}
			A  & = 3.40\,\text{m}\times 1.60\,\text{m}           \\
			   & = 5.44\,\mathsf{m^2}                            \\
			WP & = 1.6\,\text{m} + 3.4\,\text{m} + 1.6\,\text{m} \\
			   & = 6.60\,\text{m}                                \\
			R  & = A/WP                                          \\
			   & = \frac{5.44\,\mathsf{m^2}}{6.60\,\text{m}}     \\
			   & = 0.82424\,\text{m}                             
		\end{align*}
	}
	\parb
	Find the slope:
	
	\cbox[0.9]{
		\begin{align*}
			v          & = \frac{1}{n}R^{2/3}S^{0.5}                            \\
			\implies S & = \left[\frac{nv}{R^{2/3}}\right]^2                    \\
			           & = \left[\frac{0.018\times 0.1}{0.82424^{2/3}}\right]^2 \\\\
			           & = \bm{0.000419}\textbf{\%}                             
		\end{align*}
	}
}

\vfill
\pagebreak


%%%%%%%%%%%%%%%%%%%%%%%%%%%%%%%%%%%%%%%%%%%%%%%%%%%%%%%%%%%%%%%%%%%%%%%%%%%%%%%%%%%%%%%%%%%%%%%%%%%%%%%%%


\cmini[0.575]{
	\begin{myexam}{}{}
		\raggedright
		A circular culvert of diameter $2.0\,\text{m}$ is made from corrugated metal. It has a slope of $1$ in $500$.
		Determine the normal discharge and the velocity when:\parb
		\cmini{
			\begin{enumerate}[(a)]
				\item  the culvert runs half full
				\item  the culvert runs full
				\item  the culvert runs at a depth of $1.8\,\text{m}$
			\end{enumerate}
		}
	\end{myexam}
}

\mini[0.45]{
	For parts (a) and (b), we don't need the complicated circle formula.
	\cbox{
		\tikz[scale=\scale]{
			
		}
	}
}





\vfill

\pagebreak~\pagebreak

%%%%%%%%%%%%%%%%%%%%%%%%%%%%%%%%%%%%%%%%%%%%%%%%%%%%%%%%%%%%%%%%%%%%%%%%%%%%%%%%%%%%%%%%%%%%%%%%%%%%%%%%%


\textbf{Example 3}:
% left-aligned minipage with text raggedright
%mini[width]{content}
\mini[1]{
	\cfig[0.5]{../../figs/12OCF/trapEx3}
	
	Determine the normal discharge for the two channels shown. Both are in fine grained soil with an estimated $n$-value
	of $0.020$. The channel slope is 0.1\%. \par\smallskip
	Then, calculate the flow rate when the more efficient channel is lined with concrete ($n=0.013$).
}






\par\vfill
~\par



% \textbf{Solution}:\\
% \Large


\vfill
\pagebreak


%%%%%%%%%%%%%%%%%%%%%%%%%%%%%%%%%%%%%%%%%%%%%%%%%%%%%%%%%%%%%%%%%%%%%%%%%%%%%%%%%%%%%%%%%%%%%%%%%%%%%%%%%

\textbf{Example 4}:

A rectangular channel is constructed out of concrete
with $n=0.013$. The channel has a slope of 0.0005 and a base width of $b=4.0\,\text{m}$  \par\bigskip
Find the depth of flow,
$y$, if $Q=6.25\,\mathsf{m^3/s}$.

\par\vfill

\pagebreak

%%%%%%%%%%%%%%%%%%%%%%%%%%%%%%%%%%%%%%%%%%%%%%%%%%%%%%%%%%%%%%%%%%%%%%%%%%%%%%%%%%%%%%%%%%%%%%%%%%%%%%%%%

\textbf{Example 5}:

A trapezoidal channel constructed from coarse sand without clay
($n = 0.025$) has a base width of $7.0$ m and a flow depth of $1.7$ m.
\par\medskip
If the sidewall have a $z$-value of 3, find the average flow
rate and the average velocity. Check this against allowable values.\par\medskip
(Slope is 0.015\% and the maximum allowable velocity for this type of soil is $0.60$ m/s.)

\par\vfill

\pagebreak

%%%%%%%%%%%%%%%%%%%%%%%%%%%%%%%%%%%%%%%%%%%%%%%%%%%%%%%%%%%%%%%%%%%%%%%%%%%%%%%%%%%%%%%%%%%%%%%%%%%%%%%%%

\textbf{Example 6}:

Determine the specific energy in a
rectangular channel with a base of
$6.0$ m and a flow depth of $y = 1.8$ m
if the volume flow is $20 \,\mathsf{m^3/s}$

\par\vfill

\pagebreak

%%%%%%%%%%%%%%%%%%%%%%%%%%%%%%%%%%%%%%%%%%%%%%%%%%%%%%%%%%%%%%%%%%%%%%%%%%%%%%%%%%%%%%%%%%%%%%%%%%%%%%%%%

\textbf{Example 7}:


\cfig[0.35]{../../figs/12OCF/sluice1}
Water flows at a depth of 2.0 m in a rectangular channel with width 5.0 m. The flow passes under a sluice gate so that
the downstream flow has a depth of 0.5 m. Determine the flow in the channel. (There is little energy loss due to the
sluice gate.)

\par\vfill

\pagebreak

%%%%%%%%%%%%%%%%%%%%%%%%%%%%%%%%%%%%%%%%%%%%%%%%%%%%%%%%%%%%%%%%%%%%%%%%%%%%%%%%%%%%%%%%%%%%%%%%%%%%%%%%%


\textbf{Example 8}:



A channel with a triangular section is cut through rock and lined with un-trowelled shotcrete. It has a slope of
0.1201 \%.
The water depth, $y=3.5$ m and the side slope $z=1.0$. The channel has an estimated $n=0.017$.
Find:
\begin{itemize}
	\item [a)] the flow rate
	\item [b)] the velocity
	\item [c)] the critical velocity
	\item [d)] whether the slope is sub-critical or super-critical.
	\item [e)] the slope at which flow becomes critical
\end{itemize}

\cfig[0.5]{../../figs/12OCF/criticalTriangle}

\textbf{Solution}:\\

\cbox[0.8]{
	\begin{align*}
		A  & = \frac{1}{2}(7.0 \text{m})(3.5\text{ m})      \\
		   & = 12.25\;\mathsf{m^2}                          \\
		WP & = 2\sqrt{2(3.5\text{ m})^2}                    \\
		   & =9.8995\text{ m}                               \\
		R  & = \frac{A}{WP}                                 \\
		   & = \frac{12.250\;\mathsf{m^2}}{9.8995\text{ m}} \\
		   & = 1.2374\text{ m}                              
	\end{align*}
}

% centered minipage with text raggedright
%cbox[width]{content}
\cbox[0.8]{
	\begin{align*}
		Q & =\frac{1}{n}AR^{2/3}S^{1/2}                          \\
		  & =\frac{1}{0.017}(12.250)(1.2374)^{2/3}(0.0012)^{1/2} \\
		  & =28.771\;\mathsf{m^3/s}                              
	\end{align*}
	
}







\vfill
\pagebreak

%%%%%%%%%%%%%%%%%%%%%%%%%%%%%%%%%%%%%%%%%%%%%%%%%%%%%%%%%%%%%%%%%%%%%%%%%%%%%%%%


\textbf{Example 9}:
Flow over a rectangular dam spillway is approximately 1150 m$^3$/s. If the width of the spillway is
$32.0\mbox{ m}$ and the depth of the supercritical flow before the hydraulic jump is $1.4$ m,
determine the velocities of the flow before and after the jump, the depth after the jump
and the energy dissipated.
\vfill\pagebreak

\vfill
\pagebreak

%%%%%%%%%%%%%%%%%%%%%%%%%%%%%%%%%%%%%%%%%%%%%%%%%%%%%%%%%%%%%%%%%%%%%%%%%%%%%%%%%%%%%%%%%%%%%%%%%%%%%%%%%

\cfig[0.25]{../../Figs/12OCF/sluiceJumpExample10light}
\textbf{Example 10}:
Flow from a reservoir (i.e. it has no initial velocity) into a rectangular channel is controlled by a sluice gate as shown. Verify that the flow after the gate is supercritical and determine the depth of flow downstream from the jump.
\vfill\pagebreak

%%%%%%%%%%%%%%%%%%%%%%%%%%%%%%%%%%%%%%%%%%%%%%%%%%%%%%%%%%%%%%%%%%%%%%%%%%%%%%%%%%%%%%%%%%%%%%%%%%%%%%%%%


\textbf{Example 11}:
Water flows from a reservoir with a depth of 3.60 m down a trapezoidal concrete-lined channel ($n=0.017$) with a base width of $4.0$ m and a side slope value of $z=2.5$.\parm
Determine:
\begin{enumerate}
	\item The flow under critical conditions (when specific energy is at a minimum)
	\item The slope required to give critical flow
\end{enumerate}
\vfill\pagebreak




%%%%%%%%%%%%%%%%%%%%%%%%%%%%%%%%%%%%%%%%%%%%%%%%%%%%%%%%%%%%%%%%%%%%%%%%%%%%%%%%%%%%%%%%%%%%%%%%%%%%%%%%%
%\textbf{Example 12}:
%An earth channel is to be constructed in medium sand (no clay) with a trapezoidal section. \pars
%The bottom width, $b=10$ m and the side slope $z=2.0$. \pars
%The estimated $n$ value is $0.023$ and the maximum allowable velocity for this type of soil is $0.50$ m/s. \pars
%The elevation difference between the two ends is $6.3$ m and the distance directly between them is $64.2$ km. \pars
%The design flow rate is $7.044\,\mathsf{m^3/s}$ and the velocity is limited to 80\% of the maximum allowable. \parb
%
%Determine the required area of flow, the required depth, the required slope, the corresponding channel length for
%allowable velocity, and the height of a drop structure if the direct channel length is used.
%
%\vfill\newpage



\end{document}
