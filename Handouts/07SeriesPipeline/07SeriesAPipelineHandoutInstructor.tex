% override specific chktex warnings

% chktex-file 1 - ignore commands followed by a space, e.g. \\ new line here
% chktex-file 3 - enclose previous parentheses wit {}
% chktex-file 9 - sometimes messes up with ( and {
% chktex-file 36 - put a space in front of parentheses
% chktex-file 45 - don't use $$ instead of \[, etc
% chktex-file 46 - don't use $ instead of \(, etc


\documentclass[10pt, oneside]{amsart}
% \usepackage[showboxes]{textpos}

\usepackage[absolute,overlay]{textpos}
\setlength{\TPHorizModule}{1.0cm}
\setlength{\TPVertModule}{\TPHorizModule}
\textblockorigin{0.0cm}{0.0cm}  %start all at upper left corner

\usepackage{amsmath}
\usepackage{amsthm}
\usepackage{amsfonts}
\usepackage{amssymb}
\usepackage{mathpazo}
\usepackage{booktabs}
% \usepackage[usenames,x11names]{xcolor}
\usepackage{tikz}
\usepackage{textcomp}
\usepackage[letterpaper]{geometry}
\geometry{verbose,tmargin=0.5in,bmargin=0.5in,lmargin=1in,rmargin=0.5in}
\usepackage{multicol}
\usepackage{bm}
\usepackage{comment}
\usepackage{cancel}
\usepackage{array}
\usepackage{gensymb}
\usepackage{enumerate}
\usepackage[many]{tcolorbox}

\pagestyle{plain}
\raggedright
\renewcommand{\familydefault}{\sfdefault}
\setlength{\parskip}{\medskipamount}
\setlength{\columnsep}{1cm}

\everymath{\displaystyle}
\setlength{\parskip}{\bigskipamount}

% counter for resuming enumerated list numbers
\newcounter{resumeenumi}
\newcommand{\suspend}{\setcounter{resumeenumi}{\theenumi}}
\newcommand{\resume}{\setcounter{enumi}{\theresumeenumi}}

\newcommand\lb{\linebreak}
\newcommand\pars{\par\smallskip}
\newcommand\parm{\par\medskip}
\newcommand\parb{\par\bigskip}

\makeatletter
\providecommand{\gettikzxy}[3]{%
	\tikz@scan@one@point\pgfutil@firstofone#1\relax
	\edef#2{\the\pgf@x}%
	\edef#3{\the\pgf@y}%
}
\makeatother



% full width colored block but color specifiable
%\cb[body bg strength]{header bg}{header text}{body text}
\newcommand{\cb}[4][15]{
	\setbeamercolor{block title}{bg = #2}
	\setbeamercolor{block body}{bg = #2!#1}
	\setbeamercolor{item projected}{bg=#2, fg=white}
	\begin{center}
		\begin{block}{#3}
			#4
		\end{block}
	\end{center}
}

% colored block with width specified
% \cbw[body bg strength]{header bg}{width}{header text}{body text}
\newcommand{\cbw}[5][15]{
	\begin{center}
		%\vspace{-0.35cm}
		\begin{minipage}{#3\textwidth}
			\setbeamercolor{block title}{bg= #2}
			\setbeamercolor{block body}{bg= #2!#1}
			\setbeamercolor{item projected}{bg=#2, fg=white}
			\begin{block}{#4}
				\raggedright
				#5
			\end{block}
		\end{minipage}
	\end{center}
}

% centered minipage with text \raggedright
%\cmini[width]{content}
\newcommand{\cmini}[2][0.8]{
	\begin{center}
		\begin{minipage}{#1\columnwidth}
			\raggedright
			#2
		\end{minipage}
	\end{center}
}

%left flushed minipage
\newcommand{\mini}[2][0.8]{
	\begin{minipage}{#1\columnwidth}
		\raggedright
		#2
	\end{minipage}
}

%left flushed minipage, top aligned
\newcommand{\minit}[2][0.8]{
	\begin{minipage}[t]{#1\columnwidth}
		\raggedright
		#2
	\end{minipage}
}

%left flushed minipage
% \newcommand{\miniT}[2][0.8]{
%  \begin{minipage}[T]{#1\columnwidth}
%   \raggedright
%   #2
%  \end{minipage}
% }

%left flushed minipage
\newcommand{\minib}[2][0.8]{
	\begin{minipage}[b]{#1\columnwidth}
		\raggedright
		#2
	\end{minipage}
}

\newcommand{\cfig}[2][1]{% centred, scaled graphic
	\begin{center}
		\includegraphics[scale=#1]{#2}
	\end{center}
}
% figure with tight border for photos
% \cfigb[saitMaroon]{borderwidth with unit}{scale}{image}
\newcommand{\cfigb}[4][structure]{
	% \usepackage{adjustbox}
	\setlength{\fboxrule}{1pt}
	\begin{center}
		\includegraphics[scale=#3, cframe= #1 #2]{#4}
	\end{center}
}

\newcommand{\imgbox}[3]{
	% \setlength{\fboxsep}{12pt}
	\includegraphics[scale=#1, cframe= structure #3]{#2}
}

% \imgboxbg[bg color=white]{scale}{path/to/img}{border color}{border, e.g. 2pt}{margin, e.g. 4pt}
\newcommand{\imgboxbg}[6][white]{
	\setlength{\fboxrule}{#5}
	\setlength{\fboxsep}{#6}
	\centering
	\fcolorbox{#4}{#1}{\includegraphics[scale=#2]{#3}}
}

\newcommand{\fig}[2][1]{% scaled graphic
	\includegraphics[scale=#1]{#2}
}

% centred framed  box black border
%\cbox[width]{content}
\newcommand{\cbox}[2][0.9]{% framed centered  box
	\setlength\fboxsep{0.042\columnwidth}
	\setlength\fboxrule{0.0015\columnwidth}
	\begin{center}
		\fcolorbox{black}{white}{
			\vspace{-0.5cm}
			\begin{minipage}{#1\columnwidth}
				\raggedright
				#2
			\end{minipage}
		}
	\end{center}
	\setlength\fboxsep{0cm}
}



\newtcolorbox{mybox}[1][]
{
	colback=white,
	top=0.25cm,
	bottom=0.25cm,
	left=0.25cm,
	right=0.25cm,
	colframe=structure,
	fonttitle=\bfseries,
	enhanced, drop fuzzy shadow,
	% attach boxed title to top left={yshift=-2mm, xshift=5mm},
	attach boxed title to top left={yshift=-2mm, xshift=5mm}, colbacktitle=structure!80!white, #1}

\newtcolorbox{plainbox}[1][]{colback=white, sharp corners, top=0.125cm, bottom=0.125cm, left=0pt, right=0pt, boxrule=0.5pt,colframe=structure,fonttitle=\bfseries, colbacktitle=structure, arc=0mm, #1}
%
\newtcbtheorem{myexam}{Example}%
{
	enhanced,
	colback=white,
	top=0.375cm,
	bottom=0.25cm,
	left=0.375cm,
	right=0.375cm,
	colframe=structure,
	fonttitle=\bfseries,
	drop fuzzy shadow,
	%description font=\mdseries\itshape,
	attach boxed title to top left={yshift=-2mm, xshift=5mm},
	colbacktitle=structure!80!white
	}{exam}% then \pageref{exer:theoexample} references the theo

\newcommand{\myexample}[2][red]{
	% \tcb\tcbset{theostyle/.style={colframe=red,colbacktitle=yellow}}
	\begin{myexam}{}{}
		\raggedright
		#2
	\end{myexam}
	% \tcbset{colframe=structure,colbacktitle=structure}
}

\newtcbtheorem{myexer}{Exercise}%
{
	enhanced,
	colback=white,
	top=0.375cm,
	bottom=0.25cm,
	left=0.375cm,
	right=0.375cm,
	colframe=structure,
	fonttitle=\bfseries,
	drop fuzzy shadow,
	%description font=\mdseries\itshape,
	attach boxed title to top left={yshift=-2mm, xshift=5mm},
	colbacktitle=structure!80!white
	}{exer}

\newcommand{\myexercise}[2][red]{
	% \tcb\tcbset{theostyle/.style={colframe=red,colbacktitle=yellow}}
	\begin{myexer}{}{}
		\raggedright
		#2
	\end{myexer}
	% \tcbset{colframe=structure,colbacktitle=structure}
}


\begin{document}

\thispagestyle{empty}
\vspace{-7cm}
\begin{center}
	\textbf{\Large Module 7: Series A Pipeline (CIVL 318)}
\end{center}
\parb
%%%%%%%%%%%%%%%%%%%%%%%%%%%%%%%%%%%%%%%%%%%%%%%%%%%%%%%%%%%%%%%%%%%%%%%%%%%%%%%%%%%%%%%%%%%%%%%%%%%%%%%%%

\minit[0.45]{
	\textbf{Example 1}:
	\cfig[0.4]{../../Figs/07SeriesPipeline/ElevStorA}
	A pump delivers $13.5\,\text{L/s}$ of kerosene at $25^\circ$C from an underground vented storage tank to an elevated storage tank pressurized to $745\,\text{kPa}$.
	\parm
	The suction pipe is $6$-in Schedule $40$ steel pipe and is $5.0\,\text{m}$ long. It has a round-edged entrance with a radius of $r=15$~mm.
	\parm
	The discharge pipe is $3$-in Schedule 40 steel pipe,  is $11.0\,\text{m}$ long and includes a fully open butterfly valve with $L_e/D=45$.
	\parm
	All elbows are ``standard'' with $L_e/D=30$.
	\parm
	Determine the power drawn (the power in, $P_I$) by the pump, given that the pump has an efficiency of $73\%$.
	\parb
	\Large{\textbf{Solution}}:
	\normalsize
	\parb
	\underline{\bfseries Suction Pipe}
	\cbox[0.9]{
		\begin{align*}
			v                  & = \frac{0.0135\,\mathsf{m^3/s}}{\pi(0.1540\,\text{m})^2/4} = 0.72383\,\text{m/s} \\
			\frac{v^2}{2g}     & = 0.026704\,\text{m}                                                             \\
			N_R                & = \frac{0.72383(0.1541)823}{1.64\times 10^{-3}} = 55975                          \\
			                   & = 5.5975 \times 10^4                                                             \\
			\frac{D}{\epsilon} & = \frac{0.1541}{4.6\times 10^{-5}}=3350                                          \\
		\end{align*}
		$N_R>4000$ so flow is turbulent.
	}
}
\hfill
\minit[0.45]{
	Friction Losses:
	\cbox[0.9]{
		\begin{align*}
			f   & = 0.0215\quad\text{(Moody)}                     \\
			    & = 0.02144\quad\text{(Swamee-Jain)}              \\
			\intertext{Using the Moody result:}
			h_L & = 0.0215\left(\frac{5.0}{0.1541}\right)0.026704 \\
			    & = \bm{0.018629\,\text{m}}                       
		\end{align*}
	}
	\parb
	Minor Losses:
	\cbox[0.9]{
		\begin{align*}
			f_T                  & = 0.015\quad\text{(Moody)}                                          \\
			K_{\text{entrance}}  & = 0.09\quad\text{($r/D=0.1$)}                                       \\
			K_{\text{elbow}}     & = f_T\left(\frac{L_e}{D}\right) = 0.015(30) = 0.45                  \\
			h_{L_{\text{minor}}} & = K_{\text{entrance}}\frac{v^2}{2g}+2K_{\text{elbow}}\frac{v^2}{2g} \\
			                     & = (0.09+2\times0.45)0.026704                                        \\
			                     & = \bm{0.026437\,\text{m}}                                           
		\end{align*}
	}
	\parb
	\underline{\bfseries Discharge Pipe}
	\cbox[0.9]{
		\begin{align*}
			v                  & = \frac{0.0135\,\mathsf{m^3/s}}{\pi(0.0779\,\text{m})^2/4} = 2.8325\,\text{m/s} \\
			\frac{v^2}{2g}     & = 0.40892\,\text{m}                                                             \\
			N_R                & = \frac{2.8325(0.0779)823}{1.64\times 10^{-3}} = 110730                         \\
			                   & = 1.1073 \times 10^5                                                            \\
			\frac{D}{\epsilon} & = \frac{0.0779}{4.6\times 10^{-5}}=1693                                         \\
		\end{align*}
		$N_R>4000$ so flow is turbulent.
	}
	\parb
	Friction Losses:
	\cbox[0.9]{
		\begin{align*}
			f   & = 0.0205\quad\text{(Moody)}                     \\\\
			h_L & = 0.0205\left(\frac{11.0}{0.0779}\right)0.40892 \\
			    & = \bm{1.1837\,\text{m}}                         
		\end{align*}
	}
}

\newpage

\minit[0.45]{
	Minor Losses:
	\cbox[0.9]{
		\begin{align*}
			f_T                  & = 0.018\quad\text{(Table)}                         \\
			K_{\text{elbow}}     & = f_T\left(\frac{L_e}{D}\right) = 0.018(30) = 0.54 \\
			K_{\text{valve}}     & = f_T\left(\frac{L_e}{D}\right) = 0.018(45) = 0.81 \\
			K_{\text{exit}}      & = 1                                                \\
			h_{L_{\text{minor}}} & = \left(0.54+0.81+1\right)0.40892                  \\
			                     & = \bm{0.96096\,\text{m}}                           
		\end{align*}
	}
	\parb
	Total head losses:
	\cbox[0.9]{
		\begin{align*}
			h_L & = 0.018629 + 0.026437   \\
			    & \quad +1.1837+0.96096   \\
			    & = \bm{2.1897\,\text{m}} 
		\end{align*}
	}
	\parb
	General Energy Equation:
	\cbox[0.9]{
		\begin{align*}
			\cancel{\frac{P_A}{\gamma}} + \cancel{z_A} + \cancel{\frac{v_A^2}{2g}} +h_A-h_L & = \frac{P_A}{\gamma} + z_A +                
			\cancel{\frac{v_A^2}{2g}} \\
			h_A-2.1897                                                                      & = \frac{745}{0.823\times 9.81} + 11.8       \\
			h_A                                                                             & = 106.27\,\text{m}                          \\\\
			P_{\text{added}}                                                                & = h_A \gamma Q                              \\
			                                                                                & = 106.27\left(0.823\times 9.81\right)0.0135 \\
			                                                                                & = 11.583\,\text{kW}                         \\\\
			P_I                                                                             & = \frac{11.583}{0.73}                       \\
			                                                                                & = 15.867 \,\text{kW}                        \\\\
			\bm{P_I}                                                                        & = \bm{15.87\,\textbf{kW}}                   
		\end{align*}
	}
}
\newpage
%%%%%%%%%%%%%%%%%%%%%%%%%%%%%%%%%%%%%%%%%%%%%%%%%%%%%%%%%%%%%%%%%%%%%%%%%%%%%%%%%%%%%%%%%%%%%%%%%%%%%%%%

\textbf{Example 2}:
\cfig[0.5]{../../figs/07SeriesPipeline/07SeriesFindY}
\par
\minit[0.45]{
	
	\parm
	Gasoline at $25\text{\textcelsius}$ flows under gravity from tank $A$ to tank $B$; both tanks are open to the
	atmosphere.
	\parb
	The $2$-in Schedule $40$ steel pipe has a square entrance and is $45.7\,\text{m}$ in length. The $4$-in Schedule $40$ steel
	pipe contains a fully-open globe valve and is $87.5\,\text{m}$ in length. There is a sudden enlargement between the two
	pipes, as shown. Both pipes are new commercial steel. All elbows are standard $90\degree$.
	\parb
	Determine the elevation difference, $y$, between the surfaces of tanks $A$ and $B$ that is required to maintain a flow of $425\,\text{L/min}$.
	\parb
	
	\Large{\textbf{Solution}}:
	\normalsize
	\parb
	\underline{\bfseries 2-in Pipe}
	
	\cbox[0.9]{
		\large
		\begin{align*}
			v                  & = \frac{0.425/60\,\mathsf{m^3/s}}{\pi(0.0525\,\text{m})^2/4} = 3.2721\,\text{m/s} \\
			\frac{v^2}{2g}     & = 0.54571\,\text{m}                                                               \\
			N_R                & = \frac{3.2721(0.0525)680}{2.87\times 10^{-4}}                                    \\
			                   & = 407020 = 4.0702 \times 10^5                                                     \\
			\frac{D}{\epsilon} & = \frac{0.0525}{4.6\times 10^{-5}}=1141.3                                         \\
		\end{align*}
		$N_R>4000$ so flow is turbulent.
	}
}
\hfill
\minit[0.45]{
	\parb
	Friction Losses:
	\cbox[0.9]{
		\large
		\begin{align*}
			f   & = 0.0203\quad\text{(Moody)}                     \\
			    & = 0.019945\quad\text{(Swamee-Jain)}             \\
			\intertext{Using the Moody value:}
			h_L & = 0.0203\left(\frac{45.7}{0.0525}\right)0.54571 \\
			    & = \bm{9.6431\,\text{m}}                         
		\end{align*}
	}
	\parb
	Minor Losses:
	\cbox[0.9]{
		\large
		\begin{align*}
			f_T                      & = 0.0192\quad\text{(Moody)}                                \\
			                         & = 0.019026\quad\text{(Swamee-Jain)}                        \\
			                         & = 0.019\quad\text{(from table for {\bfseries steel} pipe)} \\
			\intertext{Use $f_T=0.019$ from the steel pipe table:}
			K_{\text{entrance}}      & = 0.5                                                      \\
			2\times K_{\text{elbow}} & = 2\times f_T\left(\frac{L_e}{D}\right)                    \\
			                         & = 2\times 0.019(30) = 1.14                                 \\
			\intertext{Using $D_2/D_1\approx 1.95$ and $v\approx 3.27\,\text{m/s}$ in sudden enlargement table:}
			K_{\text{enlargement}}   & = 0.52                                                     \\\\
			\Sigma K                 & = 2.16                                                     \\\
			h_{L_{\text{minor}}}     & = \Sigma K\times\frac{v^2}{2g}                             \\
			                         & = (2.16)0.54571                                            \\
			                         & = \bm{1.1787\,\text{m}}                                    
		\end{align*}
	}
}
\newpage
\minit[0.45]{
	\underline{\bfseries 4-in Pipe}
	\cbox[0.9]{
		\large
		\begin{align*}
			v                  & = \frac{0.425/60\,\mathsf{m^3/s}}{\pi(0.1023\,\text{m})^2/4} = 0.86178\,\text{m/s} \\
			\frac{v^2}{2g}     & = 0.037852\,\text{m}                                                               \\
			N_R                & = \frac{0.86178(0.1023)680}{2.87\times 10^{-4}}                                    \\
			                   & = 208880 = 2.0888 \times 10^5                                                      \\
			\frac{D}{\epsilon} & = \frac{0.1023}{4.6\times 10^{-5}}=2223.9                                          \\
		\end{align*}
		$N_R>4000$ so flow is turbulent.
	}
	\parb
	Friction Losses:
	\cbox[0.9]{
		\large
		\begin{align*}
			f   & = 0.0177\quad\text{(Moody)}                      \\
			    & = 0.017661\quad\text{(Swamee-Jain)}              \\
			\intertext{Using $f=0.0177$}
			h_L & = 0.0177\left(\frac{87.5}{0.1023}\right)0.037852 \\
			    & = \bm{0.57305\,\text{m}}                         
		\end{align*}
	}
	\parb
	Minor Losses:
	\cbox[0.9]{
		\large
		\begin{align*}
			f_T                      & = 0.0162\quad\text{(Moody)}                    \\
			                         & = 0.016319\quad\text{(Swamee-Jain)}            \\
			                         & = 0.017\quad\text{(from table for steel pipe)} \\
			\intertext{Use $f_T=0.017$}
			K_{\text{exit}}          & = 1                                            \\
			2\times K_{\text{elbow}} & = 2\times f_T\left(\frac{L_e}{D}\right)        \\
			                         & = 2\times 0.017(30) = 1.02                     \\
			K_{\text{globe}}         & = f_T\left(\frac{L_e}{D}\right)                \\
			                         & = 0.017(340) = 5.78                            \\
			\Sigma K                 & = 7.8                                          \\\
			h_{L_{\text{minor}}}     & = \Sigma K\times\frac{v^2}{2g}                 \\
			                         & = (7.8)0.037852                                \\
			                         & = \bm{0.29525\,\text{m}}                       
		\end{align*}
	}
}
\hfill
\minit[0.45]{
	Total losses in system:
	\cbox[0.9]{
		\large
		\begin{align*}
			h_L & = 9.6431+1.1787        \\
			    & \quad +0.57305+0.29525 \\
			    & =11.690\,\text{m}      
		\end{align*}
	}
	\parb
	Find $y$:
	\cbox[0.9]{
		\large
		\begin{align*}
			\frac{P_1}{\gamma} +z_1+\frac{v_1^2}{2g}-h_L & = \frac{P_2}{\gamma} +z_2+\frac{v_2^2}{2g} \\
			0+y+0-11.690                                 & = 0+0+0                                    \\
			y                                            & =11.690\,\text{m}                          \\\\
			\bm y                                        & = \bm{11.69}\,\textbf{m}                   
		\end{align*}
	}
}
\newpage
%%%%%%%%%%%%%%%%%%%%%%%%%%%%%%%%%%%%%%%%%%%%%%%%%%%%%%%%%%%%%%%%

\textbf{Example 3}:
\cfig[0.5]{../../Figs/07SeriesPipeline/07SeriesPump}
\parb
\minit[0.45]{
	Water at $25\text{ \textcelsius}$ is pumped from tank $A$ to tank $B$. Both tanks are open to the atmosphere.
	\parb
	The suction pipe is $4\text{-in}$ Schedule $40$ steel pipe, has a well-rounded ($r/D>0.15$) entrance, contains a fully
	open globe valve, and is $17.0\,\text{m}$ long.
	\parb
	The discharge pipe is $3\text{-in}$ Schedule $40$ steel pipe, contains a fully open globe valve and three standard $90\degree$ elbows; it is $163.3\,\text{m}$ long.
	\parb
	The elevation difference between $A$ and $B$ is $y=12.75\,\text{m}$ and the volume flow rate is $Q=900\,\text{L/min}$.
	\parb
	If the pump is $78\%$ efficient, determine the electrical power it uses.
	\parb
	\Large
	\textbf{Solution}:\\
	\normalsize
	
	\parb
	\underline{\bfseries Suction Pipe}
	\cbox[0.9]{
		\large
		\begin{align*}
			v                  & = \frac{0.900/60\,\mathsf{m^3/s}}{\pi(0.1023\,\text{m})^2/4} = 1.8249\,\text{m/s} \\
			\frac{v^2}{2g}     & = 0.16974\,\text{m}                                                               \\
			N_R                & = \frac{1.8249(0.1023)997}{8.91\times 10^{-4}} = 208900                           \\
			                   & = 2.0890 \times 10^5                                                              \\
			\frac{D}{\epsilon} & = \frac{0.1023}{4.6\times 10^{-5}}	= 2223.9                                       \\
		\end{align*}
		$N_R>4000$ so flow is turbulent
	}
}
\hfill
\minit[0.45]{
	\parb
	Friction Losses
	\cbox[0.9]{
		\large
		\begin{align*}
			f   & = 0.018587   \quad\text{(Swamee-Jain)}            \\
			h_L & = f\frac{L}{D} \frac{v^2}{2g}                     \\
			    & = 0.018587\left( \frac{17}{0.1023}\right) 0.16974 \\
			    & = \bm{0.52432}\text{ m}                           
		\end{align*}
	}
	\parb
	
	Minor Losses
	\cbox[0.9]{
		\large
		\begin{align*}
			f_T       & = 0.017                                \\
			k_{ent}   & = 0.04                                 \\
			k_{valve} & = 0.017(340)= 5.78                     \\\\
			h_L       & = \left(\Sigma k \right)\frac{v^2}{2g} \\
			          & = 5.82(0.16975)                        \\
			          & = \bm{0.98795}\text{ m}                
		\end{align*}
	}
}
\newpage
\minit[0.45]{
	\underline{\bfseries Discharge Pipe}
	\cbox[0.9]{
		\large
		\begin{align*}
			v                  & = \frac{0.900/60\,\mathsf{m^3/s}}{\pi(0.0779\,\text{m})^2/4} = 3.1472\,\text{m/s} \\
			\frac{v^2}{2g}     & = 0.50484\,\text{m}                                                               \\
			N_R                & = \frac{3.1472(0.0779)997}{8.91\times 10^{-4}} = 274330                           \\
			                   & = 2.7433 \times 10^5                                                              \\
			\frac{D}{\epsilon} & = \frac{0.0779}{4.6\times 10^{-5}}	= 1693.5                                       
		\end{align*}
	}
	\parb
	Friction Losses
	\cbox[0.9]{
		\large
		\begin{align*}
			f   & = 0.018941  \quad \text{(Swamee-Jain)}               \\
			h_L & = f\frac{L}{D} \frac{v^2}{2g}                        \\
			    & = 0.018941\left( \frac{163.3}{0.0779}\right) 0.50484 \\
			    & = \bm{20.045}\,\text{m}                              
		\end{align*}
	}
	
	\parb
	Minor Losses
	\cbox[0.9]{
		\large
		\begin{align*}
			f_T        & = 0.018                                \\
			k_{exit}   & = 1                                    \\
			k_{elbows} & = 3\times 0.018(30)                    \\
			           & = 1.62                                 \\
			k_{valve}  & = 0.018(340)	= 6.12                    \\\\
			h_L        & = \left(\Sigma k \right)\frac{v^2}{2g} \\
			           & = 8.74(0.50484)                        \\
			           & = \bm{4.4123}\,\text{m}                
		\end{align*}
	}
	
	
	\parb
	Total Head-loss
	
	
	\cbox[0.9]{
		\large
		\begin{align*}
			h_L   & =  0.52432 + 0.98795 \\
			      & \quad +20.045+4.4123 \\
			h_L	= & 25.970 \,\text{m}    
		\end{align*}
	}
}
\hfill
\minit[0.45]{
	General Energy Equation
	\cbox[0.9]{
		\large
		\begin{align*}
			\frac{P_1}{\gamma} +z_1+\frac{v_1^2}{2g}+h_A -h_L & = \frac{P_2}{\gamma} +z_2+\frac{v_2^2}{2g} \\
			0+0+0+h_A-25.970                                  & = 0+12.75+0                                \\
			h_A                                               & =38.720\,\text{m}                          
		\end{align*}
		\begin{align*}
			P_{\text{added}}     & = h_A \gamma Q                 \\
			                     & = 38.720 (9.78) (0.900/60)     \\
			                     & = 5.6802\text{ kW}             \\\\
			P_{\text{in}}        & = \frac{P_{\text{added}}}{e_M} \\
			                     & = \frac{5.6802}{.78}           \\
			                     & = 7.2823\text{ kW}             \\\\
			\bm{P_{\textbf{in}}} & = \bm{7.28}\,\textbf{kW}       
		\end{align*}
	}
}
\newpage
\minit[0.475]{
	\textbf{Example 4}:
	\cfig[0.35]{../../figs/07SeriesPipeline/07SeriesCycleOil}
	Heavy machine oil (sg=$0.89$, $\eta=3.80\times10^{-2}\mathsf{\,Pa\!\cdot\! s}$) is circulated through a system repeatedly to test its stability.
	\parm
	The 8-inch Schedule steel pipe on the suction side of the pump has a square entrance and a length of $6.25\,\text{m}$ and the
	3.5-inch Schedule steel pipe on the discharge side of the pump has a length of $18.0\,\text{m}$.
	\parm
	(Note that the $3.5$-inch discharges into the atmosphere above the tank so there is no exit loss in this question!)
	\parm
	All elbows are long radius. The flow rate through the system is $13.5\,\text{L/s}$. \parm
	Determine the head added by the pump.
	\parb
	{\Large \textbf{Solution}:}
	\parb
	\underline{\textbf{Suction Pipe}}
	\cbox[0.9]{
		\large
		\begin{align*}
			v              & = \frac{0.0135\ \mathsf{m^3/s}}{\pi(0.2027\,\text{m})^2/4}=0.41835\,\text{m/s} \\
			\frac{v^2}{2g} & = 0.0089202\,\text{m}                                                          \\
			N_R            & = \frac{0.41835(0.2027)890}{3.80\times10^{-2}}                                 \\
			               & = 1986.1                                                                       
		\end{align*}
		$N_R<2000$ so flow is \textbf{laminar} (and $D/\epsilon$ is not required)
	}
	% \minit[0.45]{
	\parb
	Friction losses:
	\cbox[0.9]{
		\large
		\begin{align*}
			f   & = \frac{64}{N_R}\ =\ \frac{64}{1986.1}\ =\ 0.032224           \\
			h_L & = 0.032224\left(\frac{6.25}{0.2027}\right)0.0089202\,\text{m} \\
			    & =\bm{0.0088630}\,\text{m}                                     
		\end{align*}
	}
}
\hfill
\minit[0.45]{
	\parm
	Minor Losses:
	\cbox[0.9]{
		\large
		\begin{align*}
			k_{ent} & = 0.5                           \\
			h_L     & = 0.5\times 0.0089202\,\text{m} \\
			        & = \bm{0.0044601}\,\text{m}      
		\end{align*}
	}
	\parm
	\underline{\textbf{Discharge Pipe}}
	\cbox[0.9]{
		\large
		\begin{align*}
			v                  & = \frac{0.0135\ \mathsf{m^3/s}}{\pi(0.0901)^2/4}=2.1174 \text{ m/s} \\
			\frac{v^2}{2g}     & = 0.22850\,\text{m}                                                 \\
			N_R                & = \frac{2.1174(0.0901)890)}{3.80\times10^{-2}}=4468.2               \\
			\frac{D}{\epsilon} & = \frac{0.0901}{4.6\times 10^{-5}}                                  \\
		\end{align*}
		$N_R>4000$ so flow is turbulent.
	}
	\parm
	Friction losses:
	\cbox[0.9]{
		\large
		\begin{align*}
			f   & = \quad 0.039\,\text{(Moody)}                            \\
			    & = \text{(}0.039792\,\text{from S-J)}                     \\
			h_L & = 0.039\left(\frac{18.0}{0.0901}\right)0.22850\,\text{m} \\
			    & =\bm{1.7803}\,\text{m}                                   
		\end{align*}
	}
	\parm
	Minor losses:
	\cbox[0.9]{
		\large
		\begin{align*}
			f_T             & = 0.0165\quad\text{(Moody)} \\
			3\times k_{elb} & = 3(0.0165)(20) = 0.99      \\
			h_L             & = 0.99\times 0.22850        \\
			                & = \bm{0.22622}\,\text{m}    
		\end{align*}
	}
	\parm
	Total headloss = $ 2.0198\,\text{m}$
	\cbox[0.9]{
		\large
		\begin{align*}
			\cancel{\frac{P_A}{\gamma}}+\cancel{z_A}+\cancel{\frac{v_A^2}{2g}}+h_A-h_L & = \cancel{\frac{P_B}{\gamma}}+z_B+\frac{v_B^2}{2g} \\
			h_A-2.0198                                                                 & = 0.425+0.22850                                    \\
			                                                                           & = 2.6733\,\text{m}                                 \\\\
			\bm{h_A}                                                                   & = \bm{2.67\,\text{m}}                              
		\end{align*}
	}
}
\newpage
\textbf{Example 5}:
\cfig[0.5]{../../Figs/07SeriesPipeline/pumpedStorage}
The system illustrated is a pumped storage system. During periods of high demand for electricity, water flows down
from the upper lake and drives the turbine at $D$. (During periods of low demand when electricity is cheap, such as at
night-time, $D$ acts as a pump and pumps water back up to the upper lake.)
\parm
At times of maximum demand, the system has a maximum volume flow rate of $420\;\mathsf{m^3/s}$. Base your
calculations on this flow. The water is at $10\text{\textcelsius}$.
\parm
The difference in elevation between the surfaces of the two lakes is $542\text{	m}$. \par\medskip
The low pressure tunnel from $A$ to $B$ is $1700\text{ m}$ in length, has a diameter of $10.5\text{ m}$
and is lined with concrete. The shaft and high pressure tunnel from $B$ to $D$ is $1140\text{ m}$ in length, has a diameter of $10.5\text{ m}$ and
is lined with welded steel.
\parm
There are three tailrace tunnels from the turbine to the lower reservoir with the flow equally distributed between them. Each tailrace tunnel is $382\text{ m}$ in length, has a diameter of $8.5\text{ m}$ and is lined with concrete.
\parm
The entrance to the low pressure tunnel at the upper lake has an equivalent length ratio of $Le/D=420$. The bends at
$B$ (considered to be part of tunnel $BCD$) and $C$ are in the steel pipe and each have at equivalent length ratio of $16$. A spherical valve at the inlet of the turbine that shuts off flow when the turbines are not operating is hydraulically efficient and has no losses associated with it. Each tailrace tunnel contains a butterfly valve ($Le/D=20$).
\parm
At maximum capacity, the turbine outputs $1800 \text{ MW}$. Determine the efficiency of the turbine at this output.
\parb
\vspace{1cm}
\minit[0.45]{
	\textbf{\Large Solution}:
	\parb
	\underline{\textbf{Tunnel AB}}
	\cbox[0.9]{
		\large
		\begin{align*}
			v                                    & = \frac{420\mathsf{\;m^3/s}}{\pi(10.5\mathsf{ m})^2/4}=4.8504\,\text{m/s}         \\
			\frac{v^2}{2g}                       & = \frac{(4.8504\mathsf{\;m/s})^2}{2\times9.81\mathsf{\;m/s^2}} = 1.1991\,\text{m} \\
			N_R                                  & = \frac{4.8504(10.5)1000}{0.0013} = 39176000                                      \\
			                                     & = 3.9176\times 10^{7}                                                             \\
			\frac{D}{\epsilon_{\text{concrete}}} & =\frac{10.5\;\text{m}}{1.2\times10^{-3}\text{ m}} = 8750                          
		\end{align*}
		\parb
		$N_R>4000$ so flow is turbulent.
	}
}
\hfill
\minit[0.45]{
	Friction Losses
	\cbox[0.9]{
		\large
		\begin{align*}
			f   & = \quad 0.0123\quad\text{(Moody)}            \\
			h_L & = 0.0123\left(\frac{1700}{10.5}\right)1.1991 \\
			    & = \bm{2.3879}\,\text{m}                      
		\end{align*}
	}
	\parb
	Minor Losses
	\cbox[0.9]{
		\begin{align*}
			f_T                 & = 0.0123\quad\text{(Moody)} \\
			k_{\text{entrance}} & = 0.0123(420) = 5.1660      \\
			h_L                 & = 5.1660(1.1991)            \\
			                    & = \bm{6.1946}\,\text{m}     
		\end{align*}
	}
}
\newpage
\minit[0.45]{
	\underline{\textbf{Tunnel BCD}}
	\cbox[0.9]{
		\begin{align*}
			v                          & = 4.8504\mathsf{\,m/s}                                      \\
			\frac{v^2}{2g}             & = 1.1991\mathsf{\,m}                                        \\
			N_R                        & = 3.9176\times 10^{7}                                       \\
			\frac{D}{\epsilon_{steel}} & = \frac{10.5\;\text{m}}{4.6\times10^{-5}\text{ m}} = 228260 \\
		\end{align*}
		$N_R>4000$ so flow is turbulent.
	}
	\parb
	Friction Losses
	\parb
	Note: Readings for $f$ and $f_T$ lie outside the boundary of the Moody Diagram (because of the high relative roughness and high Reynolds Number) so the Swamee-Jain is probably more accurate than attempting extrapolation from the Moody Diagram.
	\cbox[0.9]{
		\begin{align*}
			f   & = 0.0077125                                   \\
			h_L & = 0.0077125\cdot\frac{1140}{10.5}\cdot 1.1991 \\
			    & = \bm{1.0041}\,\text{m}                       
		\end{align*}
		
	}
	\parb
	Minor Losses
	\cbox[0.9]{
		\begin{align*}
			f_T                  & = 0.0071174                \\
			k_{\text{bend at B}} & = (0.0071174)16  = 0.11388 \\
			k_{\text{bend at C}} & = (0.0071174)16  = 0.11388 \\
			h_L                  & = 2(0.11388)1.1991         \\
			                     & = \bm{0.27310}\,\text{m}   
		\end{align*}
	}
	\parb
	\underline{\textbf{A single tailrace tunnel $\bm{DE}$}}
	\cbox[0.9]{
		\begin{align*}
			v              & = \frac{140\mathsf{\, m^3/s}}{\pi(8.5\mathsf{\, m})^2/4}=2.4672\mathsf{\, m/s} \\
			\frac{v^2}{2g} & = 0.31025\mathsf{\, m}                                                         \\
			N_R            & = 	\frac{2.4672(8.5)1000}{0.0013} = 16132000                                   \\
			               & = 1.6132\times 10^7                                                            \\
			\frac{D}{\epsilon_{concrete}} = \frac{8.5\;\text{m}}{1.2\times10^{-3}\text{ m}} = 7083
		\end{align*}
		$N_R>4000$ so flow is turbulent.
	}
}


\end{document}
