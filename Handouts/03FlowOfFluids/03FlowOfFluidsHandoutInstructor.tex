% override specific chktex warnings
% chktex-file 46 - don't use $ instead of \(, etc)
% chktex-file 1 - ignore commands followed by a space, e.g. \\ new line here
% chktex-file 9 - sometimes messes up with ( and {

\documentclass[10pt]{amsart}
\usepackage[showboxes]{textpos}

\usepackage{amsmath}
\usepackage{amsthm}
\usepackage{amsfonts}
\usepackage{amssymb}
\usepackage{mathpazo}
\usepackage{booktabs}
\usepackage[usenames,x11names]{xcolor}
\usepackage{tikz}
\usepackage{textcomp}
\usepackage[letterpaper]{geometry}
\geometry{verbose,tmargin=0.5in,bmargin=0.5in,lmargin=0.75in,rmargin=0.75in}
\usepackage{multicol}
\usepackage{bm}
\usepackage{comment}
\usepackage{cancel}
\usepackage{array}

\usepackage[many]{tcolorbox}

\pagestyle{plain}
\raggedright
\renewcommand{\familydefault}{\sfdefault}
\setlength{\parskip}{\medskipamount}
\setlength{\columnsep}{2cm}

\everymath{\displaystyle}
\setlength{\parskip}{\bigskipamount}

% counter for resuming enumerated list numbers
\newcounter{resumeenumi}
\newcommand{\suspend}{\setcounter{resumeenumi}{\theenumi}}
\newcommand{\resume}{\setcounter{enumi}{\theresumeenumi}}

\newcommand\lb{\linebreak}
\newcommand\pars{\par\smallskip}
\newcommand\parm{\par\medskip}
\newcommand\parb{\par\bigskip}

\makeatletter
\providecommand{\gettikzxy}[3]{%
	\tikz@scan@one@point\pgfutil@firstofone#1\relax
	\edef#2{\the\pgf@x}%
	\edef#3{\the\pgf@y}%
}
\makeatother



% full width colored block but color specifiable
%\cb[body bg strength]{header bg}{header text}{body text}
\newcommand{\cb}[4][15]{
	\setbeamercolor{block title}{bg = #2}
	\setbeamercolor{block body}{bg = #2!#1}
	\setbeamercolor{item projected}{bg=#2, fg=white}
	\begin{center}
		\begin{block}{#3}
			#4
		\end{block}
	\end{center}
}

% colored block with width specified
% \cbw[body bg strength]{header bg}{width}{header text}{body text}
\newcommand{\cbw}[5][15]{
	\begin{center}
		%\vspace{-0.35cm}
		\begin{minipage}{#3\textwidth}
			\setbeamercolor{block title}{bg= #2}
			\setbeamercolor{block body}{bg= #2!#1}
			\setbeamercolor{item projected}{bg=#2, fg=white}
			\begin{block}{#4}
				\raggedright
				#5
			\end{block}
		\end{minipage}
	\end{center}
}

% centered minipage with text \raggedright
%\cmini[width]{content}
\newcommand{\cmini}[2][0.8]{
	\begin{center}
		\begin{minipage}{#1\columnwidth}
			\raggedright
			#2
		\end{minipage}
	\end{center}
}

%left flushed minipage
\newcommand{\mini}[2][0.8]{
	\begin{minipage}{#1\columnwidth}
		\raggedright
		#2
	\end{minipage}
}

%left flushed minipage, top aligned
\newcommand{\minit}[2][0.8]{
	\begin{minipage}[t]{#1\columnwidth}
		\raggedright
		#2
	\end{minipage}
}

%left flushed minipage
% \newcommand{\miniT}[2][0.8]{
%  \begin{minipage}[T]{#1\columnwidth}
%   \raggedright
%   #2
%  \end{minipage}
% }

%left flushed minipage
\newcommand{\minib}[2][0.8]{
	\begin{minipage}[b]{#1\columnwidth}
		\raggedright
		#2
	\end{minipage}
}

\newcommand{\cfig}[2][1]{% centred, scaled graphic
	\begin{center}
		\includegraphics[scale=#1]{#2}
	\end{center}
}
% figure with tight border for photos
% \cfigb[saitMaroon]{borderwidth with unit}{scale}{image}
\newcommand{\cfigb}[4][structure]{
	% \usepackage{adjustbox}
	\setlength{\fboxrule}{1pt}
	\begin{center}
		\includegraphics[scale=#3, cframe= #1 #2]{#4}
	\end{center}
}

\newcommand{\imgbox}[3]{
	% \setlength{\fboxsep}{12pt}
	\includegraphics[scale=#1, cframe= structure #3]{#2}
}

% \imgboxbg[bg color=white]{scale}{path/to/img}{border color}{border, e.g. 2pt}{margin, e.g. 4pt}
\newcommand{\imgboxbg}[6][white]{
	\setlength{\fboxrule}{#5}
	\setlength{\fboxsep}{#6}
	\centering
	\fcolorbox{#4}{#1}{\includegraphics[scale=#2]{#3}}
}

\newcommand{\fig}[2][1]{% scaled graphic
	\includegraphics[scale=#1]{#2}
}

% centred framed  box black border
%\cbox[width]{content}
\newcommand{\cbox}[2][0.9]{% framed centered  box
	\setlength\fboxsep{0.042\columnwidth}
	\setlength\fboxrule{0.0015\columnwidth}
	\begin{center}
		\fcolorbox{black}{white}{
			\vspace{-0.5cm}
			\begin{minipage}{#1\columnwidth}
				\raggedright
				#2
			\end{minipage}
		}
	\end{center}
	\setlength\fboxsep{0cm}
}



\newtcolorbox{mybox}[1][]
{
	colback=white,
	top=0.25cm,
	bottom=0.25cm,
	left=0.25cm,
	right=0.25cm,
	colframe=structure,
	fonttitle=\bfseries,
	enhanced, drop fuzzy shadow,
	% attach boxed title to top left={yshift=-2mm, xshift=5mm},
	attach boxed title to top left={yshift=-2mm, xshift=5mm}, colbacktitle=structure!80!white, #1}

\newtcolorbox{plainbox}[1][]{colback=white, sharp corners, top=0.125cm, bottom=0.125cm, left=0pt, right=0pt, boxrule=0.5pt,colframe=structure,fonttitle=\bfseries, colbacktitle=structure, arc=0mm, #1}
%
\newtcbtheorem{myexam}{Example}%
{
	enhanced,
	colback=white,
	top=0.375cm,
	bottom=0.25cm,
	left=0.375cm,
	right=0.375cm,
	colframe=structure,
	fonttitle=\bfseries,
	drop fuzzy shadow,
	%description font=\mdseries\itshape,
	attach boxed title to top left={yshift=-2mm, xshift=5mm},
	colbacktitle=structure!80!white
	}{exam}% then \pageref{exer:theoexample} references the theo

\newcommand{\myexample}[2][red]{
	% \tcb\tcbset{theostyle/.style={colframe=red,colbacktitle=yellow}}
	\begin{myexam}{}{}
		\raggedright
		#2
	\end{myexam}
	% \tcbset{colframe=structure,colbacktitle=structure}
}

\newtcbtheorem{myexer}{Exercise}%
{
	enhanced,
	colback=white,
	top=0.375cm,
	bottom=0.25cm,
	left=0.375cm,
	right=0.375cm,
	colframe=structure,
	fonttitle=\bfseries,
	drop fuzzy shadow,
	%description font=\mdseries\itshape,
	attach boxed title to top left={yshift=-2mm, xshift=5mm},
	colbacktitle=structure!80!white
	}{exer}

\newcommand{\myexercise}[2][red]{
	% \tcb\tcbset{theostyle/.style={colframe=red,colbacktitle=yellow}}
	\begin{myexer}{}{}
		\raggedright
		#2
	\end{myexer}
	% \tcbset{colframe=structure,colbacktitle=structure}
}


\begin{document}

\thispagestyle{empty}
\vspace{-6cm}
\centering

\textbf{\Large Module 3: Flow of Fluids and Bernoulli's Equation (CIVL 318)}
\parb
\begin{center}
	\begin{tabular}{r >{$}r<{$} >{$}c<{$} >{$}l<{$}}
		\toprule
		\addlinespace
		\textbf{ Volume Flow Rate}:     & Q                                       & = & Av                                      \\
		\addlinespace
		\midrule
		\addlinespace
		\textbf{ Mass Flow Rate}:       & M                                       & = & \rho Q                                  \\
		\addlinespace
		\midrule
		\addlinespace
		\textbf{ Weight Flow Rate}:     & W                                       & = & \gamma Q                                \\
		\addlinespace
		\midrule
		\addlinespace
		\textbf{ Continuity Equation}:  & A_1v_1                                  & = & A_2v_2                                  \\
		\addlinespace
		\midrule
		\addlinespace
		\textbf{ Bernoulli's Equation}: & \frac{p_A}{\gamma}+z_A+\frac{v_A^2}{2g} & = & \frac{p_B}{\gamma}+z_B+\frac{v_B^2}{2g} \\
		\addlinespace
		\bottomrule
	\end{tabular}
\end{center}
\vspace{2cm}

\mini[0.5]
{
	
	\textbf{\normalsize Table F: Schedule 40 Steel Pipe}\parb
	\begin{tabular}{>{$}c<{$} >{$}c<{$} >{$}c<{$} >{$}c<{$}  >{$}c<{$}}
		
		\toprule
		\addlinespace
		\text{Nominal} & \text{Inside}   & \qquad\qquad & \text{Nominal} & \text{Inside}   \\
		\text{Size}    & \text{Diameter} &              & \text{Size}    & \text{Diameter} \\
		
		\addlinespace
		\text{(in)}    & \text{(mm)}     &              & \text{(in)}    & \text{(mm)}     \\
		\addlinespace
		\midrule
		\addlinespace
		\tfrac18       & 6.8             &              & 4              & 102.3           \\ \addlinespace
		\tfrac14       & 9.2             &              & 5              & 128.2           \\ \addlinespace
		\tfrac38       & 12.5            &              & 6              & 154.1           \\ \addlinespace
		\tfrac12       & 15.8            &              & 8              & 202.7           \\ \addlinespace
		\tfrac34       & 20.9            &              & 10             & 254.5           \\ \addlinespace
		1              & 26.6            &              & 12             & 303.2           \\ \addlinespace
		1\tfrac14      & 35.1            &              & 14             & 333.4           \\	\addlinespace
		1\tfrac12      & 40.9            &              & 16             & 381.0           \\ \addlinespace
		2              & 52.5            &              & 18             & 428.7           \\ \addlinespace
		2\tfrac12      & 62.7            &              & 20             & 477.9           \\ \addlinespace
		3              & 77.9            &              & 24             & 574.7           \\ \addlinespace
		3\tfrac12 	& 90.1 \\
		
		\midrule
		\bottomrule
	\end{tabular}
}
\hfill
\mini[.4]
{
	
	\textbf{\Large Table G: Dimensions of Steel Tubing}\parb
	
	\begin{tabular}{>{$}c<{$} >{$}c<{$} >{$}c<{$} >{$}c<{$}}
		
		\toprule
		\addlinespace
		\text{Outside}  & \text{Outside}  & \text{Wall}      & \text{Inside}   \\
		\text{Diameter} & \text{Diameter} & \text{Thickness} & \text{Diameter} \\ \addlinespace
		\text{(in)}     & \text{(mm)}     & \text{(mm)}      & \text{(mm)}     \\ \addlinespace
		\midrule
		\addlinespace
		\tfrac{1}{8}    & 3.18            & 0.813            & 1.549           \\
		                &                 & 0.889            & 1.397           \\ \addlinespace
		\tfrac{3}{16}   & 4.76            & 0.813            & 3.137           \\
		                &                 & 0.889            & 2.985           \\ \addlinespace
		\tfrac{1}{4}    & 6.35            & 0.889            & 4.572           \\
		                &                 & 1.24             & 3.861           \\ \addlinespace
		\tfrac{5}{16}   & 7.94            & 0.889            & 6.160           \\
		                &                 & 1.24             & 5.448           \\ \addlinespace
		\tfrac{3}{8}    & 9.53            & 0.889            & 7.747           \\
		                &                 & 1.24             & 7.036           \\ \addlinespace
		\tfrac{1}{2}    & 12.70           & 1.24             & 10.21           \\
		                &                 & 1.65             & 9.46            \\ \addlinespace
		\tfrac{5}{8}    & 15.88           & 1.24             & 13.39           \\
		                &                 & 1.65             & 12.57           \\ \addlinespace
		\tfrac{3}{4}    & 19.05           & 1.24             & 16.56           \\
		                &                 & 1.65             & 15.75           \\ \addlinespace
		\tfrac{7}{8}    & 22.23           & 1.24             & 19.74           \\
		                &                 & 1.65             & 18.92           \\ \addlinespace
		1               & 25.40           & 1.65             & 22.10           \\
		                &                 & 2.11             & 21.18           \\ \addlinespace
		1\tfrac{1}{4}   & 31.75           & 1.65             & 28.45           \\
		                &                 & 2.11             & 27.53           \\ \addlinespace
		1\tfrac{1}{2}   & 38.10           & 1.65             & 34.80           \\
		                &                 & 2.11             & 33.88           \\ \addlinespace
		1\tfrac{3}{4}   & 44.45           & 1.65             & 41.15           \\
		                &                 & 2.11             & 40.23           \\ \addlinespace
		2               & 50.80           & 1.65             & 47.50           \\
		                &                 & 2.11             & 46.587          \\ \addlinespace
		\midrule
		\bottomrule
	\end{tabular}
}


%%%%%%%%%%%%%%%%%%%%%%%%%%%%%%%%%%%%%%%%%%%%%%%%%%%%%%%%%%%%%%%%%%%%%%%%%%%%%%%%%%%%%%%%%%%%%%%%%%%%%%%%%%%%%%%%%%%%%

\vfill
\newpage

%%%%%%%%%%%%%%%%%%%%%%%%%%%%%%%%%%%%%%%%%%%%%%%%%%%%%%%%%%%%%%%%%%%%%%%%%%%%%%%%%%%%%%%%%%%%%%%%%%%%%%%%%

\begin{minipage}[t]{0.5\textwidth}
	\raggedright
	\textbf{Example 1}:
	\begin{cfig}[0.5]{../../figs/03FlowOfFluids/04BernoulliEx01}\end{cfig}
	The average velocity of the flow at the nozzle $C$ is $4.7\,\text{m/s}$.\par
	
	Determine:
	\begin{itemize}
		\item the average flow velocity at $A$
		\item the average flow velocity at $B$
		\item the volume flow rate, $Q$, through the system in L/s.
	\end{itemize}
	\textbf{Solution}:
	\cbox{
		\Large
		\begin{align*}
			A_Av_A & = A_Cv_C \Rightarrow v_A=\frac{A_C}{A_A}v_C                                  \\
			v_A    & = \frac{\pi(0.035\,\text{m})^2/4}{\pi(0.115\,\text{m})^2/4}(4.7\,\text{m/s}) \\
			       & = \frac{(0.035\,\text{m})^2}{(0.115\,\text{m})^2}(4.7\,\text{m/s})           \\
			       & = 0.43535\,\text{m/s}                                                        \\
			       & = 0.435\,\text{m/s}                                                          
		\end{align*}
	}
	\parm
	\cbox{
		\Large
		\begin{align*}
			A_B v_B & = A_C v_C \Rightarrow v_A=\frac{A_B}{A_A}v_B                                 \\
			v_B     & = \frac{\pi(0.035\,\text{m})^2/4}{\pi(0.080\,\text{m})^2/4}(4.7\,\text{m/s}) \\
			        & = \left(\frac{0.035\,\text{m}}{0.080\,\text{m}}\right)^2(4.7\,\text{m/s})    \\
			        & = 0.89961\,\text{m/s}                                                        \\
			        & = 0.900\,\text{m/s}                                                          
		\end{align*}
	}
	\parm
	\cbox{
		\Large
		\begin{align*}
			Q & = A_Cv_C = \pi(0.035\,\text{m})^2/4\times(4.7\,\text{m/s}) \\
			  & = 0.0045219\,\text{m}^3/\text{s}                           \\
			  & = 4.52\,\text{L/s}                                         
		\end{align*}
	}
\end{minipage}
\hfill
\begin{minipage}[t]{0.45\textwidth}
	\raggedright
	\textbf{Example 2}:\parb
	Water, at $70\text{\textcelsius}$ flows through $\tfrac78\text{-in.}$ steel tubing, with $1.65\,\text{mm}$
	wall thickness, at an average velocity of $5.7\,\text{m/s}$. Determine:
	\begin{itemize}
		\item the volume flow rate, $Q$
		\item the mass flow rate, $M$
		\item the weight flow rate, $W$
	\end{itemize}
	\parb
	\textbf{Solution}:\parb
	\cbox{
		\Large
		\begin{align*}
			Q & = Av                                        \\
			  & = \frac{\pi(0.01892)^2}{4}(5.7\,\text{m/s}) \\
			  & = 0.0016025\,\text{m}^3/\text{s}            \\
			  & = 1.603\,\text{L/s}                         
		\end{align*}
	}
	\parb
	\cbox{
		\Large
		\begin{align*}
			M & = \rho_{70\text{\textcelsius}}(0.0016025\,\text{m}^3/\text{s}               \\
			  & =\left(978\,\text{kg/m}^3\right)\left(0.0016025\,\text{m}^3/\text{s}\right) \\
			  & =1.5673 \,\text{kg/s}                                                       \\
			  & = 1.567\,\text{kg/s}                                                        
		\end{align*}
	}
	\parb
	\cbox{
		\Large
		\begin{align*}
			W & = \gamma_{70\text{\textcelsius}}(0.0016025\,\text{m}^3/\text{s}             \\
			  & =\left(9.59\,\text{kN/m}^3\right)\left(0.016025\,\text{m}^3/\text{s}\right) \\
			  & =0.015368 \,\text{kN/s}                                                     \\
			  & = 15.37\,\text{N/s}                                                         
		\end{align*}
	}
	
\end{minipage}


%%%%%%%%%%%%%%%%%%%%%%%%%%%%%%%%%%%%%%%%%%%%%%%%%%%%%%%%%%%%%%%%%%%%%%%%%%%%%%%%%%%%%%%%%%%%%%%%%%%%%%%%%

\begin{minipage}[t]{0.45\textwidth}
	\raggedright
	\textbf{Example 3}:\\
	\vspace*{-0.5cm}
	\begin{cfig}[0.5]{../../figs/03FlowOfFluids/04BernoulliEx03}\end{cfig}
	Oil, with a specific gravity of $0.83$, flows under gravity from a tank, through a pipe system as shown, before
	entering the atmosphere through a nozzle at $D$. \par
	Determine:
	\begin{itemize}
		\item the pressure at $A$
		\item the pressure at $B$
		\item the pressure at $C$
		\item the volume flow rate through the system
	\end{itemize}
	\parb
	\textbf{Solution}:
	\parb
	\raggedright
	We begin by applying Bernoulli's Equation between a point $S$ on the surface of the tank and the nozzle $D$.
	(These two points are chosen since the pressure and elevations at each are known. We also know the velocity head at $S$ so the only unknown is the velocity at $D$.)
	\cbox{
		\Large
		\begin{align*}
			\frac{p_S}{\gamma} + z_S + \frac{v_S^2}{2g} & = \frac{p_D}{\gamma} + z_D + \frac{v_D^2}{2g} \\
			0 + 4.3 + 0                                 & = 0 + 0 + \frac{v_D^2}{2g}                    \\
			v_D^2                                       & = 4.3(2\times9.81)                            \\
			v_D                                         & = \sqrt{84.366}=9.1851\,\text{m/s}            
		\end{align*}
	}
	\parm
	
	Then, using the continuity equation, we can find the velocities at $A$ and $B$, and at $C$. \\(Velocities at $A$ and $B$ are the same since the diameter of the tubing is the same.)
\end{minipage}
\hfill
\begin{minipage}[t]{0.45\textwidth}
	\cbox{
		\Large
		\begin{align*}
			A_Dv_D          & = A_Av_A                                                                                 \\
			\Rightarrow v_A & = \frac{A_Dv_D}{A_A}                                                                     \\
			                & = \frac{\pi(0.0125\,\text{mm})^2/4}{\pi(0.0273\,\text{mm})^2/4}\cdot(9.1851\,\text{m/s}) \\
			                & = 1.8936\,\text{m/s}                                                                     \\
			\Rightarrow v_B & =1.8936\,\text{m/s}                                                                      \\\\
			v_C             & = \frac{A_Dv_D}{A_C}                                                                     \\
			                & =                                                                                        
			\frac{\pi(0.0125\,\text{mm})^2/4}{\pi(0.04658\,\text{mm})^2/4}\cdot(9.1851\,\text{m/s})\\
			                & = 0.66146\,\text{m/s}                                                                    
		\end{align*}
	}
	\parb
	\raggedright
	Now, we can use Bernoulli's Equation between the surface and $A$ or between $A$ and the nozzle $D$ to find $p_A$. I use the surface simply because it has no velocity head:
	\cbox{
		\begin{align*}
			\frac{p_S}{\gamma} + z_S + \frac{v_S^2}{2g} & = \frac{p_A}{\gamma} + z_A + \frac{v_A^2}{2g}                      \\
			0 + 3.7 + 0                                 & = \frac{p_A}{0.83\times9.81} + 0 + \frac{(1.8936)^2}{2\times 9.81} \\
			p_A                                         & = 0.83\times 9.81\left(3.7-\frac{(1.8936)^2}{2\times	9.81}\right)  \\
			p_A                                         & = 28.638\,\text{kPa}                                               \\
			\bm{p_A}                                    & \bm{= 28.6\,\textbf{kPa}}                                          
		\end{align*}
	}
	\parb
	This procedure, using the surface and $B$, can be repeated to find $p_B$. Alternatively, using $A$ and $B$ is a good
	choice since they have the same velocity and the velocity head terms cancel out of Bernoulli's Equation:
	\cbox{
		\begin{align*}
			\frac{p_B}{\gamma} + z_B +\cancel{\frac{v_B^2}{2g}} & = \frac{p_A}{\gamma} + z_A + \cancel{\frac{v_A^2}{2g}} \\
			\frac{p_B}{0.83\times 9.81} + 0.5                   & = \frac{p_A}{0.83\times 9.81} + 0                      \\
			p_B                                                 & = p_A - 0.5\gamma                                      \\
			                                                    & = 28.638 - 0.5(0.83\times 9.81)                        \\
			p_B                                                 & = 24.567\,\text{kPa}                                   \\
			\bm{p_B}                                            & \bm{= 24.6\,\textbf{kPa}}                              
		\end{align*}
	}
\end{minipage}
\newpage
\flushleft
\begin{minipage}{0.45\textwidth}
	Using the surface and $C$, we get:
	\cbox{
		\begin{align*}
			\frac{p_S}{\gamma} + z_S + \frac{v_S^2}{2g} & = \frac{p_C}{\gamma} + z_C + \frac{v_C^2}{2g}                       \\
			0 + 3.2 + 0                                 & = \frac{p_C}{0.83\times9.81} + 0 + \frac{(0.66146)^2}{2\times 9.81} \\
			p_C                                         & = 0.83\times 9.81\left(3.2-\frac{(0.66146)^2}{2\times 9.81}\right)  \\
			p_C                                         & = 25.874\,\text{kPa}                                                \\
			\bm{p_C}                                    & \bm{= 25.9\,\textbf{kPa}}                                           
		\end{align*}
	}
	\parb
	To find the volume flow rate, use $Q=Av$ at any of the points where $A$ and $v$ are known:
	\cbox{
		\begin{align*}
			Q      & = A_Dv_D                                                     \\
			       & = \frac{\pi(0.0125\,\text{m})^2}{4}\cdot(9.1851\,\text{m/s}) \\
			       & = 0.0011272\,\text{m}^3/\text{s}                             \\
			\bm{Q} & \bm{= 1.127\,\textbf{L/s}}                                   
		\end{align*}
	}
\end{minipage}
\newpage
%%%%%%%%%%%%%%%%%%%%%%%%%%%%%%%%%%%%%%%%%%%%%%%%%%%%%%%%%%%%%%%%%%%%%%%%%%%%%%%%%%%%%%%%%%%%%%%%%%%%%%%%%

\begin{minipage}[t]{0.4\textwidth}
	\raggedright
	\textbf{Example 4}:\\
	\vspace*{-0.5cm}
	\begin{cfig}[0.42]{../../figs/03FlowOfFluids/04BernoulliEx04a}\end{cfig}
	Determine the pressure reading  $p_2$ if $Q=25\,\text{L/s}$
	
	\parb
	\textbf{Solution}:
	\parb
	
	Find the cross-sectional areas at $1$ and $2$:
	\cbox{
		\Large
		\begin{align*}
			A_1 & = \frac{\pi(0.0737\,\text{m})^2}{4} \\
			    & = 0.0042660\,\text{m}^2             \\\\
			A_2 & = \frac{\pi(0.1223\,\text{m})^2}{4} \\
			    & = 0.011747\,\text{m}^2              
		\end{align*}
	}
	\parb
	Find the velocities at $1$ and $2$ for $Q=25\,\text{L/s}$:
	\cbox{
		\Large
		\begin{align*}
			v_1 & = \frac{Q}{A_1}                                            \\
			    & = \frac{0.025\,\text{m}^3/\text{s}}{0.0042660\,\text{m}^2} \\
			    & = 5.8603\,\text{m/s}                                       \\\\
			v_2 & = \frac{Q}{A_2}                                            \\
			    & = \frac{0.025\,\text{m}^3/\text{s}}{0.011747\,\text{m}^2}  \\
			    & = 2.1282\,\text{m/s}                                       
		\end{align*}
	}
\end{minipage}
\hfill
\begin{minipage}[t]{0.52\textwidth}
	\parb
	Calculate the velocity heads at $1$ and $2$ for $Q=25\,\text{L/s}$:
	\cbox{
		\Large
		\begin{align*}
			\frac{v_1^2}{2g} & = \frac{(5.8603)^2}{2\times 9.81}  \\
			                 & = 1.7504\,\text{m}                 \\\\
			\frac{v_2^2}{2g} & = \frac{(2.1282)^2}{2 \times 9.81} \\
			                 & = 0.23085\,\text{m}                
		\end{align*}
	}
	\parb
	Applying Bernoulli's Equation between $1$ and $2$:
	\begin{align*}
		\frac{p_1}{\gamma} + z_1 + \frac{v_1^2}{2g} & = \frac{p_2}{\gamma} + z_2 + \frac{v_2^2}{2g} \\
		\frac{72}{1.26\times9.81} + 0 + 1.7504      & = \frac{p_2}{1.26\times9.81} + 1.4 + 0.23085  \\
		7.5754                                      & = \frac{p_2}{12.361} + 1.6309                 \\
		p_2                                         & = 73.480\,\text{kPa}                          \\
		\bm{p_2}                                    & \bm{= 73.5\,\textbf{kPa}}                     
	\end{align*}
	\parb
	\centering
	(So, in this case, $\bm{p_2>p_1}$.)
	
\end{minipage}

%%%%%%%%%%%%%%%%%%%%%%%%%%%%%%%%%%%%%%%%%%%%%%%%%%%%%%%%%%%%%%%%%%%%%%%%%%%%%%%%%%%%%%%%%%%%%%%%%%%%%%%%%

\begin{minipage}[t]{0.4\textwidth}
	\raggedright
	\textbf{Exercise 1}:\\
	\vspace*{-0.5cm}
	\begin{cfig}[0.42]{../../figs/03FlowOfFluids/04BernoulliEx04a}\end{cfig}
	Determine the pressure reading  $p_2$ if $Q=20\,\text{L/s}$
	
	\parb
	\textbf{Solution}:
	\parb
	
	Find the cross-sectional areas at $1$ and $2$:
	\cbox{
		\Large
		\begin{align*}
			A_1 & = \frac{\pi(0.0737\,\text{m})^2}{4} \\
			    & = 0.0042660\,\text{m}^2             \\\\
			A_2 & = \frac{\pi(0.1223\,\text{m})^2}{4} \\
			    & = 0.011747\,\text{m}^2              
		\end{align*}
	}
	\parb
	Find the velocities at $1$ and $2$ for $Q=20\,\text{L/s}$:
	\cbox{
		\Large
		\begin{align*}
			v_1 & = \frac{Q}{A_1}                                            \\
			    & = \frac{0.020\,\text{m}^3/\text{s}}{0.0042660\,\text{m}^2} \\
			    & = 4.6882\,\text{m/s}                                       \\\\
			v_2 & = \frac{Q}{A_2}                                            \\
			    & = \frac{0.020\,\text{m}^3/\text{s}}{0.011747\,\text{m}^2}  \\
			    & = 1.7026\,\text{m/s}                                       
		\end{align*}
	}
\end{minipage}
\hfill
\begin{minipage}[t]{0.52\textwidth}
	\parb
	Calculate the velocity heads at $1$ and $2$ for $Q=25\,\text{L/s}$:
	\cbox{
		\Large
		\begin{align*}
			\frac{v_1^2}{2g} & = \frac{(4.6882)^2}{2\times 9.81}  \\
			                 & = 1.1203\,\text{m}                 \\\\
			\frac{v_2^2}{2g} & = \frac{(1.7026)^2}{2 \times 9.81} \\
			                 & = 0.14774\,\text{m}                
		\end{align*}
	}
	\parb
	Applying Bernoulli's Equation between $1$ and $2$:
	\begin{align*}
		\frac{p_1}{\gamma} + z_1 + \frac{v_1^2}{2g} & = \frac{p_2}{\gamma} + z_2 + \frac{v_2^2}{2g} \\
		\frac{72}{1.26\times9.81} + 0 + 1.1203      & = \frac{p_2}{1.26\times9.81} + 1.4 + 0.14774  \\
		6.9453                                      & = \frac{p_2}{12.361} + 1.5477                 \\
		p_2                                         & = 66.719\,\text{kPa}                          \\
		\bm{p_2}                                    & \bm{= 66.7\,\textbf{kPa}}                     
	\end{align*}
	\parb
	\centering
	(So, in this case, $\bm{p_2<p_1}$.)
	
\end{minipage}

%%%%%%%%%%%%%%%%%%%%%%%%%%%%%%%%%%%%%%%%%%%%%%%%%%%%%%%%%%%%%%%%%%%%%%%%%%%%%%%%%%%%%%%%%%%%%%%%%%%%%%%%%

\begin{minipage}[t]{0.42\textwidth}
	\raggedright
	\textbf{Example 5}:\\
	\vspace*{-0.5cm}
	\begin{cfig}[0.4]{../../figs/03FlowOfFluids/04BernoulliEx04b}\end{cfig}
	Determine $Q$, the volume flow rate.
	
	\parb
	\textbf{Solution}:
	\parb
	
	This question is subtly different from the previous one. We can solve directly
	for $Q$ or solve for one of either $v_1$ or $v_2$. Let us solve for $Q$ directly here.\parb
	First, find the velocity head values at $1$ and at $2$ in terms of $Q$:
	\cbox{
		\Large
		\begin{align*}			A_1              & = \frac{\pi(0.1023\,\text{m})^2}{4}                       \\
			                 & = 0.0082194\,\text{m}^2                                   \\
			v_1              & = \frac{Q}{A_1}                                           \\
			                 & = \frac{Q}{0.0082194}\,\text{m/s}                         \\
			                 & = 121.66Q                                                 \\
			\frac{v_1^2}{2g} & = \frac{\left(\frac{Q}{0.0082194}\right)^2}{2\times 9.81} \\
			                 & = 754.43Q^2\,\text{m}                                     \\
			\\
			A_2              & = \frac{\pi(0.1541\,\text{m})^2}{4}                       \\
			                 & = 0.018651\,\text{m}^2                                    \\
			                 & = 53.617Q                                                 \\
			v_2              & = \frac{Q}{0.018651}\,\text{m/s}                          \\
			\frac{v_2^2}{2g} & = \frac{\left(\frac{Q}{0.018651}\right)^2}{2\times 9.81}  \\
			                 & = 146.52Q^2\,\text{m}                                     
		\end{align*}}
\end{minipage}
\hfill
\begin{minipage}[t]{0.53\textwidth}
	\raggedright
	Applying Bernoulli's Equation between $1$ and $2$:
	\cbox{
		\large
		\begin{align*}
			\frac{p_1}{\gamma} + z_1 + \frac{v_1^2}{2g} & = \frac{p_2}{\gamma} + z_2 + \frac{v_2^2}{2g}     \\
			\frac{87.2}{0.87\times9.81} + 0 + 754.43Q^2 & = \frac{103.8}{0.87\times9.81} + 1.65 + 146.52Q^2 \\
			10.217 + 754.43Q^2                          & = 13.812 + 146.52Q^2                              \\
			Q^2                                         & = \frac{13.812-10.217}{754.43-146.52}             \\
			\bm{Q}                                      & \bm{\approx 76.9\,\text{L/s}}                     
		\end{align*}
	}
	\parb
	\vspace{\stretch{1}}
\end{minipage}

\newpage
%%%%%%%%%%%%%%%%%%%%%%%%%%%%%%%%%%%%%%%%%%%%%%%%%%%%%%%%%%%%%%%%%%%%%%%%%%%%%%%%%%%%%%%%%%%%%%%%%%%%%%%%%

\cmini[0.76]{
	\textbf{Example 6}:\\
	\centering
	\cfig[0.27]{../../figs/03FlowOfFluids/04BernoulliEx06a}
	Determine $Q$, the volume flow rate.
	
	\parb\flushleft
	\textbf{Solution}:
	
	\cbox{
		\large
		\vspace{-0.5cm}
		\begin{align*}
			\frac{p_A}{\gamma} + z_A + \frac{v_A^2}{2g}                   & = \frac{p_S}{\gamma} + z_S + \frac{v_S^2}{2g}                                 \\
			\frac{p_A}{0.68\times 9.81} + \cancel{z_A} + \frac{v_A^2}{2g} & = \frac{p_S}{0.68\times 9.81} + \cancel{z_S} + \cancelto{0}{\frac{v_S^2}{2g}} \\
			p_S-p_A                                                       & = (0.68\times 9.81) \frac{v_A^2}{2g}                                          
		\end{align*}
	}
	\parb
	Now look at the manometer:
	\cbox{
		\large
		\vspace{-0.5cm}
		\begin{align*}
			p_A+\cancel{(0.68\times 9.81)d}+(1.88\times 9.81)(0.371) & = p_S+(0.68\times 9.81)(\cancel{d}+0.371)                   \\
			\Rightarrow p_S-p_A                                      & = 1.88\times 9.81\times 0.371 - 0.68\times 9.81\times 0.371 \\
			                                                         & = 4.3674                                                    
		\end{align*}
	}
	\parb
	Substituting in the result above:
	\cbox{
		\large
		\vspace{-0.5cm}
		\begin{align*}
			4.3674 & = (0.68\times 9.81) \frac{v_A^2}{2g}                  \\
			v_A^2  & = 12.845                                              \\
			v_A    & = 3.584 \,\text{m/s}                                  \\\\
			Q      & = \frac{\pi d^2}{4}\cdot  3.584 \,\text{m}^3\text{/s} \\
			       & = 0.11566\,\text{m}^3\text{/s}                        \\
			\bm{Q} & \bm{\approx 116\,\text{L/s}}                          
		\end{align*}
	}
}
\newpage
%%%%%%%%%%%%%%%%%%%%%%%%%%%%%%%%%%%%%%%%%%%%%%%%%%%%%%%%%%%%%%%%%%%%%%%%%%%%%%%%%%%%%%%%%%%%%%%%%%%%%%%%%

\cmini[0.75]{
	\raggedright
	\textbf{Example 7}:\\
	\begin{cfig}[0.3]{../../figs/03FlowOfFluids/04BernoulliEx08}\end{cfig}
	Determine $Q$, the volume flow rate, if $h=210$ mm.
	
	\parb
	\textbf{Solution}:
	\parb
	Use the manometer to find the difference in pressure between the flow through the pipe and through the constriction:
	\par
	(Calculate the pressure at the join of the pipe fluid and the manometer gauge
	fluid on the right hand side in the diagram, and let $t$ be the vertical
	distance between the centre of the pipe and the pipe-fluid to gauge-fluid on
	the left hand side of the diagram.)
	\cbox{
		\vspace{-0.5cm}
		\large
		\begin{align*}
			p_{1}+\gamma_\text{pipe}\cdot t + \gamma_\text{gauge}\cdot h & =  p_{2}+\gamma_\text{pipe}\cdot t + \gamma_\text{pipe}\cdot h \\
			p_{1}+\gamma_\text{gauge}\cdot h                             & =  p_{2}+ \gamma_\text{pipe}\cdot h                            \\
			p_{1}+13.54\times9.81\times0.210                             & =  p_{2}+ 0.82\times9.81\times0.210                            \\
			p_{2}-p_{1}                                                  & = (13.54-0.82)\times9.81\times0.210                            \\
			                                                             & = 26.204\text{ kPa}                                            
		\end{align*}
		\vspace{-0.875cm}
	}
	\parm
	Find the velocity heads at $p_1$ and $p_2$ in terms of $Q$:
	\parm
	\mini[0.45]{
		\large
		\cbox{
			\begin{align*}
				v_1              & = \frac{Q_1}{\pi(0.058)^2/4}\,\text{m/s} \\
				                 & = 378.49Q   \,\text{m/s}                 \\
				\frac{v_1^2}{2g} & = 7301.5Q^2 \,\text{m}                   
			\end{align*}
		}
	}
	\hfill
	\mini[0.45]{
		\cbox{
			\large
			\begin{align*}
				v_2              & = \frac{Q_2}{\pi(0.100)^2/4} \,\text{m/s} \\
				                 & = 127.32Q  \,\text{m/s}                   \\
				\frac{v_2^2}{2g} & = 826.22Q^2     \,\text{m}                
			\end{align*}
		}
	}
	\parb
	Apply Bernoulli's Equation:
	\cbox{
		\large
		\begin{align*}
			\frac{p_1}{\gamma} + \cancel{z_1} + \frac{v_1^2}{2g} & = \frac{p_2}{\gamma} + \cancel{z_2} + \frac{v_2^2}{2g} \\
			\frac{p_2-p_1}{\gamma}                               & = \frac{v_1^2}{2g}-\frac{v_2^2}{2g}                    \\
			p_2-p_1                                              & = \gamma\left[7301.5Q^2-826.22Q^2\right]               \\
			26.204                                               & = 0.82\times 9.81\times 6475.3Q^2                      \\
			Q^2                                                  & = 0.00050307                                           \\
			Q                                                    & = 0.022429\,\mathsf{m^3/s}                             \\
			                                                     & = 22.4\text{ L/s}                                      
		\end{align*}
	}
}
\newpage


%%%%%%%%%%%%%%%%%%%%%%%%%%%%%%%%%%%%%%%%%%%%%%%%%%%%%%%%%%%%%%%%%%%%%%%%%%%%%%%%%%%%%%%%%%%%%%%%%%%%%%%%%

\begin{minipage}[t]{0.48\textwidth}
	\raggedright
	\textbf{Exercise 2}:\\
	\begin{cfig}[0.2]{../../figs/03FlowOfFluids/04BernoulliEx09}\end{cfig}
	Determine $Q$, the volume flow rate, if $h=125$ mm.
\end{minipage}
\hfill
\begin{minipage}[t]{0.48\textwidth}
	\parb
	\vspace{-0.75cm}
	\textbf{Solution}:
	\begin{cfig}[0.225]{../../figs/03FlowOfFluids/04BernoulliEx09b}\end{cfig}
\end{minipage}\parb
\begin{minipage}[t]{0.85\textwidth}
	\parb
	Since $P_A=P_B$,
	\begin{align*}
		p_{1}+\gamma_\text{pipe}(a+\cancel{b}) + \gamma_\text{gauge}\cdot h & =  p_{2}+\gamma_\text{pipe}(\cancel{b} + h)         \\
		p_{1}+(0.88\times9.81)(a) + (1.26\times9.81)(0.125)                 & =  p_{2}+(0.88\times9.81)(0.125)                    \\
		p_{2}-p_{1}                                                         & = (0.88\times9.81)(a-0.125)+(1.26\times9.81)(0.125) \\
		                                                                    & = 8.6328a+0.46598\text{ kPa}                        
	\end{align*}
	Find the velocities at $1$ and $2$ relative to the flow $Q$:
	\begin{align*}
		v_1 & = \frac{Q}{A_1} = \frac{Q}{\pi(0.0627)^2/4}=\frac{Q}{0.0030876}=323.87Q \\
		v_2 & = \frac{Q}{A_2} = \frac{Q}{\pi(0.1023)^2/4}=\frac{Q}{0.0082194}=121.66Q 
	\end{align*}
	The associated velocity heads are:
	\begin{align*}
		\frac{v_1^2}{2g} & = \frac{(328.87Q)^2}{2g} = 5346.3Q^2 \\
		\frac{v_2^2}{2g} & = \frac{(121.66Q)^2}{2g} =754.43Q^2  
	\end{align*}
	Apply Bernoulli's Equation:
	\begin{align*}
		\frac{p_1}{\gamma} + z_1 + \frac{v_1^2}{2g} & = \frac{p_2}{\gamma} + z_2 + \frac{v_2^2}{2g} \\
		\frac{p_1}{\gamma} + a + 5346.3Q^2          & = \frac{p_2}{\gamma} + 0 + 754.43Q^2          \\
		\frac{p_2-p_1}{\gamma}                      & = 5346.3Q^2-754.43Q^2+a                       \\
		p_2-p_1                                     & = 0.88\times9.81\left(4591.9Q^2+a\right)      
	\end{align*}
	Incorporating our previous manometer result,
	\begin{align*}
		8.6328a+0.46598          & = 0.88\times9.81\left(4591.9Q^2+a\right) \\
		\cancel{8.6328a}+0.46598 & = 39641Q^2+ \cancel{8.6328a}             \\\\
		Q                        & = 0.0034286\;\mathrm{m^3/s}              \\
		                         & \approx 3.43\text{ L/s}                  
	\end{align*}
	Note that the answer is independent of $a$: it does not matter what angle the venturi meter
	is inclined at, the answer will be the same (including vertical or horizontal). This was not the
	case in the previous examples where the pressure was measured directly in the pipe with pressure meters.
\end{minipage}


\end{document}
