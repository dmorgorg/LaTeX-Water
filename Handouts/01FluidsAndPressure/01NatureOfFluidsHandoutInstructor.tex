% override specific chktex warnings
% chktex-file 46 - don't use $ instead of \(, etc)
% chktex-file 1 - ignore commands followed by a space, e.g. \\ new line here

% chktex-file 9 - sometimes messes up with ( and {

\documentclass[10pt]{amsart}
\usepackage[showboxes]{textpos}
\usepackage{amsmath}
\usepackage{amsthm}
\usepackage{amsfonts}
\usepackage{mathpazo}
\usepackage{booktabs}
\usepackage[usenames,dvipsnames]{color}
\usepackage{tikz}
\usepackage{textcomp}
\usepackage[letterpaper]{geometry}
\geometry{verbose,tmargin=0.5in,bmargin=0.5in,lmargin=0.75in,rmargin=0.75in}
\usepackage{multicol}
\usepackage{bm}
\usepackage{array} % needed for setting math in table definition
\usepackage[many]{tcolorbox}

\pagestyle{empty}

\renewcommand{\familydefault}{\sfdefault}
\setlength{\parskip}{\medskipamount}
\setlength{\columnsep}{2cm}

\everymath{\displaystyle}

% counter for resuming enumerated list numbers
\newcounter{resumeenumi}
\newcommand{\suspend}{\setcounter{resumeenumi}{\theenumi}}
\newcommand{\resume}{\setcounter{enumi}{\theresumeenumi}}

\newcommand\lb{\linebreak}
\newcommand\pars{\par\smallskip}
\newcommand\parm{\par\medskip}
\newcommand\parb{\par\bigskip}

\makeatletter
\providecommand{\gettikzxy}[3]{%
	\tikz@scan@one@point\pgfutil@firstofone#1\relax
	\edef#2{\the\pgf@x}%
	\edef#3{\the\pgf@y}%
}
\makeatother



% full width colored block but color specifiable
%\cb[body bg strength]{header bg}{header text}{body text}
\newcommand{\cb}[4][15]{
	\setbeamercolor{block title}{bg = #2}
	\setbeamercolor{block body}{bg = #2!#1}
	\setbeamercolor{item projected}{bg=#2, fg=white}
	\begin{center}
		\begin{block}{#3}
			#4
		\end{block}
	\end{center}
}

% colored block with width specified
% \cbw[body bg strength]{header bg}{width}{header text}{body text}
\newcommand{\cbw}[5][15]{
	\begin{center}
		%\vspace{-0.35cm}
		\begin{minipage}{#3\textwidth}
			\setbeamercolor{block title}{bg= #2}
			\setbeamercolor{block body}{bg= #2!#1}
			\setbeamercolor{item projected}{bg=#2, fg=white}
			\begin{block}{#4}
				\raggedright
				#5
			\end{block}
		\end{minipage}
	\end{center}
}

% centered minipage with text \raggedright
%\cmini[width]{content}
\newcommand{\cmini}[2][0.8]{
	\begin{center}
		\begin{minipage}{#1\columnwidth}
			\raggedright
			#2
		\end{minipage}
	\end{center}
}

%left flushed minipage
\newcommand{\mini}[2][0.8]{
	\begin{minipage}{#1\columnwidth}
		\raggedright
		#2
	\end{minipage}
}

%left flushed minipage, top aligned
\newcommand{\minit}[2][0.8]{
	\begin{minipage}[t]{#1\columnwidth}
		\raggedright
		#2
	\end{minipage}
}

%left flushed minipage
% \newcommand{\miniT}[2][0.8]{
%  \begin{minipage}[T]{#1\columnwidth}
%   \raggedright
%   #2
%  \end{minipage}
% }

%left flushed minipage
\newcommand{\minib}[2][0.8]{
	\begin{minipage}[b]{#1\columnwidth}
		\raggedright
		#2
	\end{minipage}
}

\newcommand{\cfig}[2][1]{% centred, scaled graphic
	\begin{center}
		\includegraphics[scale=#1]{#2}
	\end{center}
}
% figure with tight border for photos
% \cfigb[saitMaroon]{borderwidth with unit}{scale}{image}
\newcommand{\cfigb}[4][structure]{
	% \usepackage{adjustbox}
	\setlength{\fboxrule}{1pt}
	\begin{center}
		\includegraphics[scale=#3, cframe= #1 #2]{#4}
	\end{center}
}

\newcommand{\imgbox}[3]{
	% \setlength{\fboxsep}{12pt}
	\includegraphics[scale=#1, cframe= structure #3]{#2}
}

% \imgboxbg[bg color=white]{scale}{path/to/img}{border color}{border, e.g. 2pt}{margin, e.g. 4pt}
\newcommand{\imgboxbg}[6][white]{
	\setlength{\fboxrule}{#5}
	\setlength{\fboxsep}{#6}
	\centering
	\fcolorbox{#4}{#1}{\includegraphics[scale=#2]{#3}}
}

\newcommand{\fig}[2][1]{% scaled graphic
	\includegraphics[scale=#1]{#2}
}

% centred framed  box black border
%\cbox[width]{content}
\newcommand{\cbox}[2][0.9]{% framed centered  box
	\setlength\fboxsep{0.042\columnwidth}
	\setlength\fboxrule{0.0015\columnwidth}
	\begin{center}
		\fcolorbox{black}{white}{
			\vspace{-0.5cm}
			\begin{minipage}{#1\columnwidth}
				\raggedright
				#2
			\end{minipage}
		}
	\end{center}
	\setlength\fboxsep{0cm}
}



\newtcolorbox{mybox}[1][]
{
	colback=white,
	top=0.25cm,
	bottom=0.25cm,
	left=0.25cm,
	right=0.25cm,
	colframe=structure,
	fonttitle=\bfseries,
	enhanced, drop fuzzy shadow,
	% attach boxed title to top left={yshift=-2mm, xshift=5mm},
	attach boxed title to top left={yshift=-2mm, xshift=5mm}, colbacktitle=structure!80!white, #1}

\newtcolorbox{plainbox}[1][]{colback=white, sharp corners, top=0.125cm, bottom=0.125cm, left=0pt, right=0pt, boxrule=0.5pt,colframe=structure,fonttitle=\bfseries, colbacktitle=structure, arc=0mm, #1}
%
\newtcbtheorem{myexam}{Example}%
{
	enhanced,
	colback=white,
	top=0.375cm,
	bottom=0.25cm,
	left=0.375cm,
	right=0.375cm,
	colframe=structure,
	fonttitle=\bfseries,
	drop fuzzy shadow,
	%description font=\mdseries\itshape,
	attach boxed title to top left={yshift=-2mm, xshift=5mm},
	colbacktitle=structure!80!white
	}{exam}% then \pageref{exer:theoexample} references the theo

\newcommand{\myexample}[2][red]{
	% \tcb\tcbset{theostyle/.style={colframe=red,colbacktitle=yellow}}
	\begin{myexam}{}{}
		\raggedright
		#2
	\end{myexam}
	% \tcbset{colframe=structure,colbacktitle=structure}
}

\newtcbtheorem{myexer}{Exercise}%
{
	enhanced,
	colback=white,
	top=0.375cm,
	bottom=0.25cm,
	left=0.375cm,
	right=0.375cm,
	colframe=structure,
	fonttitle=\bfseries,
	drop fuzzy shadow,
	%description font=\mdseries\itshape,
	attach boxed title to top left={yshift=-2mm, xshift=5mm},
	colbacktitle=structure!80!white
	}{exer}

\newcommand{\myexercise}[2][red]{
	% \tcb\tcbset{theostyle/.style={colframe=red,colbacktitle=yellow}}
	\begin{myexer}{}{}
		\raggedright
		#2
	\end{myexer}
	% \tcbset{colframe=structure,colbacktitle=structure}
}


\begin{document}

\thispagestyle{empty}
\vspace{-7cm}
\centering
\rule{\textwidth}{0.02in}
\parb
\textbf{\Large Module 1: The Nature of Fluids/Pressure Measurement (CIVL 318)}
\pars
\rule{\textwidth}{0.02in}
\parb
\raggedright

%\textbf{\Large Required Reading:}
%
%\large
%Sections: 1.3, 1.4, 1.8, 1.9, 1.11 and
%Worked Examples: 1.2, 1.5, 1.6, 1.7
%\parb

\parb
\textbf{\Large Some useful results:}
\parb
\begin{center}
	
	\begin{tabular}{rrclll}
		\toprule
		\addlinespace
		\textbf{Pressure}: & $P$ &=& $\frac{F}{A}$ && $\left(\mathsf{\frac{Force}{Area}}\right)$ \\
		\addlinespace
		\midrule
		\addlinespace
		\textbf{Pascal's Laws:}& \multicolumn{5}{l}{ Pressure acts uniformly in all directions}\\
		& \multicolumn{5}{l}{ on a small volume of liquid.}\\
		\addlinespace
		& \multicolumn{5}{l}{ Pressure acts perpendicularly }\\
		& \multicolumn{5}{l}{to the solid boundaries of a fluid.}\\
		\addlinespace
		\midrule
		\addlinespace
		\textbf{Density}: & $\rho$ (rho) &=& $\frac{m}{V}$ && $\left(\mathsf{\frac{Mass}{Volume}}\right)$\\
		\addlinespace
		\midrule
		\addlinespace
		\textbf{Specific Weight}: & $\gamma$ (gamma) &=& $\frac{w}{V}$ && $\left(\mathsf{\frac{Weight}{Volume}}\right)$\\
		\addlinespace
		\midrule
		\addlinespace
		\textbf{Specific Gravity}: & sg &=& $\frac{\rho_s}{\rho_w@4\mbox{\textcelsius}}$ &=& $\frac{\gamma_s}{\gamma_w@4\mbox{\textcelsius}}$\\
		\addlinespace
		\midrule
		\addlinespace
		\textbf{Density \& Specific Weight}: & $\gamma$ &=& $\rho$g &&\\
		\addlinespace
		\midrule
		\addlinespace
		\textbf{Pressure Relationship:}	& $p_{abs}$ &=& $p_{atm}+p_{gauge}$ \\
		\addlinespace
		\addlinespace
		\midrule
		\addlinespace
		\textbf{Pressure-Elevation Relationship:}	& $\Delta p$ &=& $\gamma h$ \\
		\addlinespace
		\bottomrule
	\end{tabular}
\end{center}

\vfill
\newpage

\begin{minipage}[t]{0.44\textwidth}
%	\small
	\begin{center}
		\textbf{\Large Table A: Properties of Water}\parb
		\begin{tabular}{>{$}c<{$} >{$}c<{$} >{$}c<{$} >{$}c<{$}}
			\toprule
			\addlinespace
			& \text{Specific} &  & \text{Dynamic } \\
			\text{Temperature} & \text{Weight} & \text{Density} & \text{Viscosity} \\
			\addlinespace
			&	 \gamma & \rho & \eta \\
			\addlinespace
			(\text{\textcelsius}) & \text{(kN/m}^3) & \text{(kg/m}^3) & \text{(Pa}\cdot\text{s)} \\
			\addlinespace 
			\midrule			
			\addlinespace
			0 & 9.81 & 1000 & 1.75 \times 10^{-3} \\ \addlinespace
			5 & 9.81 & 1000 & 1.52 \times 10^{-3} \\ \addlinespace
			10 & 9.81 & 1000 & 1.30 \times 10^{-3} \\ \addlinespace
			15 & 9.81 & 1000 & 1.15 \times 10^{-3} \\ \addlinespace
			20 & 9.79 & 998 & 1.02 \times 10^{-3} \\ \addlinespace
			25 & 9.78 & 997 & 8.91 \times 10^{-4} \\ \addlinespace
			30 & 9.77 & 996 & 9.00 \times 10^{-4} \\ \addlinespace
			35 & 9.75 & 994 & 7.18 \times 10^{-4} \\ \addlinespace
			40 & 9.73 & 992 & 6.51 \times 10^{-4} \\ \addlinespace
			45 & 9.71 & 990 & 5.94 \times 10^{-4} \\ \addlinespace
			50 & 9.69 & 988 & 5.41 \times 10^{-4} \\ \addlinespace
			55 & 9.67 & 986 & 4.98 \times 10^{-4} \\ \addlinespace
			60 & 9.65 & 984 & 4.60 \times 10^{-4} \\ \addlinespace
			65 & 9.62 & 981 & 4.31 \times 10^{-4} \\ \addlinespace
			70 & 9.59 & 978 & 4.02 \times 10^{-4} \\ \addlinespace
			75 & 9.56 & 975 & 3.73 \times 10^{-4} \\ \addlinespace
			80 & 9.53 & 971 & 3.50 \times 10^{-4} \\ \addlinespace
			85 & 9.50 & 968 & 3.30 \times 10^{-4} \\ \addlinespace
			90 & 9.47 & 965 & 3.11 \times 10^{-4} \\ \addlinespace
			95 & 9.44 & 962 & 2.92 \times 10^{-4} \\ \addlinespace
			100 & 9.40 & 958 & 2.82 \times 10^{-4} \\ \addlinespace
			
			\midrule		
			\bottomrule				
		\end{tabular}
	\end{center}
\end{minipage}
\hfill
\begin{minipage}[t]{0.54\textwidth}
	\small
	\begin{center}
		\textbf{\Large Table B: Properties of Common Liquids}\par
		(at $101$ kPa and $25$\textcelsius)\parb
		\begin{tabular}{r >{$}c<{$} >{$}c<{$} >{$}c<{$} >{$}c<{$}}
			
			\toprule
			\addlinespace
			& \text{Specific} &\text{Specific} &  & \text{Dynamic } \\
			\text{Liquid} & \text{Gravity} & \text{Weight} & \text{Density} & \text{Viscosity} \\
			\addlinespace
			&&	 \gamma & \rho & \eta \\
			\addlinespace
			& & \text{(kN/m}^3) & \text{(kg/m}^3) & \text{(Pa}\cdot\text{s)} \\
			\addlinespace
			\midrule
			\addlinespace
			Acetone 				& 0.787 	& 7.72 		& 787		& 3.16 \times 10^{-4} \\ \addlinespace
			Alcohol, Ethyl 			& 0.787 	& 7.72 		& 787		& 1.00 \times 10^{-3} \\ \addlinespace
			Alcohol, Methyl 		& 0.789 	& 7.74 		& 789		& 5.60 \times 10^{-4} \\ \addlinespace
			Alcohol, Propyl 		& 0.802 	& 7.87 		& 802		& 1.92 \times 10^{-3} \\ \addlinespace
			%Aqua Ammonia	 		& 0.910 	& 8.93 		& 910		&  \\ \addlinespace
			Benzene 				& 0.876 	& 8.59 		& 876		& 6.03 \times 10^{-4} \\ \addlinespace
			Carbon Tetrachloride 	& 1.590 	& 15.60 	& 1590		& 9.10 \times 10^{-4} \\ \addlinespace
			Castor Oil 				& 0.960 	& 9.42 		& 960		& 6.51 \times 10^{-1} \\ \addlinespace
			Ethylene Glycol 		& 1.100 	& 10.79 	& 1100		& 1.62 \times 10^{-2} \\ \addlinespace
			Gasoline 				& 0.68 		& 6.67 		& 680		& 2.87 \times 10^{-4} \\ \addlinespace
			Glycerine 				& 1.258 	& 12.34 	& 1258		& 9.60 \times 10^{-1} \\ \addlinespace
			Kerosene 				& 0.823 	& 8.07 		& 823		& 1.64 \times 10^{-3} \\ \addlinespace
			Linseed Oil 			& 0.930 	& 9.12 		& 930		& 3.31 \times 10^{-2} \\ \addlinespace
			Mercury 				& 13.54 	& 132.8 	& 13540		& 1.53 \times 10^{-3} \\ \addlinespace
			Propane 				& 0.495 	& 4.86 		& 495		& 1.10 \times 10^{-4} \\ \addlinespace
			Seawater 				& 1.030 	& 10.10 	& 1030		& 1.03 \times 10^{-3} \\ \addlinespace
			Turpentine 				& 0.870 	& 8.53 		& 870		& 1.37 \times 10^{-3} \\ \addlinespace
			Fuel Oil, medium 		& 0.852 	& 8.36 		& 852		& 2.99 \times 10^{-3} \\ \addlinespace
			Fuel Oil, heavy 		& 0.906 	& 8.89 		& 906		& 1.07 \times 10^{-1} \\ \addlinespace
			\midrule		
			\bottomrule				
		\end{tabular}
	\end{center}
\end{minipage}

\newpage

%\par\vspace{1cm}
%%%%%%%%%%%%%%%%%%%%%%%%%%%%%%%%%%%%%%%%%%%%%%%%%%%%%%%%%%%%%%%%%%%%%%%%%%%%%%%%%%%%%%%%%%%%%%%%%%%%%%%%%%%%%%%%%%%%%

%\rule{\textwidth}{0.02in}
%\parb

%%%%%%%%%%%%%%%%%%%%%%%%%%%%%%%%%%%%%%%%%%%%%%%%%%%%%%%%%%%%%%%%%%%%%%%%%%%%%%%%%%%%%%%%%%%%%%%%%%%%%%%%%
\begin{minipage}[t]{0.45\textwidth}
	\raggedright
	\textbf{Example 1}:
	\begin{cfig}[0.5]{../../figs/01NoFPM/WR1M01ex2}\end{cfig}
	A piston confines oil in a closed circular cylinder. The maximum operating pressure for the
	piston is $17.8\text{ MPa}$. The piston has a diameter of $62.5\text{ mm}$. What is the maximum
	load that the piston can support?
\end{minipage}
\hfill
\begin{minipage}[t]{0.5\textwidth}
	\textbf{Solution}:
	\parb
	\cbox[0.9]{
		\begin{align*}
			F &= P\times A\\
			&= 17.8\times 10^6\text{ N/m}^2\times \frac{\pi(0.0625\text{ m})^2}{4}\\
			&= 54610\text{ N}\\
			&\approx 54.6\text{ kN}\qquad\qquad\text{($3$ sig digs)}
		\end{align*}
	}
	\parb
	\textbf{Alternative units}:
	\parb
	\cbox[0.9]{
		\begin{align*}
			F &= P\times A\\
			&= 17.8\text{ N/mm}^2\times \frac{\pi(6.25\text{ mm})^2}{4}\\
			&= 54610\text{ N}\\
			&\approx 54.6\text{ kN}
		\end{align*}
	}
\end{minipage}
\par
\vspace{2cm}
\rule{\textwidth}{0.02in}
\parb
%%%%%%%%%%%%%%%%%%%%%%%%%%%%%%%%%%%%%%%%%%%%%%%%%%%%%%%%%%%%%%%%%%%%%%%%%%%%%%%%%%%%%%%%%%%%%%%%%%%%%%%%%

\begin{minipage}[t]{0.45\textwidth}
	\raggedright
	\textbf{Exercise 1}:\\
		A press used to produce coins requires a force of $8.20\text{ kN}$.\\
		The hydraulic cylinder has
		a diameter of $63.5\text{ mm}$. \parb
		What is the oil pressure needed to generate this force?
	\par\vspace{3.25cm}
\end{minipage}
\hfill
\begin{minipage}[t]{0.5\textwidth}
	\textbf{Solution}:
	\parm
	\cbox[0.9]{
		\begin{align*}
			P &= \frac{F}{A}\\
			&= \frac{8.20\text{ kN}}{\pi(0.0635\text{ m})^2/4}\\
			&= 2\,589.3\text{ kN/m}^2\\
			&\approx 2.59\text{ MPa}
		\end{align*}
	}
	\parm
	\textbf{Alternative units}:
	\parm
	\cbox[0.9]{
		\begin{align*}
			P &= \frac{F}{A}\\
			&= \frac{8200\text{ N}}{\pi(6.35\text{ mm})^2/4}\\
			&= 2.5893\text{ N/mm}^2\\
			&\approx 2.59\text{ MPa}
		\end{align*}
	}
\end{minipage}
\vfill
\newpage
%%%%%%%%%%%%%%%%%%%%%%%%%%%%%%%%%%%%%%%%%%%%%%%%%%%%%%%%%%%%%%%%%%%%%%%%%%%%%%%%%%%%%%%%%%%%%%%%%%%%%%%%%
%\rule{\textwidth}{0.02in}
%\par
\begin{minipage}[t]{0.4\textwidth}
\raggedright
\textbf{Example 3}:\\
An empty barrel with an inside diameter of $900$~mm weighs $205$ N. \parb
What does the barrel weigh when it is filled to a depth of $750$ mm
with water at $25$\textcelsius{}?
\par\vspace{9cm}
\end{minipage}
\hfill
\begin{minipage}[t]{0.55\textwidth}

\textbf{Solution}:
\parb
\cbox[0.9]{
\Large
The volume of water is the volume of a cylinder with diameter $900$~mm and height $750$ mm:
\begin{align*}
	v &= \frac{\pi{}d^2}{4}\cdot h \\
	&= \frac{\pi(0.900\text{ m})^2}{4}\cdot (0.75\text{ m})\\
	&= 0.47713\text{ m}^3
\end{align*}
(Use $5$ significant digits for interim calculations and $3$ significant
digits for solutions.)
\parm
The specific weight of water at $25$\textcelsius{} is $9.78\text{ kN/m}^3$ (Table A.1) so
\begin{align*}
w &= \gamma{}v \\
&= 9.78\text{ kN/m}^3\times 0.47713\text{ m}^3 \\
&=4.6663\text{ kN}
\end{align*}
The combined weight of the barrel and the water is given by:
\[0.205\text{ kN}+4.6663\text{ kN}\approx 4.87\text{ kN}\]
}
\end{minipage}

\par\vspace{2cm}

\rule{\textwidth}{0.02in}
\par
\begin{minipage}[t]{0.45\textwidth}
\raggedright
\textbf{Example 4}:\\
	Calculate the density and the specific weight of benzene if it has a specific gravity of 0.876.
\par\vspace{6cm}
\end{minipage}
\hfill
\begin{minipage}[t]{0.5\textwidth}
\textbf{Solution}:
\parb
\cbox[0.9]{
	\Large
	\begin{align*}
		0.876  &=  \frac{\rho_{b}}{\rho_{water@4^{\circ}C}}\\
		\rho_{b} &= 0.876\times1000\; \text{kg/m}^3\\
		& = 876\; \text{kg/m}^3\\
		0.876 &= \frac{\gamma_{b}}{\gamma_{water@4^{\circ}C}}\\
		\gamma_{b} &= 0.876\times9.81\; \text{kN/m}^3\\
		& =8.59\; \text{kN/m}^3
	\end{align*}
}
\end{minipage}

\newpage

\begin{minipage}[c]{0.45\textwidth}
\raggedright
\textbf{Example 5}:\\
An open cylindrical tank with diameter $5.75$ m and depth $3.30$ m is filled to the top with water at $10$\textcelsius.
The water is heated to $55$\textcelsius. Assuming that the tank dimensions remain constant and there are no losses due to evaporation, calculate the mass of water that overflows.
\parb
\textbf{Solution}:
\parb
\cbox[0.9]{
	Volume of tank:
	\begin{align*}
		v_{tank} &= \frac{\pi(5.75 \text{ m})^2}{4}\times 3.30\text{ m} \\
		&= 85.692\text{ m}^3
	\end{align*}
	}
\par\vspace{2cm}
\end{minipage}
\hfill
\begin{minipage}[c]{0.5\textwidth}
\cbox[0.9]{
Mass of the water in the tank at $10$\textcelsius:
\begin{align*}
	m &= \rho_{\large{10\text{\textcelsius}}} \times v_{\large{10\text{\textcelsius}}} \\
	&= 1000\text{ kg/m}^3 \times 85.692\text{ m}^3 \\
	&= 85\,692\text{ kg}
\end{align*}
Mass of the water in the tank at $70$\textcelsius:
\begin{align*}
	m &= \rho_{\large{55\text{\textcelsius}}} \times v_{\large{55\text{\textcelsius}}} \\
	&= 986\text{ kg/m}^3 \times 85.692\text{ m}^3 \\
	&= 84\,492\text{ kg}
\end{align*}
Mass of water that overflows:
\begin{align*}
	m_{overflow} &=85\,692\text{ kg}-84\,492\text{ kg} \\
	&= 1\,200\text{ kg}\\
\end{align*}

}
\end{minipage}

\newpage

%\begin{minipage}[c]{0.45\textwidth}
%\raggedright
%\textbf{Example 6}:\\
%A vertical glass cylinder contains $2.30$ litres of water at $10$\textcelsius. The height of the water in the
%cylinder is $1.05$ m. The water and the cylinder are heated. Assuming that there are no losses due
%to evaporation, find the change in water height when the temperature reaches $75$\textcelsius. \\(The coefficient of
%thermal expansion, $\alpha$, for the glass is given by $\alpha = 3.6 \times 10^{-6}$ m/m/\textcelsius.)
%\parb
%\textbf{Solution}:
%\parb
%\cbox[0.9]{
%	Find the cylinder inside diameter at $10$\textcelsius:
%	\begin{align*}
%		d_{10} &= \sqrt{\frac{4v}{\pi h}}\\
%		&= \sqrt{\frac{4\times 0.0023}{1.05\pi}}\\
%		&=0.052811\text{ m}
%	\end{align*}
%}
%\par\vspace{4cm}
%\end{minipage}
%\hfill
%\begin{minipage}[c]{0.5\textwidth}
%\cbox[0.9]{white}{
%The diameter increases linearly with temperature change:
%\begin{align*}
%\Delta d &= \alpha\times d_{10}\times\Delta T\\
%&= 3.6\times 10^{-6}\times 0.052811\times 65\\
%&= 0.000012358\text{ m}\\\\
%d_{75} &= 0.052811 + 0.000012358\\
%&= 0.052823\text{ m}
%\end{align*}
%Since there is no evaporation, mass remains constant:
%\begin{align*}
%m=\rho_{10}v_{10} &= \rho_{75}v_{75}\\
%v_{75} &= \rho_{10}v_{10}/\rho_{75}\\
%&= 1000\times 0.0023/975\\
%&= 0.002359\text{ m}^3
%\end{align*}
%The height of the water at $75$\textcelsius:
%\begin{align*}
%h_{75} &= \frac{4v}{\pi d^2_{75}}\\
%&= \frac{4\times 0.002359}{\pi(0.052823)^2}\\
%&= 1.0764\text{ m}
%\end{align*}
%The change in height due to the temperature change is:
%\[ \Delta_h = h_{75}-h_{10}=1.0764-1.05=0.026445\text{ m}\]
%Change in height is $26.4$ mm.
%
%}
%\end{minipage}

\begin{minipage}[t]{0.45\textwidth}
	\raggedright	
	\textbf{Example 6}:
	\cfig[0.55]{../../figs/01NoFPM/02tank1}
	A tank, open to the atmosphere in the centre, contains medium fuel
	oil. Atmospheric pressure is 102.1 kPa. Calculate the gauge pressure
	and the absolute pressure for locations \emph{A, B, C, D} and \emph{E}.
	\parb
	\textbf{Solution}:
	\parb

	\cbox[0.9]{
		\Large
		\textbf{Pressure at }$\bm B$:
		$B$ is open to the atmosphere so 
		\[P_B=0 \text{ and } P_{B(abs)}=P_{(atm)}=102.1\text{ kPa}\]		
	}
	\parb
	Note that pressure is assumed to be gauge pressure unless otherwise specified.\parb
	Also, atmospheric pressure is generally specified to four significant digits; there is a distinct
	difference in pressure between 100.5 kPa and 101.4 kPa.
	

				
	\parb
	\cbox[0.9]{
		\Large
		\textbf{Pressure at }$\bm A$:\par
		\begin{eqnarray*}
			P_A & = & P_B-\Delta p\\
			& = & 0-\gamma h\\
			& = & -(8.36\text{ kN/m}^3)(0.30\text{ m})\\
			& = & -2.5080\text{ kPa}\\
			& \approx & -2.51\text{ kPa}\\
			\\
			P_{A(abs)} & = & P_{atm}+P_{A(gauge)}\\
			& = & 102.1\text{ kPa}-2.5080\text{ kPa}\\
			& = & 99.592\text{ kPa} \\
			& \approx & 99.6\text{ kPa} 
		\end{eqnarray*}		
	}
	
\end{minipage}
\hfill
\begin{minipage}[t]{0.5\textwidth}		
		
	\parb
	\cbox[0.9]{
		\Large
		\textbf{Pressure at }$\bm C$:\par
		\begin{eqnarray*}
			P_C & = & P_C-\Delta p\\
			& = & 0-\gamma h\\
			& = & -(8.36\text{ kN/m}^3)(0.950\text{ m})\\
			& = & -7.9420\text{ kN/m}^2\\
			& \approx & -7.94\text{ kPa}\\
			\\
			P_{C(abs)} & = & P_{atm}+P_{C(gauge)}\\
			& = & 102.1\text{ kPa}-7.9420\text{ kPa}\\
			& = & 94.158\text{ kPa}\\
			& \approx & 94.2\text{ kPa}
		\end{eqnarray*}	
	}
	\parb
	\cbox[0.9]{
		\Large
		\textbf{Pressure at }$\bm D$:\par
		\begin{eqnarray*}
			P_D & = & P_B+\Delta p\\
			& = & 0+\gamma h\\
			& = & (8.36\text{ kN/m}^3)(1.375\text{ m})\\
			& = & 11.495\text{ kPa}\\
			& \approx & 11.50\text{ kPa}
			\\
			P_{D(abs)} & = & P_{atm}+P_{D(gauge)}\\
			& = & 102.1\text{ kPa}+11.495\text{ kPa}\\
			& = & 113.60\text{ kPa}\\
			& = & 113.6\text{ kPa}
		\end{eqnarray*}
		}
		\parb
		\cbox[0.9]{
			\Large
			\textbf{Pressure at }$\bm E$:\par
			\begin{eqnarray*}
				P_E & = & P_B+\Delta p\\
				& = & 0+\gamma h\\
				& = & (8.36\text{ kN/m}^3)(2.0\text{ m})\\
				& = & 16.720\text{ kPa}\\
				\\
				P_{E(abs)} & = & P_{atm}+P_{D(gauge)}\\
				& = & 102.1\text{ kPa}+16.72\text{ kPa}\\
				& = & 118.82\text{ kPa}\\
				& \approx & 118.8\text{ kPa}
			\end{eqnarray*}
		}	
\end{minipage}
\newpage
%%%%%%%%%%%%%%%%%%%%%%%%%%%%%%%%%%%%%%%%%%%%%%%%%%%%%%%%%%%%%%%%%%%%%%%%%%%%%%%%%%%%%%%%%%%%%%%%%%%%%%%%%
\begin{minipage}[t]{0.45\textwidth}	
	\raggedright
	\textbf{Example 7}:\\
	Determine the pressure at $A$ given that the temperature of the
	water is $25$\textcelsius.
	\cfig[0.55]{../../figs/01NoFPM/imgCIVL250M02QB}
	\par\vspace{5cm}
\end{minipage}
\hfill
\begin{minipage}[t]{0.5\textwidth}		
	\textbf{Solution}:
	\parb	
	\cbox[0.9]{
		\Large
		\begin{eqnarray*}
			P_3 & = & 0\\
			P_2 & = & P_3+\gamma h\\
			& = & 0+(1.258)(9.81\text{ kN/m}^3)(0.105\text{ m})\\
			& = & 1.2958\text{ kPa}\\
			P_1 & = & 1.2958\text{ kPa}\\
			P_A & = & P_1-\gamma h\\
			& = & 1.2958\text{ kPa}-(9.78)(0.152)\text{ kPa}\\
			& = & -0.1907\text{ kPa}\\
		\end{eqnarray*}
	}
	\parb
	\textbf{Note}:
	\raggedright
	There is not much difference in pressure for a difference in levels
	of 0.105 m. For this reason, a gauge fluid with a higher specific
	gravity, such as mercury, is usually used to measure larger pressure
	differences.	
\end{minipage}

%\vspace{1cm}
\begin{minipage}[t]{0.45\textwidth}	
	\raggedright
	\textbf{Example 8}:\\	
	Find the pressure difference between $A$ and $B$	
	\cfig[0.55]{../../figs/01NoFPM/imgCIVL250M02QFsolved.pdf}
	\par\vspace{6cm}
\end{minipage}
\hfill 
\begin{minipage}[t]{0.5\textwidth}		
	\textbf{Solution}:
	\parb	
	\cbox[0.9]{
		\Large
		\begin{eqnarray*}
			P_1 & = & P_A+\gamma h\\
			& = & P_A+(9.81\text{ kN/m}^3)(0.195\text{ m})\\
			& = & P_A+1.913\text{ kPa}\\
			P_2 & = & P_1+(13.54)(9.81\text{ kN/m}^3)(0.15\text{ m})\\
			& = & P_A+(1.913+19.924)\text{ kPa}\\
			& = & P_A+21.837\text{ kPa}\\
			P_3 & = & P_A+21.837\text{ kPa}\\
			P_B & = & P_3-(0.90)(9.81\text{ kN/m}^3)(0.240\text{ m})\\
			& = & P_A+21.837\text{ kPa}-2.119\text{ kPa}\\
			\Delta p & = & 19.72\text{ kPa}
		\end{eqnarray*}
	}
\end{minipage}
	

\end{document}
