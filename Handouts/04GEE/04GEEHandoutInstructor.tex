% override specific chktex warnings

% chktex-file 1 - ignore commands followed by a space, e.g. \\ new line here
% chktex-file 3 - enclose previous parentheses wit {}
% chktex-file 9 - sometimes messes up with ( and {
% chktex-file 36 - put a space in front of parentheses
% chktex-file 45 - don't use $$ instead of \[, etc
% chktex-file 46 - don't use $ instead of \(, etc

\documentclass[10pt, onesided]{amsart}
% \usepackage[showboxes]{textpos}

\usepackage[absolute,overlay]{textpos}
\setlength{\TPHorizModule}{1.0cm}
\setlength{\TPVertModule}{\TPHorizModule}
\textblockorigin{0.0cm}{0.0cm}  %start all at upper left corner

\usepackage{amsmath}
\usepackage{amsthm}
\usepackage{amsfonts}
\usepackage{amssymb}
\usepackage{mathpazo}
\usepackage{booktabs}
\usepackage[usenames,x11names]{xcolor}
\usepackage{tikz}
\usepackage{textcomp}
\usepackage[letterpaper]{geometry}
\geometry{verbose,tmargin=0.5in,bmargin=0.5in,lmargin=0.75in,rmargin=0.5in}
\usepackage{multicol}
\usepackage{bm}
\usepackage{comment}
\usepackage{cancel}
\usepackage{array}
\usepackage{gensymb}
\usepackage{enumerate}

\usepackage[many]{tcolorbox}

\pagestyle{plain}
\raggedright
\renewcommand{\familydefault}{\sfdefault}
\setlength{\parskip}{\medskipamount}
\setlength{\columnsep}{1cm}

\everymath{\displaystyle}
\setlength{\parskip}{\bigskipamount}

% counter for resuming enumerated list numbers
\newcounter{resumeenumi}
\newcommand{\suspend}{\setcounter{resumeenumi}{\theenumi}}
\newcommand{\resume}{\setcounter{enumi}{\theresumeenumi}}

\newcommand\lb{\linebreak}
\newcommand\pars{\par\smallskip}
\newcommand\parm{\par\medskip}
\newcommand\parb{\par\bigskip}

\makeatletter
\providecommand{\gettikzxy}[3]{%
	\tikz@scan@one@point\pgfutil@firstofone#1\relax
	\edef#2{\the\pgf@x}%
	\edef#3{\the\pgf@y}%
}
\makeatother



% full width colored block but color specifiable
%\cb[body bg strength]{header bg}{header text}{body text}
\newcommand{\cb}[4][15]{
	\setbeamercolor{block title}{bg = #2}
	\setbeamercolor{block body}{bg = #2!#1}
	\setbeamercolor{item projected}{bg=#2, fg=white}
	\begin{center}
		\begin{block}{#3}
			#4
		\end{block}
	\end{center}
}

% colored block with width specified
% \cbw[body bg strength]{header bg}{width}{header text}{body text}
\newcommand{\cbw}[5][15]{
	\begin{center}
		%\vspace{-0.35cm}
		\begin{minipage}{#3\textwidth}
			\setbeamercolor{block title}{bg= #2}
			\setbeamercolor{block body}{bg= #2!#1}
			\setbeamercolor{item projected}{bg=#2, fg=white}
			\begin{block}{#4}
				\raggedright
				#5
			\end{block}
		\end{minipage}
	\end{center}
}

% centered minipage with text \raggedright
%\cmini[width]{content}
\newcommand{\cmini}[2][0.8]{
	\begin{center}
		\begin{minipage}{#1\columnwidth}
			\raggedright
			#2
		\end{minipage}
	\end{center}
}

%left flushed minipage
\newcommand{\mini}[2][0.8]{
	\begin{minipage}{#1\columnwidth}
		\raggedright
		#2
	\end{minipage}
}

%left flushed minipage, top aligned
\newcommand{\minit}[2][0.8]{
	\begin{minipage}[t]{#1\columnwidth}
		\raggedright
		#2
	\end{minipage}
}

%left flushed minipage
% \newcommand{\miniT}[2][0.8]{
%  \begin{minipage}[T]{#1\columnwidth}
%   \raggedright
%   #2
%  \end{minipage}
% }

%left flushed minipage
\newcommand{\minib}[2][0.8]{
	\begin{minipage}[b]{#1\columnwidth}
		\raggedright
		#2
	\end{minipage}
}

\newcommand{\cfig}[2][1]{% centred, scaled graphic
	\begin{center}
		\includegraphics[scale=#1]{#2}
	\end{center}
}
% figure with tight border for photos
% \cfigb[saitMaroon]{borderwidth with unit}{scale}{image}
\newcommand{\cfigb}[4][structure]{
	% \usepackage{adjustbox}
	\setlength{\fboxrule}{1pt}
	\begin{center}
		\includegraphics[scale=#3, cframe= #1 #2]{#4}
	\end{center}
}

\newcommand{\imgbox}[3]{
	% \setlength{\fboxsep}{12pt}
	\includegraphics[scale=#1, cframe= structure #3]{#2}
}

% \imgboxbg[bg color=white]{scale}{path/to/img}{border color}{border, e.g. 2pt}{margin, e.g. 4pt}
\newcommand{\imgboxbg}[6][white]{
	\setlength{\fboxrule}{#5}
	\setlength{\fboxsep}{#6}
	\centering
	\fcolorbox{#4}{#1}{\includegraphics[scale=#2]{#3}}
}

\newcommand{\fig}[2][1]{% scaled graphic
	\includegraphics[scale=#1]{#2}
}

% centred framed  box black border
%\cbox[width]{content}
\newcommand{\cbox}[2][0.9]{% framed centered  box
	\setlength\fboxsep{0.042\columnwidth}
	\setlength\fboxrule{0.0015\columnwidth}
	\begin{center}
		\fcolorbox{black}{white}{
			\vspace{-0.5cm}
			\begin{minipage}{#1\columnwidth}
				\raggedright
				#2
			\end{minipage}
		}
	\end{center}
	\setlength\fboxsep{0cm}
}



\newtcolorbox{mybox}[1][]
{
	colback=white,
	top=0.25cm,
	bottom=0.25cm,
	left=0.25cm,
	right=0.25cm,
	colframe=structure,
	fonttitle=\bfseries,
	enhanced, drop fuzzy shadow,
	% attach boxed title to top left={yshift=-2mm, xshift=5mm},
	attach boxed title to top left={yshift=-2mm, xshift=5mm}, colbacktitle=structure!80!white, #1}

\newtcolorbox{plainbox}[1][]{colback=white, sharp corners, top=0.125cm, bottom=0.125cm, left=0pt, right=0pt, boxrule=0.5pt,colframe=structure,fonttitle=\bfseries, colbacktitle=structure, arc=0mm, #1}
%
\newtcbtheorem{myexam}{Example}%
{
	enhanced,
	colback=white,
	top=0.375cm,
	bottom=0.25cm,
	left=0.375cm,
	right=0.375cm,
	colframe=structure,
	fonttitle=\bfseries,
	drop fuzzy shadow,
	%description font=\mdseries\itshape,
	attach boxed title to top left={yshift=-2mm, xshift=5mm},
	colbacktitle=structure!80!white
	}{exam}% then \pageref{exer:theoexample} references the theo

\newcommand{\myexample}[2][red]{
	% \tcb\tcbset{theostyle/.style={colframe=red,colbacktitle=yellow}}
	\begin{myexam}{}{}
		\raggedright
		#2
	\end{myexam}
	% \tcbset{colframe=structure,colbacktitle=structure}
}

\newtcbtheorem{myexer}{Exercise}%
{
	enhanced,
	colback=white,
	top=0.375cm,
	bottom=0.25cm,
	left=0.375cm,
	right=0.375cm,
	colframe=structure,
	fonttitle=\bfseries,
	drop fuzzy shadow,
	%description font=\mdseries\itshape,
	attach boxed title to top left={yshift=-2mm, xshift=5mm},
	colbacktitle=structure!80!white
	}{exer}

\newcommand{\myexercise}[2][red]{
	% \tcb\tcbset{theostyle/.style={colframe=red,colbacktitle=yellow}}
	\begin{myexer}{}{}
		\raggedright
		#2
	\end{myexer}
	% \tcbset{colframe=structure,colbacktitle=structure}
}


\begin{document}

\thispagestyle{empty}
\vspace{-7cm}
\centering

\textbf{\Large Module 4: The General Energy Equation (CIVL 318)}
\par\medskip
\begin{center}
	\begin{tabular}{r >{$}r<{$} >{$}c<{$} >{$}l<{$}}
		\toprule
		\addlinespace
		\textbf{ GEE}:                        & \frac{p_A}{\gamma}+z_A+\frac{v_A^2}{2g} +h_A-h_R-h_L & = & \frac{p_B}{\gamma}+z_B+\frac{v_B^2}{2g}                                                  \\
		\addlinespace
		\midrule
		\addlinespace
		\textbf{ Power added by a pump}:      & P_A                                                  & = & h_A\gamma Q                                                                              \\
		\addlinespace
		\midrule
		\addlinespace
		\textbf{ Power removed by a turbine}: & P_R                                                  & = & h_R\gamma Q                                                                              \\
		\addlinespace
		\midrule
		\addlinespace
		\textbf{ Efficiency of a pump}:       & e_M                                                  & = & \frac{\text{power delivered to fluid}}{\text{power input to pump}}=\frac{P_A}{P_I}       \\
		\addlinespace
		\addlinespace
		\midrule
		\addlinespace
		\textbf{ Efficiency of a turbine}:    & e_M                                                  & = & \frac{\text{power output from turbine}}{\text{power removed from fluid}}=\frac{P_O}{P_R} \\
		\addlinespace
		\bottomrule
	\end{tabular}
\end{center}

%%%%%%%%%%%%%%%%%%%%%%%%%%%%%%%%%%%%%%%%%%%%%%%%%%%%%%%%%%%%%%%%%%%%%%%%%%%%%%%%%%%%%%%%%%%%%%%%%%%%%%%%%%%%%%%%%%%%%

%\rule{\textwidth}{0.02in}
%%%%%%%%%%%%%%%%%%%%%%%%%%%%%%%%%%%%%%%%%%%%%%%%%%%%%%%%%%%%%%%%%%%%%%%%%%%%%%%%%%%%%%%%%%%%%%%%%%%%%%%%%%%%%%%%%%%%%



\raggedright

%%%%%%%%%%%%%%%%%%%%%%%%%%%%%%%%%%%%%%%%%%%%%%%%%%%%%%%%%%%%%%%%%%%%%%%%%%%%%%%%%%%%%%%%%%%%%%%%%%%%%%%%%
% \begin{minipage}[t]{0.45\columnwidth}
\minit[0.425]{
	\textbf{Example 1}:
	\cfig[0.325]{../../figs/04GEE/05GEE-Ex01}
	Liquid with a specific gravity of 0.9 flows from a tank, pressurized to $57\,\text{kPa}$, through the pipe system shown, before entering the atmosphere through a nozzle with diameter $125\,\text{mm}$. \parm
	If the volume flow rate is $Q=89\,\text{L/s}$, determine $h_L$, the head loss due to friction and fittings.
	\parb
	\textbf{Solution}:
	\parb
	\cbox[0.9]{
		\large
		Find the average flow velocity at the nozzle:
		\begin{align*}
			v_N              & = \frac{Q}{A_N}=\frac{0.089}{\pi(0.125)^2/4}=7.2524\text{ m/s} \\
			\frac{v_N^2}{2g} & = 2.6808\text{ m}                                              
		\end{align*}
		There is no head added or head removed in this problem so those terms may be omitted from the GEE. Apply the GEE
		to the liquid surface in the tank and to the nozzle:
	}
	\vfill
	% \end{minipage}
}
\hfill
% \begin{minipage}[t]{0.5\columnwidth}
\minit[0.5]{
	\vspace{0cm} %wtf? why do i need this?
	\cbox[0.8]{
		\begin{align*}
			\frac{P_S}{\gamma}+z_S+\frac{v_S^2}{2g}-h_L & = \frac{P_N}{\gamma}+z_N+\frac{v_N^2}{2g} \\
			\frac{57}{0.9\times9.81}+3.7+0-h_L          & = 0+0+\frac{v_N^2}{2g}                    \\
			h_L                                         & = \frac{57}{0.9\times9.81}+3.7-2.6808     \\
			                                            & = 7.4752                                  \\
			\bm{h_L}                                    & = \bm{7.48}\,\textbf{m}                   
		\end{align*}
	}
	% \end{minipage}
}

\newpage
%%%%%%%%%%%%%%%%%%%%%%%%%%%%%%%%%%%%%%%%%%%%%%%%%%%%%%%%%%%%%%%%%%%%%%%%%%%%%%%%%%%%%%%%%%%%%%%%%%%%%%%%%%%%%%%%%%%%%


\minit[0.45]{
	\textbf{Example 2}:
	\begin{cfig}[0.35]{../../figs/04GEE/05GEE-Ex02}\end{cfig}
	Liquid with a specific gravity of 0.87 is pumped from Tank $1$; the liquid exits the pipe at C before dropping into Tank~$2$ at $180\;\text{L/s}$.
	\parb
	Determine the head added by the pump and the pressure at $A$.
	\parb
	(Assume that friction losses are not significant.)
	\parb
	\textbf{Solution}:
	
	\cbox[0.8]{
		\large
		The diameter of the $10''$ S40 Steel Pipe is $254.5\text{ mm}$ and the diameter
		of the $6''$ S40 Steel Pipe is $154.1\text{ mm}$.\par
		Then, the average flow velocity at the pipe outlet, $C$, above Tank $2$ is:
		\[ v_C = \frac{Q}{A_C}=\frac{0.180}{\pi(0.1541)^2/4}=9.6511\text{ m/s} \]
		The velocity head at $C$ is:
		\[ \frac{v_C^2}{2g}=4.7474\text{ m}\]
		Apply the GEE to the surface, $S$, of the liquid in Tank 1 and to the outlet,
		$N$, above above Tank $2$:
		\begin{align*}
			\frac{P_S}{\gamma}+z_S+\frac{v_S^2}{2g} + h_A & = \frac{P_C}{\gamma}+z_C+\frac{v_C^2}{2g} \\
			0+0+0 + h_A                                   & = 0+(1.30+1.85)+4.7474                    \\
			h_A                                           & = 7.8974\,\text{m}                        \\
			\bm{h_A}                                      & = \bm{7.90\,}\textbf{m}                   
		\end{align*}
	}
}
\hfill
\minit[0.45]{
	\cbox{
		\large
		The average flow velocity at $A$ is:
		\[ v_A = \frac{Q}{A_A}=\frac{0.180}{\pi(0.2545)^2/4}=3.5384\text{ m/s} \]
		The velocity head at $A$ is:
		\[ \frac{v_A^2}{2g}=0.63814\text{ m}\]
		Now, apply the GEE to the surface, $S$, of the liquid in Tank 1 and to $A$. Note that $A$ is before the pump
		so no head has been added by the time fluid reaches $A$.
		\begin{align*}
			\frac{P_S}{\gamma}+z_S+\frac{v_S^2}{2g} & = \frac{P_A}{\gamma}+z_A+\frac{v_A^2}{2g}     \\
			0+0+0                                   & = \frac{P_A}{0.87\times9.81}+(1.30+1.85+0.75) \\
			                                        & \qquad+0.63814                                \\
			P_A                                     & = -38.732\text{ kPa}                          \\
			\bm{P_A}                                & = \bm{-38.7\,}\textbf{kPa}                    
		\end{align*}
	}
	\parb
	\cbox[1]{
		\large
		Note that if the GEE had been applied between $A$ and $C$, then the $7.8974\,$m of head added would have had to be included in the GEE since that head is added to the flow between $A$ and $C$.
	}
}




% }

\newpage

%%%%%%%%%%%%%%%%%%%%%%%%%%%%%%%%%%%%%%%%%%%%%%%%%%%%%%%%%%%%%%%%

\mini[0.5]{
	\textbf{Exercise 1}:
	\begin{cfig}[0.35]{../../figs/04GEE/05GEE-Ex02}\end{cfig}
	Liquid with a specific gravity of 0.87 is pumped from Tank $1$; the liquid exits the pipe at C before dropping into Tank~$2$ at $180\;\text{L/s}$. (Neglect any head losses due to friction and valves.)
	\parb
	Determine the pressure at $B$:
	\begin{enumerate}
		\item First, by applying the GEE between the surface of Tank 1 and $B$;
		\item Second, by applying the GEE between $A$ and $B$;
		\item Finally, by applying the GEE between $B$ and $C$.
	\end{enumerate}
	\parb
	\textbf{Solution}:
	\parm
	Note that the velocities at $B$ and at $C$ are the same.
}
\parm
% \vspace{-0cm}
\minit[0.05]{
	(1)
}
\minit[0.6]{
	\vspace{-1em}
	\cbox{
		\begin{align*}
			\frac{P_1}{\gamma}+z_1+\frac{v_1^2}{2g}+ h_A & = 	\frac{P_B}{\gamma}+z_B+\frac{v_B^2}{2g} \\
			0+0+0+ 7.8974                                & = 	\frac{P_B}{0.87\times 9.81}+5.85+4.7474 \\
			\implies P_B                                 & = -23.044\,\text{kPa}                      \\
			\implies \bm{P_B}                            & = \bm{-23.0}\,\textbf{kPa}                 
		\end{align*}
	}
}
\hfill
\parm
\minit[0.05]{
	(2)
}
\minit[0.6]{
	\vspace{-1em}
	\cbox{
		\begin{align*}
			\frac{P_A}{\gamma}+z_A+\frac{v_A^2}{2g} + h_A   & = \frac{P_B}{\gamma}+z_B+\frac{v_B^2}{2g}         \\
			\frac{-38.732}{0.87\times9.81}+0+0.63814+7.8974 & = \frac{P_B}{0.87\times9.81}+1.95         +4.7474 \\
			\implies P_B                                    & = -23.044\text{ kPa}                              \\
			\implies \bm{P_B}                               & = \bm{-23.0}\,\textbf{kPa}                        
		\end{align*}
	}
}
\hfill
\parm
\minit[0.05]{
	(3)
}
\minit[0.6]{
	\vspace{-1em}
	\cbox{
		\begin{align*}
			\frac{P_B}{\gamma}+z_B+\frac{v_B^2}{2g}                  & = \frac{P_C}{\gamma}+z_C+\frac{v_C^2}{2g} \\
			\frac{P_B}{0.87\times9.81}+2.7+\cancel{\frac{v_B^2}{2g}} & = 0+0+\cancel{\frac{v_C^2}{2g}}           \\
			\implies P_B                                             & = -23.044\,\text{kPa}                     \\
			\implies \bm{P_B}                                        & = \bm{-23.0}\,\textbf{kPa}                
		\end{align*}
	}
}



\newpage

%%%%%%%%%%%%%%%%%%%%%%%%%%%%%%%%%%%%%%%%%%%%%%%%%%%%%%%%%%%%%%%%%%%%%%%%%%%%%%%%%%%%%%%%%%%%%%%%%%%%%%%%%%%%%%%%%%%%%
\mini[0.5]{\textbf{Example 3}:
	\begin{cfig}[0.35]{../../figs/04GEE/05GEE-Ex03}\end{cfig}
	Water flows from $A$ to $B$ at the rate of $120\text{ L/s}$\\
	Determine the head removed by the turbine.
	\parb
	\textbf{Solution}:
	
	\cbox[0.9]{
		\large
		The velocity heads at $A$ and $B$ are found as follows:
		\begin{align*}
			v_A              & = \frac{0.120\,\mathsf{m^3}/s}{\pi(0.300\text{ m})^2/4} \\
			                 & = 1.6977\text{ m/s}                                     \\
			\frac{v_A^2}{2g} & = 0.14689\text{ m}                                      \\\\
			v_B              & = \frac{0.120\,\mathsf{m^3}/s}{\pi(0.600\text{ m})^2/4} \\
			                 & = 0.42441\,\text{m/s}                                   \\
			\frac{v_B^2}{2g} & = 0.0091808\,\text{m}                                   
		\end{align*}
	}
	\parb
	\cbox[0.9]{
		\large
		Apply the GEE between $A$ and $B$:
		\begin{align*}
			\frac{P_A}{\gamma}+z_A+\frac{v_A^2}{2g}-h_R & = \frac{P_B}{\gamma}+\cancelto{0}{z_B}\ \ +\frac{v_B^2}{2g} \\
			\frac{150}{9.81}+1.0+0.14689-h_R            & = \frac{-35}{9.81}+0.0091808                                \\\\
			h_R                                         & = 19.996\text{ m}                                           \\
			\bm{h_R}                                    & =  \bm{20.0}\,\textbf{m}                                    
		\end{align*}
	}
}
\newpage
%%%%%%%%%%%%%%%%%%%%%%%%%%%%%%%%%%%%%%%%%%%%%%%%%%%%%%%%%%%%%%%%%%%%%%%%%%%%%%%%%%%%%%%%%%%%%%%%%%%%%%%%%%%%%%%%%%%%%
\minit[0.45]{
	\textbf{Example 4}:
	\begin{cfig}[0.35]{../../figs/04GEE/05GEE-Ex04a}\end{cfig}
	A pump produces a flow of $1024\text{ L/min}$ of kerosene with a specific gravity of $0.823$ from vented underground
	storage to an elevated tank pressurized to $512\text{ kPa}$. Energy loss between the underground storage and
	the pump is $0.95\text{ m}$ and energy loss between the pump and the elevated tank is $4.9\text{ m}$.
	\begin{enumerate}[(a)]
		\item Determine the power added to the fluid by the pump.
		\item If the pump has an efficiency of $73\%$, determine the (electrical) power drawn by the pump.
		\item Determine the gauge and the absolute pressure at the pump inlet.
	\end{enumerate}
	\parm
	\textbf{Solution}:
	
	\cbox[0.9]{
		Find the power added by the pump:
		\begin{align*}
			Q & = \frac{1024\,\text{L/min}}{60\,\text{s/min}\times 1000\,\mathsf{L/m^3}} \\
			  & =0.017067\;\mathsf{m^3/s}                                                
		\end{align*}
		\parm
		\begin{align*}
			\frac{P_A}{\gamma}+z_A +\frac{v_A^2}{2g}+h_A-h_L & = \frac{P_B}{\gamma}+z_B+\frac{v_B^2}{2g} \\
			0+0 +0+h_A-5.85                                  & = \frac{512}{0.823\times 9.81}+15.1+0     \\
			h_A                                              & = 84.366\,\text{m}                        
			% & = 84.4\,\text{m}
		\end{align*}
		\parm
		\begin{align*}
			P_A & = h_A \gamma Q                                                                                         \\
			    & =  (84.366\text{ m})\left(0.823\times9.81\;\mathsf{kN/m^3}\right)\left(0.017067\;\mathsf{m^3/s}\right) \\
			    & = 11.625\;\mathsf{ kN\cdot m/s}                                                                        
		\end{align*}
		The power added by the pump, $\bm{P_A = 11.63\,\textbf{kW}}$
	}
}
\hfill
\minit[0.45]{
	
	\cbox[0.9]{
		\large
		
		Find the power drawn by the pump:
		\begin{align*}
			P_I      & = \frac{P_A}{e_M}         \\
			         & = \frac{11.652}{73\%}     \\
			         & = 15.924\,\text{kW}       \\
			\bm{P_I} & = \bm{15.92}\,\textbf{kW} 
		\end{align*}
	}
	
	\vfill\pagebreak
	
	\cbox[0.9]{
		\large
		Find the velocity head at $C$, the pump inlet:
		\begin{align*}
			v_C              & = Q/A_C                                                     \\
			                 & = \frac{0.017067\;\mathsf{m^3/s}}{\pi(0.1023\text{ m})^2/4} \\
			                 & = 2.0764\text{ m/s}                                         \\
			\frac{v_C^2}{2g} & = 0.21975\text{ m}                                          \\
		\end{align*}
		
		Using the GEE, find the pressure at $C$:
		\begin{align*}
			\frac{P_A}{\gamma}+z_A +\frac{v_A^2}{2g}+h_A-h_L & = \frac{P_C}{\gamma}+z_C+\frac{v_C^2}{2g} \\
			0+0+0-0.95                                       & = \frac{P_C}{0.823\times9.81}             \\
			                                                 & \qquad+4.3+0.21975                        \\
			P_C                                              & = -44.161\,\text{kPa}                     \\
			\bm{P_{C(\textbf{gauge})}}                       & = \bm{-44.2}\,\textbf{kPa}                
		\end{align*}
		\parb
		Note that this is the \textbf{gauge} pressure.
		\begin{align*}
			p_{\text{abs}}             & = p_{\text{atm}}+p_{\text{gauge}} \\
			                           & = 101.8-44.2\text{ kPa}           \\
			                           & = 57.639\text{ kPa}               \\
			\bm{P_{C{\textbf{(abs)}}}} & = \bm{57.6}\,\textbf{kPa}         
		\end{align*}
		
	}
}
\newpage

%%%%%%%%%%%%%%%%%%%%%%%%%%%%%%%%%%%%%%%%%%%%%%%%%%%%%%%%%%%%%%%%%%%%%%%%%%%%%%%%%%%%%%%%%%%%%%%%%%%%%%%%%%%%%%%%%%%%%%%%%%%%%%%%%%%
\minit[0.525]{
	\textbf{Exercise 2}:
	\begin{cfig}[0.4]{../../figs/04GEE/05GEE-Ex05}\end{cfig}
	Oil, with $\text{sg}=0.87$, flows from a tank pressurized at $130\text{ kPa}$ at a rate of $72\text{ L/s}$ and
	powers a fluid motor as shown. Energy losses due to friction and fittings between the tank and $B$ are estimated to
	be $1.81\text{ m}$.
	\par\medskip
	If the pressure at $B$ is found to be $-56\text{ kPa}$ and the motor has an efficiency of $78\%$,
	determine the power output from the motor.
	\parb
	\textbf{Solution}:
	
	\cbox[1]{
		\large
		Find the velocity head at $B$:
		\begin{align*}
			v_B              & = \frac{Q}{A_B}                                          \\
			                 & = \frac{0.072\;\mathsf{m^3/s}}{\pi(0.275\,\text{m})^2/4} \\
			                 & = 1.2122\,\text{m/s}                                     \\
			\frac{v_B^2}{2g} & = 0.074895\,\text{m}                                     
		\end{align*}
		\parb
		Apply the GEE between the oil surface in the tank and $B$:
		\begin{align*}
			\frac{P_A}{\gamma}+z_A+\cancelto{0}{\frac{v_A^2}{2g}}-h_L-h_R & = \frac{P_b}{\gamma}                    
			+\cancelto{0}{z_B}\ \ +\frac{v_B^2}{2g}\\
			\frac{130}{0.87\times 9.81}+12.6-1.81-h_R                     & = \frac{-56}{0.87\times 9.81}+ 0.074895 \\
			\frac{130+56}{0.87\times 9.81}+12.6-1.81-0.0749               & = h_R                                   \\
			h_R                                                           & = 32.508\text{ m}                       
		\end{align*}
	}
}
\hfill
\minit[0.425]{
	
	
	\cbox[0.9]{
		\large
		The power removed is:
		\begin{align*}
			P_R & = h_R \gamma Q                                                                                           \\
			    & =  (32.508\,\text{m})\!\left(0.87\times 9.81\,\mathsf{kN/m^3}\right)\!\left(0.072\,\mathsf{m^3/s}\right) \\
			    & = 19.976\,\mathsf{kN\!\cdot\!m/s}                                                                        \\
			    & = 19.976\;\text{ kW}                                                                                     
		\end{align*}
		The power output by the motor is:
		\begin{align*}
			P_O      & = P_R\times e_M           \\
			         & = 19.976\times 78\%       \\
			         & = 15.581\text{ kW}        \\
			\bm{P_O} & = \bm{15.58}\,\textbf{kW} 
		\end{align*}
	}
}
\newpage

%%%%%%%%%%%%%%%%%%%%%%%%%%%%%%%%%%%%%%%%%%%%%%%%%%%%%%%%%%%%%%%%%%%%%%%%%%%%%%%%%%%%%%%%%%%%%%%%%%%%%%%%%%%%%%%%%%%%%%%%%%%%%%%%%%%
\minit[0.45]{
	\textbf{Example 5}:
	\begin{cfig}[0.35]{../../figs/04GEE/05GEE-Ex06}\end{cfig}
	A car fuel pump pumps $1\text{ L}$ of gasoline every $45\text{ s}$ when is has a suction pressure of
	$155\text{ mm}$ of mercury vacuum and a discharge pressure of $32\text{ kPa}$. Both the suction and the discharge
	lines have the same diameter.
	\parm
	If the pump efficiency is $68\%$, determine the power drawn from the engine.
	\parb
	\textbf{Solution}:
	
	\cbox[0.9]{
		\large
		First, calculate the input pressure:\\
		``$155\text{ mm}$ of mercury vacuum'' means a pressure that is below atmospheric by an amount due to a height of
		$155\text{ mm}$ of mercury.
		\begin{align*}
			\Delta p & = \gamma h                                            \\
			         & = (13.54\times 9.81\;\mathsf{kN/m^3})(0.155\text{ m}) \\
			         & = 20.588\text{ kPa}                                   
		\end{align*}
		Since this pressure is below atmospheric, it is negative. \lb Thus the inlet pressure is $-20.588\text{ kPa}$.
	}
	\parb
	\cbox[0.9]{
		\large
		Find the head added by the pump. The lines are of the same diameter so velocity heads cancel out. Similarly, the
		lines are at the same elevation so elevation head cancels out too.
		\begin{align*}
			h_A & = \frac{P_{\text{dischage}}-P_{\text{suction}}}{\gamma} \\
			    & = \frac{32-(-20.588)}{0.68\times 9.81}                  \\
			    & = 7.8833\text{ m}                                       
		\end{align*}
	}
}
\hfill
\minit[0.45]{
	
	\cbox[0.9]{
		\large
		
		Then, the power added by the pump is:
		\begin{align*}
			p_A      & = h_A\gamma Q                                                                                                 \\
			         & = (7.8833\,\text{m})(0.68\times 9.81\;\mathsf{kN/m^3})\!\left(\frac{0.001\,\mathsf{m^3}}{45\,\text{s}}\right) \\
			         & = 0.0011686\text{ kW}                                                                                         \\
			         & = 1.1686\text{ W}                                                                                             \\\\
			P_I      & = \frac{1.1686\text{ W}}{0.68}                                                                                \\
			         & = 1.7186\text{ W}                                                                                             \\
			\bm{P_I} & = \bm{1.72}\,\textbf{W}                                                                                       
		\end{align*}
		
	}
}
\newpage

%%%%%%%%%%%%%%%%%%%%%%%%%%%%%%%%%%%%%%%%%%%%%%%%%%%%%%%%%%%%%%%%%%%%%%%%%%%%%%%%%%%%%%%%%%%%%%%%%%%%%%%%%%%%%%%%%%%%%%%%%%%%%%%%%%%
\minit[0.5]{
	\textbf{Example 6}:
	\cfig[0.4]{../../figs/04GEE/05GEE-Ex07}
	Water flows at a steady rate in a vertical pipe. Two pressure gauges are set $10\,\text{m}$	apart, as shown. There are
	no pumps or turbines and the pipe is of constant diameter. \parb Determine which of the following is true:
	\begin{enumerate}[(a)]
		\item flow is upward
		\item flow is downward
		\item there is no flow
	\end{enumerate}
	\parb
	\textbf{Solution}:
	
	\cbox[0.95]{
		\large
		
		The difference in pressure between the two gauges is $$102.0-5.0= 97.0\text{ kPa}$$
		
		If there is no flow, then the difference in pressures should be given by
		\[\Delta p = \gamma h	= 9.81\;\mathsf{kN/m^3}\times10\text{ m}=98.1\text{ kPa}  \]
		
	}
	\parb
	\cbox[0.95]{
		\large
		If the flow is upward, apply the GEE from $B$ to $A$:
		\begin{align*}
			\frac{P_B}{\gamma} + \cancelto{0}{z_B}\ \  + \cancel{\frac{v_B^2}{2g}}-h_L & = \frac{P_A}{\gamma} + \cancelto{10}{z_A}\ \  + \cancel{\frac{v_A^2}{2g}} \\
			h_L                                                                        & = \frac{P_B-P_A}{\gamma}-10                                               \\
			                                                                           & = \frac{102.0-5.0}{9.81}-10                                               \\
			                                                                           & = -0.11213\text{ m}                                                       
		\end{align*}
		A negative headloss is not possible!
	}
	\parb
	% \cbox[0.9]{
	% 	\large
	%
	% 	If the flow is downward, apply the GEE from $A$ to $B$:
	% 	\begin{align*}
	% 		\frac{P_A}{\gamma} + \cancelto{10}{z_A}\ \  + \cancel{\frac{v_A^2}{2g}}-h_L & = \frac{P_B}{\gamma} + \cancelto{0}{z_B}\ \  + \cancel{\frac{v_B^2}{2g}} \\
	% 		h_L                                                                         & = \frac{P_A-P_B}{\gamma}+10                                              \\
	% 		                                                                            & = \frac{5.0-102.0}{9.81}+10                                              \\
	% 		                                                                            & = 0.11213\text{ m}
	% 	\end{align*}
	% }
	%
	% Thus the flow is downward since there cannot be a negative head loss.
	
	Applying the GEE from $A$ to $B$ (i.e., downward flow) yields a headloss of $+0.11213\,\text{m}$.
}
\parb
\textbf{Flow is downward!}
\newpage


%%%%%%%%%%%%%%%%%%%%%%%%%%%%%%%%%%%%%%%%%%%%%%%%%%%%%%%%%%%%%%%%%%%%%%%%%%%%%%%%%%%%%%%%%%%%%%%%%%%%%%%%%%%%%%%%%%%%%%%%%%%%%%%%%%%
\minit[0.5]{
	\textbf{Exercise 3}:
	\parb
	\begin{cfig}[0.35]{../../figs/04GEE/05GEE-Ex08}\end{cfig}
	A rural house relies upon a shallow well for its water supply. The pump at the well is required to supply
	$210\,\text{L/min}$ of water. The water tank at the house maintains a pressure of $280\,\text{kPa}$. Friction losses in the
	pipe amount to $4.35\,$m.
	\parb
	If the pump is $72\%$ efficient, determine the power delivered to the pump by the electrical supply and the
	power added to the water by the pump.
	\parb
	\textbf{Solution}:\\
	
	\cbox[0.9]{
		\large
		Find the head added by the pump:
		\begin{align*}
			\frac{P_P}{\gamma} +z_P+ \frac{v_P^2}{2g} + h_A - h_L & = \frac{P_R}{\gamma} + z_R + \frac{v_R^2}{2g} \\
			0 +0  + 0 + h_A - 4.35                                & = \frac{280}{9.81} + 30.25+ 0                 \\
			h_A                                                   & = 63.142\text{ m}                             
		\end{align*}
		Find the volume flow rate:
		\begin{align*}
			Q & = 210\,\text{L/min}                        \\
			  & = \frac{210}{1000\times60}\;\mathsf{m^3/s} \\
			  & = 0.0035\,\mathsf{m^3/s}                   
		\end{align*}
		Now, the power added to the pump:
		\begin{align*}
			P_{\text{added}}      & = h_A\gamma Q                                                                                 \\
			                      & = (63.142\,\text{m})\!\left(9.81\,\mathsf{kN/m^3}\right)\!\left(0.0035\,\mathsf{m^3/s}\right) \\
			                      & = 2.1680\,\mathsf{kN\!\cdot\!m/s}                                                             \\
			\bm{P_{\text{added}}} & = \bm{2.17}\,\textbf{kW}                                                                      
		\end{align*}
		And the power delivered to the pump:
		\begin{align*}
			P_{I}      & = \frac{P_{\text{added}}}{e_M}   \\
			           & = \frac{P_{\text{added}}}{e_M}   \\
			           & = \frac{2.1680\,\text{kW}}{0.72} \\
			           & = 3.0111\,\text{kW}              \\
			\bm{P_{I}} & = \bm{3.01}\,\textbf{kW}         
		\end{align*}
	}
}



\end{document}
