% override specific chktex warnings

% chktex-file 1 - ignore commands followed by a space, e.g. \\ new line here
% chktex-file 3 - enclose previous parentheses wit {}
% chktex-file 9 - sometimes messes up with ( and {
% chktex-file 36 - put a space in front of parentheses
% chktex-file 45 - don't use $$ instead of \[, etc
% chktex-file 46 - don't use $ instead of \(, etc

\documentclass[10pt, onesided]{amsart}
% \usepackage[showboxes]{textpos}

\usepackage[absolute,overlay]{textpos}
\setlength{\TPHorizModule}{1.0cm}
\setlength{\TPVertModule}{\TPHorizModule}
\textblockorigin{0.0cm}{0.0cm}  %start all at upper left corner

\usepackage{amsmath}
\usepackage{amsthm}
\usepackage{amsfonts}
\usepackage{amssymb}
\usepackage{mathpazo}
\usepackage{booktabs}
\usepackage[usenames,x11names]{xcolor}
\usepackage{tikz}
\usepackage{textcomp}
\usepackage[letterpaper]{geometry}
\geometry{verbose,tmargin=0.5in,bmargin=0.5in,lmargin=0.75in,rmargin=0.5in}
\usepackage{multicol}
\usepackage{bm}
\usepackage{comment}
\usepackage{cancel}
\usepackage{array}
\usepackage{gensymb}
\usepackage{enumerate}

\usepackage[many]{tcolorbox}

\pagestyle{plain}
\raggedright
\renewcommand{\familydefault}{\sfdefault}
\setlength{\parskip}{\medskipamount}
\setlength{\columnsep}{1cm}

\everymath{\displaystyle}
\setlength{\parskip}{\bigskipamount}

% counter for resuming enumerated list numbers
\newcounter{resumeenumi}
\newcommand{\suspend}{\setcounter{resumeenumi}{\theenumi}}
\newcommand{\resume}{\setcounter{enumi}{\theresumeenumi}}

\newcommand\lb{\linebreak}
\newcommand\pars{\par\smallskip}
\newcommand\parm{\par\medskip}
\newcommand\parb{\par\bigskip}

\makeatletter
\providecommand{\gettikzxy}[3]{%
	\tikz@scan@one@point\pgfutil@firstofone#1\relax
	\edef#2{\the\pgf@x}%
	\edef#3{\the\pgf@y}%
}
\makeatother



% full width colored block but color specifiable
%\cb[body bg strength]{header bg}{header text}{body text}
\newcommand{\cb}[4][15]{
	\setbeamercolor{block title}{bg = #2}
	\setbeamercolor{block body}{bg = #2!#1}
	\setbeamercolor{item projected}{bg=#2, fg=white}
	\begin{center}
		\begin{block}{#3}
			#4
		\end{block}
	\end{center}
}

% colored block with width specified
% \cbw[body bg strength]{header bg}{width}{header text}{body text}
\newcommand{\cbw}[5][15]{
	\begin{center}
		%\vspace{-0.35cm}
		\begin{minipage}{#3\textwidth}
			\setbeamercolor{block title}{bg= #2}
			\setbeamercolor{block body}{bg= #2!#1}
			\setbeamercolor{item projected}{bg=#2, fg=white}
			\begin{block}{#4}
				\raggedright
				#5
			\end{block}
		\end{minipage}
	\end{center}
}

% centered minipage with text \raggedright
%\cmini[width]{content}
\newcommand{\cmini}[2][0.8]{
	\begin{center}
		\begin{minipage}{#1\columnwidth}
			\raggedright
			#2
		\end{minipage}
	\end{center}
}

%left flushed minipage
\newcommand{\mini}[2][0.8]{
	\begin{minipage}{#1\columnwidth}
		\raggedright
		#2
	\end{minipage}
}

%left flushed minipage, top aligned
\newcommand{\minit}[2][0.8]{
	\begin{minipage}[t]{#1\columnwidth}
		\raggedright
		#2
	\end{minipage}
}

%left flushed minipage
% \newcommand{\miniT}[2][0.8]{
%  \begin{minipage}[T]{#1\columnwidth}
%   \raggedright
%   #2
%  \end{minipage}
% }

%left flushed minipage
\newcommand{\minib}[2][0.8]{
	\begin{minipage}[b]{#1\columnwidth}
		\raggedright
		#2
	\end{minipage}
}

\newcommand{\cfig}[2][1]{% centred, scaled graphic
	\begin{center}
		\includegraphics[scale=#1]{#2}
	\end{center}
}
% figure with tight border for photos
% \cfigb[saitMaroon]{borderwidth with unit}{scale}{image}
\newcommand{\cfigb}[4][structure]{
	% \usepackage{adjustbox}
	\setlength{\fboxrule}{1pt}
	\begin{center}
		\includegraphics[scale=#3, cframe= #1 #2]{#4}
	\end{center}
}

\newcommand{\imgbox}[3]{
	% \setlength{\fboxsep}{12pt}
	\includegraphics[scale=#1, cframe= structure #3]{#2}
}

% \imgboxbg[bg color=white]{scale}{path/to/img}{border color}{border, e.g. 2pt}{margin, e.g. 4pt}
\newcommand{\imgboxbg}[6][white]{
	\setlength{\fboxrule}{#5}
	\setlength{\fboxsep}{#6}
	\centering
	\fcolorbox{#4}{#1}{\includegraphics[scale=#2]{#3}}
}

\newcommand{\fig}[2][1]{% scaled graphic
	\includegraphics[scale=#1]{#2}
}

% centred framed  box black border
%\cbox[width]{content}
\newcommand{\cbox}[2][0.9]{% framed centered  box
	\setlength\fboxsep{0.042\columnwidth}
	\setlength\fboxrule{0.0015\columnwidth}
	\begin{center}
		\fcolorbox{black}{white}{
			\vspace{-0.5cm}
			\begin{minipage}{#1\columnwidth}
				\raggedright
				#2
			\end{minipage}
		}
	\end{center}
	\setlength\fboxsep{0cm}
}



\newtcolorbox{mybox}[1][]
{
	colback=white,
	top=0.25cm,
	bottom=0.25cm,
	left=0.25cm,
	right=0.25cm,
	colframe=structure,
	fonttitle=\bfseries,
	enhanced, drop fuzzy shadow,
	% attach boxed title to top left={yshift=-2mm, xshift=5mm},
	attach boxed title to top left={yshift=-2mm, xshift=5mm}, colbacktitle=structure!80!white, #1}

\newtcolorbox{plainbox}[1][]{colback=white, sharp corners, top=0.125cm, bottom=0.125cm, left=0pt, right=0pt, boxrule=0.5pt,colframe=structure,fonttitle=\bfseries, colbacktitle=structure, arc=0mm, #1}
%
\newtcbtheorem{myexam}{Example}%
{
	enhanced,
	colback=white,
	top=0.375cm,
	bottom=0.25cm,
	left=0.375cm,
	right=0.375cm,
	colframe=structure,
	fonttitle=\bfseries,
	drop fuzzy shadow,
	%description font=\mdseries\itshape,
	attach boxed title to top left={yshift=-2mm, xshift=5mm},
	colbacktitle=structure!80!white
	}{exam}% then \pageref{exer:theoexample} references the theo

\newcommand{\myexample}[2][red]{
	% \tcb\tcbset{theostyle/.style={colframe=red,colbacktitle=yellow}}
	\begin{myexam}{}{}
		\raggedright
		#2
	\end{myexam}
	% \tcbset{colframe=structure,colbacktitle=structure}
}

\newtcbtheorem{myexer}{Exercise}%
{
	enhanced,
	colback=white,
	top=0.375cm,
	bottom=0.25cm,
	left=0.375cm,
	right=0.375cm,
	colframe=structure,
	fonttitle=\bfseries,
	drop fuzzy shadow,
	%description font=\mdseries\itshape,
	attach boxed title to top left={yshift=-2mm, xshift=5mm},
	colbacktitle=structure!80!white
	}{exer}

\newcommand{\myexercise}[2][red]{
	% \tcb\tcbset{theostyle/.style={colframe=red,colbacktitle=yellow}}
	\begin{myexer}{}{}
		\raggedright
		#2
	\end{myexer}
	% \tcbset{colframe=structure,colbacktitle=structure}
}


\begin{document}

\thispagestyle{empty}
\vspace{-7cm}
\centering

\textbf{\Large Module 4: The General Energy Equation (CIVL 318)}
\par\medskip
\begin{center}
	\begin{tabular}{r >{$}r<{$} >{$}c<{$} >{$}l<{$}}
		\toprule
		\addlinespace
		\textbf{ GEE}:                        & \frac{p_A}{\gamma}+z_A+\frac{v_A^2}{2g} +h_A-h_R-h_L & = & \frac{p_B}{\gamma}+z_B+\frac{v_B^2}{2g}                                                  \\
		\addlinespace
		\midrule
		\addlinespace
		\textbf{ Power added by a pump}:      & P_A                                                  & = & h_A\gamma Q                                                                              \\
		\addlinespace
		\midrule
		\addlinespace
		\textbf{ Power removed by a turbine}: & P_R                                                  & = & h_R\gamma Q                                                                              \\
		\addlinespace
		\midrule
		\addlinespace
		\textbf{ Efficiency of a pump}:       & e_M                                                  & = & \frac{\text{power delivered to fluid}}{\text{power input to pump}}=\frac{P_A}{P_I}       \\
		\addlinespace
		\addlinespace
		\midrule
		\addlinespace
		\textbf{ Efficiency of a turbine}:    & e_M                                                  & = & \frac{\text{power output from turbine}}{\text{power removed from fluid}}=\frac{P_O}{P_R} \\
		\addlinespace
		\bottomrule
	\end{tabular}
\end{center}

%%%%%%%%%%%%%%%%%%%%%%%%%%%%%%%%%%%%%%%%%%%%%%%%%%%%%%%%%%%%%%%%%%%%%%%%%%%%%%%%%%%%%%%%%%%%%%%%%%%%%%%%%%%%%%%%%%%%%

%\rule{\textwidth}{0.02in}
%%%%%%%%%%%%%%%%%%%%%%%%%%%%%%%%%%%%%%%%%%%%%%%%%%%%%%%%%%%%%%%%%%%%%%%%%%%%%%%%%%%%%%%%%%%%%%%%%%%%%%%%%%%%%%%%%%%%%



\raggedright

%%%%%%%%%%%%%%%%%%%%%%%%%%%%%%%%%%%%%%%%%%%%%%%%%%%%%%%%%%%%%%%%%%%%%%%%%%%%%%%%%%%%%%%%%%%%%%%%%%%%%%%%%
\mini[0.425]{
	\textbf{Example 1}:
	\cfig[0.325]{../../figs/04GEE/05GEE-Ex01}
	Liquid with a specific gravity of 0.9 flows from a tank, pressurized to $57\,\text{kPa}$, through the pipe system shown, before entering the atmosphere through a nozzle with diameter $125\,\text{mm}$. \parm
	If the volume flow rate is $Q=89\,\text{L/s}$, determine $h_L$, the head loss due to friction and fittings.
}

\newpage
%%%%%%%%%%%%%%%%%%%%%%%%%%%%%%%%%%%%%%%%%%%%%%%%%%%%%%%%%%%%%%%%%%%%%%%%%%%%%%%%%%%%%%%%%%%%%%%%%%%%%%%%%%%%%%%%%%%%%


\minit[0.45]{
	\textbf{Example 2}:
	\begin{cfig}[0.35]{../../figs/04GEE/05GEE-Ex02}\end{cfig}
	Liquid with a specific gravity of 0.87 is pumped from Tank $1$; the liquid exits the pipe at C before dropping into Tank~$2$ at $180\;\text{L/s}$.
	\parb
	Determine the head added by the pump and the pressure at $A$.
	\parb
	(Assume that friction losses are not significant.)
}
\newpage

%%%%%%%%%%%%%%%%%%%%%%%%%%%%%%%%%%%%%%%%%%%%%%%%%%%%%%%%%%%%%%%%

\mini[0.5]{
	\textbf{Exercise 1}:
	\begin{cfig}[0.35]{../../figs/04GEE/05GEE-Ex02}\end{cfig}
	Liquid with a specific gravity of 0.87 is pumped from Tank $1$; the liquid exits the pipe at C before dropping into Tank~$2$ at $180\;\text{L/s}$. (Neglect any head losses due to friction and valves.)
	\parb
	Determine the pressure at $B$:
	\begin{enumerate}
		\item First, by applying the GEE between the surface of Tank 1 and $B$;
		\item Second, by applying the GEE between $A$ and $B$;
		\item Finally, by applying the GEE between $B$ and $C$.
	\end{enumerate}
	
}
\newpage

%%%%%%%%%%%%%%%%%%%%%%%%%%%%%%%%%%%%%%%%%%%%%%%%%%%%%%%%%%%%%%%%%%%%%%%%%%%%%%%%%%%%%%%%%%%%%%%%%%%%%%%%%%%%%%%%%%%%%
\mini[0.5]{\textbf{Example 3}:
	\begin{cfig}[0.35]{../../figs/04GEE/05GEE-Ex03}\end{cfig}
	Water flows from $A$ to $B$ at the rate of $120\text{ L/s}$\\
	Determine the head removed by the turbine.
	
}
\newpage
%%%%%%%%%%%%%%%%%%%%%%%%%%%%%%%%%%%%%%%%%%%%%%%%%%%%%%%%%%%%%%%%%%%%%%%%%%%%%%%%%%%%%%%%%%%%%%%%%%%%%%%%%%%%%%%%%%%%%
\minit[0.45]{
	\textbf{Example 4}:
	\begin{cfig}[0.35]{../../figs/04GEE/05GEE-Ex04a}\end{cfig}
	A pump produces a flow of $1024\text{ L/min}$ of kerosene with a specific gravity of $0.823$ from vented underground
	storage to an elevated tank pressurized to $512\text{ kPa}$. Energy loss between the underground storage and
	the pump is $0.95\text{ m}$ and energy loss between the pump and the elevated tank is $4.9\text{ m}$.
	\begin{enumerate}[(a)]
		\item Determine the power added to the fluid by the pump.
		\item If the pump has an efficiency of $73\%$, determine the (electrical) power drawn by the pump.
		\item Determine the gauge and the absolute pressure at the pump inlet.
	\end{enumerate}
	
}
\newpage

%%%%%%%%%%%%%%%%%%%%%%%%%%%%%%%%%%%%%%%%%%%%%%%%%%%%%%%%%%%%%%%%%%%%%%%%%%%%%%%%%%%%%%%%%%%%%%%%%%%%%%%%%%%%%%%%%%%%%%%%%%%%%%%%%%%
\minit[0.525]{
	\textbf{Exercise 2}:
	\begin{cfig}[0.4]{../../figs/04GEE/05GEE-Ex05}\end{cfig}
	Oil, with $\text{sg}=0.87$, flows from a tank pressurized at $130\text{ kPa}$ at a rate of $72\text{ L/s}$ and
	powers a fluid motor as shown. Energy losses due to friction and fittings between the tank and $B$ are estimated to
	be $1.81\text{ m}$.
	\par\medskip
	If the pressure at $B$ is found to be $-56\text{ kPa}$ and the motor has an efficiency of $78\%$,
	determine the power output from the motor.
	
}
\newpage

%%%%%%%%%%%%%%%%%%%%%%%%%%%%%%%%%%%%%%%%%%%%%%%%%%%%%%%%%%%%%%%%%%%%%%%%%%%%%%%%%%%%%%%%%%%%%%%%%%%%%%%%%%%%%%%%%%%%%%%%%%%%%%%%%%%
\mini[0.45]{
	\textbf{Example 5}:
	\begin{cfig}[0.35]{../../figs/04GEE/05GEE-Ex06}\end{cfig}
	A car fuel pump pumps $1\text{ L}$ of gasoline every $45\text{ s}$ when is has a suction pressure of
	$155\text{ mm}$ of mercury vacuum and a discharge pressure of $32\text{ kPa}$. Both the suction and the discharge
	lines have the same diameter.
	\parm
	If the pump efficiency is $68\%$, determine the power drawn from the engine.
	
}

\newpage

%%%%%%%%%%%%%%%%%%%%%%%%%%%%%%%%%%%%%%%%%%%%%%%%%%%%%%%%%%%%%%%%%%%%%%%%%%%%%%%%%%%%%%%%%%%%%%%%%%%%%%%%%%%%%%%%%%%%%%%%%%%%%%%%%%%
\minit[0.5]{
	\textbf{Example 6}:
	\cfig[0.4]{../../figs/04GEE/05GEE-Ex07}
	Water flows at a steady rate in a vertical pipe. Two pressure gauges are set $10\,\text{m}$	apart, as shown. There are
	no pumps or turbines and the pipe is of constant diameter. \parb Determine which of the following is true:
	\begin{enumerate}[(a)]
		\item flow is upward
		\item flow is downward
		\item there is no flow
	\end{enumerate}
	
}
\newpage


%%%%%%%%%%%%%%%%%%%%%%%%%%%%%%%%%%%%%%%%%%%%%%%%%%%%%%%%%%%%%%%%%%%%%%%%%%%%%%%%%%%%%%%%%%%%%%%%%%%%%%%%%%%%%%%%%%%%%%%%%%%%%%%%%%%
\minit[0.5]{
	\textbf{Exercise 3}:
	\parb
	\begin{cfig}[0.35]{../../figs/04GEE/05GEE-Ex08}\end{cfig}
	A rural house relies upon a shallow well for its water supply. The pump at the well is required to supply
	$210\,\text{L/min}$ of water. The water tank at the house maintains a pressure of $280\,\text{kPa}$. Friction losses in the
	pipe amount to $4.35\,$m.
	\parb
	If the pump is $72\%$ efficient, determine the power delivered to the pump by the electrical supply and the
	power added to the water by the pump.
}



\end{document}
