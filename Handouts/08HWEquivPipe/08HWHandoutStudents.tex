% override specific chktex warnings

% chktex-file 1 - ignore commands followed by a space, e.g. \\ new line here
% chktex-file 3 - enclose previous parentheses wit {}
% chktex-file 9 - sometimes messes up with ( and {
% chktex-file 36 - put a space in front of parentheses
% chktex-file 45 - don't use $$ instead of \[, etc
% chktex-file 46 - don't use $ instead of \(, etc


\documentclass[10pt, oneside]{amsart}
% \usepackage[showboxes]{textpos}

\usepackage[absolute,overlay]{textpos}
\setlength{\TPHorizModule}{1.0cm}
\setlength{\TPVertModule}{\TPHorizModule}
\textblockorigin{0.0cm}{0.0cm}  %start all at upper left corner

\usepackage{amsmath}
\usepackage{amsthm}
\usepackage{amsfonts}
\usepackage{amssymb}
\usepackage{mathpazo}
\usepackage{booktabs}
\usepackage[usenames,x11names]{xcolor}
\usepackage{tikz}
\usepackage{textcomp}
\usepackage[letterpaper]{geometry}
\geometry{verbose,tmargin=0.5in,bmargin=0.5in,lmargin=1in,rmargin=0.5in}
\usepackage{multicol}
\usepackage{bm}
\usepackage{comment}
\usepackage{cancel}
\usepackage{array}
\usepackage{gensymb}
\usepackage{enumerate}
\usepackage[many]{tcolorbox}

\pagestyle{plain}
\raggedright
\renewcommand{\familydefault}{\sfdefault}
\setlength{\parskip}{\medskipamount}
\setlength{\columnsep}{1cm}

\everymath{\displaystyle}
\setlength{\parskip}{\bigskipamount}
\definecolor{structure}{RGB}{100,100,100}
% counter for resuming enumerated list numbers
\newcounter{resumeenumi}
\newcommand{\suspend}{\setcounter{resumeenumi}{\theenumi}}
\newcommand{\resume}{\setcounter{enumi}{\theresumeenumi}}

\newcommand\lb{\linebreak}
\newcommand\pars{\par\smallskip}
\newcommand\parm{\par\medskip}
\newcommand\parb{\par\bigskip}

\makeatletter
\providecommand{\gettikzxy}[3]{%
	\tikz@scan@one@point\pgfutil@firstofone#1\relax
	\edef#2{\the\pgf@x}%
	\edef#3{\the\pgf@y}%
}
\makeatother



% full width colored block but color specifiable
%\cb[body bg strength]{header bg}{header text}{body text}
\newcommand{\cb}[4][15]{
	\setbeamercolor{block title}{bg = #2}
	\setbeamercolor{block body}{bg = #2!#1}
	\setbeamercolor{item projected}{bg=#2, fg=white}
	\begin{center}
		\begin{block}{#3}
			#4
		\end{block}
	\end{center}
}

% colored block with width specified
% \cbw[body bg strength]{header bg}{width}{header text}{body text}
\newcommand{\cbw}[5][15]{
	\begin{center}
		%\vspace{-0.35cm}
		\begin{minipage}{#3\textwidth}
			\setbeamercolor{block title}{bg= #2}
			\setbeamercolor{block body}{bg= #2!#1}
			\setbeamercolor{item projected}{bg=#2, fg=white}
			\begin{block}{#4}
				\raggedright
				#5
			\end{block}
		\end{minipage}
	\end{center}
}

% centered minipage with text \raggedright
%\cmini[width]{content}
\newcommand{\cmini}[2][0.8]{
	\begin{center}
		\begin{minipage}{#1\columnwidth}
			\raggedright
			#2
		\end{minipage}
	\end{center}
}

%left flushed minipage
\newcommand{\mini}[2][0.8]{
	\begin{minipage}{#1\columnwidth}
		\raggedright
		#2
	\end{minipage}
}

%left flushed minipage, top aligned
\newcommand{\minit}[2][0.8]{
	\begin{minipage}[t]{#1\columnwidth}
		\raggedright
		#2
	\end{minipage}
}

%left flushed minipage
% \newcommand{\miniT}[2][0.8]{
%  \begin{minipage}[T]{#1\columnwidth}
%   \raggedright
%   #2
%  \end{minipage}
% }

%left flushed minipage
\newcommand{\minib}[2][0.8]{
	\begin{minipage}[b]{#1\columnwidth}
		\raggedright
		#2
	\end{minipage}
}

\newcommand{\cfig}[2][1]{% centred, scaled graphic
	\begin{center}
		\includegraphics[scale=#1]{#2}
	\end{center}
}
% figure with tight border for photos
% \cfigb[saitMaroon]{borderwidth with unit}{scale}{image}
\newcommand{\cfigb}[4][structure]{
	% \usepackage{adjustbox}
	\setlength{\fboxrule}{1pt}
	\begin{center}
		\includegraphics[scale=#3, cframe= #1 #2]{#4}
	\end{center}
}

\newcommand{\imgbox}[3]{
	% \setlength{\fboxsep}{12pt}
	\includegraphics[scale=#1, cframe= structure #3]{#2}
}

% \imgboxbg[bg color=white]{scale}{path/to/img}{border color}{border, e.g. 2pt}{margin, e.g. 4pt}
\newcommand{\imgboxbg}[6][white]{
	\setlength{\fboxrule}{#5}
	\setlength{\fboxsep}{#6}
	\centering
	\fcolorbox{#4}{#1}{\includegraphics[scale=#2]{#3}}
}

\newcommand{\fig}[2][1]{% scaled graphic
	\includegraphics[scale=#1]{#2}
}

% centred framed  box black border
%\cbox[width]{content}
\newcommand{\cbox}[2][0.9]{% framed centered  box
	\setlength\fboxsep{0.042\columnwidth}
	\setlength\fboxrule{0.0015\columnwidth}
	\begin{center}
		\fcolorbox{black}{white}{
			\vspace{-0.5cm}
			\begin{minipage}{#1\columnwidth}
				\raggedright
				#2
			\end{minipage}
		}
	\end{center}
	\setlength\fboxsep{0cm}
}



\newtcolorbox{mybox}[1][]
{
	colback=white,
	top=0.25cm,
	bottom=0.25cm,
	left=0.25cm,
	right=0.25cm,
	colframe=structure,
	fonttitle=\bfseries,
	enhanced, drop fuzzy shadow,
	% attach boxed title to top left={yshift=-2mm, xshift=5mm},
	attach boxed title to top left={yshift=-2mm, xshift=5mm}, colbacktitle=structure!80!white, #1}

\newtcolorbox{plainbox}[1][]{colback=white, sharp corners, top=0.125cm, bottom=0.125cm, left=0pt, right=0pt, boxrule=0.5pt,colframe=structure,fonttitle=\bfseries, colbacktitle=structure, arc=0mm, #1}
%
\newtcbtheorem{myexam}{Example}%
{
	enhanced,
	colback=white,
	top=0.375cm,
	bottom=0.25cm,
	left=0.375cm,
	right=0.375cm,
	colframe=structure,
	fonttitle=\bfseries,
	drop fuzzy shadow,
	%description font=\mdseries\itshape,
	attach boxed title to top left={yshift=-2mm, xshift=5mm},
	colbacktitle=structure!80!white
	}{exam}% then \pageref{exer:theoexample} references the theo

\newcommand{\myexample}[2][red]{
	% \tcb\tcbset{theostyle/.style={colframe=red,colbacktitle=yellow}}
	\begin{myexam}{}{}
		\raggedright
		#2
	\end{myexam}
	% \tcbset{colframe=structure,colbacktitle=structure}
}

\newtcbtheorem{myexer}{Exercise}%
{
	enhanced,
	colback=white,
	top=0.375cm,
	bottom=0.25cm,
	left=0.375cm,
	right=0.375cm,
	colframe=structure,
	fonttitle=\bfseries,
	drop fuzzy shadow,
	%description font=\mdseries\itshape,
	attach boxed title to top left={yshift=-2mm, xshift=5mm},
	colbacktitle=structure!80!white
	}{exer}

\newcommand{\myexercise}[2][red]{
	% \tcb\tcbset{theostyle/.style={colframe=red,colbacktitle=yellow}}
	\begin{myexer}{}{}
		\raggedright
		#2
	\end{myexer}
	% \tcbset{colframe=structure,colbacktitle=structure}
}


\begin{document}

\thispagestyle{empty}
\vspace{-7cm}
\centering
\textbf{\Large Module 8: Hazen Williams Equation and Equivalent Pipes (CIVL 318)}
\par\medskip
\centering
\cbox[0.8]{
	\centering
	\textbf{\large Hazen-Williams Equations}
	\[
		Q = \frac{C\,D^{2.63}\left(\frac{h_L}{L}\right)^{0.54}}{279000},\qquad
		h_L = L\,\left(\frac{279000\,Q}{C\,D^{2.63}}\right)^{1.852},\qquad
		D = \left(\frac{279000\,Q}{C\,\left(\frac{h_L}{L}\right)^{0.54}}\right)^{0.3802}
	\]
}
\cbox[0.7]{
	\centering
	\textbf{\Large Equivalent-Length Ratios for Fittings}
	\parb
	\begin{tabular}{r >{$}r<{$} >{$}l<{$} >{$}c<{$} }
		\toprule
		\text{Type}                                                    & \quad & L_e/D \\
		\midrule
		\addlinespace
		Globe valve --- fully open                                     &       & 340   \\
		\addlinespace
		Angle valve --- fully open                                     &       & 150   \\
		\addlinespace
		Gate valve --- fully open                                      &       & 8     \\
		\addlinespace
		--- $3/4$ open                                                 &       & 35    \\
		\addlinespace
		--- $1/2$ open                                                 &       & 160   \\
		\addlinespace
		--- $1/4$ open                                                 &       & 900   \\
		\addlinespace
		Check valve --- swing type                                     &       & 100   \\
		\addlinespace
		Check valve --- ball type                                      &       & 150   \\
		\addlinespace
		Butterfly valve --- fully open --- 2-8''                       &       & 45    \\
		\addlinespace
		--- 10-14''                                                    &       & 35    \\
		\addlinespace
		--- 16-24''                                                    &       & 25    \\
		\addlinespace
		Foot valve --- poppet disc type                                &       & 420   \\
		\addlinespace
		Foot valve --- hinged disc type                                &       & 75    \\
		\addlinespace
		$90\degree$ standard elbow                                     &       & 30    \\
		\addlinespace
		$90\degree$ long radius elbow                                  &       & 20    \\
		\addlinespace
		$90\degree$ street elbow                                       &       & 50    \\
		\addlinespace
		$45\degree$ standard elbow                                     &       & 16    \\
		\addlinespace
		$45\degree$ street elbow                                       &       & 26    \\
		\addlinespace
		Close return bend                                              &       & 50    \\
		\addlinespace
		Standard tee --- flow through run                              &       & 20    \\
		\addlinespace
		Standard tee --- flow through branch                           &       & 60    \\
		\addlinespace
		Gradual enlargement --- $15^\circ$ cone angle                  &       & 8     \\
		\addlinespace
		Gradual enlargement --- $20^\circ$ cone angle                  &       & 15    \\
		\addlinespace
		Gradual enlargement --- $30^\circ$ cone angle                  &       & 23    \\
		\addlinespace
		Gradual reduction --- $15^\circ\text{ to }40^\circ$ cone angle &       & 2     \\
		\addlinespace
		Pipe entrance --- inward projecting                            &       & 50    \\
		\addlinespace
		Pipe entrance --- square                                       &       & 25    \\
		\addlinespace
		Pipe entrance --- rounded                                      &       & 10    \\
		\addlinespace
		Venturi meter                                                  &       & 100   \\
		\addlinespace
	\end{tabular}
}
\raggedright
\newpage



%%%%%%%%%%%%%%%%%%%%%%%%%%%%%%%%%%%%%%%%%%%%%%%%%%%%%%%%%%%%%%%%%%%%%%%%%%%%%%%%%%%%%%%%%%%%%%%%%%%%%%%%%

\mini[0.75]{
	\begin{myexam}{}{}
		\raggedright
		For the pipeline shown, calculate the pressure at $B$, given that the pressure at $A$ is
		$700\,\text{kPa}$.\parm
		The pipes are cement-lined Hyprescon with a diameter of $400\,\text{mm}$ and a roughness coefficient of $C=140$. Flow through the system is $200\,\text{L/s}$.
		\parm
		Elevations are as indicated.
		\cfig[0.5]{../../figs/08HWEquivPipe/HW1}
		
	\end{myexam}
}
\newpage

%%%%%%%%%%%%%%%%%%%%%%%%%%%%%%%%%%%%%%%%%%%%%%%%%%%%%%%%%%%%%%%%%%%%%%%%%%%%%%%%%%%%%%%%%%%%%%%%%%%%%%%%%

\mini[0.75]{
	\begin{myexer}{}{}
		
		For the pipeline shown, calculate the pressure at $C$ and $D$, given that the pressure at $A$ is
		$700\,\text{kPa}$.\parm
		The pipes are cement-lined Hyprescon with a diameter of $400\,\text{mm}$ and a roughness coefficient of $C=140$. Flow through the system is $200\,\text{L/s}$.
		\parm
		Elevations are as indicated.
		
		\cfig[0.5]{../../figs/08HWEquivPipe/HW1}
		
	\end{myexer}
}

\newpage
%%%%%%%%%%%%%%%%%%%%%%%%%%%%%%%%%%%%%%%%%%%%%%%%%%%%%%%%%%%%%%%%%%%%%%%%%%%%%%%%%%%%%%%%%%%%%%%%%%%%%%%%%


\begin{myexam}{}{}
	\minit[0.475]{
		\parb\raggedright
		Water flows from a storage tank through a welded steel pipe that is 1200 m long and 350 mm in diameter, entering a
		distribution grid at point 'B'. Assume C=100. Determine:
		\begin{enumerate}
			\item The pressure at `B' when the flow is 150 L/s
			\item The maximum flow rate into the grid when the minimum allowable pressure at `B' is 400~kPa.
		\end{enumerate}
		Minor losses are negligible compared to friction losses.
	}
	\hfill
	\minit[0.475]{
		\cfig[0.5]{../../figs/08HWEquivPipe/HW2}
	}
\end{myexam}

\newpage

~

\newpage

%%%%%%%%%%%%%%%%%%%%%%%%%%%%%%%%%%%%%%%%%%%%%%%%%%%%%%%%%%%%%%%%%%%%%%%%%%%%%%%%%%%%%%%%%%%%%%%%%%%%%%%%%


\begin{myexer}{}{}
	\mini[0.425]{
		Water flows from one reservoir down to another, through a $500$ mm diameter pipe that is $2000$ m in length. The difference in elevation between the surfaces of the two reservoirs is $30$ m.
		\par\bigskip
		Determine:
		\begin{enumerate}
			\item The flow with high density polyethylene pipe (HDPE) with $C=140$
			\item The flow with welded steel with $C=100$
			\item The diameter of HDPE pipe required for a flow of $1200$ L/s
		\end{enumerate}
		Disregard minor losses.
	}
	\hfill
	\mini[0.525]{
		\cfig[0.475]{../../figs/08HWEquivPipe/HW3}
	}
\end{myexer}

\newpage


%%%%%%%%%%%%%%%%%%%%%%%%%%%%%%%%%%%%%%%%%%%%%%%%%%%%%%%%%%%%%%%%%%%%%%%%%%%%%%%%%%%%%%%%%%%%%%%%%%%%%%%%%
\minit[0.55]{
	\begin{myexam}{}{}
		\parb
		In a water treatment plant, water flows from a filter down to a clear well through the pipe system shown. The pipe is
		welded steel with a diameter of $300\,\text{mm}$ and roughness coefficient $C=130$. The total length of pipe is
		$50\,\text{m}$. Elevation difference $h_1$ between the tanks is $5$ m.
		\parm
		Equivalent length ratios, $L_e/D$, are:\parm
		\begin{tabular}{rlcrl}
			Entrance and exit losses: & 50 &   & Butterfly valve: & 35  \\
			Large radius elbows:      & 25 &   & Venturi meter:   & 100 
		\end{tabular}
		
		\parb
		Determine the flow through the system.
		\parb
		
		\cfig[0.525]{../../figs/08HWEquivPipe/HW4}
		
	\end{myexam}
}

\newpage


%%%%%%%%%%%%%%%%%%%%%%%%%%%%%%%%%%%%%%%%%%%%%%%%%%%%%%%%%%%%%%%%%%%%%%%%%%%%%%%%%%%%%%%%%%%%%%%%%%%%%%%%%
\begin{myexam}{}{}
	\cfig[0.6]{../../figs/08HWEquivPipe/HW5}
	\cmini[0.95]{
		\minit[0.45]{
			In a water treatment plant, backwash water is pumped from the clear well through the pipe system shown to the filter.
			The required backwash flow is $10\,\text{L/s}$ per square meter of filter area (the filter dimensions are
			$10\,\text{m}$ by $15\,\text{m}$.
			The inlet pipe is made of welded steel $(C=130)$, has a diameter of $1000\,\text{mm}$ and a total length
			$\left(L_1+L_2+L_3\right)$ of $10\,\text{m}$.
			The outlet pipe, from the pump to the filter, is also welded steel, has a diameter of $700\,\text{mm}$ and a length
			of $70\,\text{m}$.
			\parb
			The two elevation differences are $h_1=2\,\text{m}$ and $h_2=10\,\text{m}$.
		}
		\hfill
		\minit[0.475]{
			Equivalent length ratios, $L_e/D$, are:
			\parb
			\begin{tabular}{rlcrl}
				Entrance:          & 10  &   & Elbow (inlet):   & 25 \\
				Eccentric Reducer: & 2   &   & Butterfly Valve: & 40 \\
				Check Valve:       & 120 &   & Elbow (outlet):  & 35 \\
				Tee Connection: & 60
			\end{tabular}
			\parb
			Neglect exit losses into the filter.
			\parb
			\vspace{0.25cm}
			Determine:
			\begin{enumerate}
				\item The head losses on the inlet side (clear well to pump)
				\item The head losses on the outlet side (pump to filter)
				      \suspend
			\end{enumerate}
		}
	}
	\pars
\end{myexam}


\newpage
%%%%%%%%%%%%%%%%%%%%%%%%%%%%%%%%%%%%%%%%%%%%%%%%%%%%%%%%%%%%%%%%%%%%%%%%%%%%%%%%%%%%%%%%%%%%%%%%%%%%%%%%%%%%
\mini[0.6]{
	\begin{myexer}{}{}
		% \parb
		This exercise is a continuation of the previous example. Determine:
		\begin{enumerate}
			\resume
			\item The head added by the pump
			\item The pressure at the pump outlet
		\end{enumerate}
	\end{myexer}
}

\newpage


%%%%%%%%%%%%%%%%%%%%%%%%%%%%%%%%%%%%%%%%%%%%%%%%%%%%%%%%%%%%%%%%%%%%%%%%%%%%%%%%%%%%%%%%%%%%%%%%%%%%%%%%%


\begin{myexam}{}{}
	\mini[0.4]{
		The pumps and piping system are used to supply a municipal grid. Pump $P_1$ runs continuously and maintains the basic pressure in the distribution grid beyond point $D$. There is no flow from pumps $P_2$ and $P_3$. (Pump $P_2$ is, in addition to $P_1$, used during periods of high demand and all pumps are used during fire flow demands.)
		\parb
		The elevations are the same at the pump and the discharge point $D$. The outlet pipe, from the pump to point $D$, is welded steel $(C = 130)$ with a diameter of $200\,\text{mm}$ and a total length between fittings of $10\,\text{m}$.
		\parb
		The minimum pressure required at $D$ is $500\,\text{kPa}$ for a design flow of $150\,\text{L/s}$.
		\parb
		Equivalent length ratios, $L_e/D$, are:\parm
		\begin{tabular}{rlcrl}
			Check Valve:    & 120 &   & Gate Valve:    & 15  \\
			Tee Connection: & 60  &   & Venturi Meter: & 100 
		\end{tabular}
	}
	\hfill
	\mini[0.475]{
		\cfig[0.5]{../../figs/08HWEquivPipe/HW6}
		Determine:
		\parm
		\begin{enumerate}
			\item the head losses between $A$ and $D$
			\item the pressure at $A$ required for the required pressure and flow at $D$
		\end{enumerate}
	}
\end{myexam}

\newpage

%%%%%%%%%%%%%%%%%%%%%%%%%%%%%%%%%%%%%%%%%%%%%%%%%%%%%%%%%%%%%%%%%%%%%%%%%%%%%%%%%%%%%%%%%%%%%%%%%%%%%%%%%
\mini{
	\begin{myexam}{}{}
		\cfig[0.325]{../../figs/08HWEquivPipe/HW14ink.pdf}
		\pars
		\cmini[0.85]{
			Determine $Q$, the volume flow rate from $A$ to $D$, through the system shown. Ignore minor losses and assume that $A$ and $D$ are at the same elevation.
			\parm
		}
	\end{myexam}
}

\newpage

%%%%%%%%%%%%%%%%%%%%%%%%%%%%%%%%%%%%%%%%%%%%%%%%%%%%%%%%%%%%%%%%
\mini[0.5]{
	\begin{myexam}{}{}
		\begin{enumerate}[a)]
			\item Determine the diameter of a pipe with length $L=1000\,\text{m}$ and resistance coefficient $C=100$ that is equivalent to $785\,\text{m}$ of new Schedule $40$ $12$-in steel pipe ($D=303.2\,\text{mm}, C=130$).\parb
			\item Verify that this equivalent pipe has the same headloss as the $12$-in steel pipe for two arbitrary flows (choose a couple of flows at random, different from the flow used in part a).
		\end{enumerate}
	\end{myexam}
}

\newpage
%%%%%%%%%%%%%%%%%%%%%%%%%%%%%%%%%%%%%%%%%%%%%%%%%%%%%%%%%%%%%%%%
\begin{myexam}{}{}
	\cfig[0.35]{../../figs/08HWEquivPipe/HW14ink.pdf}
	\parb
	\cmini[0.85]{
		Use the equivalent pipe technique to determine $Q$, the volume flow rate from $A$ to $D$, through the system shown. Ignore minor losses and assume that $A$ and $D$ are at the same elevation.
		\parb
	}
\end{myexam}

\newpage
%%%%%%%%%%%%%%%%%%%%%%%%%%%%%%%%%%%%%%%%%%%%%%%%%%%%%%%%%%%%%%%%%%%%%%%%%%%%%%%%%%%%%%%%%%%%%%%%%%%%%%%%%
%%%%%%%%%%%%%%%%%%%%%%%%%%%%%%%%%%%%%%%%%%%%%%%%%%%%%%%%%%%%%%%%
\begin{myexer}{}{}
	\cfig[0.4]{../../figs/08HWEquivPipe/HW16ink.pdf}
	
	\cmini[0.675]{
		Use the equivalent pipe technique to determine $Q$, the volume flow rate from $A$ to $D$, through the system shown. Ignore minor losses and assume that $A$ and $D$ are at the same elevation.
		\parb
	}
\end{myexer}

\newpage
%%%%%%%%%%%%%%%%%%%%%%%%%%%%%%%%%%%%%%%%%%%%%%%%%%%%%%%%%%%%%%%%

\mini[0.7]{
	
	\begin{myexam}{}{}
		\cfig[0.7]{../../figs/08HWEquivPipe/HW18ink.pdf}
		\parb
		\cmini[0.8]{
			Given a flow of $18\,\text{L/s}$ and ignoring minor losses:\pars
			\begin{enumerate}[a)]
				\item Determine the volume flow rate through each of the parallel pipes between $B$ and $C$.
				\item Determine the headloss due to friction between $B$ and $C$.
			\end{enumerate}
		}
	\end{myexam}
}
\newpage
~
\newpage
%%%%%%%%%%%%%%%%%%%%%%%%%%%%%%%%%%%%%%%%%%%%%%%%%%%%%%%%%%%%%%%%


\mini{
	\begin{myexam}{}{}
		\cfig[0.7]{../../figs/08HWEquivPipe/HW18ink.pdf}
		\parb
		
		\cmini{
			Given a flow of $18\,\text{L/s}$ and ignoring minor losses:
			\begin{enumerate}[a)]
				\item Determine the percentage of the flow that goes through each parallel pipe by choosing a convenient headloss between $B$ and $C$.
				\item Determine the volume flow rate through each of the parallel pipes.
				      % \item Determine the headloss due to friction between $B$ and $C$ for the given flow.
			\end{enumerate}
		}
	\end{myexam}
}

\newpage
%%%%%%%%%%%%%%%%%%%%%%%%%%%%%%%%%%%%%%%%%%%%%%%%%%%%%%%%%%%%%%%%


\mini[1]{
	\begin{myexam}{}{}
		\cfig[0.6]{../../figs/08HWEquivPipe/HW20ink.pdf}
		\parb
		
		\cmini[0.6]{
			$A$, $B$, $C$ and $D$ are at the same elevation. Determine the flow through the system from $A$ to $D$. (Ignore minor losses.)
		}
	\end{myexam}
}
\newpage
~

\end{document}
