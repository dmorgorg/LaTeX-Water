\documentclass[9pt,oneside]{amsart}




\usepackage{amsfonts}
\usepackage{amsmath}
\usepackage{amssymb}
\usepackage{amsthm}
\usepackage{array} % needed for setting math in table definition
\usepackage{mathpazo}
\usepackage{booktabs}
\usepackage{bm}
\usepackage[many]{tcolorbox}
\usepackage{gensymb}
\usepackage[letterpaper]{geometry}
	\geometry{verbose,tmargin=0.5in,bmargin=0.5in,lmargin=0.75in,rmargin=0.75in}
\usepackage{graphicx}
\usepackage{hyperref}



\input{./includes/macros}

\renewcommand{\familydefault}{\sfdefault}
\setlength{\parskip}{\medskipamount}
\setlength{\parindent}{0pt}
\pagestyle{plain}

\begin{document}
\Huge
\thispagestyle{empty}

\begin{center}
 \textbf{CIVL 318 - Notes for Fraser}
\end{center}
\normalsize
\parb
\vspace{1cm}

To  contact Dave: dave@dmorg.org
\parb

\underline{00 Introduction}
\begin{enumerate}
	\item You may choose not to use my online assignments - students struggle initially following instructions about significant digits. If you do want to use them, they are at \url{http://qwizm.org}. I'd suggest working through them yourself ahead of time to get an idea of how they work so that you can troubleshoot for students.
	\item There are only four online assignments created.
	\item Online assignments are super quick to mark - a quick scan to make sure the written work looks professional, then enter the mark on the summary sheet into D2L. Just about a minute for each student.
	\item For traditional problem sets, just mark for professionalism (make clear what you mean by professionalism: computer printed cover sheet, engineering - or graph? - paper, complete). This is quick too.
\end{enumerate}
\parb

\underline{01 The Nature of Fluids and Pressure Measurement}
\begin{enumerate}
	\item Basic, introductory content
\end{enumerate}
\parb

\underline{02 Forces due to Static Fluids}
\begin{enumerate}
	\item Students find this module difficult, maybe due to the fact that it involves statics from semester 1.
	\item There is an online assignment at \url{http://qwizm.org/fluids/03staticForces/quizzes/assignment/index.html}. It is problems taken from the two sets of exercises with links on \url{http://qwizm.org}
	\item We only consider forces on plane surfaces; you can ignore the curved surfaces.
\end{enumerate}

\underline{03 Flow of Fluids Bernoulli's Equation}
\begin{enumerate}
	\item Derivation of Bernoulli's Equation: Students should be familiar with the equations for potential energy ($E=mgh$) and kinetic energy ($E=mv^2/2$) from physics courses in school. Pressure energy is new. I go through it but I'm not sure how many get it. Not an issue, really, since the whole Bernoulli's derivation could be dispensed with. However, I do like to include it. \parm\noindent
	In general, if a proof is relatively straightforward, I like to go through it in the class (although I don't test on proofs) - students get a lot of equations without much justification so the occasional chance to see where a proof comes from may be valuable. In my opinion...
\end{enumerate}

\underline{04 General Energy Equation}
\begin{enumerate}
	\item  Between Example 3 and Exercise 1, there is a concept check. Typically, many students instinctively feel that the pressure at $B$ should be higher than the pressure at $C$ as the liquid is `squeezed' in to a smaller pipe. This is not the case.\parm\noindent
	Total energy/head is the same at $B$ and $C$. They both have the same elevation head but $B$ has more velocity head. Thus, $C$ must have more pressure head (and more pressure) because the total energy head is the same at $B$ and $C$.\parm
	\item You can do an entertaining demo here. You need two empty beer cans (coke or pepsi cans work but this is engineering school so beer is appropriate) and a few straws (one straw is sufficient if only one student tries it). Lay the cans on their side on a desk, parallel to each other and 3-4 inches apart. Get a volunteer student to try to blow the cans apart by directing air from the straw between the cans and parallel to the desk. They may need encouragement to blow harder if the cans don't move. When they move, the cans roll towards each other, not apart. The pressure between the cans is reduced because the air has more kinetic energy. Quite an effective demo and does get the students' interest.	\parm
	\item You can find some useful videos on the Bernoulli Effect on YouTube.
\end{enumerate}

\underline{05 Friction Losses}
\begin{enumerate}
	\item Show students \url{http://eduk8r.org/Flex/fluids/FM_FrictionLosses/DarcyFrictionLoss.html}
\end{enumerate}

\underline{06 Minor Losses}
\begin{enumerate}
	\item Fairly self-explanatory.
\end{enumerate}

\underline{07 Series Pipeline}
\begin{enumerate}
	\item No presentation; no theory. Use the handout to go through the examples. First question can take up to one hour class, leaving students a little shell-shocked, but the second goes more quickly. Let them do the third as an exercise.
	\item Show students \url{http://eduk8r.org/Flex/fluids/FM_SeriesPipe/FM_Series1.html}
\end{enumerate}

\underline{Midterm 1}
\begin{enumerate}
	\item 30\% of the course
	\item 3 questions: forces due to static fluids (30\%), reading the Moody Diagram (10\%), series pipeline (60\%)
\end{enumerate}

\underline{08 Hazen-Williams, Equivalent Pipe}
\begin{enumerate}
	\item A personal favourite although not all students feel the same way.
	\item Major topic on second mid-term
	\item Some students struggle with the concept that the headloss is the same over parallel pipes and that the total headloss is the same as the individual parallel pipe loss. I.e. if three paralled pipes AB1, AB2 and AB3 each have a headloss of 6m between A and B, then the total headloss between A and B is also 6m (not 18m, as some assume).
\end{enumerate}



\end{document}
