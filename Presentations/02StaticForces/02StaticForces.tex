% override specific chktex warnings
% chktex-file 46 - don't use $ instead of \(, etc)
% chktex-file 1 - ignore commands followed by a space, e.g. \\ new line here
% chktex-file 9 - sometimes messes up with ( and {

\documentclass[9pt,xcolor={svgnames, x11names},professionalfonts, mathserif]{beamer}

\usepackage{amsmath}
\usepackage{amssymb}
\usepackage{graphicx}
\usepackage{booktabs}  % for top and bottom spacing in table cells
\usepackage{mathpazo}
\usepackage{textcomp}
\usepackage{multirow}
\usepackage{cancel}
\usepackage{array}
%\usepackage{enumerate}
% \usepackage{enumitem} %causes compile error, stack size exceeded?
\usepackage{gensymb} % for \degree
\usepackage[many]{tcolorbox}
\usepackage{verbatim}
\usepackage{bm}
\usepackage{graphicx}
\usepackage{tikz}
\usepackage{tkz-linknodes}
\usepackage[export]{adjustbox} % for tight borders around photos
\usepackage{pgf} % for sait logo in beamer
\usepackage{pgfmath}

\usepgfmodule{oo}
%\usetikzlibrary{shapes,decorations,shadows,calc}
\usetikzlibrary{shadows,calc,arrows.meta}
% \usetikzlibrary{decorations.shapes}
%\usetikzlibrary{shapes.callouts}
% bloody coils
\usetikzlibrary{decorations.pathmorphing}
\usetikzlibrary{shapes.multipart}

% counter for resuming enumerated list numbers
\newcounter{resumeenumi}
\newcommand{\suspend}{\setcounter{resumeenumi}{\theenumi}}
\newcommand{\resume}{\setcounter{enumi}{\theresumeenumi}}

\newcommand\lb{\linebreak}
\newcommand\pars{\par\smallskip}
\newcommand\parm{\par\medskip}
\newcommand\parb{\par\bigskip}

\makeatletter
\providecommand{\gettikzxy}[3]{%
	\tikz@scan@one@point\pgfutil@firstofone#1\relax
	\edef#2{\the\pgf@x}%
	\edef#3{\the\pgf@y}%
}
\makeatother



% full width colored block but color specifiable
%\cb[body bg strength]{header bg}{header text}{body text}
\newcommand{\cb}[4][15]{
	\setbeamercolor{block title}{bg = #2}
	\setbeamercolor{block body}{bg = #2!#1}
	\setbeamercolor{item projected}{bg=#2, fg=white}
	\begin{center}
		\begin{block}{#3}
			#4
		\end{block}
	\end{center}
}

% colored block with width specified
% \cbw[body bg strength]{header bg}{width}{header text}{body text}
\newcommand{\cbw}[5][15]{
	\begin{center}
		%\vspace{-0.35cm}
		\begin{minipage}{#3\textwidth}
			\setbeamercolor{block title}{bg= #2}
			\setbeamercolor{block body}{bg= #2!#1}
			\setbeamercolor{item projected}{bg=#2, fg=white}
			\begin{block}{#4}
				\raggedright
				#5
			\end{block}
		\end{minipage}
	\end{center}
}

% centered minipage with text \raggedright
%\cmini[width]{content}
\newcommand{\cmini}[2][0.8]{
	\begin{center}
		\begin{minipage}{#1\columnwidth}
			\raggedright
			#2
		\end{minipage}
	\end{center}
}

%left flushed minipage
\newcommand{\mini}[2][0.8]{
	\begin{minipage}{#1\columnwidth}
		\raggedright
		#2
	\end{minipage}
}

%left flushed minipage, top aligned
\newcommand{\minit}[2][0.8]{
	\begin{minipage}[t]{#1\columnwidth}
		\raggedright
		#2
	\end{minipage}
}

%left flushed minipage
% \newcommand{\miniT}[2][0.8]{
%  \begin{minipage}[T]{#1\columnwidth}
%   \raggedright
%   #2
%  \end{minipage}
% }

%left flushed minipage
\newcommand{\minib}[2][0.8]{
	\begin{minipage}[b]{#1\columnwidth}
		\raggedright
		#2
	\end{minipage}
}

\newcommand{\cfig}[2][1]{% centred, scaled graphic
	\begin{center}
		\includegraphics[scale=#1]{#2}
	\end{center}
}
% figure with tight border for photos
% \cfigb[saitMaroon]{borderwidth with unit}{scale}{image}
\newcommand{\cfigb}[4][structure]{
	% \usepackage{adjustbox}
	\setlength{\fboxrule}{1pt}
	\begin{center}
		\includegraphics[scale=#3, cframe= #1 #2]{#4}
	\end{center}
}

\newcommand{\imgbox}[3]{
	% \setlength{\fboxsep}{12pt}
	\includegraphics[scale=#1, cframe= structure #3]{#2}
}

% \imgboxbg[bg color=white]{scale}{path/to/img}{border color}{border, e.g. 2pt}{margin, e.g. 4pt}
\newcommand{\imgboxbg}[6][white]{
	\setlength{\fboxrule}{#5}
	\setlength{\fboxsep}{#6}
	\centering
	\fcolorbox{#4}{#1}{\includegraphics[scale=#2]{#3}}
}

\newcommand{\fig}[2][1]{% scaled graphic
	\includegraphics[scale=#1]{#2}
}

% centred framed  box black border
%\cbox[width]{content}
\newcommand{\cbox}[2][0.9]{% framed centered  box
	\setlength\fboxsep{0.042\columnwidth}
	\setlength\fboxrule{0.0015\columnwidth}
	\begin{center}
		\fcolorbox{black}{white}{
			\vspace{-0.5cm}
			\begin{minipage}{#1\columnwidth}
				\raggedright
				#2
			\end{minipage}
		}
	\end{center}
	\setlength\fboxsep{0cm}
}



\newtcolorbox{mybox}[1][]
{
	colback=white,
	top=0.25cm,
	bottom=0.25cm,
	left=0.25cm,
	right=0.25cm,
	colframe=structure,
	fonttitle=\bfseries,
	enhanced, drop fuzzy shadow,
	% attach boxed title to top left={yshift=-2mm, xshift=5mm},
	attach boxed title to top left={yshift=-2mm, xshift=5mm}, colbacktitle=structure!80!white, #1}

\newtcolorbox{plainbox}[1][]{colback=white, sharp corners, top=0.125cm, bottom=0.125cm, left=0pt, right=0pt, boxrule=0.5pt,colframe=structure,fonttitle=\bfseries, colbacktitle=structure, arc=0mm, #1}
%
\newtcbtheorem{myexam}{Example}%
{
	enhanced,
	colback=white,
	top=0.375cm,
	bottom=0.25cm,
	left=0.375cm,
	right=0.375cm,
	colframe=structure,
	fonttitle=\bfseries,
	drop fuzzy shadow,
	%description font=\mdseries\itshape,
	attach boxed title to top left={yshift=-2mm, xshift=5mm},
	colbacktitle=structure!80!white
	}{exam}% then \pageref{exer:theoexample} references the theo

\newcommand{\myexample}[2][red]{
	% \tcb\tcbset{theostyle/.style={colframe=red,colbacktitle=yellow}}
	\begin{myexam}{}{}
		\raggedright
		#2
	\end{myexam}
	% \tcbset{colframe=structure,colbacktitle=structure}
}

\newtcbtheorem{myexer}{Exercise}%
{
	enhanced,
	colback=white,
	top=0.375cm,
	bottom=0.25cm,
	left=0.375cm,
	right=0.375cm,
	colframe=structure,
	fonttitle=\bfseries,
	drop fuzzy shadow,
	%description font=\mdseries\itshape,
	attach boxed title to top left={yshift=-2mm, xshift=5mm},
	colbacktitle=structure!80!white
	}{exer}

\newcommand{\myexercise}[2][red]{
	% \tcb\tcbset{theostyle/.style={colframe=red,colbacktitle=yellow}}
	\begin{myexer}{}{}
		\raggedright
		#2
	\end{myexer}
	% \tcbset{colframe=structure,colbacktitle=structure}
}

\input{../../Includes/definedColors}
% override specific chktex warnings
% chktex-file 46 - don't use $ instead of \(, etc)
% chktex-file 36 - don't require space in front of parenthesis
% chktex-file 37 - don't require space in front of parenthesis
% chktex-file 26 - don't require space in front of punctuation
% chktex-file 1 - ignore commands followed by a space, e.g. \\ new line here
% chktex-file 9 - sometimes messes up with ( and {

\begin{comment}
Shadings are useful to give the illusion of 3D in examples and exercises presented to engineering technology students.
Vertical and horizontal shadings of rectangles are fairly straightforward to produce with the shading library included in
a recent build of Tikz.
Rotation of shaded squares is also intuitive, but rotation of a shaded rectangle appears to be both a function of the specified
rotation angle and the length to width ratio of the rectangle. This makes aligning the shading of a rotated rectangle's
fill with the stroke of a rotated rectangle a bit of an inelegant trial-and-error exercise (for me, at any rate).


\end{comment}

%http://tex.stackexchange.com/questions/33703/extract-x-y-coordinate-of-an-arbitrary-point-in-tikz
\makeatletter
\providecommand{\gettikzxy}[3]{%
	\tikz@scan@one@point\pgfutil@firstofone#1\relax
	\edef#2{\the\pgf@x}%
	\edef#3{\the\pgf@y}%
}
\makeatother


%%%%%%%%%%%%%%%%%%%%%%%%%%%%%%%% A CLASS FOR ROTATED RECTANGLES WITH A SHADED FILL %%%%%%%%%%%%%%%%%%%%%%%%%%%%%%%%%%%%%%%%
\pgfooclass{rrect}{
	% the following should be set in the calling program: \hi, \radii, \extend
	% Ax, Ay, Bx, By, outershade, innershade
	\method rrect(#1,#2,#3,#4,#5,#6) { % The constructor; everything is done in here
		\def\Ax{#1} \def\Ay{#2} \def\Bx{#3} \def\By{#4} \def\outercolor{#5} \def\innercolor{#6}
		\pgfmathparse{\Bx-\Ax} \let\deltaX\pgfmathresult
		\pgfmathparse{\By-\Ay} \let\deltaY\pgfmathresult
		\ifthenelse{\equal{\deltaX}{0.0}}
		{	% vertical rod is a special case; otherwise atan gets a div by 0 error
			\pgfmathparse{\By>\Ay} \let\ccw\pgfmathresult
			\ifthenelse{\equal{\ccw}{1}}{%
				\def\rot{90}}
			{\def\rot{-90}}}
	{	% not vertical
		\pgfmathparse{\Ax<\Bx} \let\iseast\pgfmathresult
		\ifthenelse{\equal{\iseast}{1}}{%
			\pgfmathparse{atan(\deltaY/\deltaX)} \let\rot\pgfmathresult
		} % end is east
		{
			\pgfmathparse{180+atan(\deltaY/\deltaX)} \let\rot\pgfmathresult
		} % end !east
		}
		%shading boundaries work for vertical and horizontal but otherwise ``spills'' outside it supposed boundaries,
		%particularly at multiples of 45deg
		%make some adjustments from a max at 45 to nothing at 0 or 90
		\pgfmathparse{abs(\deltaY)} \let\absdeltaY\pgfmathresult
		\pgfmathparse{abs(\deltaX)} \let\absdeltaX\pgfmathresult
		%\def\shadeangle{-42}
		\ifthenelse{\equal{\deltaX}{0.0}}
		{\def\shadeangle{0.0}}
		{\pgfmathparse{\absdeltaY > \absdeltaX} \let\foo\pgfmathresult
			\ifthenelse{\equal{\foo}{1}}
			{\pgfmathparse{90-atan(\absdeltaY/\absdeltaX)} \let\shadeangle\pgfmathresult}
			{\pgfmathparse{atan(\absdeltaY/\absdeltaX)} \let\shadeangle\pgfmathresult}
		}
		\pgfmathparse{tan(\shadeangle)} \let\fudge\pgfmathresult
		\pgfmathparse{veclen(\deltaX,\deltaY)} \let\len\pgfmathresult
		\pgfmathparse{max(\hi,\len+2*\extend)} \let\shadeboxside\pgfmathresult
		\pgfmathparse{50-25/\shadeboxside*\hi+8*\hi*\fudge/\shadeboxside} \let\mybot\pgfmathresult
		\pgfmathparse{50+25/\shadeboxside*\hi-8*\hi*\fudge/\shadeboxside} \let\mytop\pgfmathresult
		\pgfdeclareverticalshading{myshade}{100bp}{%
			color(0bp)=(\outercolor);
			color(\mybot bp)=(\outercolor);
			color(50 bp)=(\innercolor);
			color(\mytop bp)=(\outercolor);
			color(100bp)=(\outercolor)}
		\tikzset{shading=myshade}
		\begin{scope}	[rotate around = {\rot: (\Ax, \Ay)}]
			\begin{scope}
				\draw[clip, rounded corners = \scale*\radii cm] (\Ax-\extend,\Ay-\hi/2) rectangle + (\len+2*\extend,\hi);
				\shade[ shading angle=\rot] (\Ax-\extend,\Ay-\shadeboxside/2) rectangle +(\shadeboxside, \shadeboxside);
			\end{scope} %end clipping
			\draw[rounded corners=\scale*\radii cm, \stroke, \thickness] (\Ax-\extend,\Ay-\hi/2) rectangle +(\len+2*\extend,\hi);
		\end{scope}
		} % end of constructor
		} % end of rrect class

		\pgfooclass{rr}{
			\method rr (#1,#2,#3,#4,#5) { % The constructor; everything is done in here
				% Here I can get named x and y coordinates
				\def\phil{#3} \def\stroke{#4} \def\line{#5}
				\gettikzxy{(#1)}{\spx}{\spy}
				\gettikzxy{(#2)}{\epx}{\epy}
				% I'd like named points to work with
				\coordinate (Start) at (\spx, \spy);
				\coordinate (End) at (\epx, \epy);
				% Find the length between start and end. Then the angle between x axis and Diff will be the rotation to apply.
				\coordinate (Diff) at ($ (End)-(Start) $);
				\gettikzxy{(Diff)}{\dx}{\dy}
				\pgfmathparse{veclen(\dx, \dy)} \pgfmathresult
				\let\length\pgfmathresult
				\pgfmathparse{\dx==0}%
				% \ifnum low-level TeX for integers
				\ifnum\pgfmathresult=1 % \dx == 0
					\pgfmathsetmacro{\rot}{\dy > 0 ? 90 : -90}
				\else% \dx != 0
					\pgfmathsetmacro{\rot}{\dx > 0 ? atan(\dy /\dx) : 180 + atan(\dy / \dx)}
				\fi
				\begin{scope}	[rotate around = {\rot:(\spx, \spy )}]
					% \filldraw[ultra thick, fill=\phil, draw=\stroke] ($ (Start)+(0,\hi) $) arc(90:270:\hi) -- +(\length pt, 0) arc(-90:90:\hi) -- cycle;
					\filldraw[rounded corners=\scale*\radii cm, line width=\line mm, fill=\phil, draw=\stroke] (\spx-\extend cm,\spy-\hi cm) rectangle +(2*\extend cm + \length pt, 2*\hi cm);
				\end{scope}
			}
		}

		\pgfooclass{beam}{
			\method beam(#1,#2,#3,#4,#5) { % The constructor; everything is done in here
				% Here I can get named x and y coordinates
				\def\phil{#3} \def\stroke{#4} \def\line{#5}
				\gettikzxy{(#1)}{\spx}{\spy}
				\gettikzxy{(#2)}{\epx}{\epy}
				% I'd like named points to work with
				\coordinate (Start) at (\spx, \spy);
				\coordinate (End) at (\epx, \epy);
				% Find the length between start and end. Then the angle between x axis and Diff will be the rotation to apply.
				\coordinate (Diff) at ($ (End)-(Start) $);
				\gettikzxy{(Diff)}{\dx}{\dy}
				\pgfmathparse{veclen(\dx, \dy)} \pgfmathresult
				\let\length\pgfmathresult
				\pgfmathparse{\dx==0}%
				% \ifnum low-level TeX for integers
				\ifnum\pgfmathresult=1 % \dx == 0
					\pgfmathsetmacro{\rot}{\dy > 0 ? 90 : -90}
				\else% \dx != 0
					\pgfmathsetmacro{\rot}{\dx > 0 ? atan(\dy / \dx) : 180 + atan(\dy / \dx)}
				\fi
				\begin{scope}	[rotate around = {\rot:(\spx, \spy )}]
					\fill[\phil] (\spx-\extend cm,\spy-\hi cm) rectangle +(2*\extend cm + \length pt, 2*\hi cm);
					\draw[draw=\stroke, line width=\line mm] (\spx-\extend cm,\spy-\hi cm) -- +(2*\extend cm + \length pt, 0);
					\draw[draw=\stroke, line width=\line mm] (\spx-\extend cm,\spy+\hi cm) -- +(2*\extend cm + \length pt, 0);
				\end{scope}
			}
		}


\usefonttheme[onlymath]{serif}

\usepackage[absolute,overlay]{textpos}
\setlength{\TPHorizModule}{1.0cm}
\setlength{\TPVertModule}{\TPHorizModule}
\textblockorigin{0.0cm}{0.0cm}  %start all at upper left corner
\usepackage{hyperref}
\hypersetup{colorlinks=true, urlcolor=structure}
% \hypersetup{urlcolor=Blue4}

\setlength{\parskip}{\medskipamount}
\setlength{\parindent}{0pt}

\usetheme{Antibes}

\usecolortheme[rgb={0, 0.6,0.6}]{structure}
% \definecolor{structurecolor}{rgb}{0.55,0.53,0.31}
\setbeamertemplate{items}[triangle]
% \setbeamertemplate{blocks}[rounded][shadow=false]
%\setbeamertemplate{background canvas}[vertical shading][bottom=Cyan1!50, middle=white, top=white, midpoint=0.05]
\setbeamertemplate{headline}{\vspace{.05cm}}
\setbeamertemplate{footline}{\textcolor{black}{ \hfill \insertshorttitle \quad
	\insertshortsubtitle
	\quad \insertframenumber/\inserttotalframenumber \quad{ }\vspace{0.125cm}}}
\addtobeamertemplate{footline}{\hypersetup{linkcolor=.}}{}
\setbeamertemplate{navigation symbols}{} % empty braces suppresses all navigation symbols
\setbeamercolor{frametitle}{fg=gray!5!white}
% \setbeamercolor{footline}{fg=black}
\setbeamercolor{block title}{fg=gray!15!white,bg=structure}
\setbeamercolor{block body}{bg=white, fg=black}
\setbeamercolor{background canvas}{bg=gray!20!white}
\setbeamersize{text margin left = 1cm, text margin right=1cm}
%\useinnertheme[shadow]{rounded}
%\raggedright
\setbeamerfont{block title}{family=mathserif}

\everymath{\displaystyle}
\newcounter{itemcount}

\logo{\pgfputat{\pgfxy(-11.85,-0.5)}{\pgfbox[right,base]{\includegraphics[height=1cm]{../../figs/rb_logo}}}}


\resetcounteronoverlays{tcb@cnt@myexam}
\resetcounteronoverlays{tcb@cnt@myexer}


% itemize indent
\setlength{\leftmargini}{1.5em}
\def\scale{1} % initialisation for pikz
\raggedright

%%%%%%%%%%%%%%%%%%%%%%%%%%%%%%%%%%%%%%%%%%%%%%%%%%%%%%%%%%%%%%%%%%%%%%%%%%%%%%%%%%%%%%%%%%%%%%%%%%%%%%%%%%%%%%%%%%%%%%%%
\title[Static Fluids]{\Huge \textcolor{white}{02 --- Forces Due to Static Fluids}}
\subtitle[CIVL318]{\Large\textcolor{white}{Water Resources, CIVL318}}
\author{}
\institute{}
\date{Last revision on \today}



%%%%%%%%%%%%%%%%%%%%%%%%%%%%%%%%%%%%%%%%%%%%%%%%%%%%%%%%%%%%%%%%%%%%%%%%%%%%%%%%%%%%%%%%%%%%%%%%%%%%%%%%%%%%%%%%%%%%%%%%%%%
\begin{document}

\begin{frame}[plain]    %don't need footer on titlepage
	\titlepage
\end{frame}

%%%%%%%%%%%%%%%%%%%%%%%%%%%%%%%%%%%%%%%%%%%%%%%%%%%%%%%%%%%%%%%%%%%%%%%%%%%%%%%%%%%%%%%%%%%%%%%%%%%%%%%%%%%%%%%%%%%%%%%%%%%

\begin{frame}{Forces on Horizontal Plane Areas}
	\begin{itemize}
		\item Pressure on a horizontal flat plane area due to a static fluid is
		      uniform over the plane area since the whole plane area is at the same depth\par\bigskip
		      \pause
		\item The force on the horizontal flat plane area is given by $F=pA$, where $p$ is the
		      uniform pressure and $A$ is the area of the plane area (this follows from $p=F/A$,
		      the definition for pressure).
	\end{itemize}
\end{frame}

%%%%%%%%%%%%%%%%%%%%%%%%%%%%%%%%%%%%%%%%%%%%%%%%%%%%%%%%%%%%%%%%


\begin{frame}
	\definecolor{example}{RGB}{133,205,235}
	\cmini[0.6]{
		\begin{mybox}[colframe=example!85!black]
			\begin{cfig}[0.4]{../../figs/02StaticForces/03staticforces01}\end{cfig}
		\end{mybox}
	}
	\cmini{
		\begin{myexam}[colframe=example!80!black, colbacktitle=example]{}{}
			Determine the force exerted by the oil and water upon the bottom plane
			surface of the barrel.
		\end{myexam}
	}
\end{frame}



%%%%%%%%%%%%%%%%%%%%%%%%%%%%%%%%%%%%%%%%%%%%%%%%%%%%%%%%%%%%%%%%%%%%%%%%%%%%%%%%%%%%%%%%%%%%%%%%%%%%%%%%%%%%%%%%%%%%%%%

\begin{frame}
	\cmini[0.9]{
		\definecolor{example}{RGB}{168,19,168}
		\begin{mybox}[colframe=example]
			\begin{cfig}[0.45]{../../figs/02StaticForces/03staticforces02}\end{cfig}
		\end{mybox}
		\parb
		
		\begin{myexam}[colframe=example, colbacktitle=example!80!white]{}{}
			A pressurized tank contains liquid with a specific gravity of $1.59$.\\
			The inspection hatch at \emph{A} has dimensions $400\, \text{mm}\times 250\, \text{mm}$.\\
			The access hatch at \emph{B} has dimensions $500\, \text{mm}\times 750\, \text{mm}$.\par\medskip
			
			Determine the force exerted by the fluid on the hatch at \emph{A}.
		\end{myexam}
	}
\end{frame}

%%%%%%%%%%%%%%%%%%%%%%%%%%%%%%%%%%%%%%%%%%%%%%%%%%%%%%%%%%%%%%%%%%%%%%%%%%%%%%%%%%%%%%%%%%%%%%%%%%%%%%%%%%%%%%%%%%%%%%%

\begin{frame}
	\cmini[0.9]{
		\definecolor{example}{RGB}{168,19,168}
		\begin{mybox}[colframe=example]
			\begin{cfig}[0.45]{../../figs/02StaticForces/03staticforces02}\end{cfig}
		\end{mybox}
		\parb
		
		\begin{myexer}[colframe=example, colbacktitle=example!80!white]{}{}
			A pressurized tank contains liquid with a specific gravity of $1.59$.\\
			The inspection hatch at \emph{A} has dimensions $400\, \text{mm}\times 250\, \text{mm}$.\\
			The access hatch at \emph{B} has dimensions $500\, \text{mm}\times 750\, \text{mm}$.\par\medskip
			
			Determine the force exerted by the fluid on the hatch at \emph{B}.
		\end{myexer}
	}
\end{frame}

%%%%%%%%%%%%%%%%%%%%%%%%%%%%%%%%%%%%%%%%%%%%%%%%%%%%%%%%%%%%%%%%%%%%%%%%%%%%%%%%%%%%%%%%%%%%%%%%%%%%%%%%%%%%%%%%%%%%%%%%%%%

%\begin{frame}
%\begin{cfig}[0.6]{../../figs/02StaticForces/03staticforces02}\end{cfig}
%\end{frame}

%%%%%%%%%%%%%%%%%%%%%%%%%%%%%%%%%%%%%%%%%%%%%%%%%%%%%%%%%%%%%%%%%%%%%%%%%%%%%%%%%%%%%%%%%%%%%%%%%%%%%%%%%%%%%%%%%%%%%%%%%%%

\begin{frame}{Forces on Vertical Rectangular Plane Areas}
	\begin{columns}
		\begin{column}[c]{0.38\textwidth}
			\only<1>{
				\begin{cfig}[0.5]{../../figs/02StaticForces/03staticforces03a}\end{cfig}
				}\only<2>{
				\begin{cfig}[0.5]{../../figs/02StaticForces/03staticforces03b}\end{cfig}
				}\only<3>{
				\begin{cfig}[0.5]{../../figs/02StaticForces/03staticforces03c}\end{cfig}
				}\only<4>{
				\begin{cfig}[0.5]{../../figs/02StaticForces/03staticforces03d}\end{cfig}
				}\only<5-6>{
				\begin{cfig}[0.5]{../../figs/02StaticForces/03staticforces03e}\end{cfig}
				}\only<7>{
				\begin{cfig}[0.5]{../../Figs/02StaticForces/03staticforces03f}\end{cfig}
			}
		\end{column}
		\begin{column}[c]{0.58\textwidth}
			
			\begin{itemize}
				\item<1-5> Consider a static volume of water retained by a vertical wall or dam.
				\item<2-5> Water pressure on the wall is 0 at the surface.
				\item<3-5> Water pressure is at a maximum at the bottom of the volume of water
				and can be calculated from $\Delta p=\gamma h$.
				\item<4-5> Water pressure is proportional to the depth (since $\Delta p=\gamma h$,
				it follows that $\Delta p\propto h$).
				\item<5> The average pressure is at half-depth (that is, at the depth, $h_C$, of the horizontal centroidal axis of
				the rectangular area).
				\vspace{-0.25cm}
				
				\[	P_{avg}  =  \gamma\cdot h_C   \]
				
				
			\end{itemize}
			\vspace{-6.5cm}
			\begin{itemize}
				\item<6-> The magnitude of the resultant force exerted on the wall by the water is
				\begin{eqnarray*}
					F_R & = & P_{avg}A\\
					& = & \gamma h_C A
				\end{eqnarray*}
				where $A$ is the area of the rectangular plane area.
				\item<7> The resultant of the force, $F_R$, acts through the centroid of the pressure triangle,
				which is at a depth of $2h/3$ in this case. \pars
				(This is true because the plane area is a rectangle and the area reaches the surface; it is not
				true in general.)
			\end{itemize}
		\end{column}
	\end{columns}
\end{frame}

%%%%%%%%%%%%%%%%%%%%%%%%%%%%%%%%%%%%%%%%%%%%%%%%%%%%%%%%%%%%%%%%%%%%%%%%%%%%%%%%%%%%%%%%%%%%%%%%%%%%%%%%%%%%%%%%%%%%%%%%%%%

\begin{frame}{Flumes}
	\begin{cfig}[0.3]{../../figs/02StaticForces/flume_3trim}\end{cfig}
	\begin{center}
		A \textbf{flume} is a man-made channel used to transport a liquid.
		
		It is often an elevated wooden box-like structure.
	\end{center}
\end{frame}

%%%%%%%%%%%%%%%%%%%%%%%%%%%%%%%%%%%%%%%%%%%%%%%%%%%%%%%%%%%%%%%%%%%%%%%%%%%%%%%%%%%%%%%%%%%%%%%%%%%%%%%%%%%%%%%%%%%%%%%%%%%

\begin{frame}{Flumes}
	\begin{cfig}[0.2]{../../figs/02StaticForces/flume_9}\end{cfig}
	\begin{center}
		This flume was built in the $1800$s.\pars
		
		{\tiny (It is north of Reno, Nevada, on Highway 395)}
	\end{center}
	
\end{frame}



%%%%%%%%%%%%%%%%%%%%%%%%%%%%%%%%%%%%%%%%%%%%%%%%%%%%%%%%%%%%%%%%%%%%%%%%%%%%%%%%%%%%%%%%%%%%%%%%%%%%%%%%%%%%%%%%%%%%%%%%%%%

\begin{frame}{Flumes}
	\begin{cfig}[0.2]{../../figs/02StaticForces/flume_1}\end{cfig}
	\begin{center}
		It transports $55$ million gallons per day, or mgd,  $\left(\approx 200,000 \text{ m}^3/\text{day}\right)$ of water,
		providing hydro-electricity and water to the residents of Reno, Nevada.
	\end{center}
\end{frame}

%%%%%%%%%%%%%%%%%%%%%%%%%%%%%%%%%%%%%%%%%%%%%%%%%%%%%%%%%%%%%%%%%%%%%%%%%%%%%%%%%%%%%%%%%%%%%%%%%%%%%%%%%%%%%%%%%%%%%%%%%%%

\begin{frame}{Flumes}
	\begin{cfig}[0.18]{../../figs/02StaticForces/flume_2}\end{cfig}
	\begin{center}
		A $60$ m section the flume was damaged by a 4.7 magnitude earthquake on \lb April 25th, 2008; part of the repaired section is
		visible on the left.
	\end{center}
\end{frame}

%%%%%%%%%%%%%%%%%%%%%%%%%%%%%%%%%%%%%%%%%%%%%%%%%%%%%%%%%%%%%%%%%%%%%%%%%%%%%%%%%%%%%%%%%%%%%%%%%%%%%%%%%%%%%%%%%%%%%%%%%%%

\begin{frame}
	\centering
	\definecolor{example}{RGB}{204,172,83}
	\begin{mybox}[colframe=example, width=8cm]
		\vspace{-0.5cm}
		\begin{cfig}[0.55]{../../figs/02StaticForces/03staticforcesEx3}\end{cfig}
	\end{mybox}
	\begin{myexam}[colframe=example, colbacktitle=example!80!white]{}{}
		\raggedright
		The tie-bars have cross-sectional dimension of $90\text{mm}\times 90\text{mm}$.
		Determine the maximum normal stress in the tie-bars, and the bearing stress
		if the vertical posts are cut halfway into each tie-bar.		\par
		(Assume a pinned connection at the bottom of the sidewalls and treat pressure as though the fluid is static.)
		
	\end{myexam}
\end{frame}



%%%%%%%%%%%%%%%%%%%%%%%%%%%%%%%%%%%%%%%%%%%%%%%%%%%%%%%%%%%%%%%%%%%%%%%%%%%%%%%%%%%%%%%%%%%%%%%%%%%%%%%%%%%%%%%%%%%%%%%%%%%

\begin{frame}{Forces on Inclined Rectangular Plane Areas}
	\vspace{-0.75cm}
	\begin{cfig}[0.55]{../../figs/02StaticForces/03staticforces04}\end{cfig}
	\vspace{-0.45cm}
	The same argument used for vertical rectangular plane areas applies
	to inclined regular plane areas and, again,
	\vspace{-0.25cm}
	\[ p_{avg}=\gamma\cdot h_C,\quad F_{R}=\gamma\cdot	h_C\cdot A \]
	\vspace{-0.25cm}
	The direction of the pressure remains \textbf{perpendicular} to the plane area
	
\end{frame}

%%%%%%%%%%%%%%%%%%%%%%%%%%%%%%%%%%%%%%%%%%%%%%%%%%%%%%%%%%%%%%%%%%%%%%%%%%%%%%%%%%%%%%%%%%%%%%%%%%%%%%%%%%%%%%%%%%%%%%%%%%%

\begin{frame}%{Forces on Inclined Rectangular Plane Areas}
	\cmini{
		\begin{mybox}
			\begin{cfig}[0.6]{../../figs/02StaticForces/03staticforcesEx04}\end{cfig}
		\end{mybox}
	}
	\begin{myexam}{}{}
		The wall has a rectangular plane area in contact with the water, is
		inclined at $60^\circ$ to the horizontal and is $17$ m long.
		\par\smallskip
		Determine the force exerted on the dam plane area by the water.
	\end{myexam}
\end{frame}



%%%%%%%%%%%%%%%%%%%%%%%%%%%%%%%%%%%%%%%%%%%%%%%%%%%%%%%%%%%%%%%%%%%%%%%%%%%%%%%%%%%%%%%%%%%%%%%%%%%%%%%%%%%%%%%%%%%%%%%%%%%

\begin{frame}{Forces on Submerged Rectangles}
	\begin{columns}
		\begin{column}{0.45\textwidth}
			\only<1>{
				\begin{cfig}[0.5]{../../figs/02StaticForces/03staticforces06a}\end{cfig}
				}\only<2>{
				\begin{cfig}[0.5]{../../figs/02StaticForces/03staticforces06b}\end{cfig}
				}\only<3>{
				\begin{cfig}[0.5]{../../figs/02StaticForces/03staticforces06c}\end{cfig}
				}\only<4>{
				\begin{cfig}[0.5]{../../figs/02StaticForces/03staticforces06d}\end{cfig}
			}
		\end{column}
		\begin{column}{0.55\textwidth}
			\begin{itemize}
				\item Find the force on a submerged rectangular plane area \pause
				\item The pressure triangle is as before, except the pressure isn't 0 at
				      the top of the rectangle so our previous reasoning no longer holds \pause
				\item We are only interested in the pressure that the fluid exerts on the plane surface \pause
				\item The average pressure on the plane is at its mid-height (the horizontal centroidal axis of the rectangular
				      plane area, $h_C$), which can be easily calculated if the dimensions and location are known
			\end{itemize}
		\end{column}
	\end{columns}
\end{frame}

%%%%%%%%%%%%%%%%%%%%%%%%%%%%%%%%%%%%%%%%%%%%%%%%%%%%%%%%%%%%%%%%%%%%%%%%%%%%%%%%%%%%%%%%%%%%%%%%%%%%%%%%%%%%%%%%%%%%%%%%%%%

\begin{frame}%{Forces on Submerged Rectangle (Exercise)}
	\cmini{
		\begin{myexam}{}{}
			\raggedright
			A vertical retaining wall supports water to a depth of 4.75 m. There
			is a rectangular hatch in the wall. The top of the hatch is at a depth of 1.25~m; the hatch is 2.25~m wide
			$\times$ 1.5~m high.
			\parb
			What is the magnitude of the force exerted upon the hatch by the water?
		\end{myexam}
	}
\end{frame}

%%%%%%%%%%%%%%%%%%%%%%%%%%%%%%%%%%%%%%%%%%%%%%%%%%%%%%%%%%%%%%%%%%%%%%%%%%%%%%%%%%%%%%%%%%%%%%%%%%%%%%%%%%%%%%%%%%%%%%%%%%%

\begin{frame}%{Forces on Submerged Rectangle (Exercise)}
	\cmini{
		
		\begin{myexer}{}{}
			\raggedright
			A vertical retaining wall supports water to a depth of 6.25 m. There
			is a rectangular hatch in the wall.
			
			The hatch has dimensions $3.75\,\text{m wide}\times 1.6\,\text{m high}$.
			\parb
			At what depth below the surface can the top of this hatch be placed if
			the maximum allowable force on the hatch is 128 kN?
		\end{myexer}
	}
\end{frame}

%%%%%%%%%%%%%%%%%%%%%%%%%%%%%%%%%%%%%%%%%%%%%%%%%%%%%%%%%%%%%%%%%%%%%%%%%%%%%%%%%%%%%%%%%%%%%%%%%%%%%%%%%%%%%%%%%%%%%%%%%%%

\begin{frame}{Average Pressure on a Submerged Plane Area}
	\begin{columns}
		\begin{column}{0.45\textwidth}
			\only<1>{
				\begin{cfig}[0.5]{../../figs/02StaticForces/03staticforces08a}\end{cfig}
				}\only<2>{
				\begin{cfig}[0.5]{../../figs/02StaticForces/03staticforces08a2}\end{cfig}
				}\only<3>{
				\begin{cfig}[0.5]{../../figs/02StaticForces/03staticforces08b}\end{cfig}
				}\only<4>{
				\begin{cfig}[0.5]{../../figs/02StaticForces/03staticforces08c}\end{cfig}
			}
		\end{column}
		\begin{column}{0.5\textwidth}
			\begin{itemize}
				\item The average pressure acting on a submerged rectangular plane area is the pressure that acts at its mid-height. \pause
				\item This is not true for all shapes; it is true for a rectangle because that is where the centroidal $x$-axis is located.  \pause
				\item For a triangular plane area, $p_{avg}$ is the pressure at the   centroidal $x$-axis of the triangle.  \pause
				\item For all plane areas, $p_{avg}$ is the pressure at the   centroidal $x$-axis of the area.
			\end{itemize}
		\end{column}
	\end{columns}
	
\end{frame}

%%%%%%%%%%%%%%%%%%%%%%%%%%%%%%%%%%%%%%%%%%%%%%%%%%%%%%%%%%%%%%%%%%%%%%%%%%%%%%%%%%%%%%%%%%%%%%%%%%%%%%%%%%%%%%%%%%%%%%%%%%%
\begin{frame}{Average Pressure a Submerged Plane Area}
	\centering
	\begin{cfig}[0.55]{../../figs/02StaticForces/03staticforces08c2}\end{cfig}
	\begin{mybox}[width=10cm, title=Average Pressure on a Submerged Plane Area]
		
		\[\bm{ p_{avg} = \gamma h_c }\]
		\centering
		where $h_c$ is the depth of the centroidal $x$-axis of the shape
		
	\end{mybox}
\end{frame}

%%%%%%%%%%%%%%%%%%%%%%%%%%%%%%%%%%%%%%%%%%%%%%%%%%%%%%%%%%%%%%%%%%%%%%%%%%%%%%%%%%%%%%%%%%%%%%%%%%%%%%%%%%%%%%%%%%%%%%%%%%%%%%


\begin{frame}
	\centering
	\definecolor{example}{RGB}{168,145,82}
	\begin{mybox}[width=9cm,colframe=example]
		\begin{cfig}[0.55]{../../figs/02StaticForces/03staticforcesEx06a}\end{cfig}
	\end{mybox}
	
	\begin{myexam}[width=10.875cm,colframe=example,colbacktitle=example!80!white]{}{}
		\raggedright
		A tank containing kerosene (sg=$0.823$) has a triangular inspection hatch in a vertical sidewall. The hatch has
		a base of $400\,\text{mm}$ and a height of $600\,\text{mm}$. The top of the hatch is located at a depth of $700\,\text{mm}$.
		\par\smallskip
		Determine the force exerted on the hatch by the kerosene.
	\end{myexam}
	
\end{frame}

%%%%%%%%%%%%%%%%%%%%%%%%%%%%%%%%%%%%%%%%%%%%%%%%%%%%%%%%%%%%%%%%%%%%%%%%%%%%%%%%%%%%%%%%%%%%%%%%%%%%%%%%%%%%%%%%%%%%%%%%

\begin{frame}
	\centering
	\begin{mybox}[width=7cm]
		\begin{cfig}[0.25]{../../figs/02StaticForces/03staticforcesEx07a}\end{cfig}
	\end{mybox}
	
	\begin{myexam}[width=10cm]{}{}
		A water tank has a semi-circular inspection hatch, as illustrated.
		\par\smallskip
		Determine the force exerted on the hatch by the water. \par
		\begin{center}
			(For a semi-circle, $\bar{y}=\frac{4r}{3\pi}$, measured from the `diameter'.)
		\end{center}
	\end{myexam}
	
\end{frame}


%%%%%%%%%%%%%%%%%%%%%%%%%%%%%%%%%%%%%%%%%%%%%%%%%%%%%%%%%%%%%%%%%%%%%%%%%%%%%%%%%%%%%%%%%%%%%%%%%%%%%%%%%%%%%%%%%%%%%%%%

\begin{frame}{Centre of Pressure}
	\begin{columns}
		\begin{column}{0.45\textwidth}
			\only<1>{
				\begin{mybox}
					\begin{cfig}[0.4]{../../figs/02StaticForces/03staticforces09a}\end{cfig}
				\end{mybox}
				}\only<2>{
				\begin{mybox}
					\begin{cfig}[0.7]{../../figs/02StaticForces/03staticforces09b}\end{cfig}
				\end{mybox}
				}\only<3>{
				\begin{mybox}
					\begin{cfig}[0.7]{../../figs/02StaticForces/03staticforces09c}\end{cfig}
				\end{mybox}
				}\only<4>{
				\begin{mybox}
					\begin{cfig}[0.7]{../../figs/02StaticForces/03staticforces09d}\end{cfig}
				\end{mybox}
				}\only<5>{
				\begin{mybox}
					\begin{cfig}[0.7]{../../figs/02StaticForces/03staticforces09e}\end{cfig}
				\end{mybox}
			}
		\end{column}
		\begin{column}{0.55\textwidth}
			\begin{itemize}
				\item The centre of pressure (the location where the resultant force can be considered to act) for a rectangular plane area reaching the surface is at two-thirds depth \pause
				\item But where is the centre of pressure for a submerged rectangular plane area?  \pause
				\item Using techniques learned in statics, we can break to pressure down into a uniformly
				      distributed force and a uniformly varying force...  \pause
				\item ... and then calculate two forces \pause
				\item This is complicated - and only works for rectangular areas!
			\end{itemize}
		\end{column}
	\end{columns}
	
\end{frame}

%%%%%%%%%%%%%%%%%%%%%%%%%%%%%%%%%%%%%%%%%%%%%%%%%%%%%%%%%%%%%%%%%%%%%%%%%%%%%%%%%%%%%%%%%%%%%%%%%%%%%%%%%%%%%%%%%%%%%%%%

\begin{frame}{Centre of Pressure}
	\begin{columns}
		\begin{column}{0.5\textwidth}
			\begin{mybox}
				\begin{cfig}[0.56]{../../figs/02StaticForces/03staticforces12}\end{cfig}
			\end{mybox}
		\end{column}
		\begin{column}{0.5\textwidth}
			\smallskip
			\begin{itemize}
				\item There is a formula that locates the centre of pressure more readily.
				\item Its use is not restricted to rectangles.
				\item $L_c$ is the length, {\bfseries parallel} to the plane surface, from the centroid of the plane  to the surface of the liquid
				\item $A$ is the area of the plane
				\item $I_c$ is the moment of inertia of the plane about its horizontal centroidal axis
				\item Then,\pars
				      \cmini{
				      	\begin{mybox}
				      		\vspace{-0.25cm}
				      		\[\bm{ L_p - L_c =\frac{I_c}{L_c\cdot A} }\]
				      	\end{mybox}
				      }
			\end{itemize}
		\end{column}
	\end{columns}
	
\end{frame}



%%%%%%%%%%%%%%%%%%%%%%%%%%%%%%%%%%%%%%%%%%%%%%%%%%%%%%%%%%%%%%%%%%%%%%%%%%%%%%%%%%%%%%%%%%%%%%%%%%%%%%%%%%%%%%%%%%%%%%%%%%%


\begin{frame}
	% 	\setbeamercolor{block title}{bg=GreenYellow}
	\definecolor{example}{RGB}{149,149,47}
	\begin{columns}
		\begin{column}{0.45\textwidth}
			
			\begin{mybox}[colframe=example]
				\begin{cfig}[0.55]{../../figs/02StaticForces/03staticforcesEx09a}\end{cfig}
			\end{mybox}
		\end{column}
		\begin{column}{0.575\textwidth}
			\begin{myexam}[colbacktitle=example!80!white, colframe=example]{}{}
				\raggedright
				A tank containing castor oil has a $1.0\,\text{m wide}\times 600 \,\text{mm}$ high rectangular inspection hatch.\par\bigskip
				The top of the hatch is $2.5\,\text{m}$ below the surface of the castor oil. The hatch cover is attached to
				the tank by $8$ bolts, four at the top of the hatch and four at the bottom.\parm
				The bolts are offset the from the hatch opening by 	$100\,\text{mm}$, as shown.\parm
				Calculate the tension in each of the top and in each of the bottom bolts.\parm
				(Assume that all the top bolts have the same tension and that all the bottom bolts have the same tension.)
			\end{myexam}
		\end{column}
	\end{columns}
	
\end{frame}

%%%%%%%%%%%%%%%%%%%%%%%%%%%%%%%%%%%%%%%%%%%%%%%%%%%%%%%%%%%%%%%%%%%%%%%%%%%%%%%%%%%%%%%%%%%%%%%%%%%%%%%%%%%%%%%%%%%%%%%%

\begin{frame}
	\begin{columns}
		\begin{column}{0.6\textwidth}
			\begin{mybox}
				\begin{cfig}[0.6]{../../figs/02StaticForces/03staticforcesEx08a}\end{cfig}
			\end{mybox}
			
		\end{column}
		\begin{column}{0.4\textwidth}
			\begin{myexam}{}{}
				\raggedright
				This is an example of a ``self-levelling'' gate. It is hinged along its top edge \par\bigskip
				When the water exceeds a certain height, the hydrostatic force on the gate is sufficient to
				open it. Water drains until the level allows the gate to close again. \par\bigskip
				Find the value $d$ for which the gate opens.
			\end{myexam}
		\end{column}
	\end{columns}
	
\end{frame}


%%%%%%%%%%%%%%%%%%%%%%%%%%%%%%%%%%%%%%%%%%%%%%%%%%%%%%%%%%%%%%%%%%%%%%%%%%%%%%%%%%%%%%%%%%%%%%%%%%%%%%%%%%%%%%%%%%%%%%%%

\begin{frame}{Centre of Pressure}
	\begin{columns}
		\begin{column}{0.45\textwidth}
			\par\vspace{-1cm}
			\begin{cfig}[0.6]{../../figs/02StaticForces/03staticforces12b}\end{cfig}
		\end{column}
		\begin{column}{0.4\textwidth}
			\par\vspace{3cm}
			The formula works for inclined planes also:
			\[ L_p -L_c=\frac{I_c}{L_c\cdot A} \]
		\end{column}
	\end{columns}
	\begin{itemize}
		\item $L_c$ is the distance \textbf{along the slope of the plane surface} from the centroid of the plane to the surface of the liquid
		\item $L_p$ is the distance \textbf{along the slope of the plane surface} from the centre of pressure to the surface of the liquid
	\end{itemize}
\end{frame}

%%%%%%%%%%%%%%%%%%%%%%%%%%%%%%%%%%%%%%%%%%%%%%%%%%%%%%%%%%%%%%%%%%%%%%%%%%%%%%%%%%%%%%%%%%%%%%%%%%%%%%%%%%%%%%%%%%%%%%%%%%%

\begin{frame}
	% 	\setbeamercolor{block title}{bg=RawSienna}
	\definecolor{example}{RGB}{125,103,68}
	\begin{textblock*}{8cm} (4.5cm, 0.325cm)
		\begin{mybox}[colframe=example, colbacktitle=example!80!white]
			\vspace{-0.25cm}
			\begin{cfig}[0.65]{../../figs/02StaticForces/03staticforcesEx10}\end{cfig}
			\vspace{-0.5cm}
		\end{mybox}
	\end{textblock*}
	\begin{textblock*}{5.5cm} (1cm, 6.875cm)
		\begin{myexam}[colframe=example, colbacktitle=example!80!white]{}{}
			\raggedright
			Determine the tension $T$ in the upper bolt and tension $S$ in each of the lower bolts.
		\end{myexam}
	\end{textblock*}
\end{frame}

%%%%%%%%%%%%%%%%%%%%%%%%%%%%%%%%%%%%%%%%%%%%%%%%%%%%%%%%%%%%%%%%%%%%%%%%%%%%%%%%%%%%%%%%%%%%%%%%%%%%%%%%%%%%%%%%%%%%%%%%

\begin{frame}
	
	\begin{columns}
		\begin{column}{0.525\textwidth}
			\begin{mybox}
				\begin{cfig}[0.4]{../../figs/02StaticForces/03staticforcesEx11}\end{cfig}
			\end{mybox}
		\end{column}
		\begin{column}{0.475\textwidth}
			\begin{myexam}{}{}
				\raggedright
				A rectangular steel gate ($1.5\,\text{m}\times2.15\,\text{m}$) is used to regulate the level of a water storage pond. The gate has a weight of $1.9\,\text{kN}$ which can be thought of as acting through the centre of the gate.\parb
				Determine the force $T$ required to open the gate when the pond has a depth of $3.1\,\text{m}$.
			\end{myexam}
		\end{column}
	\end{columns}
	
\end{frame}



%%%%%%%%%%%%%%%%%%%%%%%%%%%%%%%%%%%%%%%%%%%%%%%%%%%%%%%%%%%%%%%%%%%%%%%%%%%%%%%%%%%%%%%%%%%%%%%%%%%%%%%%%%%%%%%%%%%%%%%%%%%\definecolor{example}{RGB}{125,103,68}

% \begin{mybox}[colframe=example, colbacktitle=example!80!white]
%   \vspace{-0.25cm}
%   \begin{cfig}[0.65]{../../figs/02StaticForces/03staticforcesEx10}\end{cfig}
%   \vspace{-0.5cm}
% \end{mybox}

\begin{frame}
	\definecolor{example}{RGB}{145,152,86}
	\begin{columns}
		\begin{column}{0.55\textwidth}
			\vspace{0.375cm}
			\begin{mybox}[colframe=example, colbacktitle=example!80!white]
				\begin{cfig}[0.35]{../../figs/02StaticForces/03staticforcesEx12}\end{cfig}
			\end{mybox}
		\end{column}
		\begin{column}{0.45\textwidth}
			\begin{myexam}[colframe=example, colbacktitle=example!80!white]{}{}
				\raggedright
				A $L$-shaped steel gate $ABC$ is used to regulate the level of a water storage pond.
				Both the vertical and the horizontal parts of the gate are rectangular with a width of $1.75\,\text{m}$.
				When the water level exceeds a certain level, the whole gate $ABC$ rotates clockwise about the frictionless hinge at $B$.
				\par \bigskip
				Determine the value of $h$ at which the gate begins to open.
			\end{myexam}
		\end{column}
	\end{columns}
	
\end{frame}



\end{document}
