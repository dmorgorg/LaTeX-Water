% override specific chktex warnings
% chktex-file 46 - don't use $ instead of \(, etc)
% chktex-file 1 - ignore commands followed by a space, e.g. \\ new line here
% chktex-file 9 - sometimes messes up with ( and {

\documentclass[9pt,xcolor={svgnames, x11names},professionalfonts, mathserif]{beamer}

\usepackage{amsmath}
\usepackage{amssymb}
\usepackage{graphicx}
\usepackage{booktabs}  % for top and bottom spacing in table cells
\usepackage{mathpazo}
\usepackage{textcomp}
\usepackage{multirow}
\usepackage{cancel}
\usepackage{array}
%\usepackage{enumerate}
% \usepackage{enumitem} %causes compile error, stack size exceeded?
\usepackage{gensymb} % for \degree
\usepackage[many]{tcolorbox}
\usepackage{verbatim}
\usepackage{bm}
\usepackage{graphicx}
\usepackage{tikz}
\usepackage{tkz-linknodes}
\usepackage[export]{adjustbox} % for tight borders around photos
\usepackage{pgf} % for sait logo in beamer
\usepackage{pgfmath}
\usepgfmodule{oo}
%\usetikzlibrary{shapes,decorations,shadows,calc}
\usetikzlibrary{shadows,calc,arrows.meta}
% \usetikzlibrary{decorations.shapes}
%\usetikzlibrary{shapes.callouts}
% bloody coils
\usetikzlibrary{decorations.pathmorphing}
\usetikzlibrary{shapes.multipart}

% counter for resuming enumerated list numbers
\newcounter{resumeenumi}
\newcommand{\suspend}{\setcounter{resumeenumi}{\theenumi}}
\newcommand{\resume}{\setcounter{enumi}{\theresumeenumi}}

\newcommand\lb{\linebreak}
\newcommand\pars{\par\smallskip}
\newcommand\parm{\par\medskip}
\newcommand\parb{\par\bigskip}

\makeatletter
\providecommand{\gettikzxy}[3]{%
	\tikz@scan@one@point\pgfutil@firstofone#1\relax
	\edef#2{\the\pgf@x}%
	\edef#3{\the\pgf@y}%
}
\makeatother



% full width colored block but color specifiable
%\cb[body bg strength]{header bg}{header text}{body text}
\newcommand{\cb}[4][15]{
	\setbeamercolor{block title}{bg = #2}
	\setbeamercolor{block body}{bg = #2!#1}
	\setbeamercolor{item projected}{bg=#2, fg=white}
	\begin{center}
		\begin{block}{#3}
			#4
		\end{block}
	\end{center}
}

% colored block with width specified
% \cbw[body bg strength]{header bg}{width}{header text}{body text}
\newcommand{\cbw}[5][15]{
	\begin{center}
		%\vspace{-0.35cm}
		\begin{minipage}{#3\textwidth}
			\setbeamercolor{block title}{bg= #2}
			\setbeamercolor{block body}{bg= #2!#1}
			\setbeamercolor{item projected}{bg=#2, fg=white}
			\begin{block}{#4}
				\raggedright
				#5
			\end{block}
		\end{minipage}
	\end{center}
}

% centered minipage with text \raggedright
%\cmini[width]{content}
\newcommand{\cmini}[2][0.8]{
	\begin{center}
		\begin{minipage}{#1\columnwidth}
			\raggedright
			#2
		\end{minipage}
	\end{center}
}

%left flushed minipage
\newcommand{\mini}[2][0.8]{
	\begin{minipage}{#1\columnwidth}
		\raggedright
		#2
	\end{minipage}
}

%left flushed minipage, top aligned
\newcommand{\minit}[2][0.8]{
	\begin{minipage}[t]{#1\columnwidth}
		\raggedright
		#2
	\end{minipage}
}

%left flushed minipage
% \newcommand{\miniT}[2][0.8]{
%  \begin{minipage}[T]{#1\columnwidth}
%   \raggedright
%   #2
%  \end{minipage}
% }

%left flushed minipage
\newcommand{\minib}[2][0.8]{
	\begin{minipage}[b]{#1\columnwidth}
		\raggedright
		#2
	\end{minipage}
}

\newcommand{\cfig}[2][1]{% centred, scaled graphic
	\begin{center}
		\includegraphics[scale=#1]{#2}
	\end{center}
}
% figure with tight border for photos
% \cfigb[saitMaroon]{borderwidth with unit}{scale}{image}
\newcommand{\cfigb}[4][structure]{
	% \usepackage{adjustbox}
	\setlength{\fboxrule}{1pt}
	\begin{center}
		\includegraphics[scale=#3, cframe= #1 #2]{#4}
	\end{center}
}

\newcommand{\imgbox}[3]{
	% \setlength{\fboxsep}{12pt}
	\includegraphics[scale=#1, cframe= structure #3]{#2}
}

% \imgboxbg[bg color=white]{scale}{path/to/img}{border color}{border, e.g. 2pt}{margin, e.g. 4pt}
\newcommand{\imgboxbg}[6][white]{
	\setlength{\fboxrule}{#5}
	\setlength{\fboxsep}{#6}
	\centering
	\fcolorbox{#4}{#1}{\includegraphics[scale=#2]{#3}}
}

\newcommand{\fig}[2][1]{% scaled graphic
	\includegraphics[scale=#1]{#2}
}

% centred framed  box black border
%\cbox[width]{content}
\newcommand{\cbox}[2][0.9]{% framed centered  box
	\setlength\fboxsep{0.042\columnwidth}
	\setlength\fboxrule{0.0015\columnwidth}
	\begin{center}
		\fcolorbox{black}{white}{
			\vspace{-0.5cm}
			\begin{minipage}{#1\columnwidth}
				\raggedright
				#2
			\end{minipage}
		}
	\end{center}
	\setlength\fboxsep{0cm}
}



\newtcolorbox{mybox}[1][]
{
	colback=white,
	top=0.25cm,
	bottom=0.25cm,
	left=0.25cm,
	right=0.25cm,
	colframe=structure,
	fonttitle=\bfseries,
	enhanced, drop fuzzy shadow,
	% attach boxed title to top left={yshift=-2mm, xshift=5mm},
	attach boxed title to top left={yshift=-2mm, xshift=5mm}, colbacktitle=structure!80!white, #1}

\newtcolorbox{plainbox}[1][]{colback=white, sharp corners, top=0.125cm, bottom=0.125cm, left=0pt, right=0pt, boxrule=0.5pt,colframe=structure,fonttitle=\bfseries, colbacktitle=structure, arc=0mm, #1}
%
\newtcbtheorem{myexam}{Example}%
{
	enhanced,
	colback=white,
	top=0.375cm,
	bottom=0.25cm,
	left=0.375cm,
	right=0.375cm,
	colframe=structure,
	fonttitle=\bfseries,
	drop fuzzy shadow,
	%description font=\mdseries\itshape,
	attach boxed title to top left={yshift=-2mm, xshift=5mm},
	colbacktitle=structure!80!white
	}{exam}% then \pageref{exer:theoexample} references the theo

\newcommand{\myexample}[2][red]{
	% \tcb\tcbset{theostyle/.style={colframe=red,colbacktitle=yellow}}
	\begin{myexam}{}{}
		\raggedright
		#2
	\end{myexam}
	% \tcbset{colframe=structure,colbacktitle=structure}
}

\newtcbtheorem{myexer}{Exercise}%
{
	enhanced,
	colback=white,
	top=0.375cm,
	bottom=0.25cm,
	left=0.375cm,
	right=0.375cm,
	colframe=structure,
	fonttitle=\bfseries,
	drop fuzzy shadow,
	%description font=\mdseries\itshape,
	attach boxed title to top left={yshift=-2mm, xshift=5mm},
	colbacktitle=structure!80!white
	}{exer}

\newcommand{\myexercise}[2][red]{
	% \tcb\tcbset{theostyle/.style={colframe=red,colbacktitle=yellow}}
	\begin{myexer}{}{}
		\raggedright
		#2
	\end{myexer}
	% \tcbset{colframe=structure,colbacktitle=structure}
}

\input{../../Includes/definedColors}
% override specific chktex warnings
% chktex-file 46 - don't use $ instead of \(, etc)
% chktex-file 36 - don't require space in front of parenthesis
% chktex-file 37 - don't require space in front of parenthesis
% chktex-file 26 - don't require space in front of punctuation
% chktex-file 1 - ignore commands followed by a space, e.g. \\ new line here
% chktex-file 9 - sometimes messes up with ( and {

\begin{comment}
Shadings are useful to give the illusion of 3D in examples and exercises presented to engineering technology students.
Vertical and horizontal shadings of rectangles are fairly straightforward to produce with the shading library included in
a recent build of Tikz.
Rotation of shaded squares is also intuitive, but rotation of a shaded rectangle appears to be both a function of the specified
rotation angle and the length to width ratio of the rectangle. This makes aligning the shading of a rotated rectangle's
fill with the stroke of a rotated rectangle a bit of an inelegant trial-and-error exercise (for me, at any rate).


\end{comment}

%http://tex.stackexchange.com/questions/33703/extract-x-y-coordinate-of-an-arbitrary-point-in-tikz
\makeatletter
\providecommand{\gettikzxy}[3]{%
	\tikz@scan@one@point\pgfutil@firstofone#1\relax
	\edef#2{\the\pgf@x}%
	\edef#3{\the\pgf@y}%
}
\makeatother


%%%%%%%%%%%%%%%%%%%%%%%%%%%%%%%% A CLASS FOR ROTATED RECTANGLES WITH A SHADED FILL %%%%%%%%%%%%%%%%%%%%%%%%%%%%%%%%%%%%%%%%
\pgfooclass{rrect}{
	% the following should be set in the calling program: \hi, \radii, \extend
	% Ax, Ay, Bx, By, outershade, innershade
	\method rrect(#1,#2,#3,#4,#5,#6) { % The constructor; everything is done in here
		\def\Ax{#1} \def\Ay{#2} \def\Bx{#3} \def\By{#4} \def\outercolor{#5} \def\innercolor{#6}
		\pgfmathparse{\Bx-\Ax} \let\deltaX\pgfmathresult
		\pgfmathparse{\By-\Ay} \let\deltaY\pgfmathresult
		\ifthenelse{\equal{\deltaX}{0.0}}
		{	% vertical rod is a special case; otherwise atan gets a div by 0 error
			\pgfmathparse{\By>\Ay} \let\ccw\pgfmathresult
			\ifthenelse{\equal{\ccw}{1}}{%
				\def\rot{90}}
			{\def\rot{-90}}}
	{	% not vertical
		\pgfmathparse{\Ax<\Bx} \let\iseast\pgfmathresult
		\ifthenelse{\equal{\iseast}{1}}{%
			\pgfmathparse{atan(\deltaY/\deltaX)} \let\rot\pgfmathresult
		} % end is east
		{
			\pgfmathparse{180+atan(\deltaY/\deltaX)} \let\rot\pgfmathresult
		} % end !east
		}
		%shading boundaries work for vertical and horizontal but otherwise ``spills'' outside it supposed boundaries,
		%particularly at multiples of 45deg
		%make some adjustments from a max at 45 to nothing at 0 or 90
		\pgfmathparse{abs(\deltaY)} \let\absdeltaY\pgfmathresult
		\pgfmathparse{abs(\deltaX)} \let\absdeltaX\pgfmathresult
		%\def\shadeangle{-42}
		\ifthenelse{\equal{\deltaX}{0.0}}
		{\def\shadeangle{0.0}}
		{\pgfmathparse{\absdeltaY > \absdeltaX} \let\foo\pgfmathresult
			\ifthenelse{\equal{\foo}{1}}
			{\pgfmathparse{90-atan(\absdeltaY/\absdeltaX)} \let\shadeangle\pgfmathresult}
			{\pgfmathparse{atan(\absdeltaY/\absdeltaX)} \let\shadeangle\pgfmathresult}
		}
		\pgfmathparse{tan(\shadeangle)} \let\fudge\pgfmathresult
		\pgfmathparse{veclen(\deltaX,\deltaY)} \let\len\pgfmathresult
		\pgfmathparse{max(\hi,\len+2*\extend)} \let\shadeboxside\pgfmathresult
		\pgfmathparse{50-25/\shadeboxside*\hi+8*\hi*\fudge/\shadeboxside} \let\mybot\pgfmathresult
		\pgfmathparse{50+25/\shadeboxside*\hi-8*\hi*\fudge/\shadeboxside} \let\mytop\pgfmathresult
		\pgfdeclareverticalshading{myshade}{100bp}{%
			color(0bp)=(\outercolor);
			color(\mybot bp)=(\outercolor);
			color(50 bp)=(\innercolor);
			color(\mytop bp)=(\outercolor);
			color(100bp)=(\outercolor)}
		\tikzset{shading=myshade}
		\begin{scope}	[rotate around = {\rot: (\Ax, \Ay)}]
			\begin{scope}
				\draw[clip, rounded corners = \scale*\radii cm] (\Ax-\extend,\Ay-\hi/2) rectangle + (\len+2*\extend,\hi);
				\shade[ shading angle=\rot] (\Ax-\extend,\Ay-\shadeboxside/2) rectangle +(\shadeboxside, \shadeboxside);
			\end{scope} %end clipping
			\draw[rounded corners=\scale*\radii cm, \stroke, \thickness] (\Ax-\extend,\Ay-\hi/2) rectangle +(\len+2*\extend,\hi);
		\end{scope}
		} % end of constructor
		} % end of rrect class

		\pgfooclass{rr}{
			\method rr (#1,#2,#3,#4,#5) { % The constructor; everything is done in here
				% Here I can get named x and y coordinates
				\def\phil{#3} \def\stroke{#4} \def\line{#5}
				\gettikzxy{(#1)}{\spx}{\spy}
				\gettikzxy{(#2)}{\epx}{\epy}
				% I'd like named points to work with
				\coordinate (Start) at (\spx, \spy);
				\coordinate (End) at (\epx, \epy);
				% Find the length between start and end. Then the angle between x axis and Diff will be the rotation to apply.
				\coordinate (Diff) at ($ (End)-(Start) $);
				\gettikzxy{(Diff)}{\dx}{\dy}
				\pgfmathparse{veclen(\dx, \dy)} \pgfmathresult
				\let\length\pgfmathresult
				\pgfmathparse{\dx==0}%
				% \ifnum low-level TeX for integers
				\ifnum\pgfmathresult=1 % \dx == 0
					\pgfmathsetmacro{\rot}{\dy > 0 ? 90 : -90}
				\else% \dx != 0
					\pgfmathsetmacro{\rot}{\dx > 0 ? atan(\dy /\dx) : 180 + atan(\dy / \dx)}
				\fi
				\begin{scope}	[rotate around = {\rot:(\spx, \spy )}]
					% \filldraw[ultra thick, fill=\phil, draw=\stroke] ($ (Start)+(0,\hi) $) arc(90:270:\hi) -- +(\length pt, 0) arc(-90:90:\hi) -- cycle;
					\filldraw[rounded corners=\scale*\radii cm, line width=\line mm, fill=\phil, draw=\stroke] (\spx-\extend cm,\spy-\hi cm) rectangle +(2*\extend cm + \length pt, 2*\hi cm);
				\end{scope}
			}
		}

		\pgfooclass{beam}{
			\method beam(#1,#2,#3,#4,#5) { % The constructor; everything is done in here
				% Here I can get named x and y coordinates
				\def\phil{#3} \def\stroke{#4} \def\line{#5}
				\gettikzxy{(#1)}{\spx}{\spy}
				\gettikzxy{(#2)}{\epx}{\epy}
				% I'd like named points to work with
				\coordinate (Start) at (\spx, \spy);
				\coordinate (End) at (\epx, \epy);
				% Find the length between start and end. Then the angle between x axis and Diff will be the rotation to apply.
				\coordinate (Diff) at ($ (End)-(Start) $);
				\gettikzxy{(Diff)}{\dx}{\dy}
				\pgfmathparse{veclen(\dx, \dy)} \pgfmathresult
				\let\length\pgfmathresult
				\pgfmathparse{\dx==0}%
				% \ifnum low-level TeX for integers
				\ifnum\pgfmathresult=1 % \dx == 0
					\pgfmathsetmacro{\rot}{\dy > 0 ? 90 : -90}
				\else% \dx != 0
					\pgfmathsetmacro{\rot}{\dx > 0 ? atan(\dy / \dx) : 180 + atan(\dy / \dx)}
				\fi
				\begin{scope}	[rotate around = {\rot:(\spx, \spy )}]
					\fill[\phil] (\spx-\extend cm,\spy-\hi cm) rectangle +(2*\extend cm + \length pt, 2*\hi cm);
					\draw[draw=\stroke, line width=\line mm] (\spx-\extend cm,\spy-\hi cm) -- +(2*\extend cm + \length pt, 0);
					\draw[draw=\stroke, line width=\line mm] (\spx-\extend cm,\spy+\hi cm) -- +(2*\extend cm + \length pt, 0);
				\end{scope}
			}
		}


\usefonttheme[onlymath]{serif}

\usepackage[absolute,overlay]{textpos}
\setlength{\TPHorizModule}{1.0cm}
\setlength{\TPVertModule}{\TPHorizModule}
\textblockorigin{0.0cm}{0.0cm}  %start all at upper left corner
\usepackage{hyperref}
\hypersetup{colorlinks=true, urlcolor=structure}
% \hypersetup{urlcolor=Blue4}

\setlength{\parskip}{\medskipamount}
\setlength{\parindent}{0pt}

\usetheme{Antibes}

\usecolortheme[rgb={0, 0.6,0.6}]{structure}
% \definecolor{structurecolor}{rgb}{0.55,0.53,0.31}
\setbeamertemplate{items}[triangle]
\setbeamertemplate{blocks}[rounded][shadow=false]
%\setbeamertemplate{background canvas}[vertical shading][bottom=Cyan1!50, middle=white, top=white, midpoint=0.05]
\setbeamertemplate{headline}{\vspace{.05cm}}
\setbeamertemplate{footline}{\textcolor{black}{ \hfill \insertshorttitle \quad
	\insertshortsubtitle
	\quad \insertframenumber/\inserttotalframenumber \quad{ }\vspace{0.125cm}}}
\addtobeamertemplate{footline}{\hypersetup{linkcolor=.}}{}
\setbeamertemplate{navigation symbols}{} % empty braces suppresses all navigation symbols
\setbeamercolor{frametitle}{fg=gray!10!white}
% \setbeamercolor{footline}{fg=black}
\setbeamercolor{block title}{fg=gray!15!white,bg=structure}
\setbeamercolor{block body}{bg=white, fg=black}
\setbeamercolor{background canvas}{bg=white}
\setbeamersize{text margin left = 1cm, text margin right=1cm}
%\useinnertheme[shadow]{rounded}
%\raggedright
\setbeamerfont{block title}{family=mathserif}

\everymath{\displaystyle}
\newcounter{itemcount}

\logo{\pgfputat{\pgfxy(-11.85,-0.5)}{\pgfbox[right,base]{\includegraphics[height=1cm]{../../figs/rb_logo}}}}


\resetcounteronoverlays{tcb@cnt@myexam}
\resetcounteronoverlays{tcb@cnt@myexer}


% itemize indent
\setlength{\leftmargini}{1.5em}
\def\scale{1} % initialisation for pikz
\raggedright



%define title content
\title[Flow of Fluids]{\Huge 03 --- Flow of Fluids in Closed Pipes}
% \title[Static Fluids]{\Huge \textcolor{white}{02 --- Forces Due to Static Fluids}}
\subtitle[CIVL318]{\Large\textcolor{white}{Water Resources, CIVL318}}
\author{}
\institute{}
\date{Last revision on \today}

%%%%%%%%%%%%%%%%%%%%%%%%%%%%%%%%%%%%%%%%%%%%%%%%%%%%%%%%%%%%%%%%%%%%%%%%%%%%%%%%%%%%%%%%%%%%%%%%%%%%%%%%%%%%%%%%%%%%%%%%%%%

\begin{document}

\begin{frame}[plain]    %don't need footer on titlepage
	\titlepage
\end{frame}

%%%%%%%%%%%%%%%%%%%%%%%%%%%%%%%%%%%%%%%%%%%%%%%%%%%%%%%%%%%%%%%%%%%%%%%%%%%%%%%%%%%%%%%%%%%%%%%%%%%%%%%%%%%%%%%%%%%%%%%%%%%

%%%%%%%%%%%%%%%%%%%%%%%%%%%%%%%%%%%%%%%%%%%%%%%%%%%%%%%%%%%%%%%%%%%%%%%%%%%%%%%%%%%%%%%%%%%%%%%%%%%%%%%%%%%%%%%%%%%%%%%%%%%

\begin{frame}{Fluid Dynamics}
	\begin{itemize}
		\item In this module, and for the rest of this course, we shall be concerned with \textbf{fluid dynamics} \pause
		\item This is the study of the \text{flow} of fluids through full pressurised pipes
		      \cfig[0.4]{../../figs/03FlowOfFluids/04FullPipe-1}
	\end{itemize}
	
\end{frame}

%%%%%%%%%%%%%%%%%%%%%%%%%%%%%%%%%%%%%%%%%%%%%%%%%%%%%%%%%%%%%%%%%%%%%%%%%%%%%%%%%%%%%%%%%%%%%%%%%%%%%%%%%%%%%%%%%%%%%%%%%%%

\begin{frame}{Flow Rates}
	
	\begin{mybox}[title=Mass Flow Rate]
		The mass $M$ of fluid flowing past a section in unit time (kg/s)
	\end{mybox}
	\parb
	\visible<2->{
		\begin{mybox}[title=Weight Flow Rate]
			The weight $W$ of fluid flowing past a section in unit time (N/s)
		\end{mybox}
	}
	\parb
	\visible<3->{
		\begin{mybox}[title=Volume Flow Rate]
			The volume $Q$ of fluid flowing past a section in unit time ($\text{m}^3/\text{s}$)
		\end{mybox}
	}
	\centering
	\visible<4>{We shall be primarily interested in the volume flow rate.}
\end{frame}

%%%%%%%%%%%%%%%%%%%%%%%%%%%%%%%%%%%%%%%%%%%%%%%%%%%%%%%%%%%%%%%%%%%%%%%%%%%%%%%%%%%%%%%%%%%%%%%%%%%%%%%%%%%%%%%%%%%%%%%%%%%

\begin{frame}{Volume Flow Rate, $Q$}
	\begin{center}
		\begin{minipage}{0.65\textwidth}
			\raggedright
			The volume flow rate, $Q$, is calculated from
			\begin{center}
				\begin{mybox}	[width=3cm]
					\vspace{-0.5cm}
					\[\bm{ Q=Av }\]
				\end{mybox}
			\end{center}
			where $A$ is the cross-sectional area of the section and $v$ is the average velocity of the flow.
			\[ \left(\text{m}^3/\text{s} = \text{m}^2\times\text{m/s}\right) \]
			
		\end{minipage}
	\end{center}
\end{frame}

%%%%%%%%%%%%%%%%%%%%%%%%%%%%%%%%%%%%%%%%%%%%%%%%%%%%%%%%%%%%%%%%%%%%%%%%%%%%%%%%%%%%%%%%%%%%%%%%%%%%%%%%%%%%%%%%%%%%%%%%%%%

\begin{frame}{Weight Flow Rate, W}
	\begin{center}
		\begin{minipage}{0.65\textwidth}
			The weight flow rate, $W$, is related to the volume flow rate by
			\begin{center}
				\begin{mybox}[width=3cm]
					\vspace{-0.5cm}
					\[\bm{ W=\gamma Q } \]
				\end{mybox}
			\end{center}
			where $\gamma$ is the specific weight of the fluid  and $Q$ is the volume flow rate.
			
			\[ \left(\text{kN/s} = \mathsf{kN/m^3}\times\mathsf{m^3/s}\right) \]
			
		\end{minipage}
	\end{center}
\end{frame}

%%%%%%%%%%%%%%%%%%%%%%%%%%%%%%%%%%%%%%%%%%%%%%%%%%%%%%%%%%%%%%%%%%%%%%%%%%%%%%%%%%%%%%%%%%%%%%%%%%%%%%%%%%%%%%%%%%%%%%%%%%%

\begin{frame}{Mass Flow Rate}
	\begin{center}
		\begin{minipage}{0.65\textwidth}
			\raggedright
			The mass flow rate, $M$, is related to the volume flow rate by
			\begin{center}
				\begin{mybox}[width=3cm]
					\vspace{-0.5cm}
					\[\bm{ M=\rho Q } \]
				\end{mybox}
			\end{center}
			where $\rho$ is the density of the fluid and $Q$ is the volume flow rate.
			\[ \left(\text{kg/s} = \mathsf{kg/m^3}\times\mathsf{m^3/s}\right) \]
		\end{minipage}
	\end{center}
\end{frame}

%%%%%%%%%%%%%%%%%%%%%%%%%%%%%%%%%%%%%%%%%%%%%%%%%%%%%%%%%%%%%%%%%%%%%%%%%%%%%%%%%%%%%%%%%%%%%%%%%%%%%%%%%%%%%%%%%%%%%%%%%%%

\begin{frame}{The Continuity Equation}
	\begin{textblock*}{5cm}(1cm,0.5cm)
		\cfig[0.45]{../../figs/03FlowOfFluids/04Bernoulli01}
	\end{textblock*}
	\only<1->{\begin{textblock*}{3.5cm}(1cm,1.5cm)
		Fluid is flowing from section $1$ to section $2$ at a constant rate;\\this is called\\\textbf{steady flow}.
		\end{textblock*}
		}\only<2->{
		\begin{textblock*}{4cm}(8cm,1cm)
			\begin{flushright}
				No fluid is added, stored or removed between $1$ and $2$. The mass flow rate
				at $2$ is the same as that at $1$.
			\end{flushright}
		\end{textblock*}
		\begin{textblock*}{6cm}(6cm,2.5cm)
			\begin{flushright}
				$A_1$ and $A_2$ are the cross-\\sectional areas at $1$ and $2$.
			\end{flushright}
		\end{textblock*}
		}\only<3->{
		\begin{textblock*}{4cm}(7.75cm,3.35cm)
			\begin{flushright}
				\begin{align*}
					M_1            & = M_2            \\
					\rho_1 Q_1     & = \rho_2 Q_2     \\
					\rho_1 A_1 v_1 & = \rho_2 A_2 v_2 
				\end{align*}
			\end{flushright}
		\end{textblock*}
		}\only<4->{
		\begin{textblock*}{6cm}(5.85cm,5.75cm)
			\begin{flushright}
				This is true for all fluids; this equation is known as the \textbf{continuity equation}.
			\end{flushright}
		\end{textblock*}
		}\only<5->{
		\begin{textblock*}{10cm}(1.85cm,7.0cm)
			\begin{flushright}
				Liquids are incompressible, the density does not change and $\rho_1=\rho_2$. \par
			\end{flushright}
		\end{textblock*}
		}\only<6>{
		\begin{textblock*}{10cm}(1.8cm,7.9cm)
			\begin{flushright}
				The continuity equation for liquids is $\bm{ A_1v_1 = A_2v_2} $
			\end{flushright}
		\end{textblock*}
	}
\end{frame}

%%%%%%%%%%%%%%%%%%%%%%%%%%%%%%%%%%%%%%%%%%%%%%%%%%%%%%%%%%%%%%%%%%%%%%%%%%%%%%%%%%%%%%%%%%%%%%%%%%%%%%%%%%%%%%%%%%%%%%%%%%%

\begin{frame}
	\begin{textblock*}{4cm} (0.25cm,2.15cm)
		\cfig[0.5]{../../figs/03FlowOfFluids/04BernoulliEx01}
	\end{textblock*}
	\definecolor{example}{RGB}{189,189,115}
	\setbeamertemplate{itemize item}{\color{example}$\blacktriangleright$}
	\begin{textblock*}{5.75cm} (6.5cm,0.75cm)
		\begin{myexam}[colframe=example!80!black, colbacktitle=example]{}{}
			\raggedright
			The average velocity of the flow at the nozzle $C$ is $4.7\,\text{m/s}$.\parb
			Determine:
			\begin{itemize}
				\item the average flow velocity at $A$
				\item the average flow velocity at $B$
				\item the volume flow rate, $Q$, through the system in L/s.
			\end{itemize}
		\end{myexam}
		
	\end{textblock*}
\end{frame}



%%%%%%%%%%%%%%%%%%%%%%%%%%%%%%%%%%%%%%%%%%%%%%%%%%%%%%%%%%%%%%%%%%%%%%%%%%%%%%%%%%%%%%%%%%%%%%%%%%%%%%%%%%%%%%%%%%%%%%%%%%%

\begin{frame}{Steel Pipe}
	\begin{itemize}
		\item Used for general purpose pipelines
		\item Has standard pipe sizes
		\item Size specified by \textbf{nominal size} and \textbf{schedule number}:
		      \begin{itemize}
		      	\item[]\item[] $4$-inch Schedule 40 steel pipe has a nominal size of $4$ inches but has an actual inside
		      	      diameter of $4.026''$ or, for our purposes, $102.3\,\text{mm}$.
		      	\item[]\item[] $12$-inch Schedule 80 steel pipe has a nominal size of $12$ inches but has an actual inside
		      	      diameter of $11.376''$ or $289.0\,\text{mm}$.
		      	\item[]
		      \end{itemize}
		\item Nominal sizes range from $\tfrac{1}{8}\,\text{in.}$ to $24\,\text{in.}$
		\item Higher schedules have a greater thickness. Schedules range from $10$ to $160$
	\end{itemize}
\end{frame}

%%%%%%%%%%%%%%%%%%%%%%%%%%%%%%%%%%%%%%%%%%%%%%%%%%%%%%%%%%%%%%%%%%%%%%%%%%%%%%%%%%%%%%%%%%%%%%%%%%%%%%%%%%%%%%%%%%%%%%%%%%%

\begin{frame}{}
	\begin{center}
		\footnotesize
		\textbf{\normalsize Table F: Schedule 40 Steel Pipe}\parb
		\begin{tabular}{>{$}c<{$} >{$}c<{$} >{$}c<{$} >{$}c<{$}  >{$}c<{$}}
			
			\toprule
			\addlinespace
			\text{Nominal} & \text{Inside}   & \qquad\qquad & \text{Nominal} & \text{Inside}   \\
			\text{Size}    & \text{Diameter} &              & \text{Size}    & \text{Diameter} \\
			
			\addlinespace
			\text{(in)}    & \text{(mm)}     &              & \text{(in)}    & \text{(mm)}     \\
			\addlinespace
			\midrule
			\addlinespace
			\tfrac18       & 6.8             &              & 4              & 102.3           \\ \addlinespace
			\tfrac14       & 9.2             &              & 5              & 128.2           \\ \addlinespace
			\tfrac38       & 12.5            &              & 6              & 154.1           \\ \addlinespace
			\tfrac12       & 15.8            &              & 8              & 202.7           \\ \addlinespace
			\tfrac34       & 20.9            &              & 10             & 254.5           \\ \addlinespace
			1              & 26.6            &              & 12             & 303.2           \\ \addlinespace
			1\tfrac14      & 35.1            &              & 14             & 333.4           \\	\addlinespace
			1\tfrac12      & 40.9            &              & 16             & 381.0           \\ \addlinespace
			2              & 52.5            &              & 18             & 428.7           \\ \addlinespace
			2\tfrac12      & 62.7            &              & 20             & 477.9           \\ \addlinespace
			3              & 77.9            &              & 24             & 574.7           \\ \addlinespace
			3\tfrac12 	& 90.1 \\\\
			
			\midrule
			\bottomrule
		\end{tabular}
	\end{center}
\end{frame}

%%%%%%%%%%%%%%%%%%%%%%%%%%%%%%%%%%%%%%%%%%%%%%%%%%%%%%%%%%%%%%%%%%%%%%%%%%%%%%%%%%%%%%%%%%%%%%%%%%%%%%%%%%%%%%%%%%%%%%%%%%%
\begin{frame}{Steel Tubing}
	\begin{itemize}
		\item Used in fluid power systems, condensers, heat exchangers, engine fuel systems and
		      industrial fluid processing systems.
		\item[]
		\item Size specified by \textbf{outside diameter} and \text{wall thickness}:
		      \begin{itemize}
		      	\item[]\item[] $\tfrac{3}{4}\text{-in.}$ steel tubing with wall thickness $1.65\,\text{mm}$ has \\an
		      	      outside diameter $\tfrac{3}{4}\text{-in.}$ ($19.05\,\text{mm}$) and an inside diameter of $15.75\,\text{mm}$
		      	\item[]
		      \end{itemize}
		\item[]
		\item Available in sizes from $\tfrac{1}{8}\text{-in.}$ to $2\text{-in.}$
		\item[]
		\item Smoother, and less resistant to flow, than steel pipe
	\end{itemize}
\end{frame}

%%%%%%%%%%%%%%%%%%%%%%%%%%%%%%%%%%%%%%%%%%%%%%%%%%%%%%%%%%%%%%%%%%%%%%%%%%%%%%%%%%%%%%%%%%%%%%%%%%%%%%%%%%%%%%%%%%%%%%%%%%%
\begin{frame}{Copper Tubing}
	\begin{itemize}
		\item The choice of which type of copper tubing depends upon use: \item[]
		      \begin{itemize}
		      	\item \textbf{Type K}: Water service, fuel oil, natural gas, compressed air \item[]
		      	\item \textbf{Type L}: Same uses as Type K but with smaller wall thickness \item[]
		      	\item \textbf{Type M}: Same uses as Type K and Type L but with yet smaller wall thickness; useful for water
		      	      services or heating applications at moderate pressures \item[]
		      	\item \textbf{Type DWV}: Drain, waste and venting uses in plumbing \item[]
		      	\item \textbf{Type ACR}: Air conditioning, refrigeration, natural gas, liquefied petroleum gas (LPG)
		      	      and compressed air \item[]
		      	\item \textbf{Type OXY/MED}: Oxygen or medical gases. Special processing for increased cleanliness.
		      \end{itemize}
		\item Available in annealed (soft) form or drawn (hard) tempers \item[]
		\item Tube supplied to the American Society for Testing and Materials (ASTM) is $99.9\%$ pure copper
	\end{itemize}
\end{frame}

%%%%%%%%%%%%%%%%%%%%%%%%%%%%%%%%%%%%%%%%%%%%%%%%%%%%%%%%%%%%%%%%%%%%%%%%%%%%%%%%%%%%%%%%%%%%%%%%%%%%%%%%%%%%%%%%%%%%%%%%%%%
\begin{frame}{Ductile Iron Pipe}
	\begin{itemize}
		\item Used in water, gas and sewage lines as a replacement for cast iron.
		\item[]
		\item Strong, ductile and easy to work with.
		\item[]
		\item Less common nowadays.
		      
	\end{itemize}
\end{frame}

%%%%%%%%%%%%%%%%%%%%%%%%%%%%%%%%%%%%%%%%%%%%%%%%%%%%%%%%%%%%%%%%%%%%%%%%%%%%%%%%%%%%%%%%%%%%%%%%%%%%%%%%%%%%%%%%%%%%%%%%%%%
\begin{frame}{Plastic Pipe and Tubing}
	\begin{itemize}
		\item Used in water and gas distribution, sewer and wastewater, oil and gas production, irrigation,  mining and
		      many other industrial applications.
		\item[]
		\item Light, easy to work with, corrosion resistant, smooth and with good flow characteristics.
		\item[]
		\item High density polyethylene (HDPE) pipe is quickly becoming the pipe of choice, replacing ductile iron pipe in
		      many situations.
		      
	\end{itemize}
\end{frame}

%%%%%%%%%%%%%%%%%%%%%%%%%%%%%%%%%%%%%%%%%%%%%%%%%%%%%%%%%%%%%%%%%%%%%%%%%%%%%%%%%%%%%%%%%%%%%%%%%%%%%%%%%%%%%%%%%%%%%%%%%%%

\begin{frame}
	\begin{center}
		\begin{minipage}{0.6\textwidth}
			
			\begin{myexam}{}{}
				Water, at $70\text{\textcelsius}$ flows through $\tfrac78\text{-in.}$ steel tubing, with $1.65\,\text{mm}$ wall thickness,
				at an average velocity of $5.7\,\text{m/s}$.
				
				Determine:
				\begin{itemize}
					\item the volume flow rate, $Q$
					\item the mass flow rate, $M$
					\item the weight flow rate, $W$
				\end{itemize}
			\end{myexam}
		\end{minipage}
	\end{center}
\end{frame}

%%%%%%%%%%%%%%%%%%%%%%%%%%%%%%%%%%%%%%%%%%%%%%%%%%%%%%%%%%%%%%%%%%%%%%%%%%%%%%%%%%%%%%%%%%%%%%%%%%%%%%%%%%%%%%%%%%%%%%%%%%%

\begin{frame}{Energy Considerations}
	\begin{textblock*}{.95\textwidth}(.75cm,1cm)
		\begin{itemize}
			\item<1-> Consider an element of fluid flowing under pressure in a pipe.\\
			Three types of energy are considered when analysing this element: \parb
			\begin{itemize}
				\item<2->\textbf{Potential energy}, due to the elevation $z$ of an element with mass $m$ of the fluid.  Then
				$\textbf{PE} \bm{=mgz}$	\parb
				\item<3-> \textbf{Kinetic energy}, due to the velocity $v$ of the element of fluid. \lb Then
				$\textbf{KE} \bm{=mv^2/2}$	\parb
				\item<4-> \textbf{Flow (or pressure) energy} which is a measure of the work done moving an element of the fluid
				against the pressure $p$ in the fluid. \lb Then $\textbf{FE} \bm{=pmg/\gamma}$
			\end{itemize}
		\end{itemize}
	\end{textblock*}
	\begin{textblock*}{4cm}(2cm,5cm)
		\only<1>{
			\cfig[0.27]{../../figs/03FlowOfFluids/04Bernoulli03a}
			}\only<2>{
			\cfig[0.27]{../../figs/03FlowOfFluids/04Bernoulli03b}
			}\only<3>{
			\cfig[0.27]{../../figs/03FlowOfFluids/04Bernoulli03c}
			}\only<4>{
			\cfig[0.27]{../../figs/03FlowOfFluids/04Bernoulli03d}
		}
	\end{textblock*}
\end{frame}

%%%%%%%%%%%%%%%%%%%%%%%%%%%%%%%%%%%%%%%%%%%%%%%%%%%%%%%%%%%%%%%%%%%%%%%%%%%%%%%%%%%%%%%%%%%%%%%%%%%%%%%%%%%%%%%%%%%%%%%%%%%

\begin{frame}{Calculation of Pressure Energy}
	This calculation is a little more complicated than the calculation for potential or kinetic energy:
	\par\vspace{-0.5cm}
	\cfig[0.27]{../../figs/03FlowOfFluids/04Bernoulli04}
	\par\vspace{-0.75cm}
	\begin{itemize}
		\item The element we consider is of length $L$
		\item To move the element completely across some section in the pipe, the element has to move a distance of $L$.
		      
		\item The work done moving the element against the pressure $p$ in the pipe is:
		      \begin{align*}
		      	\text{Work} & = \text{force}\times\text{distance}                                              \\
		      	            & = pAL                                                                            \\
		      	            & = pV \text{\footnotesize{ (where $V$ is the volume of element)}}                 \\
		      	            & = pmg/\gamma\quad \text{\footnotesize{ ($\gamma =w/V=mg/V$  so $ V=mg/\gamma)$}} \\
		      \end{align*}
	\end{itemize}
\end{frame}

%%%%%%%%%%%%%%%%%%%%%%%%%%%%%%%%%%%%%%%%%%%%%%%%%%%%%%%%%%%%%%%%%%%%%%%%%%%%%%%%%%%%%%%%%%%%%%%%%%%%%%%%%%%%%%%%%%%%%%%%%%%

\begin{frame}{Conservation of Energy}
	\begin{textblock*}{1\textwidth} (3.5cm, 0.5cm)
		\cfig[0.45]{../../figs/03FlowOfFluids/04Bernoulli05}
	\end{textblock*}
	
	\begin{textblock*}{12cm} (0.5cm, 4.25cm)
		
		\begin{itemize}
			\only<1->{
				\item The total energy of the element of fluid at $1$ is:
				\par\vspace{-0.25cm}
				\[E_1=\frac{p_1mg}{\gamma}+mgz_1+\frac{mv_1^2}{2}\qquad\qquad \qquad\qquad\qquad\qquad\]\pars
				}\only<2->{
				\item The total energy of the same element of fluid when it reaches $2$ is:
				\par\vspace{-0.25cm}
				\[E_2=\frac{p_2mg}{\gamma}+mgz_2+\frac{mv_2^2}{2}\qquad\qquad \qquad\qquad\qquad\qquad\]\pars
				}\only<3->{
				\item But energy cannot be created or destroyed \\(although it may be converted from one form to another)
			}
		\end{itemize}
	\end{textblock*}
\end{frame}

%%%%%%%%%%%%%%%%%%%%%%%%%%%%%%%%%%%%%%%%%%%%%%%%%%%%%%%%%%%%%%%%%%%%%%%%%%%%%%%%%%%%%%%%%%%%%%%%%%%%%%%%%%%%%%%%%%%%%%%%%%%

\begin{frame}{Bernoulli's Equation}
	\begin{textblock*}{1\textwidth} (3.5cm, 0.5cm)
		\cfig[0.45]{../../figs/03FlowOfFluids/04Bernoulli05}
	\end{textblock*}
	
	\begin{textblock*}{9cm} (0.5cm, 4cm)
		\begin{itemize}
			\only<1->{
				\item Then $E_1=E_2$ so
				\[ \frac{p_1mg}{\gamma}+mgz_1+\frac{mv_1^2}{2}=\frac{p_2mg}{\gamma}+mgz_2+\frac{mv_2^2}{2}\qquad\qquad\qquad\qquad \]
				}\only<2->{
				\item Dividing throughout by the weight of the element ($mg$), we get:
			}
		\end{itemize}
	\end{textblock*}
	\only<2->{
		\begin{textblock*}{7cm} (2cm, 6.75cm)
			\begin{mybox}[title=Bernoulli's Equation]
				\centering\Large
				$\bm{ \frac{p_1}{\gamma}+z_1+\frac{v_1^2}{2g}=\frac{p_2}{\gamma}+z_2+\frac{v_2^2}{2g}} $
			\end{mybox}
		\end{textblock*}
	}
\end{frame}

%%%%%%%%%%%%%%%%%%%%%%%%%%%%%%%%%%%%%%%%%%%%%%%%%%%%%%%%%%%%%%%%%%%%%%%%%%%%%%%%%%%%%%%%%%%%%%%%%%%%%%%%%%%%%%%%%%%%%%%%%%%

\begin{frame}{Bernoulli's Equation}
	\begin{center}
		\begin{minipage}{0.7\textwidth}
			\begin{mybox}[title=Bernoulli's Equation]
				\centering\Large
				$\bm{ \frac{p_1}{\gamma}+z_1+\frac{v_1^2}{2g}=\frac{p_2}{\gamma}+z_2+\frac{v_2^2}{2g}} $
			\end{mybox}
		\end{minipage}
		\begin{itemize}
			\item Bernoulli's Equation applies only to fluids that are incompressible.\pause
			\item It is assumed that no fluid is added or removed between the points $1$ and $2$ under consideration\pause
			\item It is assumed that there are no energy``losses'' between $1$ and $2$; that is, no energy is lost to friction
			      and no heat is added to or removed from the fluid\pause
			\item No energy is added to the system (by a pump, for example) or removed from the system (by a turbine, for example).\pause
			\item In theory, there is alway some energy loss due to friction but, for a large class of situations,
			      Bernoulli's Equation yields accurate results.
		\end{itemize}
	\end{center}
	
\end{frame}

%%%%%%%%%%%%%%%%%%%%%%%%%%%%%%%%%%%%%%%%%%%%%%%%%%%%%%%%%%%%%%%%%%%%%%%%%%%%%%%%%%%%%%%%%%%%%%%%%%%%%%%%%%%%%%%%%%%%%%%%%%%

\begin{frame}{Energy Head}
	\begin{center}
		\begin{minipage}{0.7\textwidth}
			\begin{mybox}[title=Bernoulli's Equation]
				\centering
				$\bm{ \frac{p_1}{\gamma}+z_1+\frac{v_1^2}{2g}=\frac{p_2}{\gamma}+z_2+\frac{v_2^2}{2g}} $
			\end{mybox}
		\end{minipage}
		\begin{itemize}
			\item The $\bm{\frac{ p}{\gamma}}$ term is known as the \textbf{pressure head}.
			      Units are $\frac{\text{kN/m}^2}{\text{kN/m}^3}=\text{m}$. \pause
			\item The $\bm{z}$ term is known as the \textbf{elevation head}.
			      Units are $\text{m}$. \pause
			\item The $\bm{\frac{v^2}{2g}}$ term is known as the \textbf{velocity head}.
			      Units are $\frac{\text{(m/s)}^2}{\text{m/s}^2}=\text{m}$. \pause
			\item Thus, \textbf{total head} is measured in metres.
		\end{itemize}
	\end{center}
	
\end{frame}

%%%%%%%%%%%%%%%%%%%%%%%%%%%%%%%%%%%%%%%%%%%%%%%%%%%%%%%%%%%%%%%%%%%%%%%%%%%%%%%%%%%%%%%%%%%%%%%%%%%%%%%%%%%%%%%%%%%%%%%%%%%

\begin{frame}{Applying Bernoulli's Equation}
	
	\begin{enumerate}
		\item Determine what is already known or given
		\item Determine what is to be calculated
		\item Choose two appropriate or convenient sections at which to apply Bernoulli's Equation
		\item Use the equation \textbf{in the direction of the flow}. That is, section $1$ should be upstream
		      of section $2$ (or, equivalently, fluid flow should go from the left hand side of the equation to the right
		      hand side of the equation).
		\item Cancel terms that are $0$ or that are the same on both sides of the equation
		      \begin{itemize}
		      	\item Tanks, reservoirs or nozzles exposed to the atmosphere have zero pressure so the pressure head term can be removed
		      	      from the calculation
		      	\item The velocity head at the surface of a tank or reservoir is assumed to be zero
		      \end{itemize}
	\end{enumerate}
\end{frame}

%%%%%%%%%%%%%%%%%%%%%%%%%%%%%%%%%%%%%%%%%%%%%%%%%%%%%%%%%%%%%%%%%%%%%%%%%%%%%%%%%%%%%%%%%%%%%%%%%%%%%%%%%%%%%%%%%%%%%%%%%%%


\begin{frame}
	\begin{textblock*}{4.75cm} (0.75cm,0.25cm)
		\definecolor{example}{RGB}{229,194,155}
		\setbeamertemplate{itemize item}{\color{example}$\blacktriangleright$}
		\begin{myexam}[colframe=example, colbacktitle=example!80!white]{}{}
			\raggedright
			Oil, with a specific gravity of $0.83$, flows under gravity from a tank, through a pipe system as shown, before
			entering the atmosphere through a nozzle at $D$. \par
			Determine:
			\begin{itemize}
				\item the pressure at $A$
				\item the pressure at $B$
				\item the pressure at $C$
				\item the volume flow rate through the system
			\end{itemize}
		\end{myexam}
	\end{textblock*}
	
	\begin{textblock*}{6cm}(4.75cm,1.75cm)
		\cfig[0.5]{../../figs/03FlowOfFluids/04BernoulliEx03}
	\end{textblock*}
\end{frame}


%%%%%%%%%%%%%%%%%%%%%%%%%%%%%%%%%%%%%%%%%%%%%%%%%%%%%%%%%%%%%%%%%%%%%%%%%%%%%%%%%%%%%%%%%%%%%%%%%%%%%%%%%%%%%%%%%%%%%%%%%%%

% \begin{frame}
% 	\cfig[0.65]{../../figs/03FlowOfFluids/04BernoulliEx03}
% \end{frame}

%%%%%%%%%%%%%%%%%%%%%%%%%%%%%%%%%%%%%%%%%%%%%%%%%%%%%%%%%%%%%%%%%%%%%%%%%%%%%%%%%%%%%%%%%%%%%%%%%%%%%%%%%%%%%%%%%%%%%%%%%%%

\begin{frame}
	\begin{textblock*}{6cm} (.5cm,-0.5cm)
		\cfig[0.4]{../../figs/03FlowOfFluids/04BernoulliEx04}
	\end{textblock*}
	\begin{textblock*}{6cm }(5.5cm,5cm)
		
		\only<1->{
			Which is the higher reading? $p_1$ or $p_2$?
		}
		
		\only<2>{\par
			We can't tell without more information.
		}
	\end{textblock*}
\end{frame}

%%%%%%%%%%%%%%%%%%%%%%%%%%%%%%%%%%%%%%%%%%%%%%%%%%%%%%%%%%%%%%%%%%%%%%%%%%%%%%%%%%%%%%%%%%%%%%%%%%%%%%%%%%%%%%%%%%%%%%%%%%%

\begin{frame}
	\begin{textblock*}{6cm} (.5cm,-0.5cm)
		\cfig[0.4]{../../figs/03FlowOfFluids/04BernoulliEx04a}
	\end{textblock*}
	\begin{textblock*}{6cm} (5.5cm,6cm)
		\definecolor{example}{RGB}{0, 138, 0}
		\begin{myexam}[colframe=example, colbacktitle=example!80!white]{}{}
			
			Determine the pressure reading  $p_2$ if $Q=25\,\text{L/s}$.
			
		\end{myexam}
	\end{textblock*}
\end{frame}

%%%%%%%%%%%%%%%%%%%%%%%%%%%%%%%%%%%%%%%%%%%%%%%%%%%%%%%%%%%%%%%%%%%%%%%%%%%%%%%%%%%%%%%%%%%%%%%%%%%%%%%%%%%%%%%%%%%%%%%%%%%

\begin{frame}
	\begin{textblock*}{6cm} (.5cm,-0.5cm)
		\cfig[0.4]{../../figs/03FlowOfFluids/04BernoulliEx04a}
	\end{textblock*}
	\begin{textblock*}{6cm} (5.5cm,6cm)
		\definecolor{example}{RGB}{0, 138, 0}
		\begin{myexer}[colframe=example, colbacktitle=example!80!white]{}{}
			
			Determine the pressure reading  $p_2$ if $Q=20\,\text{L/s}$.
			
		\end{myexer}
	\end{textblock*}
\end{frame}

%%%%%%%%%%%%%%%%%%%%%%%%%%%%%%%%%%%%%%%%%%%%%%%%%%%%%%%%%%%%%%%%%%%%%%%%%%%%%%%%%%%%%%%%%%%%%%%%%%%%%%%%%%%%%%%%%%%%%%%%%%%

% \begin{frame}
% 	\cfig[0.5]{../../figs/03FlowOfFluids/04BernoulliEx04a}
% \end{frame}

%%%%%%%%%%%%%%%%%%%%%%%%%%%%%%%%%%%%%%%%%%%%%%%%%%%%%%%%%%%%%%%%%%%%%%%%%%%%%%%%%%%%%%%%%%%%%%%%%%%%%%%%%%%%%%%%%%%%%%%%%%%

\begin{frame}
	\begin{textblock*}{6cm}(.5cm,-0.5cm)
		\cfig[0.5]{../../figs/03FlowOfFluids/04BernoulliEx04b}
	\end{textblock*}
	\begin{textblock*}{5cm}(6.5cm,6.5cm)
		\definecolor{example}{RGB}{170, 89, 84}
		\begin{myexam}[colframe=example, colbacktitle=example!80!white]{}{}
			\raggedright
			Determine $Q$, the volume flow rate.
		\end{myexam}
		
	\end{textblock*}
\end{frame}

%%%%%%%%%%%%%%%%%%%%%%%%%%%%%%%%%%%%%%%%%%%%%%%%%%%%%%%%%%%%%%%%%%%%%%%%%%%%%%%%%%%%%%%%%%%%%%%%%%%%%%%%%%%%%%%%%%%%%%%%%%%

% \begin{frame}
% 	\cfig[0.5]{../../figs/03FlowOfFluids/04BernoulliEx04b}
% \end{frame}

%%%%%%%%%%%%%%%%%%%%%%%%%%%%%%%%%%%%%%%%%%%%%%%%%%%%%%%%%%%%%%%%%%%%%%%%%%%%%%%%%%%%%%%%%%%%%%%%%%%%%%%%%%%%%%%%%%%%%%%%%%%
\begin{frame}{Pitot Tube Meter}
	\cfig[0.4]{../../figs/03FlowOfFluids/pitot-tube.jpg}
	\vspace*{-0.25cm}
	\begin{center}
		\begin{minipage}{0.9\textwidth}
			\raggedright
			A pitot-tube, on an aircraft nose or wing, is used to measure the pressure of the stationary air immediately
			at the front of the pitot tube. \par \medskip
			This can then be compared to atmospheric pressure to measure the air-speed of the aircraft.
		\end{minipage}
	\end{center}
\end{frame}

%%%%%%%%%%%%%%%%%%%%%%%%%%%%%%%%%%%%%%%%%%%%%%%%%%%%%%%%%%%%%%%%%%%%%%%%%%%%%%%%%%%%%%%%%%%%%%%%%%%%%%%%%%%%%%%%%%%%%%%%%%%

\begin{frame}
	\begin{textblock*}{6cm}(5.25cm,-1cm)
		\cfig[0.3]{../../figs/03FlowOfFluids/pitot-cockpit}
	\end{textblock*}
	\begin{textblock*}{6cm}(-0.35cm,2cm)
		\cfig[0.35]{../../figs/03FlowOfFluids/a330}
	\end{textblock*}
\end{frame}

%%%%%%%%%%%%%%%%%%%%%%%%%%%%%%%%%%%%%%%%%%%%%%%%%%%%%%%%%%%%%%%%%%%%%%%%%%%%%%%%%%%%%%%%%%%%%%%%%%%%%%%%%%%%%%%%%%%%%%%%%%%
\begin{frame}{Pitot Tube Meter}
	\cfig[0.7]{../../figs/03FlowOfFluids/tail}
	\vspace*{-0.25cm}
	\begin{center}
		\begin{minipage}{0.6\textwidth}
			\raggedright
			Iced-up pitot-tube meters are thought to be one of the reasons for the crash of an Air France Airbus A330-200
			into the Atlantic, on June 1st, 2009, with the death of all 228 aboard.
		\end{minipage}
	\end{center}
\end{frame}

%%%%%%%%%%%%%%%%%%%%%%%%%%%%%%%%%%%%%%%%%%%%%%%%%%%%%%%%%%%%%%%%%%%%%%%%%%%%%%%%%%%%%%%%%%%%%%%%%%%%%%%%%%%%%%%%%%%%%%%%%%%

\begin{frame}{Pitot Tube Meter}
	\begin{textblock*}{1\textwidth}(1.25cm, 0.35cm)
		\only<1>{\cfig[0.27]{../../figs/03FlowOfFluids/04Bernoulli06a}
			}\only<2-3>{\cfig[0.27]{../../figs/03FlowOfFluids/04Bernoulli06b}
			}\only<4>{\cfig[0.27]{../../figs/03FlowOfFluids/04Bernoulli06c}
		}
	\end{textblock*}
	\begin{textblock*}{1\textwidth}(1.25cm, 5cm)
		\begin{center}
			\begin{itemize}
				\item  Flow  is steady  except at $S$, where it comes to a halt at a hollow tube
				      pointed directly into the flow: $S$ is called the \textbf{stationary point}.
				      \uncover<2->{\item All the velocity head at $A$ is converted to extra pressure head at $S$,
				      as is shown by the different levels at.}
				      \uncover<3->{\par Since $\Delta p = \gamma h$, we have $h_A=\frac{p_A}{\gamma}$
				      	and $h_S=\frac{p_S}{\gamma}$.}
				      \uncover<4->{\item The difference in levels, $h$, together with the specific gravity of the
				      liquid and the diameter of the pipe can be used to determine the flow rate through the pipe.}
			\end{itemize}
		\end{center}
	\end{textblock*}
\end{frame}

%%%%%%%%%%%%%%%%%%%%%%%%%%%%%%%%%%%%%%%%%%%%%%%%%%%%%%%%%%%%%%%%%%%%%%%%%%%%%%%%%%%%%%%%%%%%%%%%%%%%%%%%%%%%%%%%%%%%%%%%%%%

\begin{frame}
	\begin{textblock*}{6cm} (.5cm,-0.5cm)
		\cfig[0.3]{../../figs/03FlowOfFluids/04BernoulliEx06a}
	\end{textblock*}
	\begin{textblock*}{4.5cm} (7cm,6cm)
		\begin{myexam}[colframe=Azure4, colbacktitle=Azure4!80!white]{}{}
			\raggedright
			Determine $Q$, the volume flow rate.
		\end{myexam}
	\end{textblock*}
\end{frame}

%%%%%%%%%%%%%%%%%%%%%%%%%%%%%%%%%%%%%%%%%%%%%%%%%%%%%%%%%%%%%%%%%%%%%%%%%%%%%%%%%%%%%%%%%%%%%%%%%%%%%%%%%%%%%%%%%%%%%%%%%%%
%
% \begin{frame}
% 	\cfig[0.3]{../../figs/03FlowOfFluids/04BernoulliEx06a}
% \end{frame}

%%%%%%%%%%%%%%%%%%%%%%%%%%%%%%%%%%%%%%%%%%%%%%%%%%%%%%%%%%%%%%%%%%%%%%%%%%%%%%%%%%%%%%%%%%%%%%%%%%%%%%%%%%%%%%%%%%%%%%%%%%%

\begin{frame}{Venturi Meter}
	\begin{textblock*}{1\textwidth} (1.25cm, 1.5cm)
		\only<1>{\cfig[0.35]{../../figs/03FlowOfFluids/04BernoulliVenturi01}
			}\only<2->{\cfig[0.35]{../../figs/03FlowOfFluids/04BernoulliVenturi01a}
			
		}
	\end{textblock*}
	\begin{textblock*}{1\textwidth} (1.25cm, 5.5cm)
		\begin{center}
			\begin{itemize}
				\item  A \textbf{venturi meter} is a common way of measuring flow in pipes.
				      \uncover<2->{\item A constriction in the pipe causes an increase in velocity and, as a result, a decrease in pressure.}
				      \uncover<3>{\item The difference in pressure can be used to determine the average flow velocity, and thereby the volume flow rate,
				      through the system.}
			\end{itemize}
		\end{center}
		
	\end{textblock*}
	
\end{frame}

%%%%%%%%%%%%%%%%%%%%%%%%%%%%%%%%%%%%%%%%%%%%%%%%%%%%%%%%%%%%%%%%%%%%%%%%%%%%%%%%%%%%%%%%%%%%%%%%%%%%%%%%%%%%%%%%%%%%%%%%%%%

\begin{frame}
	
	\cfig[0.35]{../../figs/03FlowOfFluids/04BernoulliEx08}
	\cmini[0.75]{
		\begin{myexam}{}{}
			Determine $Q$, the volume flow rate, if $h=210$ mm.
		\end{myexam}
	}
\end{frame}

%%%%%%%%%%%%%%%%%%%%%%%%%%%%%%%%%%%%%%%%%%%%%%%%%%%%%%%%%%%%%%%%%%%%%%%%%%%%%%%%%%%%%%%%%%%%%%%%%%%%%%%%%%%%%%%%%%%%%%%%%%%

\begin{frame}
	\cfig[0.28]{../../figs/03FlowOfFluids/04BernoulliEx09}
	\centering
	\begin{myexer}[width=6.5cm, colframe=Bisque3, colbacktitle=Bisque3!80!white]{}{}
		\raggedright
		Determine $Q$, the volume flow rate, if $h=125\,\text{mm}$.
	\end{myexer}
\end{frame}

%%%%%%%%%%%%%%%%%%%%%%%%%%%%%%%%%%%%%%%%%%%%%%%%%%%%%%%%%%%%%%%%%%%%%%%%%%%%%%%%%%%%%%%%%%%%%%%%%%%%%%%%%%%%%%%%%%%%%%%%%%%

% \begin{frame}
% 	\cfig[0.28]{../../figs/03FlowOfFluids/04BernoulliEx09b}
%
% \end{frame}


%%%%%%%%%%%%%%%%%%%%%%%%%%%%%%%%%%%%%%%%%%%%%%%%%%%%%%%%%%%%%%%%%%%%%%%%%%%%%%%%%%%%%%%%%%%%%%%%%%%%%%%%%%%%%%%%%%%%%%%%%%%

%%%%%%%%%%%%%%%%%%%%%%%%%%%%%%%%%%%%%%%%%%%%%%%%%%%%%%%%%%%%%%%%%%%%%%%%%%%%%%%%%%%%%%%%%%%%%%%%%%%%%%%%%%%%%%%%%%%%%%%%%%%


\end{document}
