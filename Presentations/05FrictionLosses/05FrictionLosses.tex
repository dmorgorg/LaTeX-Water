\documentclass[9pt,xcolor={svgnames, x11names},professionalfonts, mathserif]{beamer}

\usepackage{amsmath}
\usepackage{amssymb}
\usepackage{graphicx}
\usepackage{booktabs}  % for top and bottom spacing in table cells
\usepackage{mathpazo}
\usepackage{textcomp}
\usepackage{multirow}
\usepackage{cancel}
\usepackage{array}
%\usepackage{enumerate}
% \usepackage{enumitem} %causes compile error, stack size exceeded?
\usepackage{gensymb} % for \degree
\usepackage[many]{tcolorbox}
\usepackage{verbatim}
\usepackage{bm}
\usepackage{graphicx}
\usepackage{tikz}
\usepackage{tkz-linknodes}
\usepackage[export]{adjustbox} % for tight borders around photos
\usepackage{pgf} % for sait logo in beamer
\usepackage{pgfmath}
\usepgfmodule{oo}
%\usetikzlibrary{shapes,decorations,shadows,calc}
\usetikzlibrary{shadows,calc,arrows.meta}
% \usetikzlibrary{decorations.shapes}
%\usetikzlibrary{shapes.callouts}
% bloody coils
\usetikzlibrary{decorations.pathmorphing}
\usetikzlibrary{shapes.multipart}

\input{../../Includes/macros.tex}
\input{../../Includes/definedColors}
% override specific chktex warnings
% chktex-file 46 - don't use $ instead of \(, etc)
% chktex-file 36 - don't require space in front of parenthesis
% chktex-file 37 - don't require space in front of parenthesis
% chktex-file 26 - don't require space in front of punctuation
% chktex-file 1 - ignore commands followed by a space, e.g. \\ new line here
% chktex-file 9 - sometimes messes up with ( and {

\begin{comment}
Shadings are useful to give the illusion of 3D in examples and exercises presented to engineering technology students.
Vertical and horizontal shadings of rectangles are fairly straightforward to produce with the shading library included in
a recent build of Tikz.
Rotation of shaded squares is also intuitive, but rotation of a shaded rectangle appears to be both a function of the specified
rotation angle and the length to width ratio of the rectangle. This makes aligning the shading of a rotated rectangle's
fill with the stroke of a rotated rectangle a bit of an inelegant trial-and-error exercise (for me, at any rate).


\end{comment}

%http://tex.stackexchange.com/questions/33703/extract-x-y-coordinate-of-an-arbitrary-point-in-tikz
\makeatletter
\providecommand{\gettikzxy}[3]{%
	\tikz@scan@one@point\pgfutil@firstofone#1\relax
	\edef#2{\the\pgf@x}%
	\edef#3{\the\pgf@y}%
}
\makeatother


%%%%%%%%%%%%%%%%%%%%%%%%%%%%%%%% A CLASS FOR ROTATED RECTANGLES WITH A SHADED FILL %%%%%%%%%%%%%%%%%%%%%%%%%%%%%%%%%%%%%%%%
\pgfooclass{rrect}{
	% the following should be set in the calling program: \hi, \radii, \extend
	% Ax, Ay, Bx, By, outershade, innershade
	\method rrect(#1,#2,#3,#4,#5,#6) { % The constructor; everything is done in here
		\def\Ax{#1} \def\Ay{#2} \def\Bx{#3} \def\By{#4} \def\outercolor{#5} \def\innercolor{#6}
		\pgfmathparse{\Bx-\Ax} \let\deltaX\pgfmathresult
		\pgfmathparse{\By-\Ay} \let\deltaY\pgfmathresult
		\ifthenelse{\equal{\deltaX}{0.0}}
		{	% vertical rod is a special case; otherwise atan gets a div by 0 error
			\pgfmathparse{\By>\Ay} \let\ccw\pgfmathresult
			\ifthenelse{\equal{\ccw}{1}}{%
				\def\rot{90}}
			{\def\rot{-90}}}
	{	% not vertical
		\pgfmathparse{\Ax<\Bx} \let\iseast\pgfmathresult
		\ifthenelse{\equal{\iseast}{1}}{%
			\pgfmathparse{atan(\deltaY/\deltaX)} \let\rot\pgfmathresult
		} % end is east
		{
			\pgfmathparse{180+atan(\deltaY/\deltaX)} \let\rot\pgfmathresult
		} % end !east
		}
		%shading boundaries work for vertical and horizontal but otherwise ``spills'' outside it supposed boundaries,
		%particularly at multiples of 45deg
		%make some adjustments from a max at 45 to nothing at 0 or 90
		\pgfmathparse{abs(\deltaY)} \let\absdeltaY\pgfmathresult
		\pgfmathparse{abs(\deltaX)} \let\absdeltaX\pgfmathresult
		%\def\shadeangle{-42}
		\ifthenelse{\equal{\deltaX}{0.0}}
		{\def\shadeangle{0.0}}
		{\pgfmathparse{\absdeltaY > \absdeltaX} \let\foo\pgfmathresult
			\ifthenelse{\equal{\foo}{1}}
			{\pgfmathparse{90-atan(\absdeltaY/\absdeltaX)} \let\shadeangle\pgfmathresult}
			{\pgfmathparse{atan(\absdeltaY/\absdeltaX)} \let\shadeangle\pgfmathresult}
		}
		\pgfmathparse{tan(\shadeangle)} \let\fudge\pgfmathresult
		\pgfmathparse{veclen(\deltaX,\deltaY)} \let\len\pgfmathresult
		\pgfmathparse{max(\hi,\len+2*\extend)} \let\shadeboxside\pgfmathresult
		\pgfmathparse{50-25/\shadeboxside*\hi+8*\hi*\fudge/\shadeboxside} \let\mybot\pgfmathresult
		\pgfmathparse{50+25/\shadeboxside*\hi-8*\hi*\fudge/\shadeboxside} \let\mytop\pgfmathresult
		\pgfdeclareverticalshading{myshade}{100bp}{%
			color(0bp)=(\outercolor);
			color(\mybot bp)=(\outercolor);
			color(50 bp)=(\innercolor);
			color(\mytop bp)=(\outercolor);
			color(100bp)=(\outercolor)}
		\tikzset{shading=myshade}
		\begin{scope}	[rotate around = {\rot: (\Ax, \Ay)}]
			\begin{scope}
				\draw[clip, rounded corners = \scale*\radii cm] (\Ax-\extend,\Ay-\hi/2) rectangle + (\len+2*\extend,\hi);
				\shade[ shading angle=\rot] (\Ax-\extend,\Ay-\shadeboxside/2) rectangle +(\shadeboxside, \shadeboxside);
			\end{scope} %end clipping
			\draw[rounded corners=\scale*\radii cm, \stroke, \thickness] (\Ax-\extend,\Ay-\hi/2) rectangle +(\len+2*\extend,\hi);
		\end{scope}
		} % end of constructor
		} % end of rrect class

		\pgfooclass{rr}{
			\method rr (#1,#2,#3,#4,#5) { % The constructor; everything is done in here
				% Here I can get named x and y coordinates
				\def\phil{#3} \def\stroke{#4} \def\line{#5}
				\gettikzxy{(#1)}{\spx}{\spy}
				\gettikzxy{(#2)}{\epx}{\epy}
				% I'd like named points to work with
				\coordinate (Start) at (\spx, \spy);
				\coordinate (End) at (\epx, \epy);
				% Find the length between start and end. Then the angle between x axis and Diff will be the rotation to apply.
				\coordinate (Diff) at ($ (End)-(Start) $);
				\gettikzxy{(Diff)}{\dx}{\dy}
				\pgfmathparse{veclen(\dx, \dy)} \pgfmathresult
				\let\length\pgfmathresult
				\pgfmathparse{\dx==0}%
				% \ifnum low-level TeX for integers
				\ifnum\pgfmathresult=1 % \dx == 0
					\pgfmathsetmacro{\rot}{\dy > 0 ? 90 : -90}
				\else% \dx != 0
					\pgfmathsetmacro{\rot}{\dx > 0 ? atan(\dy /\dx) : 180 + atan(\dy / \dx)}
				\fi
				\begin{scope}	[rotate around = {\rot:(\spx, \spy )}]
					% \filldraw[ultra thick, fill=\phil, draw=\stroke] ($ (Start)+(0,\hi) $) arc(90:270:\hi) -- +(\length pt, 0) arc(-90:90:\hi) -- cycle;
					\filldraw[rounded corners=\scale*\radii cm, line width=\line mm, fill=\phil, draw=\stroke] (\spx-\extend cm,\spy-\hi cm) rectangle +(2*\extend cm + \length pt, 2*\hi cm);
				\end{scope}
			}
		}

		\pgfooclass{beam}{
			\method beam(#1,#2,#3,#4,#5) { % The constructor; everything is done in here
				% Here I can get named x and y coordinates
				\def\phil{#3} \def\stroke{#4} \def\line{#5}
				\gettikzxy{(#1)}{\spx}{\spy}
				\gettikzxy{(#2)}{\epx}{\epy}
				% I'd like named points to work with
				\coordinate (Start) at (\spx, \spy);
				\coordinate (End) at (\epx, \epy);
				% Find the length between start and end. Then the angle between x axis and Diff will be the rotation to apply.
				\coordinate (Diff) at ($ (End)-(Start) $);
				\gettikzxy{(Diff)}{\dx}{\dy}
				\pgfmathparse{veclen(\dx, \dy)} \pgfmathresult
				\let\length\pgfmathresult
				\pgfmathparse{\dx==0}%
				% \ifnum low-level TeX for integers
				\ifnum\pgfmathresult=1 % \dx == 0
					\pgfmathsetmacro{\rot}{\dy > 0 ? 90 : -90}
				\else% \dx != 0
					\pgfmathsetmacro{\rot}{\dx > 0 ? atan(\dy / \dx) : 180 + atan(\dy / \dx)}
				\fi
				\begin{scope}	[rotate around = {\rot:(\spx, \spy )}]
					\fill[\phil] (\spx-\extend cm,\spy-\hi cm) rectangle +(2*\extend cm + \length pt, 2*\hi cm);
					\draw[draw=\stroke, line width=\line mm] (\spx-\extend cm,\spy-\hi cm) -- +(2*\extend cm + \length pt, 0);
					\draw[draw=\stroke, line width=\line mm] (\spx-\extend cm,\spy+\hi cm) -- +(2*\extend cm + \length pt, 0);
				\end{scope}
			}
		}


\usefonttheme[onlymath]{serif}

\usepackage[absolute,overlay]{textpos}
\setlength{\TPHorizModule}{1.0cm}
\setlength{\TPVertModule}{\TPHorizModule}
\textblockorigin{0.0cm}{0.0cm}  %start all at upper left corner
\usepackage{hyperref}
\hypersetup{colorlinks=true, urlcolor=structure}
% \hypersetup{urlcolor=Blue4}

\setlength{\parskip}{\medskipamount}
\setlength{\parindent}{0pt}

\usetheme{Antibes}

\usecolortheme[rgb={0, 0.65,0.65}]{structure}
% \definecolor{structurecolor}{rgb}{0.55,0.53,0.31}
\setbeamertemplate{items}[triangle]
\setbeamertemplate{blocks}[rounded][shadow=false]
%\setbeamertemplate{background canvas}[vertical shading][bottom=Cyan1!50, middle=white, top=white, midpoint=0.05]
\setbeamertemplate{headline}{\vspace{.05cm}}
\setbeamertemplate{footline}{ \hfill \insertshorttitle \quad
	\insertshortsubtitle
	\quad \insertframenumber/\inserttotalframenumber \quad{ }\vspace{0.125cm}}
\addtobeamertemplate{footline}{\hypersetup{linkcolor=.}}{}
\setbeamertemplate{navigation symbols}{} % empty braces suppresses all navigation symbols
\setbeamercolor{frametitle}{fg=gray!5!white}
% \setbeamercolor{footline}{fg=black}
\setbeamercolor{block title}{fg=gray!15!white,bg=structure}
\setbeamercolor{block body}{bg=white, fg=black}
\setbeamercolor{background canvas}{bg=structure!2!white}
\setbeamersize{text margin left = 1cm, text margin right=1cm}
%\useinnertheme[shadow]{rounded}
%\raggedright
\setbeamerfont{block title}{family=mathserif}

\everymath{\displaystyle}
\newcounter{itemcount}

\logo{\pgfputat{\pgfxy(-11.85,-0.5)}{\pgfbox[right,base]{\includegraphics[height=1cm]{../../figs/rb_logo}}}}

%%%%%%%%%%%%%%%%%%%%%%%%%%%%%%%%%%%%%%%%%%%%%%%%%%%%%%%%%%%%%%%%%%%%%%%%%%%%%%%%
%
% \newtcbtheorem{myexam}{Example}%
% {
% 	enhanced,
% 	colback=white,
% 	colframe=structure,
% 	fonttitle=\bfseries,
% 	drop fuzzy shadow,
% 	attach boxed title to top left={yshift=-2mm, xshift=5mm},
% 	colbacktitle=structure
% 	}{exam}% then \pageref{exam:theoexample} references the theo
%
% \newtcbtheorem{myexer}{Exercise}%
% {
% 	enhanced,
% 	colback=white,
% 	colframe=structure,
% 	fonttitle=\bfseries,
% 	drop fuzzy shadow,
% 	attach boxed title to top left={yshift=-2mm, xshift=5mm},
% 	colbacktitle=structure
% 	}{exer}
%
%

% \newtcolorbox{mybox}[1][]{colback=white, top=0.125cm, bottom=0.125cm, left=0pt, right=0pt,
% colframe=structure,fonttitle=\bfseries, enhanced, drop fuzzy shadow, #1}


%test
\resetcounteronoverlays{tcb@cnt@myexam}
\resetcounteronoverlays{tcb@cnt@myexer}


% itemize indent
\setlength{\leftmargini}{1.5em}
\def\scale{1} % initialisation for pikz
\raggedright


%define title content
\title[Friction Losses]{\Huge \textcolor{white}{05 --- Friction Losses in Pipes}}
\subtitle[CIVL318]{\Large\textcolor{white}{Water Resources, CIVL318}}
\author{}
\institute{}
\date{Last revision on \today}


%%%%%%%%%%%%%%%%%%%%%%%%%%%%%%%%%%%%%%%%%%%%%%%%%%%%%%%%%%%%%%%%%%%%%%%%%%%%%%%%%%%%%%%%%%%%%%%%%%%%%%%%%%%%%%%%%%%%%%%%%%%

\begin{document}

\begin{frame}[plain]    %don't need footer on titlepage
 \titlepage
\end{frame}

%%%%%%%%%%%%%%%%%%%%%%%%%%%%%%%%%%%%%%%%%%%%%%%%%%%%%%%%%%%%%%%%%%%%%%%%%%%%%%%%%%%%%%%%%%%%%%%%%%%%%%%%%%%%%%%%%%%%%%%%%%%

\begin{frame}{Types of Flow}

 \begin{textblock*}{1\columnwidth}(0.75cm, 1cm)
  \begin{itemize}
   \item<1-> The flow in a pipe may be \textbf{laminar}, \textbf{turbulent} or \textbf{transitional}.\parb
   \item<2-> Osborne Reynolds (1842-1912) demonstrated the difference between these flow classifications
   by injecting dye into a pipe containing a flow of water: \parb

   \item<3-> For a sufficiently low velocity,  the dye \newline streak  will remain a well-defined line.
   \newline Adjacent layers in the flow do not mix; \newline this is laminar flow.
   \parb
   \item<4-> For a sufficiently high velocity, the dye \newline streak quickly spreads across
   the pipe, \newline mixing with the water in the pipe. \newline This is turbulent flow.
   \parb
   \item<5-> Between these two conditions is a range of velocities where flow is transitional
   %\end{itemize}
   %\end{minipage}
  \end{itemize}
 \end{textblock*}


 \begin{textblock*}{5cm}(7cm, 2.5cm)
  \only<3->{
   \begin{cfig}[0.5]{../../figs/05FrictionLosses/05FrictionLaminar}\end{cfig}
  }
 \end{textblock*}

 \begin{textblock*}{5cm}(7cm, 4.75cm)
  \only<4->{
   \begin{cfig}[0.5]{../../Figs/05FrictionLosses/05FrictionTurbulent}\end{cfig}
  }
 \end{textblock*}
\end{frame}


%%%%%%%%%%%%%%%%%%%%%%%%%%%%%%%%%%%%%%%%%%%%%%%%%%%%%%%%%%%%%%%%%%%%%%%%%%%%%%%%%%%%%%%%%%%%%%%%%%%%%%%%%%%%%%%%%%%%%%%%%%%

\begin{frame}{Velocity Profiles}

 \begin{textblock*}{1\columnwidth}(0.75cm, 1cm)
  \begin{itemize}
   \item<1-> Not all fluid particles have the same velocity within a pipe \par\vspace{0.5cm}
   \item<2-> If flow is laminar, the shape of the\newline velocity profile across the pipe is \newline parabolic.

   \item[]<3-> Velocity is $0$ at the pipe walls.
   \item[]<4->Maximum velocity is at the  centre\newline of the pipe and  is about twice the \newline average velocity of the flow
         \par\vspace{1cm}
         \item<5-> Turbulent flow has a flatter \newline velocity distribution.

   \item[]<6-> Especially in case of smooth pipe.

  \end{itemize}
 \end{textblock*}

 \begin{textblock*}{5cm}(6cm, 1.25cm)
  \only<2-3>{
   \begin{cfig}[0.35]{../../figs/05FrictionLosses/05FrictionVelocityProfileLaminarA}\end{cfig}
  }
 \end{textblock*}

 \begin{textblock*}{5cm}(6cm, 1.25cm)
  \only<4->{
   \begin{cfig}[0.35]{../../figs/05FrictionLosses/05FrictionVelocityProfileLaminarB}\end{cfig}
  }
 \end{textblock*}

 \begin{textblock*}{5cm}(6cm, 4.5cm)
  \only<5>{
   \begin{cfig}[0.35]{../../figs/05FrictionLosses/05FrictionVelocityProfileTurbulentA}\end{cfig}
  }
 \end{textblock*}

 \begin{textblock*}{5cm}(6cm, 4.5cm)
  \only<6->{
   \begin{cfig}[0.35]{../../figs/05FrictionLosses/05FrictionVelocityProfileTurbulentB}\end{cfig}
  }
 \end{textblock*}


\end{frame}


%%%%%%%%%%%%%%%%%%%%%%%%%%%%%%%%%%%%%%%%%%%%%%%%%%%%%%%%%%%%%%%%%%%%%%%%%%%%%%%%%%%%%%%%%%%%%%%%%%%%%%%%%%%%%%%%%%%%%%%%%%%

% \begin{frame}{Velocity Profiles}
% 	\begin{cfig}[0.6]{../../figs/05FrictionLosses/05FrictionVelocityProfileTurbulentLaminarA}\end{cfig}
% \end{frame}

%%%%%%%%%%%%%%%%%%%%%%%%%%%%%%%%%%%%%%%%%%%%%%%%%%%%%%%%%%%%%%%%%%%%%%%%%%%%%%%%%%%%%%%%%%%%%%%%%%%%%%%%%%%%%%%%%%%%%%%%%%%

\begin{frame}{Viscosity}
 \begin{itemize}
  \item The viscosity of a fluid is a measure of how easily it pours.\parb
  \item Heating a viscous fluid, such as cold oil, lowers its viscosity and allows it to flow more easily.\parb
  \item The flow of high viscosity fluids is more likely to be laminar.\parb
  \item The flow of low viscosity fluids (such as water) is more likely to be turbulent.
 \end{itemize}

\end{frame}

%%%%%%%%%%%%%%%%%%%%%%%%%%%%%%%%%%%%%%%%%%%%%%%%%%%%%%%%%%%%%%%%%%%%%%%%%%%%%%%%%%%%%%%%%%%%%%%%%%%%%%%%%%%%%%%%%%%%%%%%%%%
\begin{frame}{Reynolds Number}
 \begin{itemize}
  \item Energy losses within flow in pipes is dependent upon the type of flow.\parm
  \item Type of flow (in circular pipes) is dependent upon the density, viscosity and velocity of the fluid, and upon
        the inside diameter of the pipe.\parm
  \item The Reynolds Number is used to predict flow type:
 \end{itemize}

 \cmini[0.5]{

  \begin{mybox}[title=Reynolds Number]{}{}
   \[  N_R = \frac{vD\rho}{\eta} \]
  \end{mybox}
  \parm
  where:
  \parm
  \begin{tabular}{rl}
   $v$    & is the average flow velocity (m/s)                           \\
   \addlinespace
   $D$    & is the pipe inside diameter (m)                              \\
   \addlinespace
   $\rho$ & is the density of the fluid (kg/$\mathsf{m^3}$)              \\
   \addlinespace
   $\eta$ & is the dynamic viscosity of the fluid ($\mathsf{Pa\cdot s}$)
  \end{tabular}
 }

\end{frame}

%%%%%%%%%%%%%%%%%%%%%%%%%%%%%%%%%%%%%%%%%%%%%%%%%%%%%%%%%%%%%%%%%%%%%%%%%%%%%%%%%%%%%%%%%%%%%%%%%%%%%%%%%%%%%%%%%%%%%%%%%%%
\begin{frame}{Reynolds Number}
 \begin{center}
  The Reynolds Number is dimensionless:
 \end{center}
 \cmini[0.7]{
  \begin{mybox}{}{}
   \begin{align*}
    \mathsf{\frac{m/s \times m \times kg/m^3}{Pa\cdot s}} & = \mathsf{\frac{m/s \times m \times kg/m^3}{N/m^2\cdot s}}                             \\\\
                                                          & = \mathsf{\frac{m/s \times m \times kg/m^3}{\left(kg\cdot m/s^2\right)  / m^2\cdot s}} \\\\
                                                          & = 1
   \end{align*}
  \end{mybox}
 }
\end{frame}
%
%%%%%%%%%%%%%%%%%%%%%%%%%%%%%%%%%%%%%%%%%%%%%%%%%%%%%%%%%%%%%%%%%%%%%%%%%%%%%%%%%%%%%%%%%%%%%%%%%%%%%%%%%%%%%%%%%%%%%%%%%%%
\begin{frame}{Reynolds Number}
 \cmini[0.8]{
  \begin{itemize}
   \item Flows with high velocities and/or low viscosities tend to have turbulent flow.
   \item[]  Such flows have large Reynolds numbers.
   \item []
   \item Flows with low velocities and/or high viscosities tend to exhibit laminar flow.
   \item[] Such flows have low Reynolds numbers.
  \end{itemize}
 }
 \pause
 \cmini[0.8]{
  \begin{mybox}[title=Reynolds Number]{}{}
   \begin{align*}
    N_R       & < 2000\text{, flow is laminar}                     \\
    2000 <N_R & < 4000\text{, flow is in the `critical region'}    \\
    N_R       & > 4000\text{, flow can be assumed to be turbulent} \\
   \end{align*}
  \end{mybox}
 }

\end{frame}
%
%%%%%%%%%%%%%%%%%%%%%%%%%%%%%%%%%%%%%%%%%%%%%%%%%%%%%%%%%%%%%%%%%%%%%%%%%%%%%%%%%%%%%%%%%%%%%%%%%%%%%%%%%%%%%%%%%%%%%%%%%%%
\begin{frame}
 \cmini[0.75]{
  \begin{myexer}{}{}
   Flow is said to be in the \textbf{critical region}, with neither fully laminar or fully turbulent flow, if the Reynolds number for the
   flow is between $2000$ and $4000$.
   \parm
   Determine the range of velocities and volume flow rates for which flow is in the critical region for:
   \parm
   \begin{enumerate}
    \item water at $5$\textcelsius{} flowing in $1/2$-in copper tubing
    \item water at $95$\textcelsius{} flowing in $1/2$-in copper tubing
    \item fuel oil at $10$\textcelsius{} ($\text{sg}=0.94$, $\eta=2.4\;\mathsf{Pa\cdot s}$), \\flowing in $12$-in Schedule $40$ steel pipe
   \end{enumerate}
   \parm
  \end{myexer}
 }
\end{frame}

%%%%%%%%%%%%%%%%%%%%%%%%%%%%%%%%%%%%%%%%%%%%%%%%%%%%%%%%%%%%%%%%%%%%%%%%%%%%%%%%%%%%%%%%%%%%%%%%%%%%%%%%%%%%%%%%%%%%%%%%%%%
\begin{frame}{Darcy's Equation}
 \begin{cmini}[0.8]{
   Darcy's equation (or Darcy-Weisbach equation) is used to calculate the head loss due to friction in long, straight
   sections of circular pipe:
   }\end{cmini}
  \vspace{-0.5cm}
  \cmini[0.5]{
   \parm
   \begin{mybox}[title=Darcy's Equation]{}{}
    \[  h_L=f\times \frac{L}{D}\times \frac{v^2}{2g} \]
   \end{mybox}\parm
   where:
   \parb
   \begin{tabular}{rl}
    $h_L$             & is energy loss due to friction (m)        \\
    \addlinespace
    $f$               & is the friction factor (dimensionless)    \\
    \addlinespace
    $L$               & is the length of the pipe (m)             \\
    \addlinespace
    $D$               & is the diameter of the pipe (m)           \\
    \addlinespace
    $\tfrac{v^2}{2g}$ & is the velocity head of the flow based on \\&
    the average flow velocity in the pipe (m)\\
   \end{tabular}

  }

  \end{frame}

  %%%%%%%%%%%%%%%%%%%%%%%%%%%%%%%%%%%%%%%%%%%%%%%%%%%%%%%%%%%%%%%%%%%%%%%%%%%%%%%%%%%%%%%%%%%%%%%%%%%%%%%%%%%%%%%%%%%%%%%%%%%
  \begin{frame}{Friction Loss in Laminar Flow}
   \cmini{
    \begin{itemize}
     \item Darcy's Equation may be used for both laminar and turbulent flow.\parm
     \item Calculation of $f$, the friction factor, depends upon the type of flow.\parm
     \item For laminar flow,\parm
           \begin{center}
            \begin{mybox}[width=3cm]{}{}
             \[ f=\frac{64}{N_R} \]
            \end{mybox}
           \end{center}
           \parm
     \item Losses are independent of the pipe wall surface (velocity is zero at the pipe wall).
     \item Losses come from overcoming the frictional (shear) forces
           between adjacent layers of liquid moving at different velocities.
    \end{itemize}
   }
  \end{frame}

  %%%%%%%%%%%%%%%%%%%%%%%%%%%%%%%%%%%%%%%%%%%%%%%%%%%%%%%%%%%%%%%%%%%%%%%%%%%%%%%%%%%%%%%%%%%%%%%%%%%%%%%%%%%%%%%%%%%%%%%%%%%
  \begin{frame}
   \cmini[0.75]{
    \begin{myexam}{}{}
     \parm
     Determine the headloss due to friction in fuel oil at $10\,$\textcelsius{} flowing through $125\,\text{m}$
     of $12$-in Schedule $40$ steel pipe with an average flow velocity of $4.5\text{ m/s}$.\lb ($\text{sg}=0.94$, $\eta=2.4\;\mathsf{Pa\cdot s}$)\parm
    \end{myexam}
   }
  \end{frame}

  %%%%%%%%%%%%%%%%%%%%%%%%%%%%%%%%%%%%%%%%%%%%%%%%%%%%%%%%%%%%%%%%%%%%%%%%%%%%%%%%%%%%%%%%%%%%%%%%%%%%%%%%%%%%%%%%%%%%%%%%%%%
  \begin{frame}{Friction Loss in Turbulent Flow}

   \begin{itemize}
    \item Turbulent flow is chaotic and varying, and the value of $f$ has been determined experimentally
          for many flow situations
    \item Experiments have shown that $f$ depends upon the Reynolds number for the flow and the \textbf{relative roughness}, the ratio $D/\epsilon$ of pipe diameter $D$ to the average wall roughness $\epsilon$.\parm
          \begin{cfig}[0.5]{../../figs/05FrictionLosses/05FrictionRelativeRoughnessA}\end{cfig}
    \item Values for $f$ can be read from the Moody Diagram
   \end{itemize}
  \end{frame}

  %%%%%%%%%%%%%%%%%%%%%%%%%%%%%%%%%%%%%%%%%%%%%%%%%%%%%%%%%%%%%%%%%%%%%%%%%%%%%%%%%%%%%%%%%%%%%%%%%%%%%%%%%%%%%%%%%%%%%%%%%%
  \begin{frame}


   \begin{textblock*}{1\columnwidth}(1cm, 0cm)
    \begin{center}
     \textbf{Roughness, $\epsilon$}:
     \parb
     \begin{tabular}{rrl}
      \toprule
      Material (new, clean)          & $\qquad$ & $\epsilon$ (m)     \\
      \midrule
      \midrule
      Glass                          &          & Smooth             \\
      \midrule
      Plastic                        &          & $3.0\times10^{-7}$ \\
      \midrule
      Copper, brass, lead (tubing)   &          & $1.5\times10^{-6}$ \\
      \midrule
      Commercial steel, welded steel &          & $4.6\times10^{-5}$ \\
      \midrule
      Wrought iron                   &          & $4.6\times10^{-5}$ \\
      \midrule
      Ductile Iron - coated          &          & $1.2\times10^{-4}$ \\
      \midrule
      Ductile Iron - uncoated        &          & $2.4\times10^{-4}$ \\
      \midrule
      Concrete                       &          & $1.2\times10^{-4}$ \\
      \midrule
      Riveted steel                  &          & $1.8\times10^{-3}$ \\
      \midrule
      \bottomrule
     \end{tabular}
     \par\end{center}
     \end{textblock*}

     % 	\only<1>{
     % 		\begin{textblock*}{1\columnwidth}(0cm, 3cm)
     % 			\begin{minicbnot}[100]{0.6}{yellow}{
     % 			\textbf{Note:}
     % 			There is a mistake in the tables document for roughness handed out earlier this term. The entries for glass and plastic were mistakenly
     % 			combined and both reported as smooth. Only glass is smooth. Please correct you copy of the table to reflect the changed entry for plastic
     % 			shown here. }
     % 			\end{minicbnot}
     % 		\end{textblock*}
     % 	}

     \end{frame}

     %%%%%%%%%%%%%%%%%%%%%%%%%%%%%%%%%%%%%%%%%%%%%%%%%%%%%%%%%%%%%%%%%%%%%%%%%%%%%%%%%%%%%%%%%%%%%%%%%%%%%%%%%%%%%%%%%%%%%%%%%%%%%

     \begin{frame}{The Moody Diagram}
      \begin{textblock*}{1\columnwidth}(0.15cm, .2cm)
       \begin{cfig}[0.6]{../../figs/05FrictionLosses/moody}\end{cfig}
      \end{textblock*}
     \end{frame}

     %%%%%%%%%%%%%%%%%%%%%%%%%%%%%%%%%%%%%%%%%%%%%%%%%%%%%%%%%%%%%%%%%%%%%%%%%%%%%%%%%%%%%%%%%%%%%%%%%%%%%%%%%%%%%%%%%%%%%%%%%%%
     \begin{frame}{Accuracy in Determination of \textbf{f}}

      \begin{cmini}{
        ``It must be recognized that any high degree of accuracy in determining $f$ is not to be expected. With smooth tubing, it is true,
        good degrees of accuracy are obtainable; a probable variation in $f$ within about $\pm5\%$ and for commercial
        steel and wrought-iron piping, a variation within about $\pm10\%$. But, in the transition and rough-pipe regions, we lack the primary and obvious essential, a technique for measuring the roughness of a pipe mechanically\ldots
        \parm
        \ldots however, fairly reasonable estimates of friction loss can be made, and, fortunately, engineering problems rarely require more than this\ldots
        \parm
        The charts apply only to new and clean piping, since the rapidity of deterioration with age, dependent upon the quality of water or fluid and that
        of the pipe material, can only be guessed in most cases; and in addition to the variation in roughness there may be, in old piping, an appreciable reduction in effective
        diameter\ldots''
        }\end{cmini}

       \hfill{\footnotesize{Friction Factors for Pipe Flow, Lewis F. Moody, Princeton, N.J. (1944)}}
       \end{frame}

       %%%%%%%%%%%%%%%%%%%%%%%%%%%%%%%%%%%%%%%%%%%%%%%%%%%%%%%%%%%%%%%%%%%%%%%%%%%%%%%%%%%%%%%%%%%%%%%%%%%%%%%%%%%%%%%%%%%%%%%%%%%
       \begin{frame}{Regions of Flow Characteristics}
        \begin{mini}{
          \begin{itemize}
           \item Up to a Reynolds number of $2000$, flow is laminar\parm
           \item For a Reynolds number between $2000$ and about $4000$, conditions depend upon a number of factors 			(such as the shape of the pipe entrance, changes in section size, pressure waves, \ldots). This is the \textbf{critical region} indicated on the Moody diagram by the shaded area, where $f$ cannot be calculated.
                 \footnote{\noindent More recent experimental data suggest that for smooth pipe, flow is laminar up to around $N_R=2700$ and completely
                  turbulent for $N_R > 3000$. We shall use the more recognized ranges where flow can not be determined for $2000 <
                 N_r < 4000$.}
           \parm
           \item Above a Reynolds number of $4000$, there are two regions:
					 \parm

                 \begin{itemize}

                  \item First there is a \textbf{transition zone} of incomplete turbulence (the extent of this depends upon the relative roughness of the pipe)
                  \parm
                  \item The region of \textbf{complete turbulence}
                 \end{itemize}
          \end{itemize}
          }\end{mini}
         \end{frame}

         %%%%%%%%%%%%%%%%%%%%%%%%%%%%%%%%%%%%%%%%%%%%%%%%%%%%%%%%%%%%%%%%%%%%%%%%%%%%%%%%%%%%%%%%%%%%%%%%%%%%%%%%%%%%%%%%%%%%%%%%%%%%%

         \begin{frame}{Regions of Flow Characteristics}
          \begin{textblock*}{1\columnwidth}(0.15cm, 0.2cm)
           \begin{cfig}[0.6]{../../figs/05FrictionLosses/moodyZones}\end{cfig}
          \end{textblock*}
         \end{frame}


         %%%%%%%%%%%%%%%%%%%%%%%%%%%%%%%%%%%%%%%%%%%%%%%%%%%%%%%%%%%%%%%%%%%%%%%%%%%%%%%%%%%%%%%%%%%%%%%%%%%%%%%%%%%%%%%%%%%%%%%%%%%

         \begin{frame}

          \begin{textblock*}{1\columnwidth}(1cm, -0.375cm)
           \begin{cmini}[0.65]{
             \begin{myexam}[bottom=0pt]{}{}
							 \raggedright
               Use the Moody diagram to determine the friction factor for flow with $N_R=2\times10^6$ and a relative roughness of $1428$.
               \end{myexam}
              }\end{cmini}
              \end{textblock*}

              \begin{textblock*}{0.95\columnwidth}(0.65cm, 1.5cm)
               \only<1>{
                \begin{cfig}[0.55]{../../figs/05FrictionLosses/moody}\end{cfig}
               }
               \only<2>{
                \begin{cfig}[0.55]{../../figs/05FrictionLosses/05FrictionLossesEx3A}\end{cfig}
               }
               \only<3>{
                \begin{cfig}[0.55]{../../figs/05FrictionLosses/05FrictionLossesEx3B}\end{cfig}
               }
               \only<4>{
                \begin{cfig}[0.55]{../../figs/05FrictionLosses/05FrictionLossesEx3C}\end{cfig}
               }
               \only<5>{
                \begin{cfig}[0.55]{../../figs/05FrictionLosses/05FrictionLossesEx3D}\end{cfig}
               }
              \end{textblock*}

              \only<2->{
               \small
               \begin{textblock*}{0.98\columnwidth}(1.1cm, 1.2cm)
                \begin{cmini}[0.95]{
                  \begin{mybox}{}{}%
                    \hspace{-1cm}
                    \begin{enumerate}
                     \item  Locate $N_R=2\times10^6$ on the bottom scale
                           \only<3-5>{\item  Locate $\tfrac{D}{\epsilon}=1428$ on the right hand scale }
                           \only<4->{\item   Find the intersection of the vertical line representing $N_R=2\times10^6$ and a line
                            \textbf{following the curve} for relative roughness,  $\tfrac{D}{\epsilon}=1428$ }
                           \only<5->{\item  From this intersection point, draw a horizontal line leftwards to the left hand scale to read the friction factor,
                            $f=0.0183$}
                    \end{enumerate}
                    \end{mybox}
                   }\end{cmini}
                   \end{textblock*}
                   }
                   \end{frame}

                   %%%%%%%%%%%%%%%%%%%%%%%%%%%%%%%%%%%%%%%%%%%%%%%%%%%%%%%%%%%%%%%%%%%%%%%%%%%%%%%%%%%%%%%%%%%%%%%%%%%%%%%%%%%%%%%%%%%%%%%%%%%%%%%%%%%%%%%%%%%%%%%

                   \begin{frame}

                    \begin{textblock*}{1\columnwidth}(1cm, -0.375cm)
                     \begin{cmini}[0.85]{
                       \begin{myexam}{}{}
                         Use the Moody diagram to determine the friction factor for flow with $N_R=1.6\times 10^5$ in new clean $1/2$-in copper
                         tubing.
                       \end{myexam}
                     }\end{cmini}
                    \end{textblock*}

                    \begin{textblock*}{1\columnwidth}(0.3cm, 1.2cm)
                     \only<1>{
                      \begin{cfig}[0.58]{../../figs/05FrictionLosses/moody}\end{cfig}
                     }
                     \only<2>{
                      \begin{cfig}[0.58]{../../figs/05FrictionLosses/05FrictionLossesEx4A}\end{cfig}
                     }
                     \only<3>{
                      \begin{cfig}[0.58]{../../figs/05FrictionLosses/05FrictionLossesEx4B}\end{cfig}
                     }
                     \only<4>{
                      \begin{cfig}[0.58]{../../figs/05FrictionLosses/05FrictionLossesEx4C}\end{cfig}
                     }
                     \only<5>{
                      \begin{cfig}[0.58]{../../figs/05FrictionLosses/05FrictionLossesEx4D}\end{cfig}
                     }
                    \end{textblock*}

                    \only<2->{
                     \small
                     \begin{textblock*}{0.98\columnwidth}(1cm, 1.2cm)
                      \begin{cmini}[1]{
                        \begin{mybox}{}{}
                         \hspace{-1cm}
                         \begin{enumerate}
                          \item  Locate $N_R=1.6\times10^5$
                                \only<3->{\item  $D=13.39\text{ mm }=0.01339\text{ m}$ and $\epsilon=1.5\times10^{-6}$ so $\tfrac{D}{\epsilon}=8927$ }
                                \only<4->{\item Find the intersection of the vertical line representing $N_R=1.6\times10^5$ and a line
                                 \textbf{following the curve} for relative roughness,  $\tfrac{D}{\epsilon}=8927$ }
                                \only<5->{\item  From this intersection point, draw a horizontal line leftwards to the left hand scale to read the friction factor,
                                 \begin{center}$f=0.0172$\end{center}}
                         \end{enumerate}
											 \end{mybox}
                        }\end{cmini}
                       \end{textblock*}
                       }
                       \end{frame}

                       %%%%%%%%%%%%%%%%%%%%%%%%%%%%%%%%%%%%%%%%%%%%%%%%%%%%%%%%%%%%%%%%%%%%%%%%%%%%%%%%%%%%%%%%%%%%%%%%%%%%%%%%%%%%%%%%%%%%%%%%%%%%%%%%%%%%%%%%%%%%%%%

                       \begin{frame}

                        \begin{textblock*}{1\columnwidth}(1cm, -0.375cm)
                         \begin{cmini}[0.8]{
                           \begin{myexam}{}{}
														 \raggedright
                             A $75\text{ m}$ section of wooden flume is replaced with $54\text{-in}$ high density polyethylene (HDPE) pipe with inside diameter of
                             $1.37\text{ m}$.  The pipe is smooth and transports $190\times10^{3}\;\mathsf{ m^3/day}$. Determine the headloss due to friction in the
                             pipe, assuming an average temperature of $10$\textcelsius.\pars
                           \end{myexam}
                           }\end{cmini}
                          \end{textblock*}

                          \only<2->{
                           \small
                           \begin{textblock*}{0.98\columnwidth}(1cm, 3.2cm)
                            \begin{cmini}[1]{
                              \begin{mybox}{}{}
                               \hspace{-1cm}
                               \begin{itemize}
                                \only<2->{\item  Flow velocity:
                                 \[ v=\frac{Q}{A}=\frac{190\times10^3/24/60/60\;\mathsf{m^3/s}}{\pi(1.37\text{ m})^2/4}=1.4918\text{ m/s}} \]

                                \only<3->{\item  Reynolds number:
                                 \[ N_r=\frac{vD\rho}{\eta}=\frac{1.4918\text{ m/s}\times 1.37\text{ m}\times
                                   1000\mathsf{\,kg/m^3}}{1.30\times10^{-3}\mathsf{\ Pa\cdot s}}=1.5721\times10^6 \] }

                                  \only<4->{\item Relative roughness: Smooth pipe
                                   } \only<5->{\item Find the friction factor from the Moody diagram\ldots}
                                  \end{itemize}
                                \end{mybox}
                                  }\end{cmini}
                                  \end{textblock*}
                                  }
                                  \end{frame}

                                  %%%%%%%%%%%%%%%%%%%%%%%%%%%%%%%%%%%%%%%%%%%%%%%%%%%%%%%%%%%%%%%%%%%%%%%%%%%%%%%%%%%%%%%%%%%%%%%%%%%%%%%%%%%%%%%%%%%%%%%%%%%%%%%%%%%%%%%%%%%%%%%

                                  \begin{frame}

                                   \begin{textblock*}{1\columnwidth}(1cm, -0.375cm)
                                    \begin{cmini}[0.8]{
                                      \begin{mybox}[title=Cont'd$\ldots$]{}{}
                                        $N_R=1.5721\times10^6$, smooth pipe
                                      \end{mybox} }
                                       \end{cmini}
                                       \end{textblock*}

                                       \begin{textblock*}{1\columnwidth}(0.3cm, 1.2cm)
                                        \only<1>{
                                         \begin{cfig}[0.58]{../../figs/05FrictionLosses/moody}\end{cfig}
                                        }
                                        \only<2>{
                                         \begin{cfig}[0.58]{../../figs/05FrictionLosses/05FrictionLossesEx5B}\end{cfig}
                                        }
                                        \only<3>{
                                         \begin{cfig}[0.58]{../../figs/05FrictionLosses/05FrictionLossesEx5C}\end{cfig}
                                        }
                                        \only<4-6>{
                                         \begin{cfig}[0.58]{../../figs/05FrictionLosses/05FrictionLossesEx5D}\end{cfig}
                                        }
                                       \end{textblock*}

                                       \only<2->{
                                        \small
                                        \begin{textblock*}{1\columnwidth}(1cm,1cm)
                                         \begin{cmini}[0.8]{
                                           \begin{mybox}{}{}
                                             \hspace{-1cm}
                                             \begin{enumerate}
                                              \item  Locate $N_R=1.57\times10^6$ and the relative roughness curve for smooth pipe
                                                    \only<3->{\item Find the intersection of the vertical line representing $N_R$ and
                                                    the curve for the relative roughness of a smooth pipe. }
                                                    \only<4->{\item  From this intersection point, draw a horizontal line leftwards to the left hand scale to read the friction factor,
                                                     $f=0.011$}
                                                    \only<5->{\item  Now, determine the head loss due to friction\ldots}
                                                    \only<6>{\item [] \begin{align*}
                                                     h_L &= f\times\frac{L}{D}\times\frac{v^2}{2g}\\
                                                     &= 0.011\times\frac{75\text{ m}}{1.37\text{ m}}\times\frac{(1.4918)^2}{19.62}\text{ m}\\
                                                     &= 0.068305\text{ m}
                                                     \end{align*}}
                                             \end{enumerate}
                                             \end{mybox}
                                            }\end{cmini}
                                            \end{textblock*}
                                            }
                                            \end{frame}

                                            %%%%%%%%%%%%%%%%%%%%%%%%%%%%%%%%%%%%%%%%%%%%%%%%%%%%%%%%%%%%%%%%%%%%%%%%%%%%%%%%%%%%%%%%%%%%%%%%%%%%%%%%%%%%%%%%%%%%%%%%%%%%%%%%%%%%%%%%%%%%%%%

                                            \begin{frame}

                                             \cmini[0.8]{
                                              \begin{myexer}{}{}
                                                Ethyl alcohol at $25$\textcelsius{} flows through $1\tfrac{1}{2}\text{-in}$ Schedule 80 steel pipe at
                                                $5\text{ L/s}$.
                                                \parb
                                                Determine the pressure drop, due to friction losses, in a $125\text{ m}$ section of
                                                pipe.

                                                \end{myexer}

																								\parm
                                               What result do you get for $f$ from the Moody Diagram\ldots
                                               }

                                               \end{frame}

                                               %%%%%%%%%%%%%%%%%%%%%%%%%%%%%%%%%%%%%%%%%%%%%%%%%%%%%%%%%%%%%%%%%%%%%%%%%%%%%%%%%%%%%%%%%%%%%%%%%%%%%%%%%%%%%%%%%%%%%%%%%%%%%

                                               \begin{frame}
                                                \begin{textblock*}{1\columnwidth}(0.15cm, 0.2cm)
                                                 \cfig[0.6]{../../figs/05FrictionLosses/05FrictionLossesEx6C}
                                                \end{textblock*}
                                               \end{frame}

                                               %%%%%%%%%%%%%%%%%%%%%%%%%%%%%%%%%%%%%%%%%%%%%%%%%%%%%%%%%%%%%%%%%%%%%%%%%%%%%%%%%%%%%%%%%%%%%%%%%%%%%%%%%%%%%%%%%%%%%%%%%%%%%%%%%%%%%%%%%%%%%

                                               \begin{frame}{Swamee-Jain Formula for $\bm f$}

                                                \begin{cmini}[0.8]{
                                                  \begin{mybox}[title=Swamee-Jain Formula for Turbulent Flow]{}{}
                                                    \parb
                                                    \[ f = \frac{0.25}{\left[\log\left(\frac{1}{3.7\left(D/\epsilon\right)}+\frac{5.74}{N_R^{0.9}}\right)\right]^2} \]
                                                    \pars
																									\end{mybox}
                                                   \parb
                                                   The Swamee-Jain formula is quite accurate, yielding values for $f$ that are within $\pm 1\%$ of the Moody Diagram
                                                   value.\parb
																									 The Swamee-Jain approximates the Moody Diagram. Results for $f$ are no more accurate than those provided by the Moody Diagram.
                                                   }\end{cmini}
                                                   \end{frame}

                                                   %%%%%%%%%%%%%%%%%%%%%%%%%%%%%%%%%%%%%%%%%%%%%%%%%%%%%%%%%%%%%%%%%%%%%%%%%%%%%%%%%%%%%%%%%%%%%%%%%%%%%%%%%%%%%%%%%%%%%%%%%%%%%%%%%%%%%%%%%%%%%%

                                                   \begin{frame}
                                                    \cmini[0.8]{
																											\begin{myexer}{}{}
																												\raggedright
                                                       % 			Repeat the previous example using $3\text{-in}$ Schedule 80 steel pipe:\parb
                                                       Ethyl alcohol at $25$\textcelsius{} flows through $3\text{-in}$ Schedule 80 steel pipe at
                                                       $5\text{ L/s}$.
                                                       \parb
                                                       Determine the pressure drop, due to friction losses, in a $125\,\text{m}$ section of pipe.
                                                     %   \pars
                                                     \end{myexer}
                                                    }
                                                   \end{frame}

                                                   %%%%%%%%%%%%%%%%%%%%%%%%%%%%%%%%%%%%%%%%%%%%%%%%%%%%%%%%%%%%%%%%%%%%%%%%%%%%%%%%%%%%%%%%%%%%%%%%%%%%%%%%%%%%%%%%%%%%%%%%%%%%%%%%%%%%%%%%%%%%%%%

                                                   \begin{frame}{Two Pipes Compared:}
                                                    \begin{center}
                                                     \begin{cmini}[0.8]{
                                                       \begin{tabular}{r >{$}c<{$} >{$}c<{$} >{$}c<{$}}
                                                        \toprule
                                                        \addlinespace
                                                                      & 1\tfrac{1}{2}\text{-in} & 3\text{-in}       & \text{Ratio:}                                                     \\
                                                        \addlinespace
                                                        \toprule
                                                        \addlinespace
                                                        Diameter      & 38.1\text{ mm}          & 73.7\text{ mm}    & \approx 2                                                         \\
                                                        \addlinespace
                                                        \midrule
                                                        \addlinespace
                                                        Velocity      & 4.3851\text{ m/s}       & 1.1720\text{ m/s} & \approx\tfrac{1}{4}                                               \\
                                                        \addlinespace
                                                        \midrule
                                                        \addlinespace
                                                        Velocity Head & 0.98031\text{ m}        & 0.070015\text{ m} & \approx\tfrac{1}{14}                                              \\
                                                                      &                         &                   & \textcolor{red}{\left(\tfrac{1}{16}\tiny\text{ if double}\right)} \\
                                                        % 	 		{\left(\tfrac{1}{16} \tiny{\text{ if exactly double\right)}}} \\
                                                        \addlinespace
                                                        \midrule
                                                        \addlinespace
                                                        Head Loss     & 72.365\text{ m}         & 2.6125\text{ m}   & \approx\tfrac{1}{28}                                              \\
                                                        \addlinespace
                                                        \bottomrule
                                                       \end{tabular}
                                                       \parb
                                                       By (approximately) doubling the diameter, the velocity is reduced to one-quarter which, in turn, reduces the velocity
                                                       head to $1/14$th and losses to $1/28$th.
                                                       }\end{cmini}
                                                      \end{center}
                                                      \end{frame}

                                                      %%%%%%%%%%%%%%%%%%%%%%%%%%%%%%%%%%%%%%%%%%%%%%%%%%%%%%%%%%%%%%%%%%%%%%%%%%%%%%%%%%%%%%%%%%%%%%%%%%%%%%%%%%%%%%%%%%%%%%%%%%%%%%%%%%%%%%%%%%%%%%%

                                                      \begin{frame}
                                                       \begin{cmini}{
                                                         \begin{myexam}{}{}{
                                                          A horizontal $12\text{-in}$ Schedule $80$ steel pipe transports oil ($\text{sg}=0.85$,
                                                          $\eta=3.0\times10^{-3}\;\mathsf{Pa\cdot s}$) at $185\text{ L/s}$. The pipe has pumping stations spaced at $6.0\text{
                                                          km}$ intervals.
                                                          \parb
                                                          Determine the power required by each pump to maintain the same pressure at each pump outlet
                                                          if all losses are due to friction. \pars
                                                         }\end{myexam}
                                                        }
                                                       \end{cmini}
                                                      \end{frame}

                                                      %%%%%%%%%%%%%%%%%%%%%%%%%%%%%%%%%%%%%%%%%%%%%%%%%%%%%%%%%%%%%%%%%%%%%%%%%%%%%%%%%%%%%%%%%%%%%%%%%%%%%%%%%%%%%%%%%%%%%%%%%%%%%

                                                      \begin{frame}
                                                       \begin{textblock*}{1\columnwidth}(0.15cm, 0.2cm)
                                                        \begin{cfig}[0.6]{../../figs/05FrictionLosses/05FrictionLossesEx8}\end{cfig}
                                                       \end{textblock*}
                                                      \end{frame}



\end{document}
