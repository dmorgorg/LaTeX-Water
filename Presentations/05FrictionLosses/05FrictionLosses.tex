\documentclass[9pt,xcolor=x11names,professionalfonts, mathserif]{beamer}
\usepackage{amsmath}
\usepackage{amssymb}
\usepackage{graphicx}
\usepackage{booktabs}  % for top and bottom spacing in table cells
\usepackage{mathpazo}
\usepackage{textcomp}
\usepackage{gensymb}
\usepackage{multirow}
\usepackage{array}
\usepackage{tikz}
\usepackage{tkz-linknodes}
\usetikzlibrary{shapes,decorations,shadows}
\usetikzlibrary{decorations.shapes}
\usetikzlibrary{shapes.callouts}
\usepackage{verbatim}
\usepackage{bm}
\usepackage{graphicx}
\usefonttheme{structureitalicserif} %make titles fancy ;-)

\usepackage[absolute,overlay]{textpos}
\setlength{\TPHorizModule}{1.0cm}
\setlength{\TPVertModule}{\TPHorizModule}
\textblockorigin{0.0cm}{0.0cm}  %start all at upper left corner
\usepackage{hyperref}
\hypersetup{ urlcolor=Green4}

\setlength{\parskip}{\medskipamount}
\setlength{\parindent}{0pt}

\usetheme{Copenhagen}

\usecolortheme[RGB={0, 128, 128}]{structure}
\definecolor{structurecolor}{rgb}{0,0.5,0.51}
\setbeamertemplate{items}[ball]
\setbeamertemplate{blocks}[rounded][shadow=true]
%\setbeamertemplate{background canvas}[vertical shading][bottom=Cyan1!50, middle=white, top=white, midpoint=0.05]
\setbeamertemplate{headline}{\vspace{.15cm}}

\setbeamertemplate{headline}{\vspace{.15cm}}
\setbeamertemplate{footline}{\hfill \insertshorttitle \quad \insertshortsubtitle \quad \insertframenumber/\inserttotalframenumber \quad{ }\vspace{0.05cm}}
\setbeamertemplate{navigation symbols}{}
\setbeamercolor{frametitle}{fg=white}
\setbeamercolor{title}{fg=black}
\setbeamersize{text margin left = 1cm, text margin right=1cm}

\everymath{\displaystyle}

% counter for resuming enumerated list numbers
\newcounter{resumeenumi}
\newcommand{\suspend}{\setcounter{resumeenumi}{\theenumi}}
\newcommand{\resume}{\setcounter{enumi}{\theresumeenumi}}

\newcommand\lb{\linebreak}
\newcommand\pars{\par\smallskip}
\newcommand\parm{\par\medskip}
\newcommand\parb{\par\bigskip}

\makeatletter
\providecommand{\gettikzxy}[3]{%
	\tikz@scan@one@point\pgfutil@firstofone#1\relax
	\edef#2{\the\pgf@x}%
	\edef#3{\the\pgf@y}%
}
\makeatother



% full width colored block but color specifiable
%\cb[body bg strength]{header bg}{header text}{body text}
\newcommand{\cb}[4][15]{
	\setbeamercolor{block title}{bg = #2}
	\setbeamercolor{block body}{bg = #2!#1}
	\setbeamercolor{item projected}{bg=#2, fg=white}
	\begin{center}
		\begin{block}{#3}
			#4
		\end{block}
	\end{center}
}

% colored block with width specified
% \cbw[body bg strength]{header bg}{width}{header text}{body text}
\newcommand{\cbw}[5][15]{
	\begin{center}
		%\vspace{-0.35cm}
		\begin{minipage}{#3\textwidth}
			\setbeamercolor{block title}{bg= #2}
			\setbeamercolor{block body}{bg= #2!#1}
			\setbeamercolor{item projected}{bg=#2, fg=white}
			\begin{block}{#4}
				\raggedright
				#5
			\end{block}
		\end{minipage}
	\end{center}
}

% centered minipage with text \raggedright
%\cmini[width]{content}
\newcommand{\cmini}[2][0.8]{
	\begin{center}
		\begin{minipage}{#1\columnwidth}
			\raggedright
			#2
		\end{minipage}
	\end{center}
}

%left flushed minipage
\newcommand{\mini}[2][0.8]{
	\begin{minipage}{#1\columnwidth}
		\raggedright
		#2
	\end{minipage}
}

%left flushed minipage, top aligned
\newcommand{\minit}[2][0.8]{
	\begin{minipage}[t]{#1\columnwidth}
		\raggedright
		#2
	\end{minipage}
}

%left flushed minipage
% \newcommand{\miniT}[2][0.8]{
%  \begin{minipage}[T]{#1\columnwidth}
%   \raggedright
%   #2
%  \end{minipage}
% }

%left flushed minipage
\newcommand{\minib}[2][0.8]{
	\begin{minipage}[b]{#1\columnwidth}
		\raggedright
		#2
	\end{minipage}
}

\newcommand{\cfig}[2][1]{% centred, scaled graphic
	\begin{center}
		\includegraphics[scale=#1]{#2}
	\end{center}
}
% figure with tight border for photos
% \cfigb[saitMaroon]{borderwidth with unit}{scale}{image}
\newcommand{\cfigb}[4][structure]{
	% \usepackage{adjustbox}
	\setlength{\fboxrule}{1pt}
	\begin{center}
		\includegraphics[scale=#3, cframe= #1 #2]{#4}
	\end{center}
}

\newcommand{\imgbox}[3]{
	% \setlength{\fboxsep}{12pt}
	\includegraphics[scale=#1, cframe= structure #3]{#2}
}

% \imgboxbg[bg color=white]{scale}{path/to/img}{border color}{border, e.g. 2pt}{margin, e.g. 4pt}
\newcommand{\imgboxbg}[6][white]{
	\setlength{\fboxrule}{#5}
	\setlength{\fboxsep}{#6}
	\centering
	\fcolorbox{#4}{#1}{\includegraphics[scale=#2]{#3}}
}

\newcommand{\fig}[2][1]{% scaled graphic
	\includegraphics[scale=#1]{#2}
}

% centred framed  box black border
%\cbox[width]{content}
\newcommand{\cbox}[2][0.9]{% framed centered  box
	\setlength\fboxsep{0.042\columnwidth}
	\setlength\fboxrule{0.0015\columnwidth}
	\begin{center}
		\fcolorbox{black}{white}{
			\vspace{-0.5cm}
			\begin{minipage}{#1\columnwidth}
				\raggedright
				#2
			\end{minipage}
		}
	\end{center}
	\setlength\fboxsep{0cm}
}



\newtcolorbox{mybox}[1][]
{
	colback=white,
	top=0.25cm,
	bottom=0.25cm,
	left=0.25cm,
	right=0.25cm,
	colframe=structure,
	fonttitle=\bfseries,
	enhanced, drop fuzzy shadow,
	% attach boxed title to top left={yshift=-2mm, xshift=5mm},
	attach boxed title to top left={yshift=-2mm, xshift=5mm}, colbacktitle=structure!80!white, #1}

\newtcolorbox{plainbox}[1][]{colback=white, sharp corners, top=0.125cm, bottom=0.125cm, left=0pt, right=0pt, boxrule=0.5pt,colframe=structure,fonttitle=\bfseries, colbacktitle=structure, arc=0mm, #1}
%
\newtcbtheorem{myexam}{Example}%
{
	enhanced,
	colback=white,
	top=0.375cm,
	bottom=0.25cm,
	left=0.375cm,
	right=0.375cm,
	colframe=structure,
	fonttitle=\bfseries,
	drop fuzzy shadow,
	%description font=\mdseries\itshape,
	attach boxed title to top left={yshift=-2mm, xshift=5mm},
	colbacktitle=structure!80!white
	}{exam}% then \pageref{exer:theoexample} references the theo

\newcommand{\myexample}[2][red]{
	% \tcb\tcbset{theostyle/.style={colframe=red,colbacktitle=yellow}}
	\begin{myexam}{}{}
		\raggedright
		#2
	\end{myexam}
	% \tcbset{colframe=structure,colbacktitle=structure}
}

\newtcbtheorem{myexer}{Exercise}%
{
	enhanced,
	colback=white,
	top=0.375cm,
	bottom=0.25cm,
	left=0.375cm,
	right=0.375cm,
	colframe=structure,
	fonttitle=\bfseries,
	drop fuzzy shadow,
	%description font=\mdseries\itshape,
	attach boxed title to top left={yshift=-2mm, xshift=5mm},
	colbacktitle=structure!80!white
	}{exer}

\newcommand{\myexercise}[2][red]{
	% \tcb\tcbset{theostyle/.style={colframe=red,colbacktitle=yellow}}
	\begin{myexer}{}{}
		\raggedright
		#2
	\end{myexer}
	% \tcbset{colframe=structure,colbacktitle=structure}
}

\input{../../includes/definedColors}
\begin{document}

%define title content
\title[Friction]{Module 5 --- Friction Losses in Pipes}
\subtitle[CIVL318]{Water Resources --- CIVL318}
\author{Instructor: Dave Morgan
	\newline Email: \texttt{dave.morgan@sait.ca}
	% \newline Office: N313A
	% \newline Phone: 403.210.5910
}
\institute{Southern Alberta Institute of Technology}
\date{\tiny \textcolor{blueGrey}{Last revised on \today}}

%%%%%%%%%%%%%%%%%%%%%%%%%%%%%%%%%%%%%%%%%%%%%%%%%%%%%%%%%%%%%%%%%%%%%%%%%%%%%%%%%%%%%%%%%%%%%%%%%%%%%%%%%%%%%%%%%%%%%%%%%%%

\begin{frame}[plain]    %don't need footer on titlepage
	\titlepage
\end{frame}

%%%%%%%%%%%%%%%%%%%%%%%%%%%%%%%%%%%%%%%%%%%%%%%%%%%%%%%%%%%%%%%%%%%%%%%%%%%%%%%%%%%%%%%%%%%%%%%%%%%%%%%%%%%%%%%%%%%%%%%%%%%

\begin{frame}{Types of Flow}
	
	\begin{textblock*}{1\columnwidth}(0.5cm, 0.4cm)
		\begin{itemize}
			\item<1-> The flow in a pipe may be \textbf{laminar}, \textbf{turbulent} or \textbf{transitional}.
			\item[]
			      \item<2-> Osborne Reynolds (1842-1912) demonstrated the difference between these flow classifications
			      by injecting dye into a pipe containing a flow of water:
			\item[]
			      \item<3-> For a sufficiently low velocity,  the dye \newline streak  will remain a well-defined line.
			      \newline Adjacent layers in the flow do not mix; \newline this is laminar flow.
			\item[]\par\vspace{0.1cm}
			      \item<4-> For a sufficiently high velocity, the dye \newline streak quickly spreads across
			      the pipe, \newline mixing with the water in the pipe. \newline This is turbulent flow.
			\item []\par\vspace{0.1cm}
			      \item<5-> Between these two conditions is a range of velocities where flow is transitional
			      %\end{itemize}
			      %\end{minipage}
		\end{itemize}
	\end{textblock*}
	
	
	\begin{textblock*}{5cm}(7cm, 2cm)
		\only<3->{
			\begin{cfig}[0.55]{../../figs/05FrictionLosses/05FrictionLaminar}\end{cfig}
		}
	\end{textblock*}
	
	\begin{textblock*}{5cm}(7cm, 4.5cm)
		\only<4->{
			\begin{cfig}[0.55]{../../Figs/05FrictionLosses/05FrictionTurbulent}\end{cfig}
		}
	\end{textblock*}
\end{frame}


%%%%%%%%%%%%%%%%%%%%%%%%%%%%%%%%%%%%%%%%%%%%%%%%%%%%%%%%%%%%%%%%%%%%%%%%%%%%%%%%%%%%%%%%%%%%%%%%%%%%%%%%%%%%%%%%%%%%%%%%%%%

\begin{frame}{Velocity Profiles}
	
	\begin{textblock*}{1\columnwidth}(0.5cm, 0.75cm)
		\begin{itemize}
			\item<1-> Not all fluid particles have the same velocity within a pipe \par\vspace{0.5cm}
			\item<2-> If flow is laminar, the shape of the\newline velocity profile across the pipe is \newline parabolic.
			
			\item[]<3-> Velocity is $0$ at the pipe walls.
			\item[]<4->Maximum velocity is at the  centre\newline of the pipe and  is about twice the \newline average velocity of the flow
			      \par\vspace{1cm}
			      \item<5-> Turbulent flow has a flatter \newline velocity distribution.
			      
			\item[]<6-> Especially in case of smooth pipe.
			      
		\end{itemize}
	\end{textblock*}
	
	\begin{textblock*}{5cm}(6cm, 1.25cm)
		\only<2-3>{
			\begin{cfig}[0.35]{../../figs/05FrictionLosses/05FrictionVelocityProfileLaminarA}\end{cfig}
		}
	\end{textblock*}
	
	\begin{textblock*}{5cm}(6cm, 1.25cm)
		\only<4->{
			\begin{cfig}[0.35]{../../figs/05FrictionLosses/05FrictionVelocityProfileLaminarB}\end{cfig}
		}
	\end{textblock*}
	
	\begin{textblock*}{5cm}(6cm, 4.5cm)
		\only<5>{
			\begin{cfig}[0.35]{../../figs/05FrictionLosses/05FrictionVelocityProfileTurbulentA}\end{cfig}
		}
	\end{textblock*}
	
	\begin{textblock*}{5cm}(6cm, 4.5cm)
		\only<6->{
			\begin{cfig}[0.35]{../../figs/05FrictionLosses/05FrictionVelocityProfileTurbulentB}\end{cfig}
		}
	\end{textblock*}
	
	
\end{frame}


%%%%%%%%%%%%%%%%%%%%%%%%%%%%%%%%%%%%%%%%%%%%%%%%%%%%%%%%%%%%%%%%%%%%%%%%%%%%%%%%%%%%%%%%%%%%%%%%%%%%%%%%%%%%%%%%%%%%%%%%%%%

% \begin{frame}{Velocity Profiles}
% 	\begin{cfig}[0.6]{../../figs/05FrictionLosses/05FrictionVelocityProfileTurbulentLaminarA}\end{cfig}
% \end{frame}

%%%%%%%%%%%%%%%%%%%%%%%%%%%%%%%%%%%%%%%%%%%%%%%%%%%%%%%%%%%%%%%%%%%%%%%%%%%%%%%%%%%%%%%%%%%%%%%%%%%%%%%%%%%%%%%%%%%%%%%%%%%

\begin{frame}{Viscosity}
	\begin{itemize}
		\item The viscosity of a fluid is a measure of how easily it pours.
		\item Heating a viscous fluid, such as cold oil, lowers its viscosity and allows it to flow more easily.
		\item The flow of high viscosity fluids is more likely to be laminar.
		\item The flow of low viscosity fluids (such as water) is more likely to be turbulent.
	\end{itemize}
	
\end{frame}

%%%%%%%%%%%%%%%%%%%%%%%%%%%%%%%%%%%%%%%%%%%%%%%%%%%%%%%%%%%%%%%%%%%%%%%%%%%%%%%%%%%%%%%%%%%%%%%%%%%%%%%%%%%%%%%%%%%%%%%%%%%
\begin{frame}{Reynolds Number}
	\begin{itemize}
		\item Energy losses within flow in pipes is dependent upon the type of flow.
		\item Type of flow (in circular pipes) is dependent upon the density, viscosity and velocity of the fluid, and upon
		      the inside diameter of the pipe.
		\item The Reynolds Number is used to predict flow type:
	\end{itemize}
	\par\vspace{-0.5cm}
	\cmini[0.5]{
		\setbeamercolor{block title}{bg = dAqua}
		\setbeamercolor{block body}{bg = dAqua!15}
		\begin{block}{Reynolds Number}
			\[  N_R = \frac{vD\rho}{\eta} \]
		\end{block}
		
		where:
		\par\bigskip
		\begin{tabular}{rl}
			$v$    & is the average flow velocity (m/s)                           \\
			\addlinespace
			$D$    & is the pipe inside diameter (m)                              \\
			\addlinespace
			$\rho$ & is the density of the fluid (kg/$\mathsf{m^3}$)              \\
			\addlinespace
			$\eta$ & is the dynamic viscosity of the fluid ($\mathsf{Pa\cdot s}$) 
		\end{tabular}
	}
	
\end{frame}

%%%%%%%%%%%%%%%%%%%%%%%%%%%%%%%%%%%%%%%%%%%%%%%%%%%%%%%%%%%%%%%%%%%%%%%%%%%%%%%%%%%%%%%%%%%%%%%%%%%%%%%%%%%%%%%%%%%%%%%%%%%
\begin{frame}{Reynolds Number}
	\begin{center}
		The Reynolds Number is dimensionless:
	\end{center}
	
	\cbw[15]{dAqua}{.8}{}{
		\begin{align*}
			\mathsf{\frac{m/s \times m \times kg/m^3}{Pa\cdot s}} & = \mathsf{\frac{m/s \times m \times kg/m^3}{N/m^2\cdot s}}                             \\
			                                                      & = \mathsf{\frac{m/s \times m \times kg/m^3}{\left(kg\cdot m/s^2\right)  / m^2\cdot s}} \\
			                                                      & = 1                                                                                    
		\end{align*}
	}
	
\end{frame}
%
%%%%%%%%%%%%%%%%%%%%%%%%%%%%%%%%%%%%%%%%%%%%%%%%%%%%%%%%%%%%%%%%%%%%%%%%%%%%%%%%%%%%%%%%%%%%%%%%%%%%%%%%%%%%%%%%%%%%%%%%%%%
\begin{frame}{Reynolds Number}
	\cmini[0.8]{
		\begin{itemize}
			\item Flows with high velocities and/or low viscosities tend to have turbulent flow.
			\item[]  Such flows have large Reynolds numbers.
			\item []
			\item Flows with low velocities and/or high viscosities tend to exhibit laminar flow.
			\item[] Such flows have low Reynolds numbers.
		\end{itemize}
	}
	\pause
	\cmini[0.8]{
		\begin{cb}[10]{dAqua}{Reynolds Number}{% optional argument is !15 for box background transparency
				\begin{align*}
					N_R       & < 2000\text{, flow is laminar}                     \\
					2000 <N_R & < 4000\text{, flow is in the `critical region'}    \\
					N_R       & > 4000\text{, flow can be assumed to be turbulent} \\
				\end{align*}
				}\end{cb}
			}
			
			\end{frame}
			%
			%%%%%%%%%%%%%%%%%%%%%%%%%%%%%%%%%%%%%%%%%%%%%%%%%%%%%%%%%%%%%%%%%%%%%%%%%%%%%%%%%%%%%%%%%%%%%%%%%%%%%%%%%%%%%%%%%%%%%%%%%%%
			\begin{frame}
				\cmini[0.75]{
					\cb[10]{dAqua}{Example 1:}{
						Flow is said to be in the \textbf{critical region}, with neither fully laminar or fully turbulent flow, if the Reynolds number for the
						flow is between $2000$ and $4000$.
						\par\medskip
						Determine the range of velocities and volume flow rates for which flow is in the critical region for:
						\par\medskip
						\begin{enumerate}
							\item water at $5$\textcelsius{} flowing in $1/2$-in copper tubing
							\item water at $95$\textcelsius{} flowing in $1/2$-in copper tubing
							\item fuel oil at $10$\textcelsius{} ($\text{sg}=0.94$, $\eta=2.4\;\mathsf{Pa\cdot s}$), \\flowing in $12$-in Schedule $40$ steel pipe
						\end{enumerate}
						\par\medskip
					}
				}
			\end{frame}
			
			%%%%%%%%%%%%%%%%%%%%%%%%%%%%%%%%%%%%%%%%%%%%%%%%%%%%%%%%%%%%%%%%%%%%%%%%%%%%%%%%%%%%%%%%%%%%%%%%%%%%%%%%%%%%%%%%%%%%%%%%%%%
			\begin{frame}{Darcy's Equation}
				\begin{cmini}[0.8]{
						Darcy's equation (or Darcy-Weisbach equation) is used to calculate the head loss due to friction in long, straight
						sections of circular pipe:
						}\end{cmini}
					\vspace{-0.5cm}
					\begin{cmini}[0.5]{
							\par\medskip
							\begin{cb}[15]{dAqua}{Darcy's Equation}{% optional argument is !15 for box background transparency
									\[  h_L=f\times \frac{L}{D}\times \frac{v^2}{2g} \]
									}\end{cb}
								\vspace{-0.5cm}
								where:
								\par\bigskip
								\begin{tabular}{rl}
									$h_L$             & is energy loss due to friction (m)        \\
									\addlinespace
									$f$               & is the friction factor (dimensionless)    \\
									\addlinespace
									$L$               & is the length of the pipe (m)             \\
									\addlinespace
									$D$               & is the diameter of the pipe (m)           \\
									\addlinespace
									$\tfrac{v^2}{2g}$ & is the velocity head of the flow based on \\&
									the average flow velocity in the pipe (m)\\
								\end{tabular}
								}\end{cmini}
								
								\end{frame}
								
								%%%%%%%%%%%%%%%%%%%%%%%%%%%%%%%%%%%%%%%%%%%%%%%%%%%%%%%%%%%%%%%%%%%%%%%%%%%%%%%%%%%%%%%%%%%%%%%%%%%%%%%%%%%%%%%%%%%%%%%%%%%
								\begin{frame}{Friction Loss in Laminar Flow}
									\cmini{
										\begin{itemize}
											\item Darcy's Equation may be used for both laminar and turbulent flow
											\item Calculation of $f$, the friction factor, depends upon the type of flow
											\item For laminar flow,
											      \begin{cbw}[15]{dAqua}{0.4}{}{%
											      		\[ f=\frac{64}{N_R} \]
											      		}\end{cbw}
											      	\item[]
											      	\item Losses are independent of the pipe wall surface; losses come from overcoming the frictional (shear) forces
											      	between different layers of liquid moving at different velocities.
											      	\end{itemize}
											      	}
											      	\end{frame}
											      	
											      	%%%%%%%%%%%%%%%%%%%%%%%%%%%%%%%%%%%%%%%%%%%%%%%%%%%%%%%%%%%%%%%%%%%%%%%%%%%%%%%%%%%%%%%%%%%%%%%%%%%%%%%%%%%%%%%%%%%%%%%%%%%
											      	\begin{frame}
											      		\begin{cmini}[0.65]{
											      				\begin{cb}[10]{dAqua}{Example 2:}{
											      						Determine the headloss due to friction in fuel oil at $10$\textcelsius{} flowing through $125\text{ m}$
											      						of $12$-in Schedule $40$ steel pipe with an average flow velocity of $4.5\text{ m/s}$ ($\text{sg}=0.94$, $\eta=2.4\;\mathsf{Pa\cdot s}$).\par\medskip
											      						}\end{cb}
											      					}\end{cmini}
											      					\end{frame}
											      					
											      					%%%%%%%%%%%%%%%%%%%%%%%%%%%%%%%%%%%%%%%%%%%%%%%%%%%%%%%%%%%%%%%%%%%%%%%%%%%%%%%%%%%%%%%%%%%%%%%%%%%%%%%%%%%%%%%%%%%%%%%%%%%
											      					\begin{frame}{Friction Loss in Turbulent Flow}
											      						
											      						\begin{itemize}
											      							\item Turbulent flow is chaotic and varying and the value of $f$ has been determined experimentally
											      							      for many flow situations
											      							\item Experiments have shown that $f$ depends upon the Reynolds number for the flow and the \textbf{relative roughness},
											      							      the ratio $\tfrac{D}{\epsilon}$ of pipe diameter $D$ to the average wall roughness $\epsilon$.\par\medskip
											      							      \begin{cfig}[0.5]{../../figs/05FrictionLosses/05FrictionRelativeRoughnessA}\end{cfig}
											      							\item Values for $f$ can be read from the Moody Diagram
											      						\end{itemize}
											      					\end{frame}
											      					
											      					%%%%%%%%%%%%%%%%%%%%%%%%%%%%%%%%%%%%%%%%%%%%%%%%%%%%%%%%%%%%%%%%%%%%%%%%%%%%%%%%%%%%%%%%%%%%%%%%%%%%%%%%%%%%%%%%%%%%%%%%%%
											      					\begin{frame}
											      						
											      						
											      						\begin{textblock*}{1\columnwidth}(1cm, 0cm)
											      							\begin{center}
											      								\textbf{Roughness, $\epsilon$}:
											      								\par\bigskip
											      								\begin{tabular}{rrl}
											      									\toprule
											      									Material (new, clean)          & $\qquad$ & $\epsilon$ (m)     \\
											      									\midrule
											      									\midrule
											      									Glass                          &          & Smooth             \\
											      									\midrule
											      									Plastic                        &          & $3.0\times10^{-7}$ \\
											      									\midrule
											      									Copper, brass, lead (tubing)   &          & $1.5\times10^{-6}$ \\
											      									\midrule
											      									Commercial steel, welded steel &          & $4.6\times10^{-5}$ \\
											      									\midrule
											      									Wrought iron                   &          & $4.6\times10^{-5}$ \\
											      									\midrule
											      									Ductile Iron - coated          &          & $1.2\times10^{-4}$ \\
											      									\midrule
											      									Ductile Iron - uncoated        &          & $2.4\times10^{-4}$ \\
											      									\midrule
											      									Concrete                       &          & $1.2\times10^{-4}$ \\
											      									\midrule
											      									Riveted steel                  &          & $1.8\times10^{-3}$ \\
											      									\midrule
											      									\bottomrule
											      								\end{tabular}
											      								\par\end{center}
											      								\end{textblock*}
											      								
											      								% 	\only<1>{
											      								% 		\begin{textblock*}{1\columnwidth}(0cm, 3cm)
											      								% 			\begin{minicbnot}[100]{0.6}{yellow}{
											      								% 			\textbf{Note:}
											      								% 			There is a mistake in the tables document for roughness handed out earlier this term. The entries for glass and plastic were mistakenly
											      								% 			combined and both reported as smooth. Only glass is smooth. Please correct you copy of the table to reflect the changed entry for plastic
											      								% 			shown here. }
											      								% 			\end{minicbnot}
											      								% 		\end{textblock*}
											      								% 	}
											      								
											      								\end{frame}
											      								
											      								%%%%%%%%%%%%%%%%%%%%%%%%%%%%%%%%%%%%%%%%%%%%%%%%%%%%%%%%%%%%%%%%%%%%%%%%%%%%%%%%%%%%%%%%%%%%%%%%%%%%%%%%%%%%%%%%%%%%%%%%%%%%%
											      								
											      								\begin{frame}{The Moody Diagram}
											      									\begin{textblock*}{1\columnwidth}(0.15cm, .2cm)
											      										\begin{cfig}[0.6]{../../figs/05FrictionLosses/moody}\end{cfig}
											      									\end{textblock*}
											      								\end{frame}
											      								
											      								%%%%%%%%%%%%%%%%%%%%%%%%%%%%%%%%%%%%%%%%%%%%%%%%%%%%%%%%%%%%%%%%%%%%%%%%%%%%%%%%%%%%%%%%%%%%%%%%%%%%%%%%%%%%%%%%%%%%%%%%%%%
											      								\begin{frame}{Accuracy in Determination of \textbf{f}}
											      									
											      									\begin{cmini}{
											      											``It must be recognized that any high degree of accuracy in determining $f$ is not to be expected. With smooth tubing, it is true,
											      											good degrees of accuracy are obtainable; a probable variation in $f$ within about $\pm5$ per cent and for commercial
											      											steel and wrought-iron piping, a variation within about $\pm10$ per cent. But, in the transition and rough-pipe regions, we lack the primary and obvious essential, a technique for measuring
											      											the roughness of a pipe mechanically\ldots
											      											\par\medskip
											      											\ldots however, fairly reasonable estimates of friction loss can be made, and, fortunately, engineering problems rarely require more than this\ldots
											      											\par\medskip
											      											The charts apply only to new and clean piping, since the rapidity of deterioration with age, dependent upon the quality of water or fluid and that
											      											of the pipe material, can only be guessed in most cases; and in addition to the variation in roughness there may be, in old piping, an appreciable reduction in effective
											      											diameter\ldots''
											      											}\end{cmini}
											      										
											      										\hfill{\footnotesize{Friction Factors for Pipe Flow, Lewis F. Moody, Princeton, N.J. (1944)}}
											      										\end{frame}
											      										
											      										%%%%%%%%%%%%%%%%%%%%%%%%%%%%%%%%%%%%%%%%%%%%%%%%%%%%%%%%%%%%%%%%%%%%%%%%%%%%%%%%%%%%%%%%%%%%%%%%%%%%%%%%%%%%%%%%%%%%%%%%%%%
											      										\begin{frame}{Regions of Flow Characteristics}
											      											\begin{mini}{
											      													\begin{itemize}
											      														\item Up to a Reynolds number of $2000$, flow is laminar \newline(where the liquid's viscous forces damp out turbulence).
											      														\item[]
											      														\item For a Reynolds number between $2000$ and about $4000$, conditions depend upon a number of factors (such as the
											      														      shape of the pipe entrance, changes in section size, pressure waves, \ldots). This is the \textbf{critical region} indicated
											      														      on the Moody diagram by the shaded area, where $f$ cannot be calculated.
											      														      \footnote{Experimental data suggest that for smooth pipe, flow is laminar up to around $N_R=2700$ and completely
											      														      	turbulent for $N_R > 3000$. We shall use the more recognized ranges where flow can not be determined for $2000 <
											      														      N_r < 4000$.}
											      														\item []
											      														\item Above a Reynolds number of $4000$, there are two regions:
											      														      
											      														      \begin{itemize}
											      														      	\item []
											      														      	\item First there is a \textbf{transition zone} of incomplete turbulence (the extent of this depends upon the relative roughness of
											      														      	      the pipe)
											      														      	\item []
											      														      	\item The region of \textbf{complete turbulence}
											      														      \end{itemize}
											      													\end{itemize}
											      													}\end{mini}
											      												\end{frame}
											      												
											      												%%%%%%%%%%%%%%%%%%%%%%%%%%%%%%%%%%%%%%%%%%%%%%%%%%%%%%%%%%%%%%%%%%%%%%%%%%%%%%%%%%%%%%%%%%%%%%%%%%%%%%%%%%%%%%%%%%%%%%%%%%%%%
											      												
											      												\begin{frame}{Regions of Flow Characteristics}
											      													\begin{textblock*}{1\columnwidth}(0.15cm, 0.2cm)
											      														\begin{cfig}[0.6]{../../figs/05FrictionLosses/moodyZones}\end{cfig}
											      													\end{textblock*}
											      												\end{frame}
											      												
											      												
											      												%%%%%%%%%%%%%%%%%%%%%%%%%%%%%%%%%%%%%%%%%%%%%%%%%%%%%%%%%%%%%%%%%%%%%%%%%%%%%%%%%%%%%%%%%%%%%%%%%%%%%%%%%%%%%%%%%%%%%%%%%%%
											      												
											      												\begin{frame}
											      													
											      													\begin{textblock*}{1\columnwidth}(1cm, -0.7cm)
											      														\begin{cmini}[0.8]{
											      																\begin{cb}[10]{dAqua}{Example 3:}{
											      																		Use the Moody diagram to determine the friction factor for flow with $N_R=2\times10^6$ and a relative roughness of $1428$.\par\medskip
											      																		}\end{cb}
											      																	}\end{cmini}
											      																	\end{textblock*}
											      																	
											      																	\begin{textblock*}{0.95\columnwidth}(0.3cm, 1.2cm)
											      																		\only<1>{
											      																			\begin{cfig}[0.58]{../../figs/05FrictionLosses/moody}\end{cfig}
											      																		}
											      																		\only<2>{
											      																			\begin{cfig}[0.58]{../../figs/05FrictionLosses/05FrictionLossesEx3A}\end{cfig}
											      																		}
											      																		\only<3>{
											      																			\begin{cfig}[0.58]{../../figs/05FrictionLosses/05FrictionLossesEx3B}\end{cfig}
											      																		}
											      																		\only<4>{
											      																			\begin{cfig}[0.58]{../../figs/05FrictionLosses/05FrictionLossesEx3C}\end{cfig}
											      																		}
											      																		\only<5>{
											      																			\begin{cfig}[0.58]{../../figs/05FrictionLosses/05FrictionLossesEx3D}\end{cfig}
											      																		}
											      																	\end{textblock*}
											      																	
											      																	\only<2->{
											      																		\small
											      																		\begin{textblock*}{0.98\columnwidth}(1cm, 1.2cm)
											      																			\begin{cmini}[1]{
											      																					\begin{cb}[10]{dAqua}{}{%
											      																							\hspace{-1cm}
											      																							\begin{enumerate}
											      																								\item  Locate $N_R=2\times10^6$ on the bottom scale
											      																								      \only<3-5>{\item  Locate $\tfrac{D}{\epsilon}=1428$ on the right hand scale }
											      																								      \only<4->{\item   Find the intersection of the vertical line representing $N_R=2\times10^6$ and a line
											      																								      	\textbf{following the curve} for relative roughness,  $\tfrac{D}{\epsilon}=1428$ }
											      																								      \only<5->{\item  From this intersection point, draw a horizontal line leftwards to the left hand scale to read the friction factor,
											      																								      	$f=0.0183$}
											      																							\end{enumerate}
											      																							}\end{cb}
											      																						}\end{cmini}
											      																						\end{textblock*}
											      																						}
											      																						\end{frame}
											      																						
											      																						%%%%%%%%%%%%%%%%%%%%%%%%%%%%%%%%%%%%%%%%%%%%%%%%%%%%%%%%%%%%%%%%%%%%%%%%%%%%%%%%%%%%%%%%%%%%%%%%%%%%%%%%%%%%%%%%%%%%%%%%%%%%%%%%%%%%%%%%%%%%%%%
											      																						
											      																						\begin{frame}
											      																							
											      																							\begin{textblock*}{1\columnwidth}(1cm, -0.7cm)
											      																								\begin{cmini}[0.8]{
											      																										\begin{cb}[10]{dAqua}{Example 4:}{
											      																												Use the Moody diagram to determine the friction factor for flow with $N_R=1.6\times 10^5$ in new clean $1/2$-in copper
											      																												tubing.\par\medskip }
											      																										\end{cb} }
											      																								\end{cmini}
											      																							\end{textblock*}
											      																							
											      																							\begin{textblock*}{1\columnwidth}(0.3cm, 1.2cm)
											      																								\only<1>{
											      																									\begin{cfig}[0.58]{../../figs/05FrictionLosses/moody}\end{cfig}
											      																								}
											      																								\only<2>{
											      																									\begin{cfig}[0.58]{../../figs/05FrictionLosses/05FrictionLossesEx4A}\end{cfig}
											      																								}
											      																								\only<3>{
											      																									\begin{cfig}[0.58]{../../figs/05FrictionLosses/05FrictionLossesEx4B}\end{cfig}
											      																								}
											      																								\only<4>{
											      																									\begin{cfig}[0.58]{../../figs/05FrictionLosses/05FrictionLossesEx4C}\end{cfig}
											      																								}
											      																								\only<5>{
											      																									\begin{cfig}[0.58]{../../figs/05FrictionLosses/05FrictionLossesEx4D}\end{cfig}
											      																								}
											      																							\end{textblock*}
											      																							
											      																							\only<2->{
											      																								\small
											      																								\begin{textblock*}{0.98\columnwidth}(1cm, 1.2cm)
											      																									\begin{cmini}[1]{
											      																											\cb[10]{dAqua}{}{%
											      																												\hspace{-1cm}
											      																												\begin{enumerate}
											      																													\item  Locate $N_R=1.6\times10^5$
											      																													      \only<3->{\item  $D=13.39\text{ mm }=0.01339\text{ m}$ and $\epsilon=1.5\times10^{-6}$ so $\tfrac{D}{\epsilon}=8927$ }
											      																													      \only<4->{\item Find the intersection of the vertical line representing $N_R=1.6\times10^5$ and a line
											      																													      	\textbf{following the curve} for relative roughness,  $\tfrac{D}{\epsilon}=8927$ }
											      																													      \only<5->{\item  From this intersection point, draw a horizontal line leftwards to the left hand scale to read the friction factor,
											      																													      	\begin{center}$f=0.0172$\end{center}}
											      																												\end{enumerate}
											      																											}
											      																											}\end{cmini}
											      																										\end{textblock*}
											      																										}
											      																										\end{frame}
											      																										
											      																										%%%%%%%%%%%%%%%%%%%%%%%%%%%%%%%%%%%%%%%%%%%%%%%%%%%%%%%%%%%%%%%%%%%%%%%%%%%%%%%%%%%%%%%%%%%%%%%%%%%%%%%%%%%%%%%%%%%%%%%%%%%%%%%%%%%%%%%%%%%%%%%
											      																										
											      																										\begin{frame}
											      																											
											      																											\begin{textblock*}{1\columnwidth}(1cm, -0.7cm)
											      																												\begin{cmini}[0.8]{
											      																														\begin{cb}[10]{dAqua}{Example 5:}{
											      																																A $75\text{ m}$ section of wooden flume is replaced with $54\text{-in}$ high density polyethylene (HDPE) pipe with inside diameter of
											      																																$1.37\text{ m}$.  The pipe is smooth and transports $190\times10^{3}\;\mathsf{ m^3/day}$. Determine the headloss due to friction in the
											      																																pipe, assuming an average temperature of $10$\textcelsius.\par\smallskip }
											      																														\end{cb}
											      																														}\end{cmini}
											      																													\end{textblock*}
											      																													
											      																													\only<2->{
											      																														\small
											      																														\begin{textblock*}{0.98\columnwidth}(1cm, 3.2cm)
											      																															\begin{cmini}[1]{
											      																																	\cb[10]{dAqua}{}{%
											      																																		\hspace{-1cm}
											      																																		\begin{itemize}
											      																																			\only<2->{\item  Flow velocity:
											      																																				\[ v=\frac{Q}{A}=\frac{190\times10^3/24/60/60\;\mathsf{m^3/s}}{\pi(1.37\text{ m})^2/4}=1.4918\text{ m/s}} \]
											      																																			
											      																																			\only<3->{\item  Reynolds number:
											      																																				\[ N_r=\frac{vD\rho}{\eta}=\frac{1.4918\text{ m/s}\times 1.37\text{ m}\times
											      																																						1000\mathsf{\,kg/m^3}}{1.30\times10^{-3}\mathsf{\ Pa\cdot s}}=1.5721\times10^6 \] }
											      																																					
											      																																					\only<4->{\item Relative roughness: Smooth pipe
											      																																						} \only<5->{\item Find the friction factor from the Moody diagram\ldots}
											      																																					\end{itemize}
											      																																					}
											      																																					}\end{cmini}
											      																																					\end{textblock*}
											      																																					}
											      																																					\end{frame}
											      																																					
											      																																					%%%%%%%%%%%%%%%%%%%%%%%%%%%%%%%%%%%%%%%%%%%%%%%%%%%%%%%%%%%%%%%%%%%%%%%%%%%%%%%%%%%%%%%%%%%%%%%%%%%%%%%%%%%%%%%%%%%%%%%%%%%%%%%%%%%%%%%%%%%%%%%
											      																																					
											      																																					\begin{frame}
											      																																						
											      																																						\begin{textblock*}{1\columnwidth}(1cm, -0.7cm)
											      																																							\begin{cmini}[0.8]{
											      																																									\begin{cb}[10]{dAqua}{Example 5 cont'd}{
											      																																											$N_R=1.5721\times10^6$, smooth pipe
											      																																											}\end{cb} }
											      																																										\end{cmini}
											      																																										\end{textblock*}
											      																																										
											      																																										\begin{textblock*}{1\columnwidth}(0.3cm, 1.2cm)
											      																																											\only<1>{
											      																																												\begin{cfig}[0.58]{../../figs/05FrictionLosses/moody}\end{cfig}
											      																																											}
											      																																											\only<2>{
											      																																												\begin{cfig}[0.58]{../../figs/05FrictionLosses/05FrictionLossesEx5B}\end{cfig}
											      																																											}
											      																																											\only<3>{
											      																																												\begin{cfig}[0.58]{../../figs/05FrictionLosses/05FrictionLossesEx5C}\end{cfig}
											      																																											}
											      																																											\only<4-5>{
											      																																												\begin{cfig}[0.58]{../../figs/05FrictionLosses/05FrictionLossesEx5D}\end{cfig}
											      																																											}
											      																																										\end{textblock*}
											      																																										
											      																																										\only<2->{
											      																																											\small
											      																																											\begin{textblock*}{1\columnwidth}(1cm,0.9cm)
											      																																												\begin{cmini}[0.8]{
											      																																														\begin{cb}[10]{dAqua}{}{%
											      																																																\hspace{-1cm}
											      																																																\begin{enumerate}
											      																																																	\item  Locate $N_R=1.57\times10^6$ and the relative roughness curve for smooth pipe
											      																																																	      \only<3->{\item Find the intersection of the vertical line representing $N_R$ and
											      																																																	      the curve for the relative roughness of a smooth pipe. }
											      																																																	      \only<4->{\item  From this intersection point, draw a horizontal line leftwards to the left hand scale to read the friction factor,
											      																																																	      	$f=0.011$}
											      																																																	      \only<5->{\item  Now, determine the head loss due to friction\ldots}
											      																																																	      \only<6>{\item [] \begin{align*}
											      																																																	      	h_L &= f\times\frac{L}{D}\times\frac{v^2}{2g}\\
											      																																																	      	&= 0.011\times\frac{75\text{ m}}{1.37\text{ m}}\times\frac{(1.4918)^2}{19.62}\text{ m}\\
											      																																																	      	&= 0.068305\text{ m}
											      																																																	      	\end{align*}}
											      																																																\end{enumerate}
											      																																																}\end{cb}
											      																																															}\end{cmini}
											      																																															\end{textblock*}
											      																																															}
											      																																															\end{frame}
											      																																															
											      																																															%%%%%%%%%%%%%%%%%%%%%%%%%%%%%%%%%%%%%%%%%%%%%%%%%%%%%%%%%%%%%%%%%%%%%%%%%%%%%%%%%%%%%%%%%%%%%%%%%%%%%%%%%%%%%%%%%%%%%%%%%%%%%%%%%%%%%%%%%%%%%%%
											      																																															
											      																																															\begin{frame}
											      																																																
											      																																																\cmini[0.8]{
											      																																																	\begin{cb}[10]{dAqua}{Exercise 6:}{
											      																																																			Ethyl alcohol at $25$\textcelsius{} flows through $1\tfrac{1}{2}\text{-in}$ Schedule 80 steel pipe at
											      																																																			$5\text{ L/s}$.
											      																																																			\par\bigskip
											      																																																			Determine the pressure drop, due to friction losses, in a $125\text{ m}$ section of
											      																																																			pipe.
											      																																																			\par\smallskip
											      																																																			}\end{cb}
											      																																																		
											      																																																		
											      																																																		What result do you get for $f$ from the Moody Diagram\ldots
											      																																																		}
											      																																																		
											      																																																		\end{frame}
											      																																																		
											      																																																		%%%%%%%%%%%%%%%%%%%%%%%%%%%%%%%%%%%%%%%%%%%%%%%%%%%%%%%%%%%%%%%%%%%%%%%%%%%%%%%%%%%%%%%%%%%%%%%%%%%%%%%%%%%%%%%%%%%%%%%%%%%%%
											      																																																		
											      																																																		\begin{frame}
											      																																																			\begin{textblock*}{1\columnwidth}(0.15cm, 0.2cm)
											      																																																				\cfig[0.6]{../../figs/05FrictionLosses/05FrictionLossesEx6C}
											      																																																			\end{textblock*}
											      																																																		\end{frame}
											      																																																		
											      																																																		%%%%%%%%%%%%%%%%%%%%%%%%%%%%%%%%%%%%%%%%%%%%%%%%%%%%%%%%%%%%%%%%%%%%%%%%%%%%%%%%%%%%%%%%%%%%%%%%%%%%%%%%%%%%%%%%%%%%%%%%%%%%%%%%%%%%%%%%%%%%%
											      																																																		
											      																																																		\begin{frame}{Swamee-Jain Formula for \textbf{f}}
											      																																																			
											      																																																			\begin{cmini}[0.8]{
											      																																																					\begin{cb}[10]{dAqua}{Swamee-Jain Formula For Turbulent Flow}{
											      																																																							\par\smallskip
											      																																																							\[ f = \frac{0.25}{\left[\log\left(\frac{1}{3.7\left(D/\epsilon\right)}+\frac{5.74}{N_R^{0.9}}\right)\right]^2} \]
											      																																																							\par\smallskip
											      																																																							}\end{cb}
											      																																																						\par\bigskip
											      																																																						The Swamee-Jain formula is quite accurate, yielding values for $f$ that are within $\pm 1\%$ of the Moody Diagram
											      																																																						value.
											      																																																						}\end{cmini}
											      																																																						\end{frame}
											      																																																						
											      																																																						%%%%%%%%%%%%%%%%%%%%%%%%%%%%%%%%%%%%%%%%%%%%%%%%%%%%%%%%%%%%%%%%%%%%%%%%%%%%%%%%%%%%%%%%%%%%%%%%%%%%%%%%%%%%%%%%%%%%%%%%%%%%%%%%%%%%%%%%%%%%%%
											      																																																						
											      																																																						\begin{frame}
											      																																																							\begin{cmini}[0.8]{
											      																																																									\cb[10]{dAqua}{Example 7:}{
											      																																																										% 			Repeat the previous example using $3\text{-in}$ Schedule 80 steel pipe:\par\bigskip
											      																																																										Ethyl alcohol at $25$\textcelsius{} flows through $3\text{-in}$ Schedule 80 steel pipe at
											      																																																										$5\text{ L/s}$.
											      																																																										\par\bigskip
											      																																																										Determine the pressure drop, due to friction losses, in a $125\text{ m}$ section of
											      																																																										pipe.
											      																																																										\par\smallskip }
											      																																																								}
											      																																																							\end{cmini}
											      																																																						\end{frame}
											      																																																						
											      																																																						%%%%%%%%%%%%%%%%%%%%%%%%%%%%%%%%%%%%%%%%%%%%%%%%%%%%%%%%%%%%%%%%%%%%%%%%%%%%%%%%%%%%%%%%%%%%%%%%%%%%%%%%%%%%%%%%%%%%%%%%%%%%%%%%%%%%%%%%%%%%%%%
											      																																																						
											      																																																						\begin{frame}{Two Pipes Compared:}
											      																																																							\begin{center}
											      																																																								\begin{cmini}[0.8]{
											      																																																										\begin{tabular}{r >{$}c<{$} >{$}c<{$} >{$}c<{$}}
											      																																																											\toprule
											      																																																											\addlinespace
											      																																																											              & 1\tfrac{1}{2}\text{-in} & 3\text{-in}       & \text{Ratio:}                                                     \\
											      																																																											\addlinespace
											      																																																											\toprule
											      																																																											\addlinespace
											      																																																											Diameter      & 38.1\text{ mm}          & 73.7\text{ mm}    & \approx 2                                                         \\
											      																																																											\addlinespace
											      																																																											\midrule
											      																																																											\addlinespace
											      																																																											Velocity      & 4.3851\text{ m/s}       & 1.1720\text{ m/s} & \approx\tfrac{1}{4}                                               \\
											      																																																											\addlinespace
											      																																																											\midrule
											      																																																											\addlinespace
											      																																																											Velocity Head & 0.98031\text{ m}        & 0.070015\text{ m} & \approx\tfrac{1}{14}                                              \\
											      																																																											              &                         &                   & \textcolor{red}{\left(\tfrac{1}{16}\tiny\text{ if double}\right)} \\
											      																																																											% 	 		{\left(\tfrac{1}{16} \tiny{\text{ if exactly double\right)}}} \\
											      																																																											\addlinespace
											      																																																											\midrule
											      																																																											\addlinespace
											      																																																											Head Loss     & 72.365\text{ m}         & 2.6125\text{ m}   & \approx\tfrac{1}{28}                                              \\
											      																																																											\addlinespace
											      																																																											\bottomrule
											      																																																										\end{tabular}
											      																																																										\par\bigskip
											      																																																										By (approximately) doubling the diameter, the velocity is reduced to one-quarter which, in turn, reduces the velocity
											      																																																										head to $1/14$th and losses to $1/28$th.
											      																																																										}\end{cmini}
											      																																																									\end{center}
											      																																																									\end{frame}
											      																																																									
											      																																																									%%%%%%%%%%%%%%%%%%%%%%%%%%%%%%%%%%%%%%%%%%%%%%%%%%%%%%%%%%%%%%%%%%%%%%%%%%%%%%%%%%%%%%%%%%%%%%%%%%%%%%%%%%%%%%%%%%%%%%%%%%%%%%%%%%%%%%%%%%%%%%%
											      																																																									
											      																																																									\begin{frame}
											      																																																										\begin{cmini}[0.8]{
											      																																																												\cb[10]{dAqua}{Example 8:}{
											      																																																													A horizontal $12\text{-in}$ Schedule $80$ steel pipe transports oil ($\text{sg}=0.85$,
											      																																																													$\eta=3.0\times10^{-3}\;\mathsf{Pa\cdot s}$) at $185\text{ L/s}$. The pipe has pumping stations spaced at $6.0\text{
											      																																																													km}$ intervals.
											      																																																													\par\bigskip
											      																																																													Determine the power required by each pump to maintain the same pressure at each pump outlet
											      																																																													if all losses are due to friction. \par\smallskip
											      																																																												}
											      																																																											}
											      																																																										\end{cmini}
											      																																																									\end{frame}
											      																																																									
											      																																																									%%%%%%%%%%%%%%%%%%%%%%%%%%%%%%%%%%%%%%%%%%%%%%%%%%%%%%%%%%%%%%%%%%%%%%%%%%%%%%%%%%%%%%%%%%%%%%%%%%%%%%%%%%%%%%%%%%%%%%%%%%%%%
											      																																																									
											      																																																									\begin{frame}
											      																																																										\begin{textblock*}{1\columnwidth}(0.15cm, 0.2cm)
											      																																																											\begin{cfig}[0.6]{../../figs/05FrictionLosses/05FrictionLossesEx8}\end{cfig}
											      																																																										\end{textblock*}
											      																																																									\end{frame}
											      																																																									
											      																																																									
											      																																																									
\end{document}
