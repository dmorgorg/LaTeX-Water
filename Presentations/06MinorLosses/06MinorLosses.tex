\documentclass[9pt,xcolor={svgnames, x11names},professionalfonts, mathserif]{beamer}

\usepackage{amsmath}
\usepackage{amssymb}
\usepackage{graphicx}
\usepackage{booktabs}  % for top and bottom spacing in table cells
\usepackage{mathpazo}
\usepackage{textcomp}
\usepackage{multirow}
\usepackage{cancel}
\usepackage{array}
%\usepackage{enumerate}
% \usepackage{enumitem} %causes compile error, stack size exceeded?
\usepackage{gensymb} % for \degree
\usepackage[many]{tcolorbox}
\usepackage{verbatim}
\usepackage{bm}
\usepackage{graphicx}
\usepackage{tikz}
\usepackage{tkz-linknodes}
\usepackage[export]{adjustbox} % for tight borders around photos
\usepackage{pgf} % for sait logo in beamer
\usepackage{pgfmath}
\usepgfmodule{oo}
%\usetikzlibrary{shapes,decorations,shadows,calc}
\usetikzlibrary{shadows,calc,arrows.meta}
% \usetikzlibrary{decorations.shapes}
%\usetikzlibrary{shapes.callouts}
% bloody coils
\usetikzlibrary{decorations.pathmorphing}
\usetikzlibrary{shapes.multipart}

\input{../../Includes/macros.tex}
\input{../../Includes/definedColors}
% override specific chktex warnings
% chktex-file 46 - don't use $ instead of \(, etc)
% chktex-file 36 - don't require space in front of parenthesis
% chktex-file 37 - don't require space in front of parenthesis
% chktex-file 26 - don't require space in front of punctuation
% chktex-file 1 - ignore commands followed by a space, e.g. \\ new line here
% chktex-file 9 - sometimes messes up with ( and {

\begin{comment}
Shadings are useful to give the illusion of 3D in examples and exercises presented to engineering technology students.
Vertical and horizontal shadings of rectangles are fairly straightforward to produce with the shading library included in
a recent build of Tikz.
Rotation of shaded squares is also intuitive, but rotation of a shaded rectangle appears to be both a function of the specified
rotation angle and the length to width ratio of the rectangle. This makes aligning the shading of a rotated rectangle's
fill with the stroke of a rotated rectangle a bit of an inelegant trial-and-error exercise (for me, at any rate).


\end{comment}

%http://tex.stackexchange.com/questions/33703/extract-x-y-coordinate-of-an-arbitrary-point-in-tikz
\makeatletter
\providecommand{\gettikzxy}[3]{%
	\tikz@scan@one@point\pgfutil@firstofone#1\relax
	\edef#2{\the\pgf@x}%
	\edef#3{\the\pgf@y}%
}
\makeatother


%%%%%%%%%%%%%%%%%%%%%%%%%%%%%%%% A CLASS FOR ROTATED RECTANGLES WITH A SHADED FILL %%%%%%%%%%%%%%%%%%%%%%%%%%%%%%%%%%%%%%%%
\pgfooclass{rrect}{
	% the following should be set in the calling program: \hi, \radii, \extend
	% Ax, Ay, Bx, By, outershade, innershade
	\method rrect(#1,#2,#3,#4,#5,#6) { % The constructor; everything is done in here
		\def\Ax{#1} \def\Ay{#2} \def\Bx{#3} \def\By{#4} \def\outercolor{#5} \def\innercolor{#6}
		\pgfmathparse{\Bx-\Ax} \let\deltaX\pgfmathresult
		\pgfmathparse{\By-\Ay} \let\deltaY\pgfmathresult
		\ifthenelse{\equal{\deltaX}{0.0}}
		{	% vertical rod is a special case; otherwise atan gets a div by 0 error
			\pgfmathparse{\By>\Ay} \let\ccw\pgfmathresult
			\ifthenelse{\equal{\ccw}{1}}{%
				\def\rot{90}}
			{\def\rot{-90}}}
	{	% not vertical
		\pgfmathparse{\Ax<\Bx} \let\iseast\pgfmathresult
		\ifthenelse{\equal{\iseast}{1}}{%
			\pgfmathparse{atan(\deltaY/\deltaX)} \let\rot\pgfmathresult
		} % end is east
		{
			\pgfmathparse{180+atan(\deltaY/\deltaX)} \let\rot\pgfmathresult
		} % end !east
		}
		%shading boundaries work for vertical and horizontal but otherwise ``spills'' outside it supposed boundaries,
		%particularly at multiples of 45deg
		%make some adjustments from a max at 45 to nothing at 0 or 90
		\pgfmathparse{abs(\deltaY)} \let\absdeltaY\pgfmathresult
		\pgfmathparse{abs(\deltaX)} \let\absdeltaX\pgfmathresult
		%\def\shadeangle{-42}
		\ifthenelse{\equal{\deltaX}{0.0}}
		{\def\shadeangle{0.0}}
		{\pgfmathparse{\absdeltaY > \absdeltaX} \let\foo\pgfmathresult
			\ifthenelse{\equal{\foo}{1}}
			{\pgfmathparse{90-atan(\absdeltaY/\absdeltaX)} \let\shadeangle\pgfmathresult}
			{\pgfmathparse{atan(\absdeltaY/\absdeltaX)} \let\shadeangle\pgfmathresult}
		}
		\pgfmathparse{tan(\shadeangle)} \let\fudge\pgfmathresult
		\pgfmathparse{veclen(\deltaX,\deltaY)} \let\len\pgfmathresult
		\pgfmathparse{max(\hi,\len+2*\extend)} \let\shadeboxside\pgfmathresult
		\pgfmathparse{50-25/\shadeboxside*\hi+8*\hi*\fudge/\shadeboxside} \let\mybot\pgfmathresult
		\pgfmathparse{50+25/\shadeboxside*\hi-8*\hi*\fudge/\shadeboxside} \let\mytop\pgfmathresult
		\pgfdeclareverticalshading{myshade}{100bp}{%
			color(0bp)=(\outercolor);
			color(\mybot bp)=(\outercolor);
			color(50 bp)=(\innercolor);
			color(\mytop bp)=(\outercolor);
			color(100bp)=(\outercolor)}
		\tikzset{shading=myshade}
		\begin{scope}	[rotate around = {\rot: (\Ax, \Ay)}]
			\begin{scope}
				\draw[clip, rounded corners = \scale*\radii cm] (\Ax-\extend,\Ay-\hi/2) rectangle + (\len+2*\extend,\hi);
				\shade[ shading angle=\rot] (\Ax-\extend,\Ay-\shadeboxside/2) rectangle +(\shadeboxside, \shadeboxside);
			\end{scope} %end clipping
			\draw[rounded corners=\scale*\radii cm, \stroke, \thickness] (\Ax-\extend,\Ay-\hi/2) rectangle +(\len+2*\extend,\hi);
		\end{scope}
		} % end of constructor
		} % end of rrect class

		\pgfooclass{rr}{
			\method rr (#1,#2,#3,#4,#5) { % The constructor; everything is done in here
				% Here I can get named x and y coordinates
				\def\phil{#3} \def\stroke{#4} \def\line{#5}
				\gettikzxy{(#1)}{\spx}{\spy}
				\gettikzxy{(#2)}{\epx}{\epy}
				% I'd like named points to work with
				\coordinate (Start) at (\spx, \spy);
				\coordinate (End) at (\epx, \epy);
				% Find the length between start and end. Then the angle between x axis and Diff will be the rotation to apply.
				\coordinate (Diff) at ($ (End)-(Start) $);
				\gettikzxy{(Diff)}{\dx}{\dy}
				\pgfmathparse{veclen(\dx, \dy)} \pgfmathresult
				\let\length\pgfmathresult
				\pgfmathparse{\dx==0}%
				% \ifnum low-level TeX for integers
				\ifnum\pgfmathresult=1 % \dx == 0
					\pgfmathsetmacro{\rot}{\dy > 0 ? 90 : -90}
				\else% \dx != 0
					\pgfmathsetmacro{\rot}{\dx > 0 ? atan(\dy /\dx) : 180 + atan(\dy / \dx)}
				\fi
				\begin{scope}	[rotate around = {\rot:(\spx, \spy )}]
					% \filldraw[ultra thick, fill=\phil, draw=\stroke] ($ (Start)+(0,\hi) $) arc(90:270:\hi) -- +(\length pt, 0) arc(-90:90:\hi) -- cycle;
					\filldraw[rounded corners=\scale*\radii cm, line width=\line mm, fill=\phil, draw=\stroke] (\spx-\extend cm,\spy-\hi cm) rectangle +(2*\extend cm + \length pt, 2*\hi cm);
				\end{scope}
			}
		}

		\pgfooclass{beam}{
			\method beam(#1,#2,#3,#4,#5) { % The constructor; everything is done in here
				% Here I can get named x and y coordinates
				\def\phil{#3} \def\stroke{#4} \def\line{#5}
				\gettikzxy{(#1)}{\spx}{\spy}
				\gettikzxy{(#2)}{\epx}{\epy}
				% I'd like named points to work with
				\coordinate (Start) at (\spx, \spy);
				\coordinate (End) at (\epx, \epy);
				% Find the length between start and end. Then the angle between x axis and Diff will be the rotation to apply.
				\coordinate (Diff) at ($ (End)-(Start) $);
				\gettikzxy{(Diff)}{\dx}{\dy}
				\pgfmathparse{veclen(\dx, \dy)} \pgfmathresult
				\let\length\pgfmathresult
				\pgfmathparse{\dx==0}%
				% \ifnum low-level TeX for integers
				\ifnum\pgfmathresult=1 % \dx == 0
					\pgfmathsetmacro{\rot}{\dy > 0 ? 90 : -90}
				\else% \dx != 0
					\pgfmathsetmacro{\rot}{\dx > 0 ? atan(\dy / \dx) : 180 + atan(\dy / \dx)}
				\fi
				\begin{scope}	[rotate around = {\rot:(\spx, \spy )}]
					\fill[\phil] (\spx-\extend cm,\spy-\hi cm) rectangle +(2*\extend cm + \length pt, 2*\hi cm);
					\draw[draw=\stroke, line width=\line mm] (\spx-\extend cm,\spy-\hi cm) -- +(2*\extend cm + \length pt, 0);
					\draw[draw=\stroke, line width=\line mm] (\spx-\extend cm,\spy+\hi cm) -- +(2*\extend cm + \length pt, 0);
				\end{scope}
			}
		}


\usefonttheme[onlymath]{serif}

\usepackage[absolute,overlay]{textpos}
\setlength{\TPHorizModule}{1.0cm}
\setlength{\TPVertModule}{\TPHorizModule}
\textblockorigin{0.0cm}{0.0cm}  %start all at upper left corner
\usepackage{hyperref}
\hypersetup{colorlinks=true, urlcolor=structure}
% \hypersetup{urlcolor=Blue4}

\setlength{\parskip}{\medskipamount}
\setlength{\parindent}{0pt}

\usetheme{Antibes}

\usecolortheme[rgb={0, 0.65,0.65}]{structure}
% \definecolor{structurecolor}{rgb}{0.55,0.53,0.31}
\setbeamertemplate{items}[triangle]
\setbeamertemplate{blocks}[rounded][shadow=false]
%\setbeamertemplate{background canvas}[vertical shading][bottom=Cyan1!50, middle=white, top=white, midpoint=0.05]
\setbeamertemplate{headline}{\vspace{.05cm}}
\setbeamertemplate{footline}{ \hfill \insertshorttitle \quad
	\insertshortsubtitle
	\quad \insertframenumber/\inserttotalframenumber \quad{ }\vspace{0.125cm}}
\addtobeamertemplate{footline}{\hypersetup{linkcolor=.}}{}
\setbeamertemplate{navigation symbols}{} % empty braces suppresses all navigation symbols
\setbeamercolor{frametitle}{fg=white}
% \setbeamercolor{footline}{fg=black}
\setbeamercolor{block title}{fg=white,bg=structure}
\setbeamercolor{block body}{bg=white, fg=black}
\setbeamercolor{background canvas}{bg=white}
\setbeamersize{text margin left = 1cm, text margin right=1cm}
%\useinnertheme[shadow]{rounded}
%\raggedright
\setbeamerfont{block title}{family=mathserif}

\everymath{\displaystyle}
\newcounter{itemcount}

\logo{\pgfputat{\pgfxy(-11.85,-0.5)}{\pgfbox[right,base]{\includegraphics[height=1cm]{../../figs/rb_logo}}}}

\begin{document}

%define title content
\title[Minor Losses]{\Huge \textcolor{white}{06 --- Minor Losses}}
\subtitle[CIVL318]{\Large\textcolor{white}{Water Resources, CIVL318}}
\author{}
\institute{}
\date{Last revision on \today}

% \institute{Southern Alberta Institute of Technology}
% \date{\tiny \textcolor{blueGrey}{Last revised on \today}}

%%%%%%%%%%%%%%%%%%%%%%%%%%%%%%%%%%%%%%%%%%%%%%%%%%%%%%%%%%%%%%%%%%%%%%%%%%%%%%%%%%%%%%%%%%%%%%%%%%%%%%%%%%%%%%%%%%%%%%%%%%%

\begin{frame}[plain]    %don't need footer on titlepage
 \titlepage
\end{frame}




%%%%%%%%%%%%%%%%%%%%%%%%%%%%%%%%%%%%%%%%%%%%%%%%%%%%%%%%%%%%%%%%%%%%%%%%%%%%%%%%%%%%%%%%%%%%%%%%%%%%%%%%%%%%%%%%%%%%%%%%%%%

%%%%%%%%%%%%%%%%%%%%%%%%%%%%%%%%%%%%%%%%%%%%%%%%%%%%%%%%%%%%%%%%%%%%%%%%%%%%%%%%%%%%%%%%%%%%%%%%%%%%%%%%%%%%%%%%%%%%%%%%%%%

\begin{frame}{Minor Losses}

 \begin{itemize}
  \item Fluid flowing in a long straight pipe with uniform diameter assumes a characteristic velocity distribution.
  \item []
  \item A change in direction in some, or all, of the flow will create some turbulence, causing energy loss in
        addition to the normal loss that is due to fluid friction:
 \end{itemize}
 \par\vspace{-0.5cm}
 \begin{cfig}[0.35]{../../figs/06MinorLosses/06Minor01}\end{cfig}



\end{frame}

%%%%%%%%%%%%%%%%%%%%%%%%%%%%%%%%%%%%%%%%%%%%%%%%%%%%%%%%%%%%%%%%%%%%%%%%%%%%%%%%%%%%%%%%%%%%%%%%%%%%%%%%%%%%%%%%%%%%%%%%%%%

\begin{frame}{Minor Losses}

 \begin{textblock*}{0.5\columnwidth}(0.5cm, 1.75cm)

  \begin{itemize}

   \item Losses due to local flow disturbances due to entering or exiting a pipe, changes in pipe cross-section,
         elbows, valves, etc., are known as \textbf{minor losses}.
   \item[]
   \item In the case of a long pipe, these losses are usually small in comparison to the frictional losses in the
         length of pipe.
   \item[]
   \item In a short pipe, such as the suction pipe for a pump with a footvalve and strainer, these losses can be
         considerable and exceed the losses due to friction

  \end{itemize}

 \end{textblock*}

 \begin{textblock*}{0.5\columnwidth}(0.55\columnwidth, 1.5cm)
  \begin{cfig}[0.25]{../../figs/06MinorLosses/06Minor02}\end{cfig}
 \end{textblock*}

\end{frame}

%%%%%%%%%%%%%%%%%%%%%%%%%%%%%%%%%%%%%%%%%%%%%%%%%%%%%%%%%%%%%%%%%%%%%%%%%%%%%%%%%%%%%%%%%%%%%%%%%%%%%%%%%%%%%%%%%%%%%%%%%%%

\begin{frame}{Resistance Coefficient}

 \begin{itemize}

  \item It has been verified experimentally that the head loss due to valves and fittings is
        proportional to the velocity head of the flow.
        \begin{itemize}
         \item For check valves, this relationship is only true for flows that are sufficient to hold the valve disc in a
               fully open position; we shall assume that this is the case in the situations we investigate
        \end{itemize}
  \item[]
  \item For $h_L\propto\tfrac{v^2}{2g}$, there exists a \textbf{resistance coefficient, $\bm K$}, such that:
        \[ h_L = K\frac{v^2}{2g} \]
  \item $\bm K$-values have been calculated for a variety of situations; they depend upon the geometry of the
        fitting. In some cases, $K$ may also depend upon the velocity of the flow.

 \end{itemize}

\end{frame}

%%%%%%%%%%%%%%%%%%%%%%%%%%%%%%%%%%%%%%%%%%%%%%%%%%%%%%%%%%%%%%%%%%%%%%%%%%%%%%%%%%%%%%%%%%%%%%%%%%%%%%%%%%%%%%%%%%%%%%%%%%%

\begin{frame}{Sudden Contraction}

 \begin{itemize}

  \item With a sudden contraction in pipe diameter, there is a marked decrease in pressure due to the increase in
        velocity and due to loss of energy caused by turbulence at the contraction.
  \item []
  \item The resistance to flow due to a sudden contraction depends upon the ratio of the diameters of the pipes
        before and after the contraction. Also, this is a case where the value of $K$ depends upon the flow. Values for $K$
        can be found in tables.
  \item []
  \item The velocity related to the head loss is in the smaller, downstream pipe.

 \end{itemize}
 \begin{cfig}[0.35]{../../figs/06MinorLosses/06Minor03}\end{cfig}

\end{frame}

%%%%%%%%%%%%%%%%%%%%%%%%%%%%%%%%%%%%%%%%%%%%%%%%%%%%%%%%%%%%%%%%%%%%%%%%%%%%%%%%%%%%%%%%%%%%%%%%%%%%%%%%%%%%%%%%%%%%%%%%%%%

\begin{frame}{Sudden Contraction }

 \begin{textblock*}{1\textwidth}(0cm, 2cm)
  \small
  \hspace{-0.5cm}
  \begin{tabular}{>{$}c<{$} >{$}c<{$} >{$}c<{$} >{$}c<{$}>{$}c<{$}>{$}c<{$} >{$}c<{$} >{$}c<{$} >{$}c<{$} >{$}c<{$}
   >{$}c<{$} }
   \toprule
   \addlinespace
   && \multicolumn{9}{c}{Velocity, $v$} \\
   D_1/D_2 &\qquad& 0.6\,\text{m/s} & 1.2\,\text{m/s} & 1.8\,\text{m/s} & 2.4\,\text{m/s}& 3\,\text{m/s} &
   4.5\,\text{m/s} & 6\,\text{m/s} & 9\,\text{m/s} & 12\,\text{m/s} \\
   \midrule 1 && 0.0 & 0.0 & 0.0 & 0.0 & 0.0 & 0.0
   & 0.0  & 0.0 & 0.0\\
   1.1    &   & 0.03 & 0.04 & 0.04 & 0.04 & 0.04 & 0.04 & 0.05 & 0.05 & 0.06 \\
   1.2    &   & 0.07 & 0.07 & 0.07 & 0.07 & 0.08 & 0.08 & 0.09 & 0.10 & 0.11 \\
   1.4    &   & 0.17 & 0.17 & 0.17 & 0.17 & 0.18 & 0.18 & 0.18 & 0.19 & 0.20 \\
   1.6    &   & 0.26 & 0.26 & 0.26 & 0.26 & 0.26 & 0.25 & 0.25 & 0.25 & 0.24 \\
   1.8    &   & 0.34 & 0.34 & 0.34 & 0.33 & 0.33 & 0.32 & 0.31 & 0.29 & 0.27 \\
   2.0    &   & 0.38 & 0.37 & 0.37 & 0.36 & 0.36 & 0.34 & 0.33 & 0.31 & 0.29 \\
   2.2    &   & 0.40 & 0.40 & 0.39 & 0.39 & 0.38 & 0.37 & 0.35 & 0.33 & 0.30 \\
   2.5    &   & 0.42 & 0.42 & 0.41 & 0.40 & 0.40 & 0.38 & 0.38 & 0.34 & 0.31 \\
   3.0    &   & 0.44 & 0.44 & 0.43 & 0.42 & 0.42 & 0.40 & 0.39 & 0.36 & 0.33 \\
   4.0    &   & 0.47 & 0.46 & 0.45 & 0.45 & 0.44 & 0.42 & 0.41 & 0.37 & 0.34 \\
   5.0    &   & 0.48 & 0.47 & 0.47 & 0.46 & 0.45 & 0.44 & 0.42 & 0.38 & 0.35 \\
   10.0   &   & 0.49 & 0.48 & 0.48 & 0.47 & 0.46 & 0.45 & 0.43 & 0.40 & 0.36 \\
   \infty &   & 0.49 & 0.48 & 0.48 & 0.47 & 0.47 & 0.45 & 0.44 & 0.41 & 0.38 \\
   \bottomrule

  \end{tabular}
 \end{textblock*}
\end{frame}

%%%%%%%%%%%%%%%%%%%%%%%%%%%%%%%%%%%%%%%%%%%%%%%%%%%%%%%%%%%%%%%%%%%%%%%%%%%%%%%%%%%%%%%%%%%%%%%%%%%%%%%%%%%%%%%%%%%%%%%%%%%

\begin{frame}
 \centering
 \begin{myexam}[width=0.7\textwidth]{}{}
  \raggedright
  Determine the head loss that occurs when $100\text{ L/min}$ of fluid flows from $3\text{-in}$ Type K copper tube $(D=73.84\,\text{mm})$ into
  $1\text{-in}$ Type K copper tube $(D=25.27\,\text{mm})$ through a sudden contraction.
 \end{myexam}

\end{frame}

%%%%%%%%%%%%%%%%%%%%%%%%%%%%%%%%%%%%%%%%%%%%%%%%%%%%%%%%%%%%%%%%%%%%%%%%%%%%%%%%%%%%%%%%%%%%%%%%%%%%%%%%%%%%%%%%%%%%%%%%%%%

\begin{frame}{Gradual Contraction}

 \begin{itemize}

  \item To reduce losses, sudden changes of cross-section should be avoided where possible.
  \item The resistance to flow due to a gradual contraction is given by:
        \begin{equation*}
         K=
         \begin{cases}
          0.5\sqrt{\sin\frac{\theta}{2}}\left(1-\left(\frac{D_2}{D_1}\right)^2\right), & 45\degree < \theta \le 180\degree
          \\
          0.8\sin\frac{\theta}{2}\left(1-\left(\frac{D_2}{D_1}\right)^2\right),        & 15\degree < \theta \le 45\degree
         \end{cases}
        \end{equation*}
        where $D_2$ is the smaller (downstream) diameter of the pipe after the contraction and $D_1$ is the diameter of the
        larger (upstream) pipe and \textbf{the head loss is based on the velocity of the smaller pipe}.
  \item Below a cone angle of $15\degree$, the value of $K$ starts to increase again as friction losses due to the
        length of the contraction become significant.	Tables should be consulted to determine the resistance coefficient for
        these contractions.
  \item Rounded ends of the conical transition can reduce the resistance coefficient further still.

 \end{itemize}
 \par\vspace{-0.5cm}
 \begin{cfig}[0.25]{../../figs/06MinorLosses/06Minor04}\end{cfig}

\end{frame}

%%%%%%%%%%%%%%%%%%%%%%%%%%%%%%%%%%%%%%%%%%%%%%%%%%%%%%%%%%%%%%%%%%%%%%%%%%%%%%%%%%%%%%%%%%%%%%%%%%%%%%%%%%%%%%%%%%%%%%%%%%%

\begin{frame}
 \centering
 \begin{myexam}[width=0.95\textwidth]{}{}
  \raggedright
  Determine the head loss for a gradual contraction from $4\text{-in}$~Schedule~$80$~pipe $(D=97.2\,\text{mm})$ to a
  $1\tfrac{1}{2}\text{-in}$ Schedule $80$ pipe $(D=38.1\,\text{mm})$ with a cone angle of $76\degree$. The flow is $450\text{ L/min}$.
 \end{myexam}

\end{frame}

%%%%%%%%%%%%%%%%%%%%%%%%%%%%%%%%%%%%%%%%%%%%%%%%%%%%%%%%%%%%%%%%%%%%%%%%%%%%%%%%%%%%%%%%%%%%%%%%%%%%%%%%%%%%%%%%%%%%%%%%%%%

\begin{frame}{Sudden Enlargement}

 \begin{itemize}
  \item Velocity abruptly decreases, causing turbulence.
 \end{itemize}
 \begin{cfig}[0.35]{../../figs/06MinorLosses/06Minor05}\end{cfig}

\end{frame}
%%%%%%%%%%%%%%%%%%%%%%%%%%%%%%%%%%%%%%%%%%%%%%%%%%%%%%%%%%%%%%%%%%%%%%%%%%%%%%%%%%%%%%%%%%%%%%%%%%%%%%%%%%%%%%%%%%%%%%%%%%%

\begin{frame}{Sudden Enlargement}

 \begin{tabular}{>{$}c<{$} >{$}c<{$} >{$}c<{$} >{$}c<{$} >{$}c<{$} >{$}c<{$} >{$}c<{$} >{$}c<{$} >{$}c<{$} }
  \toprule
  \addlinespace
  && \multicolumn{7}{c}{Velocity, $v$} \\
  D_2/D_1 &\qquad& 0.6\,\text{m/s} & 1.2\,\text{m/s} & 3\,\text{m/s} & 4.5\,\text{m/s}  & 6\,\text{m/s}  &
  9\,\text{m/s} & 12\,\text{m/s} \\ \midrule
  1      &   & 0.0  & 0.0  & 0.0  & 0.0  & 0.0  & 0.0  & 0.0  \\
  1.2    &   & 0.11 & 0.10 & 0.09 & 0.09 & 0.09 & 0.09 & 0.08 \\
  1.4    &   & 0.26 & 0.25 & 0.23 & 0.22 & 0.22 & 0.21 & 0.20 \\
  1.6    &   & 0.40 & 0.38 & 0.35 & 0.34 & 0.33 & 0.32 & 0.32 \\
  1.8    &   & 0.51 & 0.48 & 0.45 & 0.43 & 0.42 & 0.41 & 0.40 \\
  2.0    &   & 0.60 & 0.56 & 0.52 & 0.51 & 0.50 & 0.48 & 0.47 \\
  2.5    &   & 0.74 & 0.70 & 0.65 & 0.63 & 0.62 & 0.60 & 0.58 \\
  3.0    &   & 0.83 & 0.78 & 0.73 & 0.70 & 0.69 & 0.67 & 0.65 \\
  4.0    &   & 0.92 & 0.87 & 0.80 & 0.78 & 0.76 & 0.74 & 0.72 \\
  5.0    &   & 0.96 & 0.91 & 0.84 & 0.82 & 0.80 & 0.77 & 0.75 \\
  10.0   &   & 1.00 & 0.96 & 0.89 & 0.86 & 0.84 & 0.82 & 0.80 \\
  \infty &   & 1.00 & 0.98 & 0.91 & 0.88 & 0.86 & 0.83 & 0.81 \\
  \bottomrule
 \end{tabular}

\end{frame}

%%%%%%%%%%%%%%%%%%%%%%%%%%%%%%%%%%%%%%%%%%%%%%%%%%%%%%%%%%%%%%%%%%%%%%%%%%%%%%%%%%%%%%%%%%%%%%%%%%%%%%%%%%%%%%%%%%%%%%%%%%%

\begin{frame}

 \begin{myexam}{}{}
  \raggedright
  Determine the head loss that occurs when $100\text{ L/min}$ flows from $1\text{-in}$ Type K copper tube $(D=25.27\,\text{mm})$ into
  $3\text{-in}$ Type K copper tube $(D=73.48\,\text{mm})$ through a sudden enlargement.
 \end{myexam}

\end{frame}

%%%%%%%%%%%%%%%%%%%%%%%%%%%%%%%%%%%%%%%%%%%%%%%%%%%%%%%%%%%%%%%%%%%%%%%%%%%%%%%%%%%%%%%%%%%%%%%%%%%%%%%%%%%%%%%%%%%%%%%%%%%

\begin{frame}{Gradual Enlargement}

 \begin{itemize}

  \item When the enlargement is less abrupt, energy losses are reduced.
  \item []
  \item The enlargement is made with a conical section; the sloping walls of the cone reduce the amount of separation
        of flow from the pipe wall.
  \item []
  \item $K$ is dependent upon upon the ratio of the diameters and the cone angle.
  \item []
  \item Enlargements are used for pressure recovery; they are also known as \textbf{diffusers}
 \end{itemize}
 \begin{cfig}[0.35]{../../figs/06MinorLosses/06Minor06}\end{cfig}

\end{frame}

%%%%%%%%%%%%%%%%%%%%%%%%%%%%%%%%%%%%%%%%%%%%%%%%%%%%%%%%%%%%%%%%%%%%%%%%%%%%%%%%%%%%%%%%%%%%%%%%%%%%%%%%%%%%%%%%%%%%%%%%%%%

\begin{frame}{Gradual Enlargement }
 \small
 \begin{textblock*}{1\textwidth}(0.25cm, 2cm)

  \begin{tabular}{>{$}c<{$} >{$}c<{$} >{$}c<{$} >{$}c<{$} >{$}c<{$} >{$}c<{$} >{$}c<{$} >{$}c<{$} >{$}c<{$} >{$}c<{$} >{$}c<{$} >{$}c<{$} >{$}c<{$} }
   \toprule
   D_2/D_1 & 2^\circ & 6^\circ & 10^\circ & 15^\circ & 20^\circ & 25^\circ & 30^\circ & 35^\circ & 40^\circ & 45^\circ & 50^\circ & 60^\circ \\
   \midrule
   1.1     & 0.01    & 0.01    & 0.03     & 0.05     & 0.10     & 0.13     & 0.16     & 0.81     & 0.19     & 0.20 & 0.21 & 0.23 \\

   1.2     & 0.02    & 0.02    & 0.04     & 0.09     & 0.16     & 0.21     & 0.25     & 0.29     & 0.31     & 0.33 & 0.35 & 0.37 \\

   1.4     & 0.02    & 0.03    & 0.06     & 0.12     & 0.23     & 0.30     & 0.36     & 0.41     & 0.44     & 0.47 & 0.50 & 0.53 \\

   1.6     & 0.03    & 0.04    & 0.07     & 0.14     & 0.26     & 0.35     & 0.42     & 0.47     & 0.51     & 0.54 & 0.57 & 0.61 \\

   1.8     & 0.03    & 0.04    & 0.07     & 0.15     & 0.28     & 0.37     & 0.44     & 0.50     & 0.54     & 0.58 & 0.61 & 0.65 \\

   2.0     & 0.03    & 0.04    & 0.07     & 0.16     & 0.29     & 0.38     & 0.46     & 0.52     & 0.56     & 0.60 & 0.63 & 0.68 \\

   2.5     & 0.03    & 0.04    & 0.08     & 0.16     & 0.30     & 0.39     & 0.48     & 0.54     & 0.58     & 0.62 & 0.65 & 0.70 \\

   3       & 0.03    & 0.04    & 0.08     & 0.16     & 0.31     & 0.40     & 0.48     & 0.55     & 0.59     & 0.63 & 0.66 & 0.71 \\

   \infty  & 0.03    & 0.05    & 0.08     & 0.16     & 0.31     & 0.40     & 0.49     & 0.56     & 0.60     & 0.64 & 0.67 & 0.72 \\
   \bottomrule
  \end{tabular}
 \end{textblock*}
\end{frame}

%%%%%%%%%%%%%%%%%%%%%%%%%%%%%%%%%%%%%%%%%%%%%%%%%%%%%%%%%%%%%%%%%%%%%%%%%%%%%%%%%%%%%%%%%%%%%%%%%%%%%%%%%%%%%%%%%%%%%%%%%%%

\begin{frame}{Entrance Loss}
 \begin{minipage}{0.45\columnwidth}
  \begin{itemize}
   \item When fluid flows from a tank into a pipe, a form of contraction occurs. The fluid accelerates from zero
         velocity to the average flow velocity in the pipe.\parm
   \item The values for $K$ depend upon the geometry of the entrance. \\Each geometry has an associated resistance
         coefficient, $K$. \par
   \item Then, $$h_L=K\frac{v^2}{2g}$$ where $v$ is based on the velocity {\bfseries in the pipe}.
  \end{itemize}
 \end{minipage}
 \hfill
 \begin{minipage}{0.5\columnwidth}
  \begin{cfig}[0.35]{../../figs/06MinorLosses/06Minor07}\end{cfig}
 \end{minipage}

\end{frame}

%%%%%%%%%%%%%%%%%%%%%%%%%%%%%%%%%%%%%%%%%%%%%%%%%%%%%%%%%%%%%%%%%%%%%%%%%%%%%%%%%%%%%%%%%%%%%%%%%%%%%%%%%%%%%%%%%%%%%%%%%%%

\begin{frame}{Entrance Loss --- Inward-Projecting}

 \begin{itemize}
  \item An inward-projecting entrance is the most severe in terms of turbulence and resultant energy loss.
  \item []
  \item Mott (the course text) recommends a value of $K=1.0$ to be conservative. Crane Co.'s Technical Paper
        uses $K=0.78$. \\(Use the value provided in the question.)
 \end{itemize}
 \par\vspace{-0.5cm}
 \begin{cfig}[0.6]{../../figs/06MinorLosses/06Minor07inwardProjecting}\end{cfig}

\end{frame}

%%%%%%%%%%%%%%%%%%%%%%%%%%%%%%%%%%%%%%%%%%%%%%%%%%%%%%%%%%%%%%%%%%%%%%%%%%%%%%%%%%%%%%%%%%%%%%%%%%%%%%%%%%%%%%%%%%%%%%%%%%%

\begin{frame}{Entrance Loss --- Square-Edged and Chamfered}
 \begin{textblock*}{1\columnwidth}(1cm, 1cm)
  \begin{itemize}
   \item Square-edged and chamfered entrances (inlets) have $K$ values of $0.5$ and $0.25$ respectively

   \item These entrances reduce the flow separation that occurs at the sharp corner of an inward-projecting entrance and
         the vena contracta effect is reduced
  \end{itemize}
 \end{textblock*}

 \begin{textblock*}{0.55\columnwidth}(1cm, 2.75cm)
  \begin{cfig}[0.5]{../../figs/06MinorLosses/06Minor07square}\end{cfig}

  \centering
  \par\vspace{-0.75cm}
  Square-edged
 \end{textblock*}

 \begin{textblock*}{0.55\columnwidth}(0.5\columnwidth, 5.5cm)
  \begin{cfig}[0.5]{../../figs/06MinorLosses/06Minor07chamfer}\end{cfig}
  \centering
  \par\vspace{-0.75cm}
  Chamfered
 \end{textblock*}

\end{frame}

%%%%%%%%%%%%%%%%%%%%%%%%%%%%%%%%%%%%%%%%%%%%%%%%%%%%%%%%%%%%%%%%%%%%%%%%%%%%%%%%%%%%%%%%%%%%%%%%%%%%%%%%%%%%%%%%%%%%%%%%%%%

\begin{frame}{Entrance Loss --- Rounded Entrance}

 \begin{itemize}
  \item Resistance coefficients for rounded entrances are calculated using a ratio of radius $r$ of the rounded
        entrance to the diameter of the pipe, $D$
  \item []
  \item A well-rounded entrance has a very small resistance coefficient, and a fairly insignificant energy loss
 \end{itemize}
 \par\vspace{-0.5cm}
 \begin{cfig}[0.6]{../../figs/06MinorLosses/06Minor07rounded}\end{cfig}

\end{frame}

%%%%%%%%%%%%%%%%%%%%%%%%%%%%%%%%%%%%%%%%%%%%%%%%%%%%%%%%%%%%%%%%%%%%%%%%%%%%%%%%%%%%%%%%%%%%%%%%%%%%%%%%%%%%%%%%%%%%%%%%%%%

\begin{frame}{Entrance Losses Summary}
 \small
 \begin{textblock*}{0.65\columnwidth}(1cm, 0.5cm)
  \begin{cfig}[0.4]{../../figs/06MinorLosses/06Minor07}\end{cfig}
 \end{textblock*}
 \begin{textblock*}{5cm}(0.7\columnwidth, 1.8cm)
  Inward-projecting:
  $K=0.78\text{ --- }1.0$
 \end{textblock*}

 \begin{textblock*}{5cm}(0.7\columnwidth, 3.8cm)
  Square-edged inlet:
  $K=0.5$
 \end{textblock*}

 \begin{textblock*}{5cm}(0.7\columnwidth, 5.8cm)
  Chamfered Inlet:
  $K=0.25$
 \end{textblock*}
 \footnotesize
 \begin{textblock*}{5cm}(0.7\columnwidth, 6.75cm)
  Rounded inlet:
  \begin{tabular}{>{$}r<{$}|>{$}l<{$}}
   r/D     & K    \\
   \midrule
   0       & 0.5  \\
   0.02    & 0.28 \\
   0.04    & 0.24 \\
   0.06    & 0.15 \\
   0.10    & 0.09 \\
   \ge0.15 & 0.04
  \end{tabular}
 \end{textblock*}
 \normalsize{}
\end{frame}

%%%%%%%%%%%%%%%%%%%%%%%%%%%%%%%%%%%%%%%%%%%%%%%%%%%%%%%%%%%%%%%%%%%%%%%%%%%%%%%%%%%%%%%%%%%%%%%%%%%%%%%%%%%%%%%%%%%%%%%%%%%

\begin{frame}

 \begin{myexam}{}{}
  \raggedright
  Compare the headlosses between an inward-projecting entrance and rounded entrance with a radius of $25\text{ mm}$
  for water entering $6\text{-in}$ Schedule 40 steel pipe $(D=154.1\,\text{mm})$ with a flow of $75\text{L/s}$. \lb(Use $K=0.78$.)
 \end{myexam}

\end{frame}

%%%%%%%%%%%%%%%%%%%%%%%%%%%%%%%%%%%%%%%%%%%%%%%%%%%%%%%%%%%%%%%%%%%%%%%%%%%%%%%%%%%%%%%%%%%%%%%%%%%%%%%%%%%%%%%%%%%%%%%%%%%

\begin{frame}{Exit Loss}

 \begin{itemize}

  \item As liquid flows from a pipe into a large reservoir or tank, the velocity decreases to almost zero and {\bfseries all} the velocity head is lost. \parm

        Then $K=1$ and
        \[ h_L = \frac{v^2}{2g} \]
        \parm
  \item  It is important to remember this loss since, typically, no formula is provided for this in an exam
        formula sheet!

 \end{itemize}
\end{frame}

%%%%%%%%%%%%%%%%%%%%%%%%%%%%%%%%%%%%%%%%%%%%%%%%%%%%%%%%%%%%%%%%%%%%%%%%%%%%%%%%%%%%%%%%%%%%%%%%%%%%%%%%%%%%%%%%%%%%%%%%%%%


\begin{frame}{Valves and Fittings}

 \begin{itemize}

  \item Valves are used to control the amount of flow:
  \item [] e.g. globe valves, angle valves, gate valves, butterfly valves, check (non-return) valves, etc.
  \item []
  \item Fittings are used to change the direction of the flow:
  \item [] e.g. elbows, tees, reducers, nozzles, etc.

 \end{itemize}
\end{frame}

%%%%%%%%%%%%%%%%%%%%%%%%%%%%%%%%%%%%%%%%%%%%%%%%%%%%%%%%%%%%%%%%%%%%%%%%%%%%%%%%%%%%%%%%%%%%%%%%%%%%%%%%%%%%%%%%%%%%%%%%%%%
\begin{frame}{Globe Valve}

 \mini[.5]{

  \cfig[0.4]{../../figs/06MinorLosses/valves/globe2}

  \tiny http://en.wikipedia.org/wiki/File:Carbon\_steel\_globe\_valve.jpg

 }
 \hfill
 \mini[.45]{
  Used for regulating flow.

  \parm
  It has a moveable disc (or plug) and stationary disc seat in a spherical chamber that gives the valve its name.

  \parm
  The disc is screwed into the seat to close the valve

  \cfig[0.3]{../../figs/06MinorLosses/valves/globe}
  \vspace{-0.5cm}
  \tiny http://en.wikipedia.org/wiki/File:Globe\_valve\_diagram-en.svg
  \normalsize
  \par\medskip
  Due to the disruption of flow, even a fully open globe valve causes significant losses.

 }



\end{frame}

%%%%%%%%%%%%%%%%%%%%%%%%%%%%%%%%%%%%%%%%%%%%%%%%%%%%%%%%%%%%%%%%%%%%%%%%%%%%%%%%%%%%%%%%%%%%%%%%%%%%%%%%%%%%%%%%%%%%%%%%%%%
\begin{frame}{Angle Valve}

 \cmini{

  \cfig[0.3]{../../figs/06MinorLosses/valves/angle}
  \centering
  \vspace{-0.5cm}
  \tiny http://articles.compressionjobs.com/images/stories/norrie/8/nw08\_18\_32.jpg
  \normalsize
  \par\bigskip
  A globe valve with inlet and outlet ports at an angle
 }



 % 		\par\medskip
 % 		Outlet port usually orientated downwards to assist the draining of fluid that may remain in the case of a standard
 % 		globe valve, helping to prevent clogging or corrosion


\end{frame}

%%%%%%%%%%%%%%%%%%%%%%%%%%%%%%%%%%%%%%%%%%%%%%%%%%%%%%%%%%%%%%%%%%%%%%%%%%%%%%%%%%%%%%%%%%%%%%%%%%%%%%%%%%%%%%%%%%%%%%%%%%%
\begin{frame}{Gate Valve}

 \mini[.5]{

  \cfig[0.8]{../../figs/06MinorLosses/valves/gate}

  \centering \tiny http://en.wikipedia.org/wiki/File:Valve.jpg

 }
 \hfill
 \mini[.45]{
  A gate valve opens by lifting a gate out of the flow of the fluid

  \par\medskip
  Typically used to allow or prevent flow rather than to regulate, or 'throttle' the flow

  \par\medskip
  If used to regulate flow, vibration and excessive wear may occur

  \par\medskip
  When fully open, there is very little in the way of minor losses


 }



\end{frame}

%%%%%%%%%%%%%%%%%%%%%%%%%%%%%%%%%%%%%%%%%%%%%%%%%%%%%%%%%%%%%%%%%%%%%%%%%%%%%%%%%%%%%%%%%%%%%%%%%%%%%%%%%%%%%%%%%%%%%%%%%%%
\begin{frame}{Gate Valve}

 \cmini{

  \cfig[0.05]{../../figs/06MinorLosses/valves/gate2}
  \centering
  \tiny http://www.ctgclean.com/tech-blog/wp-content/uploads/gate-valve3.jpg

 }

\end{frame}

%%%%%%%%%%%%%%%%%%%%%%%%%%%%%%%%%%%%%%%%%%%%%%%%%%%%%%%%%%%%%%%%%%%%%%%%%%%%%%%%%%%%%%%%%%%%%%%%%%%%%%%%%%%%%%%%%%%%%%%%%%%
\begin{frame}{Gate Valve}

 \cmini{

  \cfig[0.27]{../../figs/06MinorLosses/valves/gate3}
  \centering \vspace{-0.5cm}
  \tiny http://www.nomenclaturo.com/wp-content/uploads/Cutaway-View-and-Terminology-of-Gate-Valve.gif

 }

\end{frame}

%%%%%%%%%%%%%%%%%%%%%%%%%%%%%%%%%%%%%%%%%%%%%%%%%%%%%%%%%%%%%%%%%%%%%%%%%%%%%%%%%%%%%%%%%%%%%%%%%%%%%%%%%%%%%%%%%%%%%%%%%%%
\begin{frame}{Check Valve - Swing Type}

 \mini[0.53]{
  \vspace{-2cm}
  \cfig[0.15]{../../figs/06MinorLosses/valves/checkBanff}

  Banff Middle Springs Pumping Station
 }
 \hfill
 \mini[0.42]{
  \vspace{2cm}
  Allows flow in only one direction
  \par\medskip
  Shuts automatically when flow reverses
  \par
  \cfig[0.3]{../../figs/06MinorLosses/valves/check}
  \vspace{-0.35cm}
  \tiny http://tapseis.anl.gov/guide/photo/images/Check\_Valve\_Diagram.jpg


 }

\end{frame}

%%%%%%%%%%%%%%%%%%%%%%%%%%%%%%%%%%%%%%%%%%%%%%%%%%%%%%%%%%%%%%%%%%%%%%%%%%%%%%%%%%%%%%%%%%%%%%%%%%%%%%%%%%%%%%%%%%%%%%%%%%%
\begin{frame}{Check Valve - Ball Type}

 \mini[0.45]{

  \cfig[0.65]{../../figs/06MinorLosses/valves/ballCheck1}
  \centering
  Open
 }
 \hfill
 \mini[0.45]{
  \cfig[0.65]{../../figs/06MinorLosses/valves/ballCheck2}
  \centering
  Closed
 }
 \cmini{
  The closing member is a spherical ball, which may or may not be spring-loaded.
  \par\medskip
  The seats are conical in shape, to guide the ball into the correct position to obstruct flow.
  \par\medskip
  Small, simple and cheap
  \par\medskip
  Not to be confused with ball valves
 }
\end{frame}

%%%%%%%%%%%%%%%%%%%%%%%%%%%%%%%%%%%%%%%%%%%%%%%%%%%%%%%%%%%%%%%%%%%%%%%%%%%%%%%%%%%%%%%%%%%%%%%%%%%%%%%%%%%%%%%%%%%%%%%%%%%
\begin{frame}{Ball Valve}

 \mini[0.45]{
  \cfig[1]{../../figs/06MinorLosses/valves/ball}
  \centering
  \tiny http://en.wikipedia.org/wiki/File:Ball.PNG


  % \hspace{2cm}
  \hfill
  \mini[0.65]{
   \small\parm
   \begin{enumerate}
    \item Body
    \item Head
    \item Ball
    \item Handle
    \item Stem
   \end{enumerate}
  }
 }
 \hfill
 \mini[0.5]{

  A sphere, with a hole (or port) through it is rotated by a handle
  \par\medskip
  The handle is lined up with the hole to show the valve's position
  \par\medskip
  Can support high pressures (up to 1000 times atmospheric pressure)
  \par\medskip
  A good choice for shut-off application, often used in preference to globe or gate valves, providing perfect shut-off
  after years of use

  \cfig[.2]{../../figs/06MinorLosses/valves/ball3}

 }

\end{frame}

%%%%%%%%%%%%%%%%%%%%%%%%%%%%%%%%%%%%%%%%%%%%%%%%%%%%%%%%%%%%%%%%%%%%%%%%%%%%%%%%%%%%%%%%%%%%%%%%%%%%%%%%%%%%%%%%%%%%%%%%%%%
\begin{frame}{Butterfly Valve}

 \mini[.5]{
  \cfig[0.06]{../../figs/06MinorLosses/valves/Butterfly-valve}
 }
 \hfill
 \mini[.45]{
  A disc, or 'butterfly', is mounted on a rod connected to an 'actuator' outside the pipe. Turning the actuator turns the disc.

  \par\medskip
  When the disc is perpendicular to the flow, flow is blocked and the valve is closed.

  \par\medskip
  When the disc is parallel to the flow,  the valve is open.

  \par\medskip
  Even when the valve is open, there are still some losses - the presence of the disc within the pipe causes some
  disruption to the flow.

 }

 \tiny http://commons.wikimedia.org/wiki/File:Butterfly-valve--The-Alloy-Valve-Stockist.JPG

\end{frame}

%%%%%%%%%%%%%%%%%%%%%%%%%%%%%%%%%%%%%%%%%%%%%%%%%%%%%%%%%%%%%%%%%%%%%%%%%%%%%%%%%%%%%%%%%%%%%%%%%%%%%%%%%%%%%%%%%%%%%%%%%%%
\begin{frame}{Butterfly Valve}

 \mini[.53]{
  \cfig[0.85]{../../figs/06MinorLosses/valves/butterfly2}
 }
 \hfill
 \mini[.42]{
  This is a large butterfly valve used on a hydroelectric power station water inlet pipe in Japan.
  \par\medskip
  Butterfly valves are lower in cost and lighter in weight compared to other valve designs/
  \par\medskip
  They may be used partially open to 'throttle' flow


  \cfig[0.25]{../../figs/06MinorLosses/valves/butterfly3}
  \tiny http://en.wikipedia.org/wiki/File:Nasa-space-18408-l.jpg
 }

 \tiny http://en.wikipedia.org/wiki/File:Yagisawa\_power\_station\_inlet\_valve.jpg

\end{frame}


%%%%%%%%%%%%%%%%%%%%%%%%%%%%%%%%%%%%%%%%%%%%%%%%%%%%%%%%%%%%%%%%%%%%%%%%%%%%%%%%%%%%%%%%%%%%%%%%%%%%%%%%%%%%%%%%%%%%%%%%%%%
\begin{frame}{Foot Valve}

 \mini[.53]{
  \cfig[0.35]{../../figs/06MinorLosses/valves/foot2}
  \centering
  \tiny http://www.cla-val.com/images/products/583%20Foot%20Valve.jpg
 }
 \hfill
 \mini[.42]{
  Attached to the end of a suction line to a pump, allowing water through the valve when the pump is on but with a check
  valve that stops water draining from the line when the flow stops (i.e. when the pump stops).
  \par\medskip
  There is often a strainer on the outside of the valve to prevent objects from blocking the valve inlet.

  \cfig[0.4]{../../figs/06MinorLosses/valves/foot1}
  \centering
  \vspace{-0.25cm}
  \tiny http://www.merrillmfg.com/product/11-CheckValves/BrassValves/images/BrassValves-01.jpg
 }

 % \tiny http://en.wikipedia.org/wiki/File:Yagisawa\_power\_station\_inlet\_valve.jpg

\end{frame}

%%%%%%%%%%%%%%%%%%%%%%%%%%%%%%%%%%%%%%%%%%%%%%%%%%%%%%%%%%%%%%%%%%%%%%%%%%%%%%%%%%%%%%%%%%%%%%%%%%%%%%%%%%%%%%%%%%%%%%%%%%%

\begin{frame}{Equivalent-Length}

 Energy losses incurred by fluid flowing through a valve or fitting is the same as for the minor losses already
 discussed but $K$ is calculated differently:
 \begin{enumerate}
  \item Friction losses in the very short sections of straight pipe at the valve entrance and exit are so small they
        may be ignored.
  \item Losses are then due to changes in direction of the flow, obstructions in the flow and changes in the
        cross-section and shape of the flow, \textbf{where the flow is fully turbulent}
  \item Let $L_e$ (called the \textbf{equivalent length}) be the length of straight pipe with fully turbulent flow that
        would have the same (equivalent) resistance as the valve or fitting. (That is, if the valve were replaced with length $L_e$ of
        straight pipe, losses would be unchanged.)
  \item Then, if we denote the friction factor for fully turbulent flow by $f_T$,
        \[ h_L = f_T\cdot\frac{L_e}{D}\cdot\frac{v^2}{2g} \]
        But the resistance factor $K$ is defined by the relationship $\displaystyle h_L=K\cdot\frac{v^2}{2g}$ so
        \[ K=f_T\left(\frac{L_e}{D}\right) \]
 \end{enumerate}
\end{frame}
%%%%%%%%%%%%%%%%%%%%%%%%%%%%%%%%%%%%%%%%%%%%%%%%%%%%%%%%%%%%%%%%%%%%%%%%%%%%%%%%%%%%%%%%%%%%%%%%%%%%%%%%%%%%%%%%%%%%%%%%%%%

\begin{frame}{Equivalent Length Ratio}

 \begin{itemize}

  \item $\frac{L_e}{D}$ is known as the \textbf{equivalent length ratio} and is considered constant for a particular
        type of fitting.\parb
  \item The friction factor for fully turbulent flow, $f_T$, is independent of the Reynolds number. It is that part of
        the Moody Diagram where the curve is horizontal depends only on the relative roughness of the pipe in question.\parb
  \item $f_T$ can be determined in a number of ways:\pars
        \begin{enumerate}
         \item   From the Moody Diagram, most conveniently by drawing a horizontal line directly leftwards from the relative
               roughness scale on the right hand border of the Diagram.\parm
         \item From the Swamee-Jain formula, using a sufficiently high value for the Reynolds number to guarantee complete
               turbulence (such as $N_R=10^8$).\parm
         \item Or, in the case of steel pipe, from the table on the next page\ldots
        \end{enumerate}




 \end{itemize}
\end{frame}


%%%%%%%%%%%%%%%%%%%%%%%%%%%%%%%%%%%%%%%%%%%%%%%%%%%%%%%%%%%%%%%%%%%%%%%%%%%%%%%%%%%%%%%%%%%%%%%%%%%%%%%%%%%%%%%%%%%%%%%%%%%

\begin{frame}{Friction Factor for Fully Turbulent Flow}
 \begin{center}
  \textbf{Friction Factor, $\bm {f_T}$}
  \\in the zone of complete turbulence\\
  for new clean commercial \textbf{steel} pipe:
  \parb\footnotesize
  \begin{tabular}{>{$}r<{$} >{$}r<{$} >{$}l<{$} >{$}c<{$} >{$}r<{$} >{$}r<{$} >{$}l<{$}}
   \toprule
   \text{Nominal}   & \quad & f_T   & \qquad\qquad\qquad & \text{Nominal}   & \quad & f_T   \\
   \text{Size (in)} &       &       &                    & \text{Size (in)} &       &       \\
   \midrule
   \midrule
   \frac{1}{2}      &       & 0.027 &                    & 3\frac{1}{2},4   &       & 0.017 \\
   \addlinespace	\midrule \addlinespace
   \frac{3}{4}      &       & 0.025 &                    & 5                &       & 0.016 \\
   \addlinespace	\midrule \addlinespace
   1                &       & 0.023 &                    & 6                &       & 0.015 \\
   \addlinespace	\midrule \addlinespace
   1\frac{1}{4}     &       & 0.022 &                    & 8-10             &       & 0.014 \\
   \addlinespace	\midrule \addlinespace
   1\frac{1}{2}     &       & 0.021 &                    & 12-16            &       & 0.013 \\
   \addlinespace	\midrule \addlinespace
   2                &       & 0.019 &                    & 18-24            &       & 0.012 \\
   \addlinespace	\midrule \addlinespace
   2\frac{1}{2},3 &&	0.018\\
   \bottomrule
  \end{tabular}
  \par\end{center}

  \end{frame}

  %%%%%%%%%%%%%%%%%%%%%%%%%%%%%%%%%%%%%%%%%%%%%%%%%%%%%%%%%%%%%%%%%%%%%%%%%%%%%%%%%%%%%%%%%%%%%%%%%%%%%%%%%%%%%%%%%%%%%%%%%%%

  \begin{frame}
\centering

   \begin{myexam}[width=0.75\textwidth]{}{}
		 \raggedright
    Find the pressure drop across a fully open globe valve ($L_e/D=340$) in $4\text{-in}$ Shedule $40$ steel pipe\lb $(D=102.3\,\text{mm})$
    carrying $1600\text{ L/min}$.
   \end{myexam}

  \end{frame}

  %%%%%%%%%%%%%%%%%%%%%%%%%%%%%%%%%%%%%%%%%%%%%%%%%%%%%%%%%%%%%%%%%%%%%%%%%%%%%%%%%%%%%%%%%%%%%%%%%%%%%%%%%%%%%%%%%%%%%%%%%%%

  \begin{frame}[plain]
   \cfig[0.55]{../../Figs/05FrictionLosses/moody}

  \end{frame}

  %%%%%%%%%%%%%%%%%%%%%%%%%%%%%%%%%%%%%%%%%%%%%%%%%%%%%%%%%%%%%%%%%%%%%%%%%%%%%%%%%%%%%%%%%%%%%%%%%%%%%%%%%%%%%%%%%%%%%%%%%%%

  \begin{frame}{Table of Equivalent-Length Ratios for Valves}
   \begin{center}
		 \small
    \parb
    \begin{tabular}{r >{$}r<{$} >{$}l<{$} >{$}c<{$} }
     \toprule
     \text{Type}                              & \quad & L_e/D \\
     \midrule
     Globe valve --- fully open               &       & 340   \\
     \addlinespace
     Angle valve --- fully open               &       & 150   \\
     \addlinespace
     Gate valve --- fully open                &       & 8     \\
     \addlinespace
     --- $3/4$ open                           &       & 35    \\
     \addlinespace
     --- $1/2$ open                           &       & 160   \\
     \addlinespace
     --- $1/4$ open                           &       & 900   \\
     \addlinespace
     Check valve --- swing type               &       & 100   \\
     \addlinespace
     Check valve --- ball type                &       & 150   \\
     \addlinespace
     Butterfly valve --- fully open --- 2-8'' &       & 45    \\
     \addlinespace
     --- 10-14''                              &       & 35    \\
     \addlinespace
     --- 16-24''                              &       & 25    \\
     \addlinespace
     Foot valve --- poppet disc type          &       & 420   \\
     \addlinespace
     Foot valve --- hinged disc type          &       & 75    \\
     \addlinespace

     \bottomrule
    \end{tabular}
    \par\end{center}

    \end{frame}

    %%%%%%%%%%%%%%%%%%%%%%%%%%%%%%%%%%%%%%%%%%%%%%%%%%%%%%%%%%%%%%%%%%%%%%%%%%%%%%%%%%%%%%%%%%%%%%%%%%%%%%%%%%%%%%%%%%%%%%%%%%%

    \begin{frame}
\centering
     \begin{myexam}[width=0.8\textwidth]{}{}
			 % \raggedright
      Calculate the headloss across a ball-type check valve placed in a $1\tfrac{1}{4}\text{-in}$ copper
      tubing $(D=31.62\,\text{mm})$ if water is flowing through the tubing with a velocity of $2.35\text{ m/s}$.
		\end{myexam}

    \end{frame}

    %%%%%%%%%%%%%%%%%%%%%%%%%%%%%%%%%%%%%%%%%%%%%%%%%%%%%%%%%%%%%%%%%%%%%%%%%%%%%%%%%%%%%%%%%%%%%%%%%%%%%%%%%%%%%%%%%%%%%%%%%%%

    \begin{frame}[plain]

     \cfig[0.55]{../../Figs/05FrictionLosses/moody}

    \end{frame}

    %%%%%%%%%%%%%%%%%%%%%%%%%%%%%%%%%%%%%%%%%%%%%%%%%%%%%%%%%%%%%%%%%%%%%%%%%%%%%%%%%%%%%%%%%%%%%%%%%%%%%%%%%%%%%%%%%%%%%%%%%%%

    \begin{frame}{Table of Equivalent-Length Ratios for Fittings}
     \begin{center}

      \par\bigskip
      \begin{tabular}{r >{$}r<{$} >{$}l<{$} >{$}c<{$} }
       \toprule
       \text{Type}                          & \quad & L_e/D \\
       \midrule
       $90\degree$ standard elbow           &       & 30    \\
       \addlinespace
       $90\degree$ long radius elbow        &       & 20    \\
       \addlinespace
       $90\degree$ street elbow             &       & 50    \\
       \addlinespace
       $45\degree$ standard elbow           &       & 16    \\
       \addlinespace
       $45\degree$ street elbow             &       & 26    \\
       \addlinespace
       Close return bend                    &       & 50    \\
       \addlinespace
       Standard tee --- flow through run    &       & 20    \\
       \addlinespace
       Standard tee --- flow through branch &       & 60    \\
       \addlinespace
       \bottomrule
      \end{tabular}
      \par\end{center}

      \end{frame}

      %%%%%%%%%%%%%%%%%%%%%%%%%%%%%%%%%%%%%%%%%%%%%%%%%%%%%%%%%%%%%%%%%%%%%%%%%%%%%%%%%%%%%%%%%%%%%%%%%%%%%%%%%%%%%%%%%%%%%%%%%%%

      \begin{frame}
				\centering
       \begin{myexam}[width=0.8\textwidth]{}{}
				 \raggedright
        Find the pressure drop across a $90\degree$ standard elbow in a $2\tfrac{1}{2}\text{-in}$ Schedule $40$ steel pipe $(D=62.7\,\text{mm})$ if water is flowing
        at the rate of $800\text{ L/min}$.
			\end{myexam}

      \end{frame}





      %%%%%%%%%%%%%%%%%%%%%%%%%%%%%%%%%%%%%%%%%%%%%%%%%%%%%%%%%%%%%%%%%%%%%%%%%%%%%%%%%%%%%%%%%%%%%%%%%%%%%%%%%%%%%%%%%%%%%%%%%%%

      %%%%%%%%%%%%%%%%%%%%%%%%%%%%%%%%%%%%%%%%%%%%%%%%%%%%%%%%%%%%%%%%%%%%%%%%%%%%%%%%%%%%%%%%%%%%%%%%%%%%%%%%%%%%%%%%%%%%%%%%%%%


\end{document}
