% override specific chktex warnings
% chktex-file 46 - don't use $ instead of \(, etc)
% chktex-file 1 - ignore commands followed by a space, e.g. \\ new line here
% chktex-file 9 - sometimes messes up with ( and {

\documentclass[9pt,xcolor={svgnames, x11names},professionalfonts, mathserif]{beamer}

% \usepackage{euler}
\usepackage{lmodern}
\usepackage{amsmath}
\usepackage{amssymb}
\usepackage{graphicx}
\usepackage{booktabs}  % for top and bottom spacing in table cells
\usepackage{mathpazo}
\usepackage{textcomp}
\usepackage{multirow}
\usepackage{cancel}
\usepackage{array}
\usepackage{gensymb} % for \degree
\usepackage{cleveref}
\usepackage[many]{tcolorbox}

\usepackage{verbatim}
\usepackage{bm}
\usepackage{graphicx}
\usepackage{tikz}
\usepackage{tkz-linknodes}
\usepackage[export]{adjustbox} % for tight borders around photos
\usepackage{pgf} % for sait logo in beamer
\usepackage{pgfmath}

\usepgfmodule{oo}
%\usetikzlibrary{shapes,decorations,shadows,calc}
\usetikzlibrary{shadows,calc,arrows.meta}
% \usetikzlibrary{decorations.shapes}
%\usetikzlibrary{shapes.callouts}
% bloody coils
\usetikzlibrary{decorations.pathmorphing}
\usetikzlibrary{shapes.multipart}

% counter for resuming enumerated list numbers
\newcounter{resumeenumi}
\newcommand{\suspend}{\setcounter{resumeenumi}{\theenumi}}
\newcommand{\resume}{\setcounter{enumi}{\theresumeenumi}}

\newcommand\lb{\linebreak}
\newcommand\pars{\par\smallskip}
\newcommand\parm{\par\medskip}
\newcommand\parb{\par\bigskip}

\makeatletter
\providecommand{\gettikzxy}[3]{%
	\tikz@scan@one@point\pgfutil@firstofone#1\relax
	\edef#2{\the\pgf@x}%
	\edef#3{\the\pgf@y}%
}
\makeatother



% full width colored block but color specifiable
%\cb[body bg strength]{header bg}{header text}{body text}
\newcommand{\cb}[4][15]{
	\setbeamercolor{block title}{bg = #2}
	\setbeamercolor{block body}{bg = #2!#1}
	\setbeamercolor{item projected}{bg=#2, fg=white}
	\begin{center}
		\begin{block}{#3}
			#4
		\end{block}
	\end{center}
}

% colored block with width specified
% \cbw[body bg strength]{header bg}{width}{header text}{body text}
\newcommand{\cbw}[5][15]{
	\begin{center}
		%\vspace{-0.35cm}
		\begin{minipage}{#3\textwidth}
			\setbeamercolor{block title}{bg= #2}
			\setbeamercolor{block body}{bg= #2!#1}
			\setbeamercolor{item projected}{bg=#2, fg=white}
			\begin{block}{#4}
				\raggedright
				#5
			\end{block}
		\end{minipage}
	\end{center}
}

% centered minipage with text \raggedright
%\cmini[width]{content}
\newcommand{\cmini}[2][0.8]{
	\begin{center}
		\begin{minipage}{#1\columnwidth}
			\raggedright
			#2
		\end{minipage}
	\end{center}
}

%left flushed minipage
\newcommand{\mini}[2][0.8]{
	\begin{minipage}{#1\columnwidth}
		\raggedright
		#2
	\end{minipage}
}

%left flushed minipage, top aligned
\newcommand{\minit}[2][0.8]{
	\begin{minipage}[t]{#1\columnwidth}
		\raggedright
		#2
	\end{minipage}
}

%left flushed minipage
% \newcommand{\miniT}[2][0.8]{
%  \begin{minipage}[T]{#1\columnwidth}
%   \raggedright
%   #2
%  \end{minipage}
% }

%left flushed minipage
\newcommand{\minib}[2][0.8]{
	\begin{minipage}[b]{#1\columnwidth}
		\raggedright
		#2
	\end{minipage}
}

\newcommand{\cfig}[2][1]{% centred, scaled graphic
	\begin{center}
		\includegraphics[scale=#1]{#2}
	\end{center}
}
% figure with tight border for photos
% \cfigb[saitMaroon]{borderwidth with unit}{scale}{image}
\newcommand{\cfigb}[4][structure]{
	% \usepackage{adjustbox}
	\setlength{\fboxrule}{1pt}
	\begin{center}
		\includegraphics[scale=#3, cframe= #1 #2]{#4}
	\end{center}
}

\newcommand{\imgbox}[3]{
	% \setlength{\fboxsep}{12pt}
	\includegraphics[scale=#1, cframe= structure #3]{#2}
}

% \imgboxbg[bg color=white]{scale}{path/to/img}{border color}{border, e.g. 2pt}{margin, e.g. 4pt}
\newcommand{\imgboxbg}[6][white]{
	\setlength{\fboxrule}{#5}
	\setlength{\fboxsep}{#6}
	\centering
	\fcolorbox{#4}{#1}{\includegraphics[scale=#2]{#3}}
}

\newcommand{\fig}[2][1]{% scaled graphic
	\includegraphics[scale=#1]{#2}
}

% centred framed  box black border
%\cbox[width]{content}
\newcommand{\cbox}[2][0.9]{% framed centered  box
	\setlength\fboxsep{0.042\columnwidth}
	\setlength\fboxrule{0.0015\columnwidth}
	\begin{center}
		\fcolorbox{black}{white}{
			\vspace{-0.5cm}
			\begin{minipage}{#1\columnwidth}
				\raggedright
				#2
			\end{minipage}
		}
	\end{center}
	\setlength\fboxsep{0cm}
}



\newtcolorbox{mybox}[1][]
{
	colback=white,
	top=0.25cm,
	bottom=0.25cm,
	left=0.25cm,
	right=0.25cm,
	colframe=structure,
	fonttitle=\bfseries,
	enhanced, drop fuzzy shadow,
	% attach boxed title to top left={yshift=-2mm, xshift=5mm},
	attach boxed title to top left={yshift=-2mm, xshift=5mm}, colbacktitle=structure!80!white, #1}

\newtcolorbox{plainbox}[1][]{colback=white, sharp corners, top=0.125cm, bottom=0.125cm, left=0pt, right=0pt, boxrule=0.5pt,colframe=structure,fonttitle=\bfseries, colbacktitle=structure, arc=0mm, #1}
%
\newtcbtheorem{myexam}{Example}%
{
	enhanced,
	colback=white,
	top=0.375cm,
	bottom=0.25cm,
	left=0.375cm,
	right=0.375cm,
	colframe=structure,
	fonttitle=\bfseries,
	drop fuzzy shadow,
	%description font=\mdseries\itshape,
	attach boxed title to top left={yshift=-2mm, xshift=5mm},
	colbacktitle=structure!80!white
	}{exam}% then \pageref{exer:theoexample} references the theo

\newcommand{\myexample}[2][red]{
	% \tcb\tcbset{theostyle/.style={colframe=red,colbacktitle=yellow}}
	\begin{myexam}{}{}
		\raggedright
		#2
	\end{myexam}
	% \tcbset{colframe=structure,colbacktitle=structure}
}

\newtcbtheorem{myexer}{Exercise}%
{
	enhanced,
	colback=white,
	top=0.375cm,
	bottom=0.25cm,
	left=0.375cm,
	right=0.375cm,
	colframe=structure,
	fonttitle=\bfseries,
	drop fuzzy shadow,
	%description font=\mdseries\itshape,
	attach boxed title to top left={yshift=-2mm, xshift=5mm},
	colbacktitle=structure!80!white
	}{exer}

\newcommand{\myexercise}[2][red]{
	% \tcb\tcbset{theostyle/.style={colframe=red,colbacktitle=yellow}}
	\begin{myexer}{}{}
		\raggedright
		#2
	\end{myexer}
	% \tcbset{colframe=structure,colbacktitle=structure}
}

\input{../../Includes/definedColors}
% override specific chktex warnings
% chktex-file 46 - don't use $ instead of \(, etc)
% chktex-file 36 - don't require space in front of parenthesis
% chktex-file 37 - don't require space in front of parenthesis
% chktex-file 26 - don't require space in front of punctuation
% chktex-file 1 - ignore commands followed by a space, e.g. \\ new line here
% chktex-file 9 - sometimes messes up with ( and {

\begin{comment}
Shadings are useful to give the illusion of 3D in examples and exercises presented to engineering technology students.
Vertical and horizontal shadings of rectangles are fairly straightforward to produce with the shading library included in
a recent build of Tikz.
Rotation of shaded squares is also intuitive, but rotation of a shaded rectangle appears to be both a function of the specified
rotation angle and the length to width ratio of the rectangle. This makes aligning the shading of a rotated rectangle's
fill with the stroke of a rotated rectangle a bit of an inelegant trial-and-error exercise (for me, at any rate).


\end{comment}

%http://tex.stackexchange.com/questions/33703/extract-x-y-coordinate-of-an-arbitrary-point-in-tikz
\makeatletter
\providecommand{\gettikzxy}[3]{%
	\tikz@scan@one@point\pgfutil@firstofone#1\relax
	\edef#2{\the\pgf@x}%
	\edef#3{\the\pgf@y}%
}
\makeatother


%%%%%%%%%%%%%%%%%%%%%%%%%%%%%%%% A CLASS FOR ROTATED RECTANGLES WITH A SHADED FILL %%%%%%%%%%%%%%%%%%%%%%%%%%%%%%%%%%%%%%%%
\pgfooclass{rrect}{
	% the following should be set in the calling program: \hi, \radii, \extend
	% Ax, Ay, Bx, By, outershade, innershade
	\method rrect(#1,#2,#3,#4,#5,#6) { % The constructor; everything is done in here
		\def\Ax{#1} \def\Ay{#2} \def\Bx{#3} \def\By{#4} \def\outercolor{#5} \def\innercolor{#6}
		\pgfmathparse{\Bx-\Ax} \let\deltaX\pgfmathresult
		\pgfmathparse{\By-\Ay} \let\deltaY\pgfmathresult
		\ifthenelse{\equal{\deltaX}{0.0}}
		{	% vertical rod is a special case; otherwise atan gets a div by 0 error
			\pgfmathparse{\By>\Ay} \let\ccw\pgfmathresult
			\ifthenelse{\equal{\ccw}{1}}{%
				\def\rot{90}}
			{\def\rot{-90}}}
	{	% not vertical
		\pgfmathparse{\Ax<\Bx} \let\iseast\pgfmathresult
		\ifthenelse{\equal{\iseast}{1}}{%
			\pgfmathparse{atan(\deltaY/\deltaX)} \let\rot\pgfmathresult
		} % end is east
		{
			\pgfmathparse{180+atan(\deltaY/\deltaX)} \let\rot\pgfmathresult
		} % end !east
		}
		%shading boundaries work for vertical and horizontal but otherwise ``spills'' outside it supposed boundaries,
		%particularly at multiples of 45deg
		%make some adjustments from a max at 45 to nothing at 0 or 90
		\pgfmathparse{abs(\deltaY)} \let\absdeltaY\pgfmathresult
		\pgfmathparse{abs(\deltaX)} \let\absdeltaX\pgfmathresult
		%\def\shadeangle{-42}
		\ifthenelse{\equal{\deltaX}{0.0}}
		{\def\shadeangle{0.0}}
		{\pgfmathparse{\absdeltaY > \absdeltaX} \let\foo\pgfmathresult
			\ifthenelse{\equal{\foo}{1}}
			{\pgfmathparse{90-atan(\absdeltaY/\absdeltaX)} \let\shadeangle\pgfmathresult}
			{\pgfmathparse{atan(\absdeltaY/\absdeltaX)} \let\shadeangle\pgfmathresult}
		}
		\pgfmathparse{tan(\shadeangle)} \let\fudge\pgfmathresult
		\pgfmathparse{veclen(\deltaX,\deltaY)} \let\len\pgfmathresult
		\pgfmathparse{max(\hi,\len+2*\extend)} \let\shadeboxside\pgfmathresult
		\pgfmathparse{50-25/\shadeboxside*\hi+8*\hi*\fudge/\shadeboxside} \let\mybot\pgfmathresult
		\pgfmathparse{50+25/\shadeboxside*\hi-8*\hi*\fudge/\shadeboxside} \let\mytop\pgfmathresult
		\pgfdeclareverticalshading{myshade}{100bp}{%
			color(0bp)=(\outercolor);
			color(\mybot bp)=(\outercolor);
			color(50 bp)=(\innercolor);
			color(\mytop bp)=(\outercolor);
			color(100bp)=(\outercolor)}
		\tikzset{shading=myshade}
		\begin{scope}	[rotate around = {\rot: (\Ax, \Ay)}]
			\begin{scope}
				\draw[clip, rounded corners = \scale*\radii cm] (\Ax-\extend,\Ay-\hi/2) rectangle + (\len+2*\extend,\hi);
				\shade[ shading angle=\rot] (\Ax-\extend,\Ay-\shadeboxside/2) rectangle +(\shadeboxside, \shadeboxside);
			\end{scope} %end clipping
			\draw[rounded corners=\scale*\radii cm, \stroke, \thickness] (\Ax-\extend,\Ay-\hi/2) rectangle +(\len+2*\extend,\hi);
		\end{scope}
		} % end of constructor
		} % end of rrect class

		\pgfooclass{rr}{
			\method rr (#1,#2,#3,#4,#5) { % The constructor; everything is done in here
				% Here I can get named x and y coordinates
				\def\phil{#3} \def\stroke{#4} \def\line{#5}
				\gettikzxy{(#1)}{\spx}{\spy}
				\gettikzxy{(#2)}{\epx}{\epy}
				% I'd like named points to work with
				\coordinate (Start) at (\spx, \spy);
				\coordinate (End) at (\epx, \epy);
				% Find the length between start and end. Then the angle between x axis and Diff will be the rotation to apply.
				\coordinate (Diff) at ($ (End)-(Start) $);
				\gettikzxy{(Diff)}{\dx}{\dy}
				\pgfmathparse{veclen(\dx, \dy)} \pgfmathresult
				\let\length\pgfmathresult
				\pgfmathparse{\dx==0}%
				% \ifnum low-level TeX for integers
				\ifnum\pgfmathresult=1 % \dx == 0
					\pgfmathsetmacro{\rot}{\dy > 0 ? 90 : -90}
				\else% \dx != 0
					\pgfmathsetmacro{\rot}{\dx > 0 ? atan(\dy /\dx) : 180 + atan(\dy / \dx)}
				\fi
				\begin{scope}	[rotate around = {\rot:(\spx, \spy )}]
					% \filldraw[ultra thick, fill=\phil, draw=\stroke] ($ (Start)+(0,\hi) $) arc(90:270:\hi) -- +(\length pt, 0) arc(-90:90:\hi) -- cycle;
					\filldraw[rounded corners=\scale*\radii cm, line width=\line mm, fill=\phil, draw=\stroke] (\spx-\extend cm,\spy-\hi cm) rectangle +(2*\extend cm + \length pt, 2*\hi cm);
				\end{scope}
			}
		}

		\pgfooclass{beam}{
			\method beam(#1,#2,#3,#4,#5) { % The constructor; everything is done in here
				% Here I can get named x and y coordinates
				\def\phil{#3} \def\stroke{#4} \def\line{#5}
				\gettikzxy{(#1)}{\spx}{\spy}
				\gettikzxy{(#2)}{\epx}{\epy}
				% I'd like named points to work with
				\coordinate (Start) at (\spx, \spy);
				\coordinate (End) at (\epx, \epy);
				% Find the length between start and end. Then the angle between x axis and Diff will be the rotation to apply.
				\coordinate (Diff) at ($ (End)-(Start) $);
				\gettikzxy{(Diff)}{\dx}{\dy}
				\pgfmathparse{veclen(\dx, \dy)} \pgfmathresult
				\let\length\pgfmathresult
				\pgfmathparse{\dx==0}%
				% \ifnum low-level TeX for integers
				\ifnum\pgfmathresult=1 % \dx == 0
					\pgfmathsetmacro{\rot}{\dy > 0 ? 90 : -90}
				\else% \dx != 0
					\pgfmathsetmacro{\rot}{\dx > 0 ? atan(\dy / \dx) : 180 + atan(\dy / \dx)}
				\fi
				\begin{scope}	[rotate around = {\rot:(\spx, \spy )}]
					\fill[\phil] (\spx-\extend cm,\spy-\hi cm) rectangle +(2*\extend cm + \length pt, 2*\hi cm);
					\draw[draw=\stroke, line width=\line mm] (\spx-\extend cm,\spy-\hi cm) -- +(2*\extend cm + \length pt, 0);
					\draw[draw=\stroke, line width=\line mm] (\spx-\extend cm,\spy+\hi cm) -- +(2*\extend cm + \length pt, 0);
				\end{scope}
			}
		}


% \usefonttheme[onlymath]{serif}

\usepackage[absolute,overlay]{textpos}
\setlength{\TPHorizModule}{1.0cm}
\setlength{\TPVertModule}{\TPHorizModule}
\textblockorigin{0.0cm}{0.0cm}  %start all at upper left corner
\usepackage{hyperref}
\hypersetup{colorlinks=true, urlcolor=structure}
% \hypersetup{urlcolor=Blue4}

\setlength{\parskip}{\medskipamount}
\setlength{\parindent}{0pt}

\usetheme{Antibes}

\usecolortheme[rgb={0, 0.6, 0.6}]{structure}
% \definecolor{structurecolor}{rgb}{0.55,0.53,0.31}
\setbeamertemplate{items}[triangle]
\setbeamertemplate{blocks}[rounded][shadow=false]
%\setbeamertemplate{background canvas}[vertical shading][bottom=Cyan1!50, middle=white, top=white, midpoint=0.05]
\setbeamertemplate{headline}{\vspace{.05cm}}
\setbeamertemplate{footline}{\textcolor{structure}{ \hfill\insertshorttitle\quad
	\insertshortsubtitle{}
	\quad \insertframenumber/\inserttotalframenumber\quad{ }\vspace{0.125cm}}}
\addtobeamertemplate{footline}{\hypersetup{linkcolor=.}}{}
\setbeamertemplate{navigation symbols}{} % empty braces suppresses all navigation symbols
\setbeamercolor{frametitle}{fg=gray!5!white}
% \setbeamercolor{footline}{fg=black}
\setbeamercolor{block title}{fg=gray!15!white,bg=structure}
\setbeamercolor{block body}{bg=white, fg=black}
\setbeamercolor{background canvas}{bg=gray!20!white}
\setbeamersize{text margin left = 1cm, text margin right=1cm}
%\useinnertheme[shadow]{rounded}
%\raggedright
\setbeamerfont{block title}{family=mathserif}

\everymath{\displaystyle}
\newcounter{itemcount}

\logo{\pgfputat{\pgfxy(-11.85,-0.5)}{\pgfbox[right,base]{\includegraphics[height=1cm]{../../figs/rb_logo}}}}

%%%%%%%%%%%%%%%%%%%%%%%%%%%%%%%%%%%%%%%%%%%%%%%%%%%%%%%%%%%%%%%%%%%%%%%%%%%%%%%%

%test
\resetcounteronoverlays{tcb@cnt@myexam}
\resetcounteronoverlays{tcb@cnt@myexer}


% itemize indent
\setlength{\leftmargini}{1.5em}
\def\scale{1} % initialisation for pikz
\raggedright

%%%%%%%%%%%%%%%%%%%%%%%%%%%%%%%%%%%%%%%%%%%%%%%%%%%%%%%%%%%%%%%%%%%%%%%%%%%%%%%%%%%%%%%%%%%%%%%%%%%%%%%%%%%%%%%%%%%%%%%%
\title[Nature Of Fluids/Pressure Measurement]{\Huge \textcolor{white}{01 --- The Nature Of Fluids \& Pressure Measurement}}
\subtitle[CIVL318]{\Large\textcolor{white}{Water Resources, CIVL318}}
\author{}
\institute{}
\date{Last revision on \today}



%%%%%%%%%%%%%%%%%%%%%%%%%%%%%%%%%%%%%%%%%%%%%%%%%%%%%%%%%%%%%%%%%%%%%%%%%%%%%%%%%%%%%%%%%%%%%%%%%%%%%%%%%%%%%%%%%%%%%%%%%%%
\begin{document}
%%%%%%%%%%%%%%%%%%%%%%%%%%%%%%%%%%%%%%%%%%%%%%%%%%%%%%%%%%%%%%%%%%%%%%%%%%%%%%%%%%%%%%%%%%%%%%%%%%%%%%%%%%%%%%%%%%%%%%%%
\begin{frame}[plain]
	\titlepage
\end{frame}



%%%%%%%%%%%%%%%%%%%%%%%%%%%%%%%%%%%%%%%%%%%%%%%%%%%%%%%%%%%%%%%%%%%%%%%%%%%%%%%%%%%%%%%%%%%%%%%%%%%%%%%%%%%%%%%%%%%%%%%%%%%%%%%

\begin{frame}{Elementary Properties of Fluids}
	\raggedright
	\begin{itemize}
		\item Fluids can be either liquid or gas \parm \pause
		\item A gas: \parm
		      \begin{itemize}
		      	\item  Tends to expand to fill the closed container it is in (or to disperse if not contained). \parm
		      	\item Is readily compressible.\parm
		      	\item We are not (much) concerned with gases in this course.\parm
		      \end{itemize}
		      \pause \parb
		\item A liquid: \parm
		      \begin{itemize}
		      	\item Tends to flow and conform to the shape of its container.\parm
		      	\item For the purpose of this course we consider a liquid to be incompressible: \parm \pause
		      	      \begin{itemize}
		      	      	\item We assume that the density of water at the bottom of the ocean is the same as the density near the surface (if at the same temperature)\parm
		      	      	\item Density is different from pressure: the pressure at the bottom of the ocean is much more than at the surface
		      	      \end{itemize}
		      \end{itemize}
	\end{itemize}
\end{frame}
%%%%%%%%%%%%%%%%%%%%%%%%%%%%%%%%%%%%%%%%%%%%%%%%%%%%%%%%%%%%%%%%%%%%%%%%%%%%%%%%%%%%%%%%%%%%%%%%%%%%%%%%%%%%%%%%%%%%%%%%%%%%%%

\begin{frame}{Pressure}
	
	\begin{center}
		\begin{mybox}[width=3cm, top=-0.05cm]
			\[p=\frac{F}{A}\]
		\end{mybox}
		Pressure is the force per unit area on a surface, where \parm
		$1\; \text{N/m}^{2}=1\; \text{Pa}$ (pascal)
	\end{center}
	
	\pause
	\textbf{Blaise Pascal} ($1623 - 1662$) determined the following principles:
	
	\mini[0.3]{
		\begin{center}
			% http://commons.wikimedia.org/wiki/File:Blaise_pascal.jpg
			% \imgboxbg[bg color=white]{scale}{path/to/img}{border, e.g. 2pt}{margin, e.g. 4pt}
			\imgboxbg{0.45}{../../figs/01NoFPM/180px-Blaise_pascal}{1pt}{5pt}
			{\scriptsize {\href{http://commons.wikimedia.org/wiki/File:Blaise_pascal.jpg}{\textcolor{black} {Wikimedia link}}}}
		\end{center}
	}
	\hfill
	\mini[0.6]{
		\begin{enumerate}
			\item Pressure acts uniformly in all directions on a ``small'' volume
			      of a fluid at rest \parm
			\item In a fluid confined by solid boundaries, pressure acts perpendicularly
			      to the boundaries
		\end{enumerate}
	}
\end{frame}
%%%%%%%%%%%%%%%%%%%%%%%%%%%%%%%%%%%%%%%%%%%%%%%%%%%%%%%%%%%%%%%%%%%%%%%%%%%%%%%%%%%%%%%%%%%%%%%%%%%%%%%%%%

\begin{frame}{Pascal's Laws}
	
	\smallskip
	\begin{center}
		\begin{mybox}[width=8cm]
			\centering
			\bfseries Pressure acts uniformly in all directions on a small volume of a fluid at rest.
		\end{mybox}
		
	\end{center}
	\begin{flushleft}
		The forces must balance out (i.e. $\Sigma\ F_{x}=\Sigma\ F_{y}=0$); otherwise
		the volume of fluid will not be in equilibrium and cannot remain at rest.
	\end{flushleft}
	\mini[0.55]{
		Also, the volume must be sufficiently ``small'' that we do not
		have to consider the mass, and therefore the weight, $w$, of the volume
		of fluid.
	}
	\hfill
	\mini[0.4]{
		\cfig{../../figs/01NoFpM/01pascal1}
	}
	
	
	
\end{frame}

%%%%%%%%%%%%%%%%%%%%%%%%%%%%%%%%%%%%%%%%%%%%%%%%%%%%%%%%%%%%%%%%%%%%%%%%%%%%%%%%%%%%%%%%%%%%%%%%%%%%%%%%%%

\begin{frame}{Pascal's Laws}
	\mini[0.6]{
		\smallskip
		\begin{mybox}
			\centering
			{\bfseries In a fluid confined by solid boundaries, pressure acts perpendicularly to the boundaries.}
		\end{mybox}
	}
	\hfill
	\mini[0.35]{
		\cfig[0.8]{../../figs/01NoFPM/01pascal2}
	}
\end{frame}


%%%%%%%%%%%%%%%%%%%%%%%%%%%%%%%%%%%%%%%%%%%%%%%%%%%%%%%%%%%%%%%%%%%%%%%%%%%
\begin{frame}{}
	
	\cmini[0.85]{
		\myexample{
		\raggedright
		A piston confines oil in a closed circular cylinder. The maximum operating pressure for the
		piston is $17.8\text{ MPa}$. The piston has a diameter of $62.5\text{ mm}$. \parb
		What is the maximum load that the piston can support?
		}
	}
	
\end{frame}

%%%%%%%%%%%%%%%%%%%%%%%%%%%%%%%%%%%%%%%%%%%%%%%%%%%%%%%%%%%%%%%%%%%%%%%%%%%
\begin{frame}
	\cmini[0.85]{
		\myexercise{
			\raggedright
			A press used to produce coins requires a force of $8.20\text{ kN}$.\\
			The hydraulic cylinder has a diameter of $63.5\text{ mm}$. \par\bigskip
			What is the oil pressure needed to generate this force?
			
		}
	}
	
\end{frame}

%%%%%%%%%%%%%%%%%%%%%%%%%%%%%%%%%%%%%%%%%%%%%%%%%%%%%%%%%%%%%%%%%%%%%%%%%%%
\begin{frame}{Density}
	\centering
	\begin{mybox}[width=3cm,top=-0.05cm]
		\[ \rho=\frac{w}{V} \]
	\end{mybox}
	
	{\bfseries Density} (denoted by $\bm\rho$, `rho') is mass per unit volume.
	\parb
	
	\cmini{
		\begin{itemize}
			\item The density of water between $0\,^{\circ}C$ and $15\,^{\circ}C$ is approximately $1000\; \text{kg/m}^{3}$. \parm
			\item Water has a maximum density close to $4\,^{\circ}C$. \parm
			\item Above $15\,^{\circ}C$, the density drops steadily to a density of $958\; \text{kg/m}^{3}$ at $100\,^{\circ}C$.
		\end{itemize}
		\parb
		\centering
		(There is a table of values for the properties of water attached to this module's handout.)
	}
	
\end{frame}
%%%%%%%%%%%%%%%%%%%%%%%%%%%%%%%%%%%%%%%%%%%%%%%%%%%%%%%%%%%%%%%%%%%%%%%%%%%%%%%%%%%%%%%%%%%%%%%%%%%%%%%%%%%%
\begin{frame}{Specific Weight}
	\centering
	\begin{mybox}[width=3cm,top=-0.05cm]
		\[\gamma=\frac{w}{V}\]
	\end{mybox}
	
	{\bfseries Specific weight}  (denoted by $\bm\gamma$, `gamma') is weight per unit volume.
	\parb
	Water has a specific weight of approximately $9.81\; \text{kN/m}^{3}$ \lb between $0\,^{\circ}C$ and  $15\,^{\circ}C$.
	\parb
	
	Since $w=mg$, it follows that:
	\[\gamma=\frac{w}{V}=\frac{mg}{V}=\rho g\]
	
	\begin{mybox}[width=3cm,top=-0.05cm]
		\[\gamma=\frac{w}{V}=\rho{}g\]
	\end{mybox}
\end{frame}

%%%%%%%%%%%%%%%%%%%%%%%%%%%%%%%%%%%%%%%%%%%%%%%%%%%%%%%%%%%%%%%%%%%%%%%%%%%%%%%%%%%%%%%%%%%%%%%%%%%%%%%%%%%%%%
\begin{frame}{}
	\cmini[0.65]{
		\myexample{
		\raggedright
		An  empty barrel with an inside diameter of~$900$~mm weighs $205$ N.\par\bigskip
		What does the barrel weigh when it is filled to a depth of $750$ mm with water at $25$\textcelsius?
		}
	}
\end{frame}

%%%%%%%%%%%%%%%%%%%%%%%%%%%%%%%%%%%%%%%%%%%%%%%%%%%%%%%%%%%%%%%%%%%%%%%%%%%%%%%%%%%%%%%%%%%%%%%%%%%%%%%%%%%%%%
\begin{frame}{Specific Gravity}
	\cmini{
		\textbf{Specific gravity}, {\bfseries sg}, is the ratio of the density (or specific weight) of a substance to the density (or specific weight) of water at $4\,^{\circ}C$.
		\parb
		The specific gravity of a substance $s$ is given by:
		\begin{center}
			\begin{mybox}[width=5cm,top=-0.05cm]
				\[sg=\frac{\rho_{s}}{\rho_{w\,@\,4\,^{\circ}C}}=\frac{\gamma_{s}}{\gamma_{w\,@\,4\,^{\circ}C}}\]
			\end{mybox}
		\end{center}
		
		\pause\parb
		The density of gasoline at $25\,^{\circ}C$ is $680\; \text{kg/m}^{3}$ and the density of water at $4\,^{\circ}C$ is $1000\; \text{kg/m}^{3}$. Therefore, the specific gravity of gasoline at $25\,^{\circ}C$   is $sg=680/1000=0.68$.
		\parb
		\pause
		The specific weight of mercury at $25\,^{\circ}C$ is $132.8\; \text{kN/m}^{3}$ and the specific weight of water at $4\,^{\circ}C$ is $9.81\; \text{kN/m}^{3}$ so the specific gravity of mercury at $25\,^{\circ}C$ is $sg=13.54$.
	}
\end{frame}
%%%%%%%%%%%%%%%%%%%%%%%%%%%%%%%%%%%%%%%%%%%%%%%%%%%%%%%%%%%%%%%%%%%%%%%%%%%%%%%%%%%%%%%%%%%%%%%%%%%%%%%%%%%%%%
\begin{frame}{}
	\cmini{
		\myexample{
		Calculate the density and the specific weight of benzene which has a specific gravity of 0.876.
		}
	}
\end{frame}

%%%%%%%%%%%%%%%%%%%%%%%%%%%%%%%%%%%%%%%%%%%%%%%%%%%%%%%%%%%%%%%%%%%%%%%%%%%%%%%%%%%%%%%%%%%%%%%%%%%%%%%%%%%%%%
\begin{frame}{}
	\cmini{
		\myexample{
		An open cylindrical tank with diameter $5.75$ m and depth $3.30$~m is filled to the top with water at $10$\textcelsius.\parm
		The water is then heated to $55$\textcelsius. Assume that the tank dimensions remain constant and there are no losses due to evaporation.
		\parm
		Calculate the mass of water that overflows.
		}
	}
\end{frame}

%%%%%%%%%%%%%%%%%%%%%%%%%%%%%%%%%%%%%%%%%%%%%%%%%%%%%%%%%%%%%%%%%%%%%%%%%%%%%%%%%%%%%%%%%%%%%%%%%%%%%%%%%%%%%%
\begin{frame}{Pressure Measurement}
	Absolute and Gauge Pressure:
	\begin{itemize}
		\item Pressure measurements are made relative to some reference pressure, usually atmospheric pressure \pause \parb
		\item Pressure relative to a perfect vacuum is called \textbf{absolute pressure }
		      \parb
		\item Pressure relative to the surrounding atmosphere is called \textbf{gauge pressure}
		      \parb \pause
		\item Absolute, atmospheric and gauge pressures are related by the following expression:
		\item[]
		      \begin{center}
		      	\begin{mybox}[width=4cm,top=-0.1cm]
		      		\[ p_{abs}=p_{atm}+p_{gauge} \]
		      	\end{mybox}
		      \end{center}
	\end{itemize}
\end{frame}

%%%%%%%%%%%%%%%%%%%%%%%%%%%%%%%%%%%%%%%%%%%%%%%%%%%%%%%%%%%%%%%%%%%%%%%%%%%%%%%%%%%%%%%%%%%%%%%%%%%%%%%%%%%%%%
\begin{frame}{Absolute and Gauge Pressure}
	\cmini{
		\begin{itemize}
			\item When using a tire-gauge to check the pressure in a car or a bicycle tire,
			      the tire-gauge reports gauge pressure; this is the amount of pressure
			      in the tire in excess of the pressure of the surrounding atmosphere.\pause\parb
			\item If a tire-gauge reports a pressure of 275 kPa and the atmospheric
			      pressure is $101.3\,\text{kPa (abs)}$, then the absolute pressure inside the
			      tire is 376.3 kPa\pause\parb
			\item Normal pressure at sea-level is $101.3$ kPa (or $14.7$ psi) but changes with the weather. The highest
			      recorded pressure (adjusted to sea-level) of 110.0 kPa was measured in the middle of Siberia and the lowest of 89.1~kPa in a typhoon in the South Pacific.
			      \pause\parb
			\item Atmospheric pressure also varies with altitude. The pressure at the top of Everest is approximately one-third of that at sea-level (and contains one-third of the oxygen available at sea-level).
		\end{itemize}
	}
\end{frame}

%%%%%%%%%%%%%%%%%%%%%%%%%%%%%%%%%%%%%%%%%%%%%%%%%%%%%%%%%%%%%%%%%%%%%%%%%%%%%%%%%%%%%%%%%%%%%%%%%%%%%%%%%%%%%%
\begin{frame}{Atmospheric Pressure}
	In January, 2017, the atmospheric pressure at Calgary International Airport was reported by Environment Canada to be $101.9$ kPa:
	\begin{center}
		\begin{imgboxbg}[white]{0.45}{../../figs/01NoFPM/YYCairportPressureC.png}{1pt}{4pt}
		\end{imgboxbg}
	\end{center}
	The US measures pressure in inches of mercury:
	\begin{center}
		\begin{imgboxbg}[white]{0.45}{../../figs/01NoFPM/YYCairportPressureF.png}{1pt}{4pt}
		\end{imgboxbg}
		\pars\footnotesize
		\href{http://weather.gc.ca/city/pages/ab-52\_metric\_e.html}{http://weather.gc.ca/city/pages/ab-52\_metric\_e.html}
	\end{center}
\end{frame}
%%%%%%%%%%%%%%%%%%%%%%%%%%%%%%%%%%%%%%%%%%%%%%%%%%%%%%%%%%%%%%%%%%%%%%%%%%%%%%%%%%%%%%%%%%%%%%%%%%%%%%%%%%%%%%
\begin{frame}{Pressure and Elevation}
	\begin{itemize}
		\item Pressure decreases with increased elevation in the atmosphere; the
		      atmospheric pressure in Leh, India, ($3500$ m above sea-level) should be around $65\,$kPa. (Pressure usually decreases about $10\,$kPa for every elevation gain of $1\,000\,\text{m}$.) \pause\parm
		\item Weather stations do not normally display this difference due to elevation;
		      the stations adjust the pressure for elevation.
		      Thus, $103.1\,$kPa was the reported pressure in Leh on Monday 9th January, 2017, at 12.05pm. This adjustment enables us to know something about the weather conditions in a particular location without knowing the elevation and making calculations.\pars
		      {\footnotesize \href{http://www.worldweatheronline.com/Leh-weather/Jammu-and-Kashmir/IN.aspx}{http://www.worldweatheronline.com/Leh-weather/Jammu-and-Kashmir/IN.aspx} }
	\end{itemize}
\end{frame}

%%%%%%%%%%%%%%%%%%%%%%%%%%%%%%%%%%%%%%%%%%%%%%%%%%%%%%%%%%%%%%%%%%%%%%%%%%%%%%%%%%%%%%%%%%%%%%%%%%%%%%%%%%%%%%
\begin{frame}{Pressure and Elevation}
	\begin{itemize}
		\item Pressure increases with increased depth in a fluid. The pressure at
		      the bottom of the deep end in a swimming pool is noticeably greater
		      than just below the surface. \pause\parb
		\item In fluid mechanics, \emph{elevation} refers to the vertical distance
		      from some reference point to a point of interest. (Death Valley is
		      $86\,\text{m}$\textbf{ below sea-level}). \pause\parb
		\item Elevation relative to some reference level is usually denoted by $\bm z$. \lb A difference in elevation between two points is usually denoted by $\bm h$.
	\end{itemize}
\end{frame}

%%%%%%%%%%%%%%%%%%%%%%%%%%%%%%%%%%%%%%%%%%%%%%%%%%%%%%%%%%%%%%%%%%%%%%%%%%%%%%%%%%%%%%%%%%%%%%%%%%%%%%%%%%%%%%
\begin{frame}{Pressure and Elevation}
	
	
	The change in pressure in a homogeneous liquid at rest due to change
	in elevation is given by:
	\begin{center}
		\begin{mybox}[width=4cm]
			\[\Delta p=\gamma h\]
		\end{mybox}
	\end{center}
	
	where
	\vspace{-0.5cm}
	\begin{eqnarray*}
		\Delta p & = & \text{change  in  pressure}\\
		\gamma & = & \text{specific weight of  liquid}\\
		h & = & \text{change in elevation}
	\end{eqnarray*} \pause
	Note:
	\begin{enumerate}
		\item This equation does not apply to gases
		\item Points at the same elevation (same horizontal level) have the same
		      pressure (see Pascal's Paradox)
		\item This equation is stated for absolute values of change in pressure. \lb Pressure increases with increased depth.
	\end{enumerate}
	
\end{frame}

%%%%%%%%%%%%%%%%%%%%%%%%%%%%%%%%%%%%%%%%%%%%%%%%%%%%%%%%%%%%%%%%%%%%%%%%%%%%%%%%%%%%%%%%%%%%%%%%%%%%%%%%%%%%%%
\begin{frame}{Derivation of $\Delta p = \gamma h$}
	\mini[0.35]{
		\only<1>{
			\cfig[0.25]{../../figs/01NoFPM/02deltaPderiv_a.pdf}
			}\only<2>{
			\cfig[0.25]{../../figs/01NoFPM/02deltaPderiv_b.pdf}
			}\only<3>{
			\cfig[0.25]{../../figs/01NoFPM/02deltaPderiv_c.pdf}
			}\only<4>{
			\cfig[0.25]{../../figs/01NoFPM/02deltaPderiv_d.pdf}
			}\only<5>{
			\cfig[0.25]{../../figs/01NoFPM/02deltaPderiv_e.pdf}
		}
	}
	\hfill
	\mini[0.6]{
		\begin{mybox}[]
			\begin{itemize}
				\item Consider a cylindrical volume of liquid within a body of liquid. Let the cylinder have height $h$ and cross-sectional area $A$. \pause
				\item The pressure, $p_1$, on the top surface of the
				      cylinder is uniform since the surface is horizontal (all points are at the same depth). \pause
				\item	The force exerted on the top surface is $F_{down}=p_1 A$. \pause
				\item Similarly the pressure, $p_2$, on the bottom surface of the cylinder is uniform \ldots \pause
				\item \ldots and $F_{up}=p_2 A$.
			\end{itemize}
		\end{mybox}
	}
\end{frame}

%%%%%%%%%%%%%%%%%%%%%%%%%%%%%%%%%%%%%%%%%%%%%%%%%%%%%%%%%%%%%%%%%%%%%%%%%%%%%%%%%%%%%%%%%%%%%%%%%%%%%%%%%%%%%%
\begin{frame}{Derivation of $\Delta p = \gamma h$}
	\mini[0.35]{
		\only<1>{
			\cfig[0.25]{../../figs/01NoFPM/02deltaPderiv_f.pdf}
			}\only<2->{
			\cfig[0.25]{../../figs/01NoFPM/02deltaPderiv_g.pdf}
		}
	}
	\hfill
	\mini[0.6]{
		\begin{mybox}[]
			\begin{itemize}
				\item The other force to be considered is the weight, $w$, of the cylindrical volume \pause
				\item $w=\gamma\cdot V$\pause
				\item	The cylinder is in equilibrium so \vspace{-0.25cm}
				      \[\Sigma F_{y}=p_{2}A-\gamma V-p_{1}A=0\]\pause\vspace{-0.75cm}
				\item $V=Ah$ so \vspace{-0.25cm}
				      \begin{eqnarray*}
				      	p_2A-\gamma\cdot Ah-p_1A & = & 0\\
				      	p_2-\gamma h-p_1 & = & 0\\
				      	p_2-p_1 & = & \gamma h\\
				      	\Delta p & = & \gamma h
				      \end{eqnarray*}
			\end{itemize}
		\end{mybox}
	}
\end{frame}



%%%%%%%%%%%%%%%%%%%%%%%%%%%%%%%%%%%%%%%%%%%%%%%%%%%%%%%%%%%%%%%%%%%%%%%%%%%%%%%%%%%%%%%%%%%%%%%%%%%%%%%%%%%%%%

\begin{frame}{Pascal's Paradox}
	\begin{center}
		\cfig[0.3]{../../figs/01NofPM/pascalsParadox}
		
		\begin{mybox}[width=8cm, title=Pascal's Paradox]
			\raggedright
			All three vessels contain the same liquid. The pressure at the bottom
			of each vessel is the same because pressure is due only to the depth
			of liquid.
			
		\end{mybox}
	\end{center}
\end{frame}

%%%%%%%%%%%%%%%%%%%%%%%%%%%%%%%%%%%%%%%%%%%%%%%%%%%%%%%%%%%%%%%%%%%%%%%%%%%%%%%%%%%%%%%%%%%%%%%%%%%%%%%%%%%%%%
\begin{frame}{Water Tower}
	\begin{center}
		\cfig[0.24]{../../figs/01NoFPM/waterTower1.png}
	\end{center}
\end{frame}
%%%%%%%%%%%%%%%%%%%%%%%%%%%%%%%%%%%%%%%%%%%%%%%%%%%%%%%%%%%%%%%%%%%%%%%%%%%%%%%%%%%%%%%%%%%%%%%%%%%%%%%%%%%%%%

\begin{frame}{Water Tower}
	\begin{center}
		\cfig[0.3]{../../figs/01NoFPM/waterTower2.png}
	\end{center}
\end{frame}

%%%%%%%%%%%%%%%%%%%%%%%%%%%%%%%%%%%%%%%%%%%%%%%%%%%%%%%%%%%%%%%%%%%%%%%%%%%%%%%%%%%%%%%%%%%%%%%%%%%%%%%%%%%%%%
\begin{frame}%{Pressure Measurement}
	\definecolor{example}{RGB}{228,164,5}
	\begin{myexam}[colframe=example, colbacktitle=example]{}{}
		A tank, open to the atmosphere in the centre, contains medium fuel
		oil. Atmospheric pressure is 102.1 kPa. Calculate the gauge pressure
		and the absolute pressure for locations \emph{A, B} and \emph{D}.
	\end{myexam}
	
	% \vspace{-0.5cm}
	\begin{center}
		\cfig[0.5]{../../figs/01NoFPM/02tank1}
	\end{center}
\end{frame}

%%%%%%%%%%%%%%%%%%%%%%%%%%%%%%%%%%%%%%%%%%%%%%%%%%%%%%%%%%%%%%%%%%%%%%%%%%%%%%%%%%%%%%%%%%%%%%%%%%%%%%%%%%%%%%
\begin{frame}%{Pressure Measurement}
	\definecolor{example}{RGB}{228,164,5}
	\begin{myexer}[colframe=example, colbacktitle=example]{}{}
		A tank, open to the atmosphere in the centre, contains medium fuel
		oil. Atmospheric pressure is 102.1 kPa. Calculate the gauge pressure
		and the absolute pressure for locations \emph{C} and \emph{E}.
	\end{myexer}
	
	% \vspace{-0.5cm}
	\begin{center}
		\cfig[0.5]{../../figs/01NoFPM/02tank1}
	\end{center}
\end{frame}

%%%%%%%%%%%%%%%%%%%%%%%%%%%%%%%%%%%%%%%%%%%%%%%%%%%%%%%%%%%%%%%%%%%%%%%%%%%%%%%%%%%%%%%%%%%%%%%%%%%%%%%%%%%%%%

\begin{frame}{Manometers}
	\mini[0.4]{
		\begin{imgboxbg}[white]{0.4}{../../figs/01NoFPM/imgCIVL250M02QBblank.pdf}{1pt}{10pt}\end{imgboxbg}
	}
	\hfill
	\mini[0.55]{
		\begin{itemize}
			\item A manometer is a pressure-measuring instrument.
			\item It uses the height of a liquid column, $h$, to measure the pressure
			      difference between two locations.
			\item This manometer is open to the atmosphere at one end; it is used to
			      measure the difference in pressure between $A$ and the atmosphere
			      (that is, it measures gauge pressure at $A$).
		\end{itemize}
	}
\end{frame}

%%%%%%%%%%%%%%%%%%%%%%%%%%%%%%%%%%%%%%%%%%%%%%%%%%%%%%%%%%%%%%%%%%%%%%%%%%%%%%%%%%%%%%%%%%%%%%%%%%%%%%%%%%%%%%
\begin{frame}
	\cmini{
		\begin{myexam}{}{}
			Determine the pressure at $A$ given that the temperature of the
			water is $25$\textcelsius.
			\cfig[0.55]{../../figs/01NoFPM/imgCIVL250M02QB}
		\end{myexam}
	}
	
	
	
\end{frame}

%%%%%%%%%%%%%%%%%%%%%%%%%%%%%%%%%%%%%%%%%%%%%%%%%%%%%%%%%%%%%%%%%%%%%%%%%%%%%%%%%%%%%%%%%%%%%%%%%%%%%%%%%%%%%%


\begin{frame}{Differential Manometers}
	\centering
	\begin{imgboxbg}[white]{0.45}{../../figs/01NoFPM/imgCIVL250M02QAblank.pdf}{1pt}{10pt}\end{imgboxbg}
	\begin{itemize}
		\item The \textbf{ differential manometer} illustrated is used to
		      measure the difference in pressure between $A$ and $B$.
	\end{itemize}
\end{frame}

%%%%%%%%%%%%%%%%%%%%%%%%%%%%%%%%%%%%%%%%%%%%%%%%%%%%%%%%%%%%%%%%%%%%%%%%%%%%%%%%%%%%%%%%%%%%%%%%%%%%%%%%%%%%%%


\begin{frame}
	\cmini{
		\begin{myexam}{}{}
			Determine the pressure difference between $A$ and $B$.
			\cfig[0.45]{../../figs/01NoFPM/imgCIVL250M02QFsolved.pdf}
		\end{myexam}
	}
	
	
	
	
\end{frame}


\end{document}
